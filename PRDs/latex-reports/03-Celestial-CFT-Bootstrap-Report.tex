\documentclass[11pt,a4paper]{article}

% ============================================================
% PACKAGES
% ============================================================
\usepackage[utf8]{inputenc}
\usepackage[T1]{fontenc}
\usepackage{amsmath,amssymb,amsthm}
\usepackage{mathtools}
\usepackage{physics}
\usepackage{geometry}
\usepackage{hyperref}
\usepackage{xcolor}
\usepackage{listings}
\usepackage{tcolorbox}
\usepackage{enumitem}
\usepackage{booktabs}
\usepackage{graphicx}
\usepackage{fancyhdr}
\usepackage{tikz}
\usepackage{tensor}
\usepackage{bbm}

% ============================================================
% PAGE SETUP
% ============================================================
\geometry{margin=1in}
\hypersetup{
    colorlinks=true,
    linkcolor=blue!70!black,
    citecolor=green!50!black,
    urlcolor=purple!70!black
}

% ============================================================
% CODE LISTINGS SETUP
% ============================================================
\definecolor{codegreen}{rgb}{0,0.6,0}
\definecolor{codegray}{rgb}{0.5,0.5,0.5}
\definecolor{codepurple}{rgb}{0.58,0,0.82}
\definecolor{backcolour}{rgb}{0.95,0.95,0.92}

\lstdefinestyle{pythonstyle}{
    backgroundcolor=\color{backcolour},
    commentstyle=\color{codegreen},
    keywordstyle=\color{magenta},
    numberstyle=\tiny\color{codegray},
    stringstyle=\color{codepurple},
    basicstyle=\ttfamily\footnotesize,
    breakatwhitespace=false,
    breaklines=true,
    captionpos=b,
    keepspaces=true,
    numbers=left,
    numbersep=5pt,
    showspaces=false,
    showstringspaces=false,
    showtabs=false,
    tabsize=2,
    frame=single,
    language=Python
}

\lstset{style=pythonstyle}

% ============================================================
% THEOREM ENVIRONMENTS
% ============================================================
\theoremstyle{definition}
\newtheorem{definition}{Definition}[section]
\newtheorem{theorem}{Theorem}[section]
\newtheorem{lemma}[theorem]{Lemma}
\newtheorem{proposition}[theorem]{Proposition}
\newtheorem{corollary}[theorem]{Corollary}
\newtheorem{remark}{Remark}[section]

% ============================================================
% CUSTOM COMMANDS
% ============================================================
\newcommand{\calA}{\mathcal{A}}
\newcommand{\calM}{\mathcal{M}}
\newcommand{\calO}{\mathcal{O}}
\newcommand{\SL}{\mathrm{SL}}
\newcommand{\Tr}{\mathrm{Tr}}
\newcommand{\Z}{\mathbb{Z}}
\newcommand{\R}{\mathbb{R}}
\newcommand{\C}{\mathbb{C}}
\newcommand{\CP}{\mathbb{CP}}
\newcommand{\scrI}{\mathscr{I}}
\newcommand{\CFT}{\mathrm{CFT}}
\newcommand{\OPE}{\mathrm{OPE}}

% ============================================================
% ANNOTATION BOX
% ============================================================
\newtcolorbox{annotation}[1][]{
    colback=blue!5!white,
    colframe=blue!75!black,
    fonttitle=\bfseries,
    title={Analysis Note},
    #1
}

\newtcolorbox{pursuitbox}[1][]{
    colback=green!5!white,
    colframe=green!60!black,
    fonttitle=\bfseries,
    title={Research Direction},
    #1
}

\newtcolorbox{warningbox}[1][]{
    colback=red!5!white,
    colframe=red!75!black,
    fonttitle=\bfseries,
    title={Critical Consideration},
    #1
}

\newtcolorbox{physicsbox}[1][]{
    colback=orange!5!white,
    colframe=orange!70!black,
    fonttitle=\bfseries,
    title={Physical Insight},
    #1
}

\newtcolorbox{mathbox}[1][]{
    colback=purple!5!white,
    colframe=purple!70!black,
    fonttitle=\bfseries,
    title={Mathematical Structure},
    #1
}

% ============================================================
% DOCUMENT BEGIN
% ============================================================
\begin{document}

% ============================================================
% TITLE PAGE
% ============================================================
\begin{titlepage}
    \centering
    \vspace*{2cm}

    {\Huge\bfseries Challenge 03:\\[0.5em]
    Celestial CFT Bootstrap\par}

    \vspace{1.5cm}

    {\Large\itshape Comprehensive Technical Report\par}

    \vspace{2cm}

    \begin{tabular}{ll}
        \textbf{Domain:} & Quantum Gravity \& Particle Physics \\
        \textbf{Difficulty:} & High \\
        \textbf{Timeline:} & 6--12 months \\
        \textbf{Prerequisites:} & Scattering amplitudes, conformal field theory, \\
        & Mellin transforms, representation theory
    \end{tabular}

    \vfill

    {\large Pure Thought AI Challenges\par}
    {\large\today\par}
\end{titlepage}

\tableofcontents
\newpage

% ============================================================
% SECTION 1: EXECUTIVE SUMMARY
% ============================================================
\section{Executive Summary}

\textbf{Celestial holography} represents a revolutionary approach to understanding quantum gravity in flat spacetime. It reformulates 4D flat-space scattering amplitudes as correlation functions in a 2D conformal field theory living on the \textbf{celestial sphere} at null infinity. This challenge aims to apply the powerful \textbf{conformal bootstrap} methodology to constrain and classify consistent celestial CFTs compatible with graviton scattering.

\begin{annotation}
Unlike AdS/CFT, which relates gravity in anti-de Sitter space to CFT on its boundary, celestial holography provides a holographic description for \emph{asymptotically flat} spacetimes---the relevant arena for actual gravitational wave observations and particle physics experiments.
\end{annotation}

The key innovation is the \textbf{Mellin transform}, which converts momentum-space amplitudes into celestial amplitudes carrying definite conformal weights. These celestial amplitudes satisfy conformal Ward identities, crossing symmetry, and soft theorems---providing a rich set of constraints amenable to bootstrap analysis.

% ============================================================
% SECTION 2: SCIENTIFIC CONTEXT
% ============================================================
\section{Scientific Context and Motivation}

\subsection{From AdS/CFT to Flat Space Holography}

The AdS/CFT correspondence has been spectacularly successful, but the real universe is not Anti-de Sitter---it is asymptotically flat (or de Sitter, accounting for dark energy). The quest for a \textbf{flat space holographic principle} has led to several approaches:
\begin{itemize}
    \item BMS symmetry and soft theorems (Strominger et al.)
    \item Carrollian CFT on null infinity
    \item \textbf{Celestial holography:} 2D CFT on the celestial sphere
\end{itemize}

\begin{physicsbox}
\textbf{The Celestial Sphere:} Consider 4D Minkowski space $\mathbb{R}^{1,3}$. Null infinity $\scrI^{\pm}$ has topology $\mathbb{R} \times S^2$. The $S^2$ factor is the \textbf{celestial sphere}---the sphere of directions from which massless particles can arrive or depart. Celestial holography posits that 4D gravity is equivalent to a 2D CFT on this $S^2$.
\end{physicsbox}

\subsection{Celestial Amplitudes via Mellin Transform}

The key mathematical operation is the \textbf{Mellin transform} from momentum space to the ``celestial basis'':

\begin{definition}[Celestial Amplitude]
Given a momentum-space amplitude $A(\omega_i, z_i, \bar{z}_i)$ for $n$ massless particles, the celestial amplitude is:
\begin{equation}
    \boxed{\tilde{A}(\Delta_i, z_i, \bar{z}_i) = \int_0^\infty \prod_{i=1}^n d\omega_i \, \omega_i^{\Delta_i - 1} \, A(\omega_i, z_i, \bar{z}_i)}
\end{equation}
where:
\begin{itemize}
    \item $\omega_i$ are energy variables (with momenta parametrized as $p_i^\mu = \omega_i (1, \hat{n}_i)$)
    \item $(z_i, \bar{z}_i)$ are stereographic coordinates on the celestial sphere
    \item $\Delta_i$ are \textbf{conformal weights} (generally complex)
\end{itemize}
\end{definition}

\begin{annotation}
The Mellin transform trades energy $\omega$ for conformal weight $\Delta$. Particles with definite energy become superpositions over the principal continuous series of $\SL(2,\C)$ representations. The resulting celestial amplitudes transform as CFT correlators under Lorentz transformations.
\end{annotation}

\subsection{The Core Question}

\begin{tcolorbox}[colback=yellow!10!white,colframe=orange!80!black,title=\textbf{Central Research Question}]
\textbf{What is the consistent space of celestial CFTs compatible with graviton scattering amplitudes?}

The celestial CFT must satisfy:
\begin{enumerate}
    \item $\SL(2,\C)$ covariance (celestial conformal symmetry)
    \item Crossing symmetry (OPE associativity)
    \item Unitarity (positive norms / positive spectral density)
    \item Weinberg soft graviton theorem
    \item Regge boundedness at high energies
\end{enumerate}
\end{tcolorbox}

\subsection{Why This Matters}

\begin{enumerate}[label=\textbf{(\arabic*)}]
    \item \textbf{New Holographic Paradigm:} Connects 4D gravity to 2D CFT without requiring AdS geometry---directly applicable to the real universe.

    \item \textbf{Rigorous Constraints:} The bootstrap approach carves out the space of consistent theories using only fundamental principles.

    \item \textbf{Testable Predictions:} Produces ``islands'' (allowed regions) and ``no-go'' exclusions that can be verified by explicit amplitude calculations.

    \item \textbf{Unification:} May connect to the broader program of flat-space holography, including BMS symmetry and memory effects.
\end{enumerate}

% ============================================================
% SECTION 3: MATHEMATICAL FORMULATION
% ============================================================
\section{Mathematical Formulation}

\subsection{Kinematics: Celestial Sphere Coordinates}

A null 4-momentum in Minkowski space can be parametrized as:
\begin{equation}
    p^\mu = \omega \, q^\mu(z, \bar{z}), \quad q^\mu = (1 + z\bar{z}, z + \bar{z}, -i(z - \bar{z}), 1 - z\bar{z})
\end{equation}
where $(z, \bar{z}) \in \C$ are stereographic coordinates on $S^2 \cong \CP^1$.

\begin{mathbox}
\textbf{Lorentz Group Action:} The Lorentz group $\SL(2,\C)$ acts on the celestial sphere via Möbius transformations:
\begin{equation}
    z \mapsto \frac{az + b}{cz + d}, \quad \begin{pmatrix} a & b \\ c & d \end{pmatrix} \in \SL(2,\C)
\end{equation}
This is precisely the global conformal group of a 2D CFT!
\end{mathbox}

\subsection{Conformal Primary Wavefunctions}

Instead of plane waves $e^{ip \cdot x}$, celestial holography uses \textbf{conformal primary wavefunctions}:
\begin{equation}
    \phi_{\Delta}^{\pm}(X; z, \bar{z}) = \int_0^\infty d\omega \, \omega^{\Delta - 1} e^{\pm i\omega q \cdot X}
\end{equation}

These transform under Lorentz transformations as:
\begin{equation}
    \phi_{\Delta}^{\pm}(X; z, \bar{z}) \to |cz + d|^{-2\Delta} \phi_{\Delta}^{\pm}(X; z', \bar{z}')
\end{equation}
which is precisely the transformation of a \textbf{CFT primary operator} with weights $(\Delta, \bar{\Delta}) = (\Delta/2, \Delta/2)$ for scalars.

\subsection{Celestial Amplitudes as CFT Correlators}

The celestial amplitude $\tilde{A}(\Delta_i, z_i, \bar{z}_i)$ transforms as an $n$-point correlator in 2D CFT:
\begin{equation}
    \tilde{A}(\Delta_i, z_i, \bar{z}_i) \sim \left\langle \calO_{\Delta_1}(z_1, \bar{z}_1) \cdots \calO_{\Delta_n}(z_n, \bar{z}_n) \right\rangle_{\CFT}
\end{equation}

\begin{annotation}
\textbf{Key Insight:} This is not a metaphor---celestial amplitudes \emph{are} CFT correlators. They satisfy conformal Ward identities, have OPE expansions, and (conjecturally) arise from some underlying 2D CFT on the celestial sphere.
\end{annotation}

\subsection{Constraints on Celestial CFT}

\subsubsection{Constraint 1: $\SL(2,\C)$ Covariance}

Celestial amplitudes must transform covariantly under Lorentz (= conformal) transformations:
\begin{equation}
    \tilde{A}(\Delta_i, z_i, \bar{z}_i) \to \prod_i |cz_i + d|^{-2\Delta_i} \tilde{A}(\Delta_i, z_i', \bar{z}_i')
\end{equation}

This determines the kinematic structure of $n$-point functions up to conformally invariant cross-ratios.

\subsubsection{Constraint 2: Crossing Symmetry (OPE Associativity)}

The \textbf{Operator Product Expansion (OPE)} in the celestial CFT:
\begin{equation}
    \calO_{\Delta_1}(z_1) \calO_{\Delta_2}(z_2) = \sum_{\Delta} C_{\Delta_1 \Delta_2}^{\Delta} \frac{\calO_\Delta(z_2)}{(z_1 - z_2)^{\Delta_1 + \Delta_2 - \Delta}}
\end{equation}

\textbf{Crossing symmetry} requires that the OPE is \emph{associative}:
\begin{equation}
    \boxed{(\calO_1 \calO_2) \calO_3 = \calO_1 (\calO_2 \calO_3)}
\end{equation}

For 4-point functions, this gives the \textbf{crossing equation}:
\begin{equation}
    \sum_{\Delta_s} C_s^2(\Delta_s) G_{\Delta_s}(z, \bar{z}) = \sum_{\Delta_t} C_t^2(\Delta_t) G_{\Delta_t}(1-z, 1-\bar{z})
\end{equation}
where $G_\Delta$ are conformal blocks.

\subsubsection{Constraint 3: Unitarity}

For a unitary CFT, the OPE coefficients squared must be non-negative:
\begin{equation}
    \boxed{|C_{\Delta_1 \Delta_2}^\Delta|^2 \geq 0}
\end{equation}

This is the \textbf{positivity condition} that makes the bootstrap a convex optimization problem.

\begin{warningbox}
\textbf{Subtlety:} Celestial CFT involves the \textbf{principal continuous series} of $\SL(2,\C)$, where $\Delta = 1 + i\lambda$ with $\lambda \in \R$. Unitarity must be properly defined for this non-compact situation.
\end{warningbox}

\subsubsection{Constraint 4: Soft Graviton Theorem}

Weinberg's \textbf{soft graviton theorem} states that in the limit where one graviton becomes soft ($\omega \to 0$):
\begin{equation}
    A_{n+1}(\omega \to 0) \sim \frac{S^{(0)}}{\omega} A_n + S^{(1)} A_n + O(\omega)
\end{equation}

In the celestial basis, this becomes a \textbf{pole structure}:
\begin{equation}
    \boxed{\tilde{A}(\Delta \to 0) \sim \frac{S^{(0)}}{\Delta} + \frac{S^{(1)}}{\Delta^2} + \cdots}
\end{equation}

where $S^{(0)}$ is the Weinberg soft factor (sum of $\varepsilon \cdot p_i / p \cdot p_i$ terms).

\subsubsection{Constraint 5: Regge Boundedness}

At high energies, the amplitude must be bounded:
\begin{equation}
    |\tilde{A}(\Delta)| \lesssim |\Delta|^N \quad \text{as } |\Delta| \to \infty
\end{equation}

This ensures convergence of OPE sums and dispersion relations.

\subsection{Bootstrap Formulation}

\begin{tcolorbox}[colback=blue!5!white,colframe=blue!75!black,title=\textbf{Celestial Bootstrap Problem}]
\textbf{Find:} Celestial OPE data $\{C_{ijk}, \Delta_i\}$

\textbf{Satisfying:}
\begin{enumerate}
    \item $\SL(2,\C)$ covariance
    \item Crossing symmetry (OPE associativity)
    \item Unitarity ($C^2 \geq 0$ or appropriate positivity for continuous spectrum)
    \item Soft theorems (pole structure at $\Delta \to 0, 1$)
    \item Regge boundedness
\end{enumerate}

\textbf{Goal:} Either construct explicit OPE data, or prove no solution exists for given assumptions (e.g., minimal gap $\Delta_{\text{gap}}$).
\end{tcolorbox}

% ============================================================
% SECTION 4: IMPLEMENTATION APPROACH
% ============================================================
\section{Implementation Approach}

\subsection{Phase 1: Celestial Amplitude Calculator (Months 1--2)}

\textbf{Goal:} Build a Mellin transform engine to compute celestial amplitudes from momentum-space expressions.

\subsubsection{Mellin Transform Implementation}

\begin{lstlisting}[language=Python, caption={Mellin transform for celestial amplitudes}]
from mpmath import mp
import numpy as np
from scipy.integrate import quad

mp.dps = 50  # High precision arithmetic

def mellin_transform(amplitude_func, omega_vars: list,
                     delta_vars: list, z_vars: list) -> complex:
    """
    Compute celestial amplitude via Mellin transform

    A_tilde(Delta_i) = integral d^n omega prod_i omega_i^{Delta_i-1} A(omega_i)

    Args:
        amplitude_func: Momentum-space amplitude A(omega, z, zbar)
        omega_vars: List of energy integration variables
        delta_vars: List of conformal weights (can be complex!)
        z_vars: List of celestial coordinates (z, zbar) pairs

    Returns:
        Celestial amplitude value (complex)
    """
    n_particles = len(omega_vars)

    def integrand(*omegas):
        # Momentum-space amplitude
        A = amplitude_func(omegas, z_vars)

        # Mellin weight factors
        weight = 1.0
        for i, (omega, delta) in enumerate(zip(omegas, delta_vars)):
            weight *= omega**(delta - 1)

        return A * weight

    # Multi-dimensional integration
    # For principal series: Delta = 1 + i*lambda, integrate along contour
    result = multi_integrate(integrand, [(0, np.inf)] * n_particles)

    return result


def multi_integrate(func, limits, method='adaptive'):
    """
    Multi-dimensional numerical integration

    Uses nested quadrature for moderate dimensions,
    Monte Carlo for high dimensions.
    """
    from scipy.integrate import nquad

    result, error = nquad(func, limits)
    return result
\end{lstlisting}

\subsubsection{Three-Point Graviton Amplitude}

\begin{lstlisting}[language=Python, caption={Celestial 3-point graviton amplitude}]
def graviton_3pt_momentum(omega1, omega2, omega3, z1, z2, z3,
                           zbar1, zbar2, zbar3, helicities):
    """
    Tree-level 3-graviton amplitude in momentum space

    For (+++) or (---) helicity configurations, this vanishes.
    For mixed helicity, involves spinor-helicity brackets.

    Args:
        omega_i: Energies
        z_i, zbar_i: Celestial coordinates
        helicities: Tuple of helicities (+1 or -1)

    Returns:
        Amplitude (includes momentum-conserving delta function)
    """
    # Spinor-helicity brackets
    # <ij> = sqrt(omega_i * omega_j) * (zbar_i - zbar_j)
    # [ij] = sqrt(omega_i * omega_j) * (z_i - z_j)

    h1, h2, h3 = helicities

    # MHV amplitude: (++-) configuration
    if h1 == 1 and h2 == 1 and h3 == -1:
        bracket_12 = np.sqrt(omega1 * omega2) * (zbar1 - zbar2)
        bracket_23 = np.sqrt(omega2 * omega3) * (zbar2 - zbar3)
        bracket_31 = np.sqrt(omega3 * omega1) * (zbar3 - zbar1)

        # Parke-Taylor-like structure
        amplitude = bracket_12**4 / (bracket_12 * bracket_23 * bracket_31)

    else:
        amplitude = 0  # Other helicity configurations

    # Momentum conservation (delta function)
    # Handled by integration measure

    return amplitude


def celestial_3pt_graviton(Delta1, Delta2, Delta3, z1, z2, z3,
                            zbar1, zbar2, zbar3, helicities):
    """
    Celestial 3-point graviton amplitude

    After Mellin transform, this should have pure conformal structure:
    A_tilde ~ C_123 / ((z12)^{a} (z23)^{b} (z31)^{c})
    with exponents determined by conformal weights.
    """
    def amplitude_func(omegas, z_data):
        omega1, omega2, omega3 = omegas
        return graviton_3pt_momentum(
            omega1, omega2, omega3,
            z1, z2, z3, zbar1, zbar2, zbar3,
            helicities
        )

    return mellin_transform(
        amplitude_func,
        [omega1, omega2, omega3],
        [Delta1, Delta2, Delta3],
        [(z1, zbar1), (z2, zbar2), (z3, zbar3)]
    )
\end{lstlisting}

\begin{annotation}
\textbf{Expected Structure:} The celestial 3-point amplitude should take the form:
\begin{equation}
    \tilde{A}_3 = C(\Delta_i) \cdot \frac{1}{z_{12}^{\Delta_1 + \Delta_2 - \Delta_3} z_{23}^{\Delta_2 + \Delta_3 - \Delta_1} z_{31}^{\Delta_3 + \Delta_1 - \Delta_2}}
\end{equation}
Verify this structure numerically as a consistency check.
\end{annotation}

\subsection{Phase 2: Conformal Block Decomposition (Months 2--4)}

\textbf{Goal:} Implement celestial conformal blocks and OPE decomposition.

\subsubsection{Celestial Conformal Blocks}

\begin{lstlisting}[language=Python, caption={Celestial conformal blocks}]
from mpmath import hyp2f1

def celestial_conformal_block(Delta, z, zbar, Delta_ext):
    """
    Conformal block for celestial CFT

    For 2D CFT on the sphere, conformal blocks are hypergeometric:
    G_Delta(z, zbar) = z^{Delta/2} zbar^{Delta/2} F(Delta, ..., z) F(..., zbar)

    For celestial CFT with continuous spectrum, use principal series.

    Args:
        Delta: Internal conformal dimension (can be complex: 1 + i*lambda)
        z, zbar: Cross-ratio coordinates
        Delta_ext: External dimensions [Delta_1, Delta_2, Delta_3, Delta_4]

    Returns:
        Conformal block value
    """
    Delta1, Delta2, Delta3, Delta4 = Delta_ext

    # Holomorphic and anti-holomorphic parts
    h = Delta / 2  # Holomorphic weight
    hbar = Delta / 2  # Anti-holomorphic weight

    # For scalar external operators
    a = (Delta1 - Delta2) / 2
    b = (Delta3 - Delta4) / 2

    # Hypergeometric function for holomorphic block
    # g_h(z) = z^h * 2F1(h-a, h+b; 2h; z)
    g_h = z**h * float(hyp2f1(h - a, h + b, 2*h, z))
    g_hbar = zbar**hbar * float(hyp2f1(hbar - a, hbar + b, 2*hbar, zbar))

    return g_h * g_hbar


def conformal_block_series(Delta, z, zbar, Delta_ext, order=20):
    """
    Conformal block via series expansion for better numerics

    Uses the Zamolodchikov recursion relation for efficiency.
    """
    # Implementation of recursion
    # G_Delta = z^Delta * sum_n c_n(Delta) z^n
    pass
\end{lstlisting}

\subsubsection{OPE Decomposition}

\begin{lstlisting}[language=Python, caption={OPE decomposition of celestial amplitudes}]
import numpy as np
from scipy.optimize import minimize

def ope_decomposition(celestial_4pt, z_grid, zbar_grid,
                       Delta_spectrum, Delta_ext):
    """
    Decompose celestial 4-point function into conformal blocks

    A_tilde_4 = sum_Delta C^2(Delta) G_Delta(z, zbar)

    Args:
        celestial_4pt: Function or array of 4-point amplitude values
        z_grid, zbar_grid: Grid of cross-ratio values
        Delta_spectrum: List of conformal dimensions to include
        Delta_ext: External operator dimensions

    Returns:
        Dictionary {Delta: C^2} of OPE coefficients squared
    """
    n_points = len(z_grid)
    n_ops = len(Delta_spectrum)

    # Build matrix of conformal blocks
    G_matrix = np.zeros((n_points, n_ops), dtype=complex)
    for i, (z, zbar) in enumerate(zip(z_grid, zbar_grid)):
        for j, Delta in enumerate(Delta_spectrum):
            G_matrix[i, j] = celestial_conformal_block(
                Delta, z, zbar, Delta_ext
            )

    # Amplitude values at grid points
    A_values = np.array([celestial_4pt(z, zbar)
                        for z, zbar in zip(z_grid, zbar_grid)])

    # Solve for OPE coefficients: A = G @ C^2
    # Use non-negative least squares for unitarity
    from scipy.optimize import nnls
    C_squared, residual = nnls(G_matrix.real, A_values.real)

    return {Delta: c2 for Delta, c2 in zip(Delta_spectrum, C_squared)}
\end{lstlisting}

\subsection{Phase 3: Crossing Equations and Soft Theorems (Months 4--6)}

\textbf{Goal:} Formulate crossing symmetry and soft theorem constraints.

\subsubsection{Crossing Equation}

\begin{lstlisting}[language=Python, caption={Crossing symmetry verification}]
def crossing_equation(ope_data_s, ope_data_t, z, zbar, Delta_ext):
    """
    Verify/impose s-channel = t-channel OPE

    sum_{Delta_s} C_s^2(Delta_s) G_{Delta_s}(z, zbar)
        = sum_{Delta_t} C_t^2(Delta_t) G_{Delta_t}(1-z, 1-zbar)

    Args:
        ope_data_s: Dict {Delta: C^2} for s-channel
        ope_data_t: Dict {Delta: C^2} for t-channel
        z, zbar: Cross-ratio point
        Delta_ext: External dimensions

    Returns:
        Crossing residual (should be ~ 0)
    """
    # s-channel sum
    lhs = sum(C2 * celestial_conformal_block(Delta, z, zbar, Delta_ext)
              for Delta, C2 in ope_data_s.items())

    # t-channel sum (note: z -> 1-z)
    rhs = sum(C2 * celestial_conformal_block(Delta, 1-z, 1-zbar, Delta_ext)
              for Delta, C2 in ope_data_t.items())

    return abs(lhs - rhs)


def verify_crossing_symmetry(ope_data, z_grid, tol=1e-6):
    """
    Verify crossing symmetry at multiple points
    """
    violations = []
    for z, zbar in z_grid:
        residual = crossing_equation(ope_data, ope_data, z, zbar, ...)
        if residual > tol:
            violations.append((z, zbar, residual))

    return len(violations) == 0, violations
\end{lstlisting}

\subsubsection{Soft Theorem Constraints}

\begin{lstlisting}[language=Python, caption={Soft graviton theorem implementation}]
def impose_soft_theorem(ope_data, Delta_soft=0):
    """
    Impose Weinberg soft graviton theorem

    In celestial basis, as Delta -> 0:
    A_tilde(Delta) ~ S^(0)/Delta + S^(1)/Delta^2 + ...

    S^(0) is the leading soft factor (sum of epsilon.p_i / p.p_i)

    Args:
        ope_data: Current OPE data
        Delta_soft: Small conformal weight (approaching soft limit)

    Returns:
        Constraint residual
    """
    # Extract behavior near Delta = 0
    def celestial_amplitude_near_soft(Delta, z, zbar):
        # Reconstruct amplitude from OPE
        return sum(C2 * celestial_conformal_block(Delta_int, z, zbar, ...)
                   for Delta_int, C2 in ope_data.items())

    # Check pole structure
    epsilon = 1e-4
    A_eps = celestial_amplitude_near_soft(epsilon, z_test, zbar_test)

    # Leading pole: should go like 1/Delta
    residue_0 = A_eps * epsilon

    # Compare to Weinberg soft factor
    weinberg_factor = compute_weinberg_soft(z_test, zbar_test, ...)

    return abs(residue_0 - weinberg_factor)


def compute_weinberg_soft(z, zbar, external_data):
    """
    Compute Weinberg soft graviton factor

    S^(0) = sum_i (epsilon.p_i) / (p_soft.p_i)

    For celestial coordinates, this becomes a sum over external
    operators weighted by their positions on the celestial sphere.
    """
    S0 = 0
    for i, (z_i, zbar_i, Delta_i) in enumerate(external_data):
        # Soft factor contribution from particle i
        S0 += (zbar - zbar_i) / ((z - z_i) * (zbar - zbar_i))

    return S0
\end{lstlisting}

\subsection{Phase 4: Bootstrap SDP (Months 6--9)}

\textbf{Goal:} Formulate and solve the celestial bootstrap as a semidefinite program.

\begin{lstlisting}[language=Python, caption={Celestial bootstrap SDP formulation}]
import cvxpy as cp
import numpy as np

def setup_celestial_bootstrap_sdp(Delta_gap, Delta_max,
                                   z_grid, Delta_ext):
    """
    Set up SDP for celestial CFT bootstrap

    Variables: OPE coefficients C^2(Delta) for Delta in spectrum
    Constraints:
        1. Positivity: C^2(Delta) >= 0 (unitarity)
        2. Crossing symmetry: sum C^2 G_s = sum C^2 G_t
        3. Soft theorem: correct pole structure
        4. Normalization

    Args:
        Delta_gap: Minimum conformal dimension (gap assumption)
        Delta_max: Maximum dimension to include
        z_grid: Cross-ratio grid for sampling constraints
        Delta_ext: External operator dimensions

    Returns:
        cvxpy Problem object
    """
    # Discretize spectrum above the gap
    # For continuous spectrum: use Gauss quadrature points
    Delta_spectrum = np.linspace(Delta_gap, Delta_max, num=100)
    n_ops = len(Delta_spectrum)

    # Variables: OPE coefficients squared
    C_squared = cp.Variable(n_ops, nonneg=True)

    constraints = []

    # 1. Positivity: automatic from nonneg=True

    # 2. Crossing symmetry at each z point
    for z, zbar in z_grid:
        # s-channel blocks
        G_s = np.array([celestial_conformal_block(D, z, zbar, Delta_ext)
                       for D in Delta_spectrum])
        # t-channel blocks
        G_t = np.array([celestial_conformal_block(D, 1-z, 1-zbar, Delta_ext)
                       for D in Delta_spectrum])

        # Crossing: sum C^2 G_s = sum C^2 G_t
        constraints.append(G_s.real @ C_squared == G_t.real @ C_squared)
        constraints.append(G_s.imag @ C_squared == G_t.imag @ C_squared)

    # 3. Soft theorem (simplified: fix coefficient at Delta near 0)
    # constraints.append(soft_theorem_constraint(C_squared, ...))

    # 4. Normalization: fix identity contribution
    constraints.append(C_squared[0] == 1)  # Identity operator

    return C_squared, constraints, Delta_spectrum


def solve_celestial_bootstrap(Delta_gap):
    """
    Solve for consistent celestial CFT with given gap

    Returns:
        Dictionary with status and OPE data (if feasible)
    """
    C_squared, constraints, spectrum = setup_celestial_bootstrap_sdp(
        Delta_gap, Delta_max=10, z_grid=generate_z_grid(20),
        Delta_ext=[1, 1, 1, 1]  # External gravitons
    )

    # Feasibility problem
    problem = cp.Problem(cp.Minimize(0), constraints)

    try:
        problem.solve(solver=cp.SCS, verbose=True)

        if problem.status == cp.OPTIMAL:
            ope_data = {D: c2 for D, c2 in zip(spectrum, C_squared.value)
                       if c2 > 1e-6}
            return {'status': 'feasible', 'ope_data': ope_data}
        else:
            return {'status': 'infeasible',
                    'dual_certificate': extract_dual(constraints)}

    except Exception as e:
        return {'status': 'error', 'message': str(e)}
\end{lstlisting}

\begin{warningbox}
\textbf{Continuous Spectrum Challenge:} Unlike typical CFT bootstrap where the spectrum is discrete, celestial CFT has a \emph{continuous} spectrum (principal series). The SDP must be discretized carefully, and results should be checked for sensitivity to discretization.
\end{warningbox}

\subsection{Phase 5: Extract Results and Certificates (Months 9--12)}

\textbf{Goal:} Map allowed regions and generate verifiable certificates.

\begin{lstlisting}[language=Python, caption={Scanning celestial CFT space}]
def scan_celestial_cft_space(Delta_gap_range, Delta_ext):
    """
    Scan over gap assumptions and map allowed vs. forbidden regions

    Args:
        Delta_gap_range: Range of gap values to test
        Delta_ext: External operator dimensions

    Returns:
        Results dictionary with phase diagram data
    """
    results = {}

    for Delta_gap in Delta_gap_range:
        print(f"\nTesting Delta_gap = {Delta_gap}")

        result = solve_celestial_bootstrap(Delta_gap)
        results[Delta_gap] = result

        if result['status'] == 'feasible':
            print(f"  FEASIBLE: Found consistent OPE data")
            # Verify and export
            verify_celestial_cft(result['ope_data'])
            export_ope_data(result['ope_data'], f'celestial_gap{Delta_gap}.json')

        elif result['status'] == 'infeasible':
            print(f"  INFEASIBLE: No solution exists")
            # Export extremal functional
            export_extremal_functional(
                result['dual_certificate'],
                f'extremal_gap{Delta_gap}.json'
            )

    # Generate phase diagram
    plot_celestial_phase_diagram(results)

    return results
\end{lstlisting}

% ============================================================
% SECTION 5: DETAILED RESEARCH DIRECTIONS
% ============================================================
\section{Detailed Research Directions}

\subsection{Direction 1: MHV Sector Bootstrap}

\begin{pursuitbox}
\textbf{Simplification:} Focus on the MHV (Maximally Helicity Violating) sector where graviton amplitudes take the simplest form.

\textbf{Approach:}
\begin{enumerate}
    \item Compute celestial MHV amplitudes explicitly for 3, 4, 5 particles
    \item Extract OPE data from these known amplitudes
    \item Bootstrap: can we \emph{derive} the MHV amplitudes from crossing + soft theorems alone?
\end{enumerate}

\textbf{Expected Outcome:} Demonstrate that celestial bootstrap reproduces known graviton scattering in the MHV sector.
\end{pursuitbox}

\subsection{Direction 2: Subleading Soft Theorems}

Beyond the leading Weinberg soft factor, there are \textbf{subleading} soft theorems:
\begin{equation}
    \tilde{A}(\Delta) = \frac{S^{(0)}}{\Delta} + S^{(1)} + \Delta \, S^{(2)} + O(\Delta^2)
\end{equation}

\begin{pursuitbox}
\textbf{Investigation:} How do subleading soft theorems constrain the celestial OPE?

These are related to:
\begin{itemize}
    \item BMS supertranslations ($S^{(0)}$)
    \item Superrotations ($S^{(1)}$)
    \item Higher memory effects ($S^{(2)}$)
\end{itemize}
\end{pursuitbox}

\subsection{Direction 3: Celestial OPE and Collinear Limits}

When two particles become collinear, the celestial amplitude should factorize:
\begin{equation}
    \lim_{z_1 \to z_2} \tilde{A}_n \sim \sum_{\Delta} \frac{C_{\Delta_1 \Delta_2}^\Delta}{(z_1 - z_2)^{\Delta_1 + \Delta_2 - \Delta}} \tilde{A}_{n-1}
\end{equation}

\begin{pursuitbox}
\textbf{Study:} Extract celestial OPE coefficients from known collinear limits of graviton amplitudes. Compare with bootstrap predictions.
\end{pursuitbox}

\subsection{Direction 4: Gluon Amplitudes and Yang-Mills}

Celestial holography applies beyond gravity:
\begin{itemize}
    \item \textbf{Gluon amplitudes} map to a celestial CFT with different structure
    \item \textbf{Color structure} introduces new OPE channels
    \item \textbf{Soft gluon theorem} differs from graviton case
\end{itemize}

\begin{pursuitbox}
\textbf{Extension:} Develop celestial bootstrap for Yang-Mills theory. Compare and contrast with gravity.
\end{pursuitbox}

\subsection{Direction 5: Loop Corrections and Infrared Divergences}

Loop amplitudes have \textbf{infrared divergences} that affect celestial amplitudes:
\begin{itemize}
    \item Soft and collinear divergences modify the Mellin transform
    \item May require IR regularization (dimensional regularization, mass regulator)
    \item Celestial interpretation of IR physics is an active research area
\end{itemize}

% ============================================================
% SECTION 6: SUCCESS CRITERIA
% ============================================================
\section{Success Criteria}

\subsection{Minimum Viable Result (6 months)}

\begin{itemize}
    \item[$\checkmark$] Mellin transform engine working for tree-level graviton amplitudes
    \item[$\checkmark$] 3-point and 4-point celestial amplitudes computed
    \item[$\checkmark$] $\SL(2,\C)$ covariance verified numerically
    \item[$\checkmark$] Soft theorems checked
    \item[$\checkmark$] \textbf{First bootstrap result:} Crossing equation setup for MHV sector
    \item[$\checkmark$] Either: Allowed OPE data found, or: no-go region certified
\end{itemize}

\subsection{Strong Result (9 months)}

\begin{itemize}
    \item[$\checkmark$] Multi-channel bootstrap including all helicity sectors
    \item[$\checkmark$] Soft + subleading soft theorems incorporated
    \item[$\checkmark$] Rigorous allowed region in OPE space determined
    \item[$\checkmark$] Or: Exclusion of certain conformal weight ranges proved
    \item[$\checkmark$] Extremal functionals extracted and verified
\end{itemize}

\subsection{Publication-Quality Result (12 months)}

\begin{itemize}
    \item[$\checkmark$] Comprehensive celestial CFT space mapped for graviton scattering
    \item[$\checkmark$] Phase diagram of consistent theories
    \item[$\checkmark$] Novel predictions or no-go theorems
    \item[$\checkmark$] Formal verification of key results
    \item[$\checkmark$] Comparison with explicit amplitude calculations
\end{itemize}

% ============================================================
% SECTION 7: VERIFICATION PROTOCOL
% ============================================================
\section{Verification Protocol}

\begin{lstlisting}[language=Python, caption={Comprehensive celestial CFT verification}]
def verify_celestial_cft(ope_data):
    """
    Comprehensive verification of celestial CFT solution

    Checks:
    1. SL(2,C) covariance of reconstructed amplitudes
    2. Crossing symmetry
    3. Soft graviton theorem
    4. Unitarity (positivity)

    Args:
        ope_data: Dictionary {Delta: C^2} of OPE coefficients

    Returns:
        "VERIFIED" if all checks pass
    """
    # 1. Verify SL(2,C) covariance
    for transformation in sl2c_generators():
        transformed = apply_lorentz_transformation(ope_data, transformation)
        if not is_equivalent(transformed, ope_data):
            return "FAILED: SL(2,C) covariance violated"

    # 2. Check crossing symmetry at multiple points
    z_test_points = generate_crossing_test_points(n=50)
    for z, zbar in z_test_points:
        s_channel = ope_sum(ope_data, 's', z, zbar)
        t_channel = ope_sum(ope_data, 't', z, zbar)
        if abs(s_channel - t_channel) > 1e-8:
            return f"FAILED: Crossing violated at z={z}"

    # 3. Verify soft theorem
    soft_behavior = extract_soft_limit(ope_data, Delta_soft=1e-3)
    weinberg_soft = compute_weinberg_soft(...)
    if not is_close(soft_behavior, weinberg_soft, rtol=1e-6):
        return "FAILED: Soft theorem violated"

    # 4. Unitarity (positive spectral density)
    for Delta, C2 in ope_data.items():
        if C2 < -1e-10:
            return f"FAILED: Negative OPE coefficient at Delta={Delta}"

    return "VERIFIED"
\end{lstlisting}

% ============================================================
% SECTION 8: COMMON PITFALLS
% ============================================================
\section{Common Pitfalls and Mitigations}

\subsection{Mellin Transform Convergence}

\begin{warningbox}
\textbf{Problem:} The Mellin integral $\int_0^\infty d\omega \, \omega^{\Delta-1} A(\omega)$ may not converge for all $\Delta$.

\textbf{Solutions:}
\begin{itemize}
    \item Work on the principal series: $\Delta = 1 + i\lambda$ with $\lambda \in \R$
    \item Use analytic continuation from convergent region
    \item Regularize with exponential damping: $\omega^{\Delta-1} e^{-\epsilon\omega}$
\end{itemize}
\end{warningbox}

\subsection{Continuous vs.\ Discrete Spectrum}

\begin{warningbox}
\textbf{Problem:} Standard bootstrap techniques assume discrete spectrum; celestial CFT has continuous principal series.

\textbf{Solutions:}
\begin{itemize}
    \item Discretize carefully using Gauss quadrature adapted to the measure
    \item Check results are stable under discretization refinement
    \item Use functional analysis techniques for continuous bootstrap
\end{itemize}
\end{warningbox}

\subsection{Shadow Operators and Redundancy}

\begin{warningbox}
\textbf{Problem:} Celestial operators $\calO_\Delta$ and their shadows $\calO_{2-\Delta}$ are related, leading to redundancy in the OPE.

\textbf{Solution:}
\begin{itemize}
    \item Impose shadow symmetry as an additional constraint
    \item Or restrict to $\mathrm{Re}(\Delta) \geq 1$ (half the spectrum)
    \item Document the convention clearly
\end{itemize}
\end{warningbox}

% ============================================================
% SECTION 9: MILESTONE CHECKLIST
% ============================================================
\section{Milestone Checklist}

\subsection{Phase 1: Amplitude Infrastructure (Months 1--2)}
\begin{itemize}
    \item[$\square$] Mellin transform calculator implemented
    \item[$\square$] 3-point celestial amplitude computed
    \item[$\square$] 4-point MHV celestial amplitude computed
    \item[$\square$] $\SL(2,\C)$ covariance verified numerically
    \item[$\square$] Spinor-helicity formalism working
\end{itemize}

\subsection{Phase 2: Conformal Blocks (Months 2--4)}
\begin{itemize}
    \item[$\square$] Celestial conformal blocks implemented
    \item[$\square$] Principal series representations handled correctly
    \item[$\square$] OPE decomposition working for test cases
    \item[$\square$] Cross-ratio dependence verified
\end{itemize}

\subsection{Phase 3: Constraints (Months 4--6)}
\begin{itemize}
    \item[$\square$] Crossing equations formulated
    \item[$\square$] Soft theorem constraints implemented
    \item[$\square$] Subleading soft theorems added
    \item[$\square$] Combined constraint system tested
\end{itemize}

\subsection{Phase 4: Bootstrap Solver (Months 6--9)}
\begin{itemize}
    \item[$\square$] SDP formulation complete
    \item[$\square$] Solver running and converging
    \item[$\square$] First feasibility/infeasibility result
    \item[$\square$] Gap scan initiated
\end{itemize}

\subsection{Phase 5: Results \& Verification (Months 9--12)}
\begin{itemize}
    \item[$\square$] Allowed/forbidden regions mapped
    \item[$\square$] Extremal functionals extracted
    \item[$\square$] Results verified against known amplitudes
    \item[$\square$] Publication draft prepared
\end{itemize}

% ============================================================
% SECTION 10: RESOURCES
% ============================================================
\section{Resources and References}

\subsection{Foundational Papers}

\begin{enumerate}
    \item Pasterski, Shao, Strominger (2017): ``Flat Space Amplitudes and Conformal Symmetry of the Celestial Sphere'' [arXiv:1701.00049]

    \item Pasterski, Shao (2017): ``Conformal Basis for Flat Space Amplitudes'' [arXiv:1705.01027]

    \item Strominger (2018): ``Lectures on the Infrared Structure of Gravity and Gauge Theory'' [arXiv:1703.05448]

    \item Raclariu (2021): ``Lectures on Celestial Holography'' [arXiv:2107.02075]

    \item Pasterski (2021): ``Celestial Amplitudes'' [TASI lectures]
\end{enumerate}

\subsection{Bootstrap References}

\begin{enumerate}
    \item Simmons-Duffin (2016): ``The Conformal Bootstrap'' [arXiv:1602.07982]

    \item Poland, Rychkov, Vichi (2019): ``The Conformal Bootstrap: Theory, Numerical Techniques, and Applications'' [arXiv:1805.04405]
\end{enumerate}

\subsection{Software}

\begin{itemize}
    \item \textbf{mpmath:} Arbitrary precision --- \texttt{pip install mpmath}
    \item \textbf{CVXPY:} Convex optimization --- \texttt{pip install cvxpy}
    \item \textbf{SymPy:} Symbolic computation --- \texttt{pip install sympy}
    \item \textbf{Spinor-Helicity packages:} Various implementations available
\end{itemize}

% ============================================================
% CONCLUSION
% ============================================================
\section{Conclusion}

Celestial holography represents a bold new paradigm for understanding quantum gravity in the real, asymptotically flat universe. The celestial bootstrap approach---constraining the space of consistent celestial CFTs using crossing symmetry, unitarity, and soft theorems---offers a rigorous path toward classifying possible theories.

Success in this challenge would:
\begin{enumerate}
    \item Establish the celestial bootstrap as a viable tool for constraining gravity
    \item Produce novel predictions for graviton scattering (or prove certain structures impossible)
    \item Connect the S-matrix bootstrap to the holographic program
    \item Open new directions in flat-space holography
\end{enumerate}

The interplay between scattering amplitudes, conformal field theory, and optimization provides a rich mathematical structure amenable to rigorous analysis and machine verification.

\end{document}
