\documentclass[11pt,a4paper]{article}

% Packages
\usepackage[utf8]{inputenc}
\usepackage[T1]{fontenc}
\usepackage{lmodern}
\usepackage[english]{babel}
\usepackage{amsmath,amssymb,amsthm}
\usepackage{mathtools}
\usepackage{physics}
\usepackage{graphicx}
\usepackage{xcolor}
\usepackage{listings}
\usepackage{hyperref}
\usepackage{geometry}
\usepackage{fancyhdr}
\usepackage{tocloft}
\usepackage{enumitem}
\usepackage{booktabs}
\usepackage{algorithm}
\usepackage{algpseudocode}

% Page geometry
\geometry{
    a4paper,
    left=25mm,
    right=25mm,
    top=30mm,
    bottom=30mm,
}

% Header and footer
\pagestyle{fancy}
\fancyhf{}
\fancyhead[L]{Pure Thought Challenge 25}
\fancyhead[R]{\thepage}
\renewcommand{\headrulewidth}{0.4pt}

% Hyperref setup
\hypersetup{
    colorlinks=true,
    linkcolor=blue,
    citecolor=blue,
    urlcolor=blue,
    pdfauthor={Pure Thought AI Challenges},
    pdftitle={PRD 25: Quantum Algorithms and Computational Complexity},
}

% Code listing style
\definecolor{codegray}{rgb}{0.95,0.95,0.95}
\definecolor{codegreen}{rgb}{0,0.6,0}
\definecolor{codepurple}{rgb}{0.58,0,0.82}

\lstdefinestyle{pythonstyle}{
    language=Python,
    backgroundcolor=\color{codegray},
    commentstyle=\color{codegreen},
    keywordstyle=\color{blue},
    stringstyle=\color{codepurple},
    basicstyle=\ttfamily\small,
    breaklines=true,
    breakatwhitespace=true,
    captionpos=b,
    frame=single,
    numbers=left,
    numberstyle=\tiny\color{gray},
    tabsize=4,
    showstringspaces=false,
}

\lstset{style=pythonstyle}

% Theorem environments
\newtheorem{theorem}{Theorem}[section]
\newtheorem{lemma}[theorem]{Lemma}
\newtheorem{proposition}[theorem]{Proposition}
\newtheorem{corollary}[theorem]{Corollary}
\theoremstyle{definition}
\newtheorem{definition}[theorem]{Definition}
\newtheorem{example}[theorem]{Example}
\theoremstyle{remark}
\newtheorem{remark}[theorem]{Remark}

% Custom commands
\newcommand{\checklist}[1]{\item[$\square$] #1}
\newcommand{\R}{\mathbb{R}}
\newcommand{\C}{\mathbb{C}}
\newcommand{\Z}{\mathbb{Z}}
\newcommand{\N}{\mathbb{N}}

% Title information
\title{\textbf{PRD 25: Quantum Algorithms and Computational Complexity} \\
\large Pure Thought AI Challenge 25}
\author{Pure Thought AI Challenges Project}
\date{\today}

\begin{document}

\maketitle
\thispagestyle{empty}

\begin{abstract}
This document presents a comprehensive Product Requirement Document (PRD) for implementing a pure-thought computational challenge. The problem can be tackled using only symbolic mathematics, exact arithmetic, and fresh code---no experimental data or materials databases required until final verification. All results must be accompanied by machine-checkable certificates.
\end{abstract}

\clearpage
\tableofcontents
\clearpage


\textbf{Domain}: Quantum Information & Computer Science

\textbf{Timeline}: 6-9 months

\textbf{Difficulty}: High

\textbf{Prerequisites}: Quantum mechanics, linear algebra, complexity theory, graph theory, optimization



\bigskip\hrule\bigskip


\subsection{1. Problem Statement}


\subsubsection{Scientific Context}

\textbf{Quantum computing} harnesses the principles of quantum mechanics—superposition, entanglement, and interference—to solve computational problems more efficiently than classical computers. The foundational quantum algorithms, \textbf{Grover's search} (1996) and \textbf{Shor's factoring} (1994), demonstrated provable quantum speedups over classical algorithms, launching the modern era of quantum information science. Grover's algorithm finds a marked item in an unsorted database of N elements in O(√N) queries versus O(N) classically—a quadratic speedup. Shor's algorithm factors integers in polynomial time, threatening RSA cryptography and providing an exponential speedup over the best known classical algorithms.


\textbf{Quantum walks} generalize classical random walks to the quantum setting, providing a powerful framework for designing quantum algorithms. The coined quantum walk uses a "coin" Hilbert space to determine transition directions, while continuous-time quantum walks evolve via unitary operators exp(-iHt) where H encodes the graph structure. Quantum walks achieve quadratic speedups for problems like element distinctness (Ambainis, 2007) and exponential speedups for certain graph traversal problems (Childs et al., 2003). The \textbf{HHL algorithm} (Harrow-Hassidim-Lloyd, 2009) solves linear systems Ax = b in time O(log N poly(κ)) where κ is the condition number, exponentially faster than classical O(N) algorithms—though caveats apply regarding state preparation and readout.


\textbf{Variational quantum algorithms}, including the \textbf{Quantum Approximate Optimization Algorithm (QAOA)} (Farhi et al., 2014) and \textbf{Variational Quantum Eigensolver (VQE)}, leverage hybrid quantum-classical optimization to solve combinatorial problems and find ground states of quantum systems. QAOA applies alternating unitary layers controlled by classical optimization of parameters, seeking approximate solutions to NP-hard problems like MaxCut and Max-SAT. While rigorous performance guarantees remain elusive, QAOA shows promise for near-term noisy intermediate-scale quantum (NISQ) devices.



\subsubsection{Core Question}

\textbf{Given the quantum circuit model and complexity-theoretic framework:}

\begin{itemize}
\item Implement canonical quantum algorithms: Grover search, quantum walks, HHL, QAOA

\item Analyze query complexity and prove optimality (via polynomial method, adversary bounds)

\item Construct oracle separations proving BQP ≠ BPP (e.g., Recursive Fourier Sampling)

\item Benchmark quantum advantage: when does quantum outperform classical for specific problems?

\item Generate certificates: success probabilities, query counts, complexity lower bounds



\subsubsection{Why This Matters}

\item \textbf{Cryptographic Impact}: Shor's algorithm threatens RSA, ECC; post-quantum cryptography urgently needed

\item \textbf{Optimization}: QAOA and VQE promise near-term applications in logistics, drug discovery, materials science

\item \textbf{Complexity Theory}: Quantum computing provides new tools to understand P vs NP, BQP vs BPP

\item \textbf{Fundamental Physics}: Computational complexity reflects fundamental limits on information processing in nature

\item \textbf{Database Search}: Grover's algorithm offers provable speedup for unstructured search, applicable to SAT solving, collision finding



\subsubsection{Pure Thought Advantages}

\item \textbf{Exact Simulation}: Small qubit systems (≤20 qubits) can be simulated exactly using linear algebra

\item \textbf{Query Complexity}: Lower bounds proven rigorously via polynomial method, adversary method

\item \textbf{Oracle Separations}: BQP ≠ BPP proven via explicit oracle constructions (no real-world assumptions)

\item \textbf{Certificate-Based}: All query complexities, success probabilities, and optimality claims are mathematically provable

\item \textbf{Benchmarking}: Compare quantum vs classical on identical problems without hardware noise



\bigskip\hrule\bigskip


\subsection{2. Mathematical Formulation}


\subsubsection{Quantum Circuit Model}

\end{itemize}

A \textbf{quantum circuit} on n qubits operates on the Hilbert space H = (ℂ²)^⊗n with basis states |x⟩ for x ∈ {0,1}ⁿ. \textbf{Quantum gates} are unitary operators U ∈ U(2ⁿ). Basic gates include:

\begin{itemize}
\item \textbf{Hadamard}: H = (1/√2)[[1,1],[1,-1]]

\item \textbf{Pauli X,Y,Z}: σ\textit{x = [[0,1],[1,0]], σ}y = [[0,-i],[i,0]], σ_z = [[1,0],[0,-1]]

\item \textbf{CNOT}: Controlled-NOT flipping target qubit if control is |1⟩

\item \textbf{Phase gate}: R_ϕ = [[1,0],[0,e^{iϕ}]]


\end{itemize}

A quantum algorithm is a sequence of gates followed by measurement in the computational basis, yielding outcome x with probability |⟨x|ψ⟩|².



\subsubsection{Complexity Classes}

\begin{itemize}
\item \textbf{BPP (Bounded-error Probabilistic Polynomial)}: Classical randomized algorithms with error probability ≤ 1/3

\item \textbf{BQP (Bounded-error Quantum Polynomial)}: Quantum circuits with poly(n) gates, error ≤ 1/3

\item \textbf{NP}: Problems with polynomial-time verifiable certificates

\item \textbf{P}: Polynomial-time deterministic algorithms


\end{itemize}

\textbf{Known relations}: P ⊆ BPP ⊆ BQP, P ⊆ NP. \textbf{Open}: BQP vs NP (neither contains the other is proven), BPP vs BQP (strongly believed BPP ⊂ BQP).



\subsubsection{Query Complexity}

\textbf{Query complexity} measures the number of oracle queries f: {0,1}ⁿ → {0,1} required to solve a problem. For Grover search (find x\textit{ with f(x}) = 1):

\begin{itemize}
\item \textbf{Classical lower bound}: Ω(N) queries (must check ~N/2 items on average)

\item \textbf{Quantum lower bound}: Ω(√N) queries (Bennett et al., 1997, via hybrid argument)

\item \textbf{Grover optimal}: Θ(√N) queries, matching lower bound



\subsubsection{Grover's Algorithm}

\end{itemize}

\textbf{Grover operator}: G = (2|ψ⟩⟨ψ| - I)O where O is the oracle (phase flip on solution) and |ψ⟩ = H^⊗n|0⟩ is uniform superposition.


\textbf{Geometric interpretation}: G rotates state vector toward |x\textit{⟩ in 2D subspace spanned by |ψ⟩ and |x}⟩. After k iterations, amplitude of |x*⟩ is sin((2k+1)θ) where sin θ = 1/√N.


\textbf{Optimal iterations}: k* = ⌊π/(4θ)⌋ ≈ π√N/4 gives success probability ≥ 1 - 1/N.



\subsubsection{Quantum Walk Model}

A \textbf{coined quantum walk} on graph G = (V,E) uses Hilbert space H\textit{coin ⊗ H}position where dim(H\textit{coin) = max}degree(G).


\textbf{Evolution}: U = S(C ⊗ I) where C is coin operator (often Grover diffusion), S is shift operator moving particle along edges.


\textbf{Hitting time}: Expected time to reach target node. Quantum walks achieve quadratic speedup for many graphs (hypercube, complete graph).



\subsubsection{Certificate Specification}

A \textbf{quantum algorithm certificate} must contain:

\begin{itemize}
\item \textbf{Circuit description}: Gate sequence, qubit count, depth

\item \textbf{Success probability}: P(correct output) ≥ 1 - ε with ε ≤ 1/3

\item \textbf{Query complexity}: Number of oracle calls, comparison to classical

\item \textbf{Lower bound proof}: Adversary method, polynomial method, or hybrid argument

\item \textbf{Complexity class}: BQP, BPP, or other; oracle separation if applicable

\item \textbf{Numerical simulation}: For ≤20 qubits, exact amplitudes at each step



\bigskip\hrule\bigskip


\subsection{3. Implementation Approach}

\end{itemize}

This is a 6-phase project spanning 6-9 months, implementing canonical quantum algorithms with complexity analysis.



\subsubsection{Phase 1: Grover's Algorithm (Months 1-2)}

\textbf{Objective}: Implement Grover search, verify O(√N) query complexity, prove optimality.


\begin{lstlisting}
import numpy as np
from typing import Callable, Dict, List
from dataclasses import dataclass

def hadamard_n(n: int) -> np.ndarray:
    """
    n-qubit Hadamard gate: H^⊗n.

    Returns: 2^n × 2^n unitary matrix.
    """
    H1 = np.array([[1, 1], [1, -1]]) / np.sqrt(2)
    H_n = H1
    for _ in range(n - 1):
        H_n = np.kron(H_n, H1)
    return H_n

def oracle_matrix(marked_items: List[int], N: int) -> np.ndarray:
    """
    Oracle O that flips phase of marked items: O|x⟩ = (-1)^{f(x)}|x⟩.

    Args:
        marked_items: List of indices x with f(x) = 1
        N: Total number of items (N = 2^n)

    Returns: N × N diagonal matrix with -1 at marked positions.
    """
    O = np.eye(N)
    for x in marked_items:
        O[x, x] = -1
    return O

def grover_operator(oracle: np.ndarray, n: int) -> np.ndarray:
    """
    Grover diffusion operator G = (2|ψ⟩⟨ψ| - I)O.

    |ψ⟩ = H^⊗n|0⟩ = (1/√N) Σ_x |x⟩ (uniform superposition)

    Returns: Grover operator G.
    """
    N = 2**n
    H = hadamard_n(n)

    # |ψ⟩ = H|0⟩
    psi = np.zeros(N)
    psi[0] = 1.0
    psi = H @ psi

    # Diffusion operator: D = 2|ψ⟩⟨ψ| - I
    D = 2 * np.outer(psi, psi) - np.eye(N)

    # Grover operator: G = D O
    G = D @ oracle

    return G

def grover_search(marked_items: List[int], n: int, verbose: bool = False) -> Dict:
    """
    Grover's algorithm: find marked item in O(√N) queries.

    Args:
        marked_items: List of marked indices (assumed |marked_items| = 1 for simplicity)
        n: Number of qubits (N = 2^n items)
        verbose: Print iteration details

    Returns: Result dictionary with final state, measurement outcome, success probability.
    """
    N = 2**n
    M = len(marked_items)  # Number of solutions

    # Optimal number of iterations
    theta = np.arcsin(np.sqrt(M / N))
    k_optimal = int(np.pi / (4 * theta)) if theta > 0 else 0

    # Initial state: uniform superposition
    H = hadamard_n(n)
    psi = np.zeros(N)
    psi[0] = 1.0
    psi = H @ psi

    # Construct Grover operator
    O = oracle_matrix(marked_items, N)
    G = grover_operator(O, n)

    # Apply G^k
    for k in range(k_optimal):
        psi = G @ psi
        if verbose:
            prob_marked = sum(abs(psi[x])**2 for x in marked_items)
            print(f"Iteration {k+1}: P(marked) = {prob_marked:.6f}")

    # Measure
    probabilities = np.abs(psi)**2
    result = int(np.argmax(probabilities))

    success_prob = sum(probabilities[x] for x in marked_items)

    return {
        'final_state': psi,
        'measurement_outcome': result,
        'success_probability': success_prob,
        'iterations': k_optimal,
        'oracle_queries': k_optimal,
        'correct': result in marked_items
    }

def verify_grover_optimality(n_range: range) -> Dict:
    """
    Verify Grover's √N scaling by running for different N.

    Returns: Dictionary mapping N to average queries.
    """
    results = {}

    for n in n_range:
        N = 2**n
        # Single marked item at random position
        marked = [np.random.randint(0, N)]

        result = grover_search(marked, n)
        queries = result['oracle_queries']

        results[N] = {
            'queries': queries,
            'sqrt_N': np.sqrt(N),
            'ratio': queries / np.sqrt(N)
        }

        print(f"N={N:5d}: {queries:4d} queries, √N={np.sqrt(N):7.2f}, ratio={queries/np.sqrt(N):.4f}")

    return results

# Example usage
if __name__ == "__main__":
    # Simple example: N=16, marked item at index 7
    n = 4
    marked = [7]

    result = grover_search(marked, n, verbose=True)

    print(f"\nGrover's Algorithm Result:")
    print(f"  Marked item: {marked[0]}")
    print(f"  Found: {result['measurement_outcome']}")
    print(f"  Success probability: {result['success_probability']:.6f}")
    print(f"  Oracle queries: {result['oracle_queries']}")
    print(f"  Correct: {result['correct']}")

    # Verify scaling
    print(f"\nVerifying √N scaling:")
    verify_grover_optimality(range(4, 10))
\end{lstlisting}


\subsubsection{Phase 2: Quantum Walks (Months 2-4)}

\textbf{Objective}: Implement coined and continuous-time quantum walks, analyze hitting times.


\begin{lstlisting}
import networkx as nx
from scipy.linalg import expm

def grover_diffusion_coin(d: int) -> np.ndarray:
    """
    Grover diffusion coin: C = 2|ψ⟩⟨ψ| - I where |ψ⟩ = (1/√d) Σ_j |j⟩.

    Args:
        d: Coin dimension (typically max degree of graph)

    Returns: d × d unitary coin operator.
    """
    psi = np.ones(d) / np.sqrt(d)
    C = 2 * np.outer(psi, psi) - np.eye(d)
    return C

def shift_operator(graph: nx.Graph, d_max: int) -> np.ndarray:
    """
    Shift operator S for coined quantum walk.

    Maps |j,v⟩ → |j,w⟩ where w is j-th neighbor of v.

    Args:
        graph: NetworkX graph
        d_max: Maximum degree (coin dimension)

    Returns: Shift operator on (d_max * N)-dimensional Hilbert space.
    """
    N = graph.number_of_nodes()
    dim = d_max * N
    S = np.zeros((dim, dim), dtype=complex)

    # Relabel nodes to 0, 1, ..., N-1
    mapping = {node: i for i, node in enumerate(graph.nodes())}
    G = nx.relabel_nodes(graph, mapping)

    for v in G.nodes():
        neighbors = list(G.neighbors(v))
        degree = len(neighbors)

        for j, w in enumerate(neighbors):
            # |j,v⟩ → |j,w⟩
            # Basis: |coin,position⟩ with index = coin + d_max * position
            idx_from = j + d_max * v
            idx_to = j + d_max * w
            S[idx_to, idx_from] = 1.0

    return S

def coined_quantum_walk(graph: nx.Graph, steps: int, start_node: int = 0) -> np.ndarray:
    """
    Coined quantum walk on graph.

    Args:
        graph: NetworkX graph
        steps: Number of walk steps
        start_node: Initial position

    Returns: Probability distribution over nodes after 'steps'.
    """
    N = graph.number_of_nodes()
    d_max = max(dict(graph.degree()).values())
    dim = d_max * N

    # Coin operator
    C = grover_diffusion_coin(d_max)

    # Shift operator
    S = shift_operator(graph, d_max)

    # Walk operator: U = S (C ⊗ I_N)
    U = S @ np.kron(C, np.eye(N))

    # Initial state: |0,start_node⟩ (coin state 0, position start_node)
    psi = np.zeros(dim, dtype=complex)
    psi[0 + d_max * start_node] = 1.0

    # Evolve for 'steps'
    for _ in range(steps):
        psi = U @ psi

    # Measure position (trace over coin space)
    prob = np.zeros(N)
    for v in range(N):
        for j in range(d_max):
            idx = j + d_max * v
            prob[v] += abs(psi[idx])**2

    return prob

def continuous_time_quantum_walk(graph: nx.Graph, t: float, start_node: int = 0) -> np.ndarray:
    """
    Continuous-time quantum walk: |ψ(t)⟩ = exp(-iHt)|ψ(0)⟩.

    H is the adjacency matrix (or Laplacian) of the graph.

    Args:
        graph: NetworkX graph
        t: Evolution time
        start_node: Initial position

    Returns: Probability distribution over nodes at time t.
    """
    N = graph.number_of_nodes()

    # Hamiltonian: adjacency matrix
    A = nx.adjacency_matrix(graph).toarray()
    H = A.astype(complex)

    # Initial state: |start_node⟩
    psi_0 = np.zeros(N, dtype=complex)
    psi_0[start_node] = 1.0

    # Evolve: |ψ(t)⟩ = exp(-iHt)|ψ(0)⟩
    U_t = expm(-1j * H * t)
    psi_t = U_t @ psi_0

    # Probability distribution
    prob = np.abs(psi_t)**2

    return prob

def analyze_quantum_walk_speedup(graph: nx.Graph, target_node: int, max_steps: int = 100) -> Dict:
    """
    Compare quantum vs classical random walk hitting time to target node.

    Returns: Dictionary with hitting times and speedup factor.
    """
    N = graph.number_of_nodes()

    # Quantum walk: find time to reach target with high probability
    hitting_time_quantum = None
    for steps in range(1, max_steps):
        prob = coined_quantum_walk(graph, steps, start_node=0)
        if prob[target_node] > 0.5:  # Threshold for "hitting"
            hitting_time_quantum = steps
            break

    # Classical random walk: expected hitting time
    # Use eigenvalue analysis or simulation
    # For simplicity, estimate as N (typical for random graphs)
    hitting_time_classical = N  # Placeholder

    speedup = hitting_time_classical / hitting_time_quantum if hitting_time_quantum else float('inf')

    return {
        'quantum_hitting_time': hitting_time_quantum,
        'classical_hitting_time': hitting_time_classical,
        'speedup': speedup,
        'graph_size': N
    }

# Example: Quantum walk on cycle graph
if __name__ == "__main__":
    # Cycle graph with 16 nodes
    G = nx.cycle_graph(16)

    # Coined quantum walk
    prob_coined = coined_quantum_walk(G, steps=20, start_node=0)
    print("Coined quantum walk probability distribution:")
    print(prob_coined)

    # Continuous-time quantum walk
    prob_ctqw = continuous_time_quantum_walk(G, t=5.0, start_node=0)
    print("\nContinuous-time quantum walk probability distribution:")
    print(prob_ctqw)

    # Analyze speedup
    speedup_result = analyze_quantum_walk_speedup(G, target_node=8)
    print(f"\nQuantum walk hitting time: {speedup_result['quantum_hitting_time']}")
    print(f"Speedup over classical: {speedup_result['speedup']:.2f}x")
\end{lstlisting}


\subsubsection{Phase 3: HHL Algorithm for Linear Systems (Months 4-5)}

\textbf{Objective}: Implement HHL algorithm for solving Ax = b, analyze complexity.


\begin{lstlisting}
from scipy.linalg import eigh

def hhl_algorithm_simulation(A: np.ndarray, b: np.ndarray, t: float = 1.0,
                             epsilon: float = 0.01) -> Dict:
    """
    Simulate HHL algorithm for solving Ax = b.

    Algorithm:
    1. Phase estimation to encode eigenvalues of A in ancilla register
    2. Controlled rotation to invert eigenvalues: R(θ_j) where sin(θ_j) ∝ 1/λ_j
    3. Uncompute phase estimation
    4. Post-select on ancilla = |1⟩

    Args:
        A: Hermitian matrix (N × N), assumed well-conditioned
        b: Input vector (N-dimensional)
        t: Evolution time for phase estimation
        epsilon: Precision parameter

    Returns: Dictionary with solution state |x⟩ and success probability.
    """
    N = A.shape[0]

    # Step 1: Eigenvalue decomposition of A
    eigvals, eigvecs = eigh(A)  # A = Σ_j λ_j |u_j⟩⟨u_j|

    # Normalize input: |b⟩ = Σ_j β_j |u_j⟩
    b_normalized = b / np.linalg.norm(b)
    betas = eigvecs.T @ b_normalized  # Coefficients β_j = ⟨u_j|b⟩

    # Step 2: Simulate controlled rotations
    # Exact solution: |x⟩ = A^{-1}|b⟩ = Σ_j (β_j / λ_j) |u_j⟩
    x_state = np.zeros(N, dtype=complex)
    success_prob = 0.0

    C = 1.0  # Normalization constant (related to condition number κ)

    for j in range(N):
        if abs(eigvals[j]) > epsilon:  # Avoid division by near-zero eigenvalues
            coeff = betas[j] / eigvals[j]
            x_state += coeff * eigvecs[:, j]

            # Success probability contribution from post-selection
            # P(ancilla=1) ∝ |1/λ_j|²
            success_prob += abs(betas[j])**2 / eigvals[j]**2

    # Normalize
    x_state /= np.linalg.norm(x_state)
    success_prob /= sum(abs(betas[j])**2 / eigvals[j]**2 for j in range(N) if abs(eigvals[j]) > epsilon)

    # Classical solution for comparison
    x_classical = np.linalg.solve(A, b)
    x_classical /= np.linalg.norm(x_classical)

    # Compare quantum vs classical
    fidelity = abs(np.vdot(x_state, x_classical))**2

    return {
        'quantum_solution_state': x_state,
        'classical_solution': x_classical,
        'fidelity': fidelity,
        'success_probability': success_prob,
        'condition_number': np.linalg.cond(A),
        'eigenvalues': eigvals
    }

def hhl_complexity_analysis(N: int, kappa: float) -> Dict:
    """
    Analyze HHL complexity: O(log N poly(κ, 1/ε)).

    Compare to classical Gaussian elimination: O(N²) for sparse, O(N³) for dense.

    Args:
        N: Matrix dimension
        kappa: Condition number κ = λ_max / λ_min

    Returns: Complexity estimates.
    """
    # Quantum complexity (gate count)
    # Phase estimation: O(poly(log N, log κ, log(1/ε)))
    # Hamiltonian simulation: O(poly(log N))
    quantum_gates = (np.log2(N))**2 * np.log2(kappa)

    # Classical complexity
    classical_ops_sparse = N**2  # Sparse solver
    classical_ops_dense = N**3   # Dense Gaussian elimination

    # Speedup (caveat: assumes efficient state preparation and readout)
    speedup_sparse = classical_ops_sparse / quantum_gates
    speedup_dense = classical_ops_dense / quantum_gates

    return {
        'quantum_gates': quantum_gates,
        'classical_ops_sparse': classical_ops_sparse,
        'classical_ops_dense': classical_ops_dense,
        'speedup_vs_sparse': speedup_sparse,
        'speedup_vs_dense': speedup_dense,
        'caveat': 'Speedup assumes O(polylog N) state preparation and measurement'
    }

# Example: Solve simple linear system
if __name__ == "__main__":
    # Construct well-conditioned Hermitian matrix
    N = 8
    A = np.random.randn(N, N)
    A = (A + A.T) / 2  # Symmetrize
    A += 5 * np.eye(N)  # Ensure positive definite (condition number ~O(1))

    b = np.random.randn(N)

    result = hhl_algorithm_simulation(A, b)

    print("HHL Algorithm Simulation:")
    print(f"  Fidelity with classical solution: {result['fidelity']:.6f}")
    print(f"  Success probability: {result['success_probability']:.6f}")
    print(f"  Condition number κ: {result['condition_number']:.2f}")

    # Complexity analysis
    complexity = hhl_complexity_analysis(N=1024, kappa=10.0)
    print(f"\nComplexity for N=1024, κ=10:")
    print(f"  Quantum gates: {complexity['quantum_gates']:.0f}")
    print(f"  Classical ops (sparse): {complexity['classical_ops_sparse']:.0e}")
    print(f"  Speedup vs sparse: {complexity['speedup_vs_sparse']:.2e}x")
\end{lstlisting}


\subsubsection{Phase 4: QAOA for Combinatorial Optimization (Months 5-6)}

\textbf{Objective}: Implement QAOA for MaxCut, analyze approximation ratio.


\begin{lstlisting}
from scipy.optimize import minimize
from itertools import combinations

def maxcut_hamiltonian(graph: nx.Graph) -> np.ndarray:
    """
    MaxCut cost Hamiltonian: H_C = Σ_{(i,j) ∈ E} ½(1 - Z_i Z_j).

    Maximizing cut size ≡ minimizing -H_C.

    Returns: 2^N × 2^N matrix.
    """
    N = graph.number_of_nodes()
    dim = 2**N
    H_C = np.zeros((dim, dim))

    for i, j in graph.edges():
        # Z_i Z_j operator
        Z_i = pauli_z_on_qubit(i, N)
        Z_j = pauli_z_on_qubit(j, N)
        ZZ = Z_i @ Z_j

        H_C += 0.5 * (np.eye(dim) - ZZ)

    return H_C

def pauli_z_on_qubit(k: int, N: int) -> np.ndarray:
    """
    Z operator on qubit k in N-qubit system.

    Z = [[1,0],[0,-1]]
    """
    Z = np.array([[1, 0], [0, -1]])
    I = np.eye(2)

    op = I
    for j in range(N):
        if j == 0:
            op = Z if k == 0 else I
        else:
            op = np.kron(op, Z if k == j else I)

    return op

def pauli_x_on_qubit(k: int, N: int) -> np.ndarray:
    """X operator on qubit k."""
    X = np.array([[0, 1], [1, 0]])
    I = np.eye(2)

    op = I
    for j in range(N):
        if j == 0:
            op = X if k == 0 else I
        else:
            op = np.kron(op, X if k == j else I)

    return op

def mixer_hamiltonian(N: int) -> np.ndarray:
    """
    Mixer Hamiltonian: H_M = Σ_i X_i.

    Returns: 2^N × 2^N matrix.
    """
    dim = 2**N
    H_M = np.zeros((dim, dim))

    for i in range(N):
        H_M += pauli_x_on_qubit(i, N)

    return H_M

def qaoa_circuit(params: np.ndarray, H_C: np.ndarray, H_M: np.ndarray, p: int) -> np.ndarray:
    """
    QAOA circuit: |ψ(γ,β)⟩ = Π_{i=1}^p e^{-iβ_i H_M} e^{-iγ_i H_C} |+⟩^⊗n.

    Args:
        params: Array of 2p parameters [γ_1,...,γ_p,β_1,...,β_p]
        H_C: Cost Hamiltonian
        H_M: Mixer Hamiltonian
        p: Number of QAOA layers

    Returns: Final state |ψ(γ,β)⟩.
    """
    N = int(np.log2(H_C.shape[0]))
    dim = 2**N

    gamma = params[:p]
    beta = params[p:]

    # Initial state: |+⟩^⊗n = H^⊗n|0⟩
    psi = np.ones(dim) / np.sqrt(dim)

    # Apply QAOA layers
    for i in range(p):
        # Cost layer: e^{-iγ_i H_C}
        U_C = expm(-1j * gamma[i] * H_C)
        psi = U_C @ psi

        # Mixer layer: e^{-iβ_i H_M}
        U_M = expm(-1j * beta[i] * H_M)
        psi = U_M @ psi

    return psi

def qaoa_maxcut(graph: nx.Graph, p: int = 1, max_iter: int = 100) -> Dict:
    """
    QAOA for MaxCut problem.

    Args:
        graph: NetworkX graph
        p: Number of QAOA layers
        max_iter: Maximum optimization iterations

    Returns: Optimal parameters, state, and approximation ratio.
    """
    N = graph.number_of_nodes()

    # Construct Hamiltonians
    H_C = maxcut_hamiltonian(graph)
    H_M = mixer_hamiltonian(N)

    # Objective function: ⟨ψ(γ,β)|H_C|ψ(γ,β)⟩
    def objective(params):
        psi = qaoa_circuit(params, H_C, H_M, p)
        expectation = np.real(psi.conj() @ H_C @ psi)
        return -expectation  # Minimize -⟨H_C⟩ to maximize cut size

    # Optimize
    init_params = np.random.uniform(0, 2*np.pi, 2*p)
    result = minimize(objective, init_params, method='COBYLA',
                     options={'maxiter': max_iter})

    # Extract solution
    optimal_params = result.x
    optimal_psi = qaoa_circuit(optimal_params, H_C, H_M, p)
    qaoa_cut_value = -result.fun

    # Classical MaxCut upper bound (brute force for small graphs)
    max_cut_classical = maxcut_brute_force(graph)

    # Approximation ratio
    approx_ratio = qaoa_cut_value / max_cut_classical if max_cut_classical > 0 else 0

    return {
        'optimal_params': optimal_params,
        'optimal_state': optimal_psi,
        'qaoa_cut_value': qaoa_cut_value,
        'max_cut_classical': max_cut_classical,
        'approximation_ratio': approx_ratio,
        'p': p
    }

def maxcut_brute_force(graph: nx.Graph) -> float:
    """
    Compute maximum cut via brute force enumeration (feasible for N ≤ 15).
    """
    N = graph.number_of_nodes()
    max_cut = 0

    for partition in range(2**(N-1)):  # Only need half due to symmetry
        cut_size = 0
        # Decode partition as bitstring
        S1 = {i for i in range(N) if (partition >> i) & 1}
        S2 = set(range(N)) - S1

        for i, j in graph.edges():
            if (i in S1 and j in S2) or (i in S2 and j in S1):
                cut_size += 1

        max_cut = max(max_cut, cut_size)

    return max_cut

# Example: QAOA on small random graph
if __name__ == "__main__":
    # Random graph with 6 nodes
    G = nx.erdos_renyi_graph(6, 0.5, seed=42)

    # Run QAOA with p=1
    result = qaoa_maxcut(G, p=1)

    print("QAOA for MaxCut:")
    print(f"  QAOA cut value: {result['qaoa_cut_value']:.4f}")
    print(f"  Optimal cut (classical): {result['max_cut_classical']:.0f}")
    print(f"  Approximation ratio: {result['approximation_ratio']:.4f}")
    print(f"  p={result['p']}")
\end{lstlisting}


\subsubsection{Phase 5: Complexity Analysis and Oracle Separations (Months 6-7)}

\textbf{Objective}: Prove query complexity lower bounds, construct BQP vs BPP oracle separations.


\begin{lstlisting}
def query_complexity_lower_bound_adversary(N: int) -> Dict:
    """
    Prove Ω(√N) lower bound for Grover search using adversary method.

    Adversary argument: any quantum algorithm distinguishing between two
    functions f, g (differing on single input) requires Ω(√N) queries.

    Returns: Lower bound certificate.
    """
    # Adversary matrix Γ with Γ[x,y] = 1 if f_x(y) ≠ g_x(y)
    # For Grover: f has solution at x, g has solution at y
    # Spectral norm ||Γ|| = √N gives lower bound

    lower_bound = np.sqrt(N)

    # Certificate: spectral norm of adversary matrix
    # (For full proof, construct Γ and compute eigenvalues)

    return {
        'problem': 'Grover search',
        'lower_bound': lower_bound,
        'method': 'Adversary method',
        'certificate': 'Spectral norm ||Γ|| = √N'
    }

def bqp_bpp_oracle_separation() -> Dict:
    """
    Construct oracle separation proving BQP^O ≠ BPP^O.

    Use Recursive Fourier Sampling (RFS) problem:
    - Quantum algorithm solves RFS in poly(n) queries
    - Classical algorithm requires exp(n) queries

    Returns: Oracle construction and complexity bounds.
    """
    # RFS problem: Given oracle access to function f: Z_2^n → Z_2,
    # output Fourier coefficient f̂(s) for uniformly random s

    # Quantum algorithm: Hadamard test in O(1) queries
    quantum_queries = 1

    # Classical algorithm: Must estimate f̂(s) = (1/2^n) Σ_x (-1)^{f(x) + s·x}
    # Requires Ω(2^n) samples to distinguish from 0
    classical_queries_lower_bound = lambda n: 2**(n-1)

    n_example = 10

    return {
        'problem': 'Recursive Fourier Sampling',
        'quantum_queries': quantum_queries,
        'classical_queries_lower_bound': classical_queries_lower_bound(n_example),
        'separation': 'Exponential',
        'conclusion': 'BQP^O ≠ BPP^O for oracle O encoding RFS'
    }

# Example: Lower bound certificates
if __name__ == "__main__":
    # Grover lower bound
    lb_grover = query_complexity_lower_bound_adversary(N=1024)
    print("Grover Search Lower Bound:")
    print(f"  Problem: {lb_grover['problem']}")
    print(f"  Lower bound: Ω({lb_grover['lower_bound']:.0f}) queries")
    print(f"  Method: {lb_grover['method']}")

    # BQP vs BPP separation
    separation = bqp_bpp_oracle_separation()
    print(f"\nBQP vs BPP Oracle Separation:")
    print(f"  Problem: {separation['problem']}")
    print(f"  Quantum: {separation['quantum_queries']} queries")
    print(f"  Classical: Ω({separation['classical_queries_lower_bound']}) queries")
    print(f"  Separation: {separation['separation']}")
\end{lstlisting}


\subsubsection{Phase 6: Certificate Generation and Export (Months 7-9)}

\textbf{Objective}: Generate machine-checkable certificates for all algorithms.


\begin{lstlisting}
from dataclasses import dataclass, asdict
import json
from datetime import datetime

@dataclass
class QuantumAlgorithmCertificate:
    """Certificate for quantum algorithm performance and correctness."""

    algorithm_name: str
    problem_size: int  # N or n (number of qubits)

    # Query complexity
    quantum_queries: int
    classical_queries_lower_bound: int
    speedup: float

    # Success probability
    success_probability: float
    error_bound: float

    # Circuit details
    qubit_count: int
    gate_count: int
    circuit_depth: int

    # Verification
    correctness_verified: bool
    optimality_proof: str  # "Adversary method", "Polynomial method", etc.

    # Metadata
    timestamp: str
    simulation_time: float

def generate_quantum_algorithm_certificate(algorithm_result: Dict,
                                          algorithm_name: str) -> QuantumAlgorithmCertificate:
    """
    Generate certificate for quantum algorithm.

    Args:
        algorithm_result: Output from quantum algorithm simulation
        algorithm_name: "Grover", "QuantumWalk", "HHL", "QAOA"

    Returns: Certificate object.
    """
    if algorithm_name == "Grover":
        N = 2**algorithm_result.get('n', 4)
        cert = QuantumAlgorithmCertificate(
            algorithm_name="Grover Search",
            problem_size=N,
            quantum_queries=algorithm_result['oracle_queries'],
            classical_queries_lower_bound=N // 2,
            speedup=N / (2 * algorithm_result['oracle_queries']),
            success_probability=algorithm_result['success_probability'],
            error_bound=1.0 / N,
            qubit_count=int(np.log2(N)),
            gate_count=algorithm_result['oracle_queries'] * N,  # Rough estimate
            circuit_depth=algorithm_result['oracle_queries'],
            correctness_verified=algorithm_result['correct'],
            optimality_proof="Adversary method (Bennett et al., 1997)",
            timestamp=datetime.now().isoformat(),
            simulation_time=0.0
        )

    # Add similar branches for QuantumWalk, HHL, QAOA

    return cert

def export_certificate_json(cert: QuantumAlgorithmCertificate, filepath: str):
    """Export certificate to JSON."""
    with open(filepath, 'w') as f:
        json.dump(asdict(cert), f, indent=2)

    print(f"Certificate exported to {filepath}")

# Example: Full pipeline
if __name__ == "__main__":
    # Run Grover
    n = 6
    marked = [42]
    grover_result = grover_search(marked, n)
    grover_result['n'] = n

    # Generate certificate
    cert = generate_quantum_algorithm_certificate(grover_result, "Grover")

    # Export
    export_certificate_json(cert, "grover_certificate.json")

    print("\nCertificate Summary:")
    print(f"  Algorithm: {cert.algorithm_name}")
    print(f"  Quantum queries: {cert.quantum_queries}")
    print(f"  Classical lower bound: {cert.classical_queries_lower_bound}")
    print(f"  Speedup: {cert.speedup:.2f}x")
    print(f"  Success probability: {cert.success_probability:.6f}")
\end{lstlisting}


\bigskip\hrule\bigskip


\subsection{4. Example Starting Prompt}

Use this prompt to initialize a long-running AI system for quantum algorithms research:


\begin{lstlisting}
You are a quantum algorithm researcher studying computational complexity and quantum advantage.
Your task is to implement canonical quantum algorithms (Grover, quantum walks, HHL, QAOA),
analyze their query complexity, and prove optimality via adversary and polynomial methods.

CONTEXT:
Quantum computing harnesses superposition and interference to solve certain problems faster
than classical computers. Grover's algorithm searches unsorted databases in O(√N) queries
versus O(N) classically—a provable quadratic speedup. Shor's algorithm factors integers in
polynomial time, threatening RSA. Quantum walks generalize random walks, achieving speedups
for graph problems. QAOA tackles combinatorial optimization on near-term devices.

Complexity theory classifies problems by resources required. BQP (Bounded-error Quantum
Polynomial) contains problems solvable by quantum computers in poly(n) time with error ≤1/3.
BPP is the classical randomized analogue. Oracle separations prove BQP ≠ BPP, but relations
to NP remain open.

OBJECTIVE:
Phase 1 (Months 1-2): Implement Grover's algorithm for N=2^n items. Verify O(√N) queries,
  success probability ≥ 1-1/N. Prove optimality via adversary method (spectral norm ||Γ||=√N).

Phase 2 (Months 2-4): Implement coined and continuous-time quantum walks on graphs (cycle,
  hypercube, complete). Analyze hitting times, compare to classical random walks. Identify
  quadratic speedups.

Phase 3 (Months 4-5): Implement HHL algorithm for linear systems Ax=b. Analyze complexity
  O(log N poly(κ)), compare to classical O(N²) for sparse, O(N³) for dense. Discuss caveats
  (state preparation, readout).

Phase 4 (Months 5-6): Implement QAOA for MaxCut on random graphs. Optimize parameters γ,β
  via classical minimization. Compute approximation ratio vs brute-force solution. Analyze
  performance vs graph structure.

Phase 5 (Months 6-7): Prove query complexity lower bounds using adversary method (Grover Ω(√N)),
  polynomial method (collision finding Ω(N^{1/3})). Construct BQP vs BPP oracle separation via
  Recursive Fourier Sampling.

Phase 6 (Months 7-9): Generate machine-checkable certificates for all algorithms:
  - Query complexity and classical lower bounds
  - Success probabilities and error bounds
  - Circuit parameters (qubits, gates, depth)
  - Optimality proofs (adversary matrix spectral norms)
  - Export as JSON with exact arithmetic where applicable

PURE THOUGHT CONSTRAINTS:
- Simulate quantum circuits exactly using numpy linear algebra (≤20 qubits)
- All complexity claims must have rigorous proofs (adversary, polynomial, hybrid arguments)
- Compare quantum vs classical on identical problems without hardware assumptions
- No approximations beyond specified error bounds (e.g., ε=10⁻⁶ for QAOA optimization)
- Export quantum states, gates, and measurements with full precision

SUCCESS CRITERIA:
- Minimum Viable Result (2-4 months): Grover working with verified √N scaling, quantum walk
  on simple graphs, basic QAOA implementation
- Strong Result (6-8 months): All algorithms operational, query complexity lower bounds proven,
  HHL analysis complete, QAOA approximation ratios measured
- Publication-Quality (9 months): BQP vs BPP oracle separation constructed, novel quantum walk
  applications, comprehensive complexity analysis, comparison with theoretical bounds

START:
Begin with Grover's algorithm (Phase 1). Implement oracle, Grover operator, and measurement.
Verify success probability ≥1-1/N for N=16,64,256. Plot queries vs √N to confirm scaling.
Prove Ω(√N) lower bound via adversary method. Export certificate with all details.
\end{lstlisting}


\bigskip\hrule\bigskip


\subsection{5. Success Criteria}


\subsubsection{Minimum Viable Result (MVR) - 2-4 Months}

\textbf{Core Functionality}:

\begin{itemize}
\item Grover's algorithm: finds marked item in ≤ ⌈π√N/4⌉ queries with P(success) ≥ 1-1/N

\item Quantum walk on cycle and hypercube graphs: hitting time measured

\item Basic QAOA implementation: MaxCut on 6-8 node graphs

\item Certificate generation: query counts, success probabilities


\end{itemize}

\textbf{Deliverables}:

\begin{itemize}
\item \texttt{grover.py}: Complete implementation with optimality verification

\item \texttt{quantum_walk.py}: Coined and continuous-time walks

\item \texttt{qaoa_maxcut.py}: QAOA with classical parameter optimization

\item \texttt{certificates.json}: Query complexity and success probability data


\end{itemize}

\textbf{Quality Metrics}:

\begin{itemize}
\item Grover: |queries - π√N/4| < 2 for all N = 2^n with n ≤ 10

\item Quantum walk: hitting time on cycle C_N is O(N) vs classical O(N²)

\item QAOA: approximation ratio ≥ 0.7 for random 3-regular graphs



\subsubsection{Strong Result - 6-8 Months}

\end{itemize}

\textbf{Extended Capabilities}:

\begin{itemize}
\item HHL algorithm: solve 2^n × 2^n systems with n ≤ 10, fidelity with classical solution >0.99

\item Query complexity lower bounds: adversary method for Grover (Ω(√N)), element distinctness (Ω(N^{2/3}))

\item Quantum walk speedups: analyze on 10+ graph families (trees, grids, expanders)

\item QAOA: test on MaxCut, Max-SAT, graph coloring; approximation ratios documented


\end{itemize}

\textbf{Deliverables}:

\begin{itemize}
\item \texttt{hhl.py}: Full HHL with complexity analysis

\item \texttt{complexity_bounds.py}: Adversary and polynomial method implementations

\item \texttt{qaoa_suite.py}: QAOA for multiple combinatorial problems

\item Research report: "Quantum vs Classical: A Pure Thought Comparison"


\end{itemize}

\textbf{Quality Metrics}:

\begin{itemize}
\item HHL: κ ≤ 100, success probability ≥0.5, quantum gates O((log N)²)

\item Lower bounds: adversary matrix spectral norms computed exactly (sympy)

\item Quantum walk: speedup factor ≥2 verified on ≥5 graph families

\item QAOA: depth p ≤ 3, approximation ratio documented vs graph size



\subsubsection{Publication-Quality Result - 9 Months}

\end{itemize}

\textbf{Novel Contributions}:

\begin{itemize}
\item BQP vs BPP oracle separation: full construction of Recursive Fourier Sampling oracle

\item New quantum walk application: novel algorithm for graph property testing

\item QAOA performance theory: approximation ratio bounds vs graph structure

\item Comprehensive database: 1000+ quantum algorithm runs with certificates


\end{itemize}

\textbf{Deliverables}:

\begin{itemize}
\item \texttt{oracle_separation.py}: Explicit BQP^O ≠ BPP^O construction

\item Research paper: "Provable Quantum Advantage: From Grover to Oracle Separations"

\item Interactive visualization: Quantum vs classical complexity comparison

\item Formal verification: Lean4 proofs for Grover optimality (optional advanced goal)


\end{itemize}

\textbf{Quality Metrics}:

\begin{itemize}
\item Oracle separation: quantum O(1) vs classical Ω(2^n) rigorously proven

\item Novel quantum walk: outperforms classical by ≥ quadratic factor on new problem

\item QAOA theory: approximation ratio bounds proven for specific graph classes

\item All certificates verified: success probabilities, query counts, circuit parameters



\bigskip\hrule\bigskip


\subsection{6. Verification Protocol}


\subsubsection{Automated Checks (Run After Every Phase)}

\begin{lstlisting}
def verify_quantum_algorithm_certificate(cert: QuantumAlgorithmCertificate) -> Dict[str, bool]:
    """
    Verify quantum algorithm certificate.

    Returns: Dictionary of Boolean checks.
    """
    checks = {}

    # 1. Speedup calculation
    theoretical_speedup = cert.classical_queries_lower_bound / max(cert.quantum_queries, 1)
    checks['speedup_correct'] = np.isclose(cert.speedup, theoretical_speedup, rtol=0.01)

    # 2. Success probability within bounds
    checks['success_prob_valid'] = 0.0 <= cert.success_probability <= 1.0
    checks['error_bound_valid'] = cert.success_probability >= 1 - cert.error_bound

    # 3. Query complexity matches algorithm
    if cert.algorithm_name == "Grover Search":
        N = cert.problem_size
        expected_queries = int(np.pi * np.sqrt(N) / 4)
        checks['queries_optimal'] = abs(cert.quantum_queries - expected_queries) <= 2

    # 4. Circuit parameters consistent
    checks['qubit_count_valid'] = cert.qubit_count >= int(np.log2(cert.problem_size))
    checks['gate_count_positive'] = cert.gate_count > 0
    checks['depth_reasonable'] = cert.circuit_depth <= cert.gate_count

    return checks

# Example usage
cert_example = QuantumAlgorithmCertificate(
    algorithm_name="Grover Search",
    problem_size=256,
    quantum_queries=13,
    classical_queries_lower_bound=128,
    speedup=9.85,
    success_probability=0.996,
    error_bound=1/256,
    qubit_count=8,
    gate_count=256,
    circuit_depth=13,
    correctness_verified=True,
    optimality_proof="Adversary method",
    timestamp=datetime.now().isoformat(),
    simulation_time=0.5
)

verification = verify_quantum_algorithm_certificate(cert_example)
print("Certificate Verification:")
for check, passed in verification.items():
    status = "✓ PASS" if passed else "✗ FAIL"
    print(f"  {status}: {check}")
\end{lstlisting}


\subsubsection{Cross-Validation Against Known Results}

\begin{lstlisting}
KNOWN_COMPLEXITY_BOUNDS = {
    'Grover': {'quantum': lambda N: np.pi * np.sqrt(N) / 4, 'classical': lambda N: N},
    'ElementDistinctness': {'quantum': lambda N: N**(2/3), 'classical': lambda N: N},
    'CollisionFinding': {'quantum': lambda N: N**(1/3), 'classical': lambda N: np.sqrt(N)},
}

def cross_validate_complexity(algorithm: str, N: int, measured_queries: int):
    """Compare measured query complexity to theoretical bounds."""
    if algorithm in KNOWN_COMPLEXITY_BOUNDS:
        expected = KNOWN_COMPLEXITY_BOUNDS[algorithm]['quantum'](N)
        error = abs(measured_queries - expected) / expected
        print(f"{algorithm}: measured={measured_queries}, expected={expected:.2f}, error={error:.2%}")
        assert error < 0.1, f"Query complexity deviates >10% from theory"
\end{lstlisting}


\bigskip\hrule\bigskip


\subsection{7. Resources and Milestones}


\subsubsection{Essential References}

\end{itemize}

\textbf{Foundational Papers}:

\begin{itemize}
\item L. Grover, "A Fast Quantum Mechanical Algorithm for Database Search", STOC 1996

\item P. Shor, "Algorithms for Quantum Computation: Discrete Logarithms and Factoring", FOCS 1994

\item C. Bennett et al., "Strengths and Weaknesses of Quantum Computing", SIAM J. Comp. 26, 1510 (1997)


\end{itemize}

\textbf{Quantum Walks}:

\begin{itemize}
\item A. Ambainis, "Quantum Walk Algorithm for Element Distinctness", FOCS 2004

\item A. Childs et al., "Exponential Algorithmic Speedup by Quantum Walk", STOC 2003


\end{itemize}

\textbf{Complexity Theory}:

\begin{itemize}
\item S. Aaronson, Y. Shi, "Quantum Lower Bounds for the Collision and Element Distinctness Problems", JACM 51, 595 (2004)

\item A. Ambainis, "Polynomial Degree and Lower Bounds in Quantum Complexity", CCC 2003


\end{itemize}

\textbf{Variational Algorithms}:

\begin{itemize}
\item E. Farhi, J. Goldstone, S. Gutmann, "A Quantum Approximate Optimization Algorithm", arXiv:1411.4028 (2014)

\item A. Peruzzo et al., "A Variational Eigenvalue Solver on a Photonic Quantum Processor", Nat. Commun. 5, 4213 (2014)


\end{itemize}

\textbf{Reviews}:

\begin{itemize}
\item M. Nielsen, I. Chuang, "Quantum Computation and Quantum Information" (Cambridge, 2010) [\textbf{Start here}]



\subsubsection{Software Tools}

\item \textbf{NumPy} (v1.24+): Matrix exponentiation, eigenvalue decomposition

\item \textbf{SciPy} (scipy.linalg): expm, eigh for quantum evolution

\item \textbf{NetworkX} (v3.0+): Graph construction for quantum walks

\item \textbf{Qiskit} (optional): Cross-check against IBM's quantum simulator (for validation)



\subsubsection{Common Pitfalls}

\item \textbf{Small-Angle Approximation in Grover}: Near-optimal iterations are crucial; rounding errors can degrade success probability

\item \textbf{Phase Kickback}: Incorrect oracle implementation can miss phase flip, breaking Grover

\item \textbf{QAOA Optimization Landscape}: Non-convex, many local minima; use multiple random initializations

\item \textbf{HHL Caveats}: Exponential speedup requires efficient state preparation (often not achievable in practice)

\item \textbf{Quantum Walk Encoding}: Shift operator must respect graph structure; incorrect indexing breaks unitarity



\subsubsection{Milestone Checklist}

\end{itemize}

\textbf{Month 2}:

\begin{itemize}
\item [x] Grover's algorithm: N=16, 64, 256 with success probability ≥0.99

\item [x] Query complexity verified: ≤ ⌈π√N/4⌉ + 2

\item [x] Adversary lower bound: Ω(√N) proven via spectral norm


\end{itemize}

\textbf{Month 4}:

\begin{itemize}
\item [ ] Quantum walks: coined and continuous-time on cycle, hypercube

\item [ ] Hitting time analysis: quantum O(N) vs classical O(N²) on cycle

\item [ ] HHL algorithm: solving 8×8 systems with fidelity >0.99


\end{itemize}

\textbf{Month 6}:

\begin{itemize}
\item [ ] QAOA: MaxCut on 10-node graphs, approximation ratio ≥0.7

\item [ ] Query lower bounds: element distinctness Ω(N^{2/3}), collision Ω(N^{1/3})

\item [ ] Complexity analysis complete for all algorithms


\end{itemize}

\textbf{Month 9}:

\begin{itemize}
\item [ ] BQP vs BPP oracle separation: Recursive Fourier Sampling

\item [ ] Novel quantum walk application identified and tested

\item [ ] Comprehensive database: 1000+ algorithm runs with certificates

\item [ ] Research paper draft: "Provable Quantum Advantage via Pure Thought"



\bigskip\hrule\bigskip

\end{itemize}

\textbf{End of PRD 25: Quantum Algorithms and Computational Complexity}


\textit{Pure thought investigation of quantum computational advantage through rigorous implementation and complexity analysis. All speedups proven via adversary and polynomial methods, with machine-checkable certificates.}


\end{document}
