\documentclass[11pt,a4paper]{article}

% Packages
\usepackage[utf8]{inputenc}
\usepackage[T1]{fontenc}
\usepackage{lmodern}
\usepackage[english]{babel}
\usepackage{amsmath,amssymb,amsthm}
\usepackage{mathtools}
\usepackage{physics}
\usepackage{graphicx}
\usepackage{xcolor}
\usepackage{listings}
\usepackage{hyperref}
\usepackage{geometry}
\usepackage{fancyhdr}
\usepackage{tocloft}
\usepackage{enumitem}
\usepackage{booktabs}
\usepackage{algorithm}
\usepackage{algpseudocode}

% Page geometry
\geometry{
    a4paper,
    left=25mm,
    right=25mm,
    top=30mm,
    bottom=30mm,
}

% Header and footer
\pagestyle{fancy}
\fancyhf{}
\fancyhead[L]{Pure Thought Challenge 03}
\fancyhead[R]{\thepage}
\renewcommand{\headrulewidth}{0.4pt}

% Hyperref setup
\hypersetup{
    colorlinks=true,
    linkcolor=blue,
    citecolor=blue,
    urlcolor=blue,
    pdfauthor={Pure Thought AI Challenges},
    pdftitle={Challenge 03: Celestial CFT Bootstrap},
}

% Code listing style
\definecolor{codegray}{rgb}{0.95,0.95,0.95}
\definecolor{codegreen}{rgb}{0,0.6,0}
\definecolor{codepurple}{rgb}{0.58,0,0.82}

\lstdefinestyle{pythonstyle}{
    language=Python,
    backgroundcolor=\color{codegray},
    commentstyle=\color{codegreen},
    keywordstyle=\color{blue},
    stringstyle=\color{codepurple},
    basicstyle=\ttfamily\small,
    breaklines=true,
    breakatwhitespace=true,
    captionpos=b,
    frame=single,
    numbers=left,
    numberstyle=\tiny\color{gray},
    tabsize=4,
    showstringspaces=false,
}

\lstset{style=pythonstyle}

% Theorem environments
\newtheorem{theorem}{Theorem}[section]
\newtheorem{lemma}[theorem]{Lemma}
\newtheorem{proposition}[theorem]{Proposition}
\newtheorem{corollary}[theorem]{Corollary}
\theoremstyle{definition}
\newtheorem{definition}[theorem]{Definition}
\newtheorem{example}[theorem]{Example}
\theoremstyle{remark}
\newtheorem{remark}[theorem]{Remark}

% Custom commands
\newcommand{\checklist}[1]{\item[$\square$] #1}
\newcommand{\R}{\mathbb{R}}
\newcommand{\C}{\mathbb{C}}
\newcommand{\Z}{\mathbb{Z}}
\newcommand{\N}{\mathbb{N}}

% Title information
\title{\textbf{Challenge 03: Celestial CFT Bootstrap} \\
\large Pure Thought AI Challenge 03}
\author{Pure Thought AI Challenges Project}
\date{\today}

\begin{document}

\maketitle
\thispagestyle{empty}

\begin{abstract}
This document presents a comprehensive Product Requirement Document (PRD) for implementing a pure-thought computational challenge. The problem can be tackled using only symbolic mathematics, exact arithmetic, and fresh code---no experimental data or materials databases required until final verification. All results must be accompanied by machine-checkable certificates.
\end{abstract}

\clearpage
\tableofcontents
\clearpage


\textbf{Domain:} Quantum Gravity & Particle Physics

\textbf{Difficulty:} High

\textbf{Timeline:} 6-12 months

\textbf{Prerequisites:} Scattering amplitudes, conformal field theory, Mellin transforms



\bigskip\hrule\bigskip


\subsection{Problem Statement}


\subsubsection{Scientific Context}
Celestial holography reformulates 4D flat-space quantum gravity as a 2D conformal field theory living on the "celestial sphere" at null infinity. Scattering amplitudes are mapped to correlation functions of operators with conformal weights (Δ, Δ̄) on the celestial sphere via Mellin transforms.



\subsubsection{The Core Question}
\textbf{What is the consistent space of celestial CFTs compatible with graviton scattering amplitudes?}


Must satisfy:

\begin{itemize}
\item SL(2,ℂ) covariance (celestial conformal symmetry)

\item Crossing symmetry

\item Unitarity (positive norms)

\item Weinberg soft graviton theorem

\item Regge boundedness at high energies



\subsubsection{Why This Matters}
\item \textbf{New holographic paradigm:} Connects 4D gravity to 2D CFT without AdS geometry

\item \textbf{Rigorous constraints:} Bootstrap approach carves out consistent theories

\item \textbf{Testable:} Produces islands/no-go regions verifiable by amplitude calculations



\bigskip\hrule\bigskip


\subsection{Mathematical Formulation}


\subsubsection{Problem Definition}

\end{itemize}

\textbf{Celestial amplitude} (conformal primary wavefunction):

\begin{lstlisting}
Ã(Δᵢ, z_i, z̄_i) = ∫ dⁿω ω_1^{Δ₁-1} ... ω_n^{Δₙ-1} A(ω_i, z_i, z̄_i)
\end{lstlisting}

where:

\begin{itemize}
\item A is the usual momentum-space amplitude

\item ω_i are energy variables

\item (z\textit{i, z̄}i) are celestial sphere coordinates

\item Δ_i are conformal weights (generically complex)


\end{itemize}

\textbf{Constraints:}


\begin{itemize}
\item \textbf{SL(2,ℂ) covariance:}

\begin{lstlisting}
   Ã transforms as CFT₂ correlator under z → (az+b)/(cz+d)
\end{lstlisting}

\item \textbf{Crossing symmetry:}

\end{itemize}

   Celestial OPE must be associative


\begin{itemize}
\item \textbf{Unitarity:}

\end{itemize}

   Positive-definite celestial inner product


\begin{itemize}
\item \textbf{Soft theorem (Δ → 0):}

\begin{lstlisting}
   Ã(Δ→0) ~ S⁰/Δ + S¹/Δ² + ...
\end{lstlisting}
\end{itemize}

   where S⁰, S¹ are Weinberg soft factors


\begin{itemize}
\item \textbf{Regge boundedness:}

\begin{lstlisting}
   Ã(Δ) bounded in Regge limit
\end{lstlisting}

\end{itemize}

\textbf{Bootstrap formulation:}


Find celestial OPE data {C\textit{ijk, Δ}i} satisfying all constraints.


Use crossing equations + positivity to bound or determine OPE coefficients.



\subsubsection{Certificate of Correctness}

\textbf{If feasible (consistent celestial CFT found):}

\begin{itemize}
\item Explicit OPE data: conformal weights {Δ\textit{i} and structure constants {C}ijk}

\item Verification: construct celestial correlators, check all constraints

\item Cross-check: compute flat-space amplitude, verify it's physical


\end{itemize}

\textbf{If infeasible:}

\begin{itemize}
\item Extremal functional α(Δ, z, z̄) proving no solution exists

\item Verification: show α applied to crossing equation gives contradiction



\bigskip\hrule\bigskip


\subsection{Implementation Approach}


\subsubsection{Phase 1: Celestial Amplitude Calculator (Months 1-2)}

\end{itemize}

\textbf{Build Mellin transform engine:}


\begin{lstlisting}
import sympy as sp
from mpmath import mp

mp.dps = 50  # High precision

def mellin_transform(amplitude, omega_vars, delta_vars):
    """
    Compute celestial amplitude via Mellin transform

    Ã(Δᵢ) = ∫ dⁿω ∏ᵢ ω_i^{Δᵢ-1} A(ωᵢ)
    """
    integrand = amplitude
    for omega, delta in zip(omega_vars, delta_vars):
        integrand *= omega**(delta - 1)

    # Integrate (numerically or symbolically)
    return mp.quad(integrand, [0, mp.inf] * len(omega_vars))
\end{lstlisting}

\textbf{Test cases:}


\begin{itemize}
\item \textbf{3-point graviton amplitude:}

\begin{lstlisting}
   def graviton_3pt_amplitude(omega1, omega2, omega3, z1, z2, z3):
       # Momentum conservation: ω₁ + ω₂ + ω₃ = 0 (on support)
       return momentum_conserving_delta * kinematic_factor(z1, z2, z3)

   celestial_3pt = mellin_transform(graviton_3pt_amplitude, ...)
   # Should give pure conformal structure
\end{lstlisting}

\item \textbf{4-point MHV amplitude:}

\end{itemize}

   Compute celestial transform and verify SL(2,ℂ) covariance



\subsubsection{Phase 2: Conformal Block Decomposition (Months 2-4)}

\textbf{Celestial conformal blocks:}


\begin{lstlisting}
def celestial_conformal_block(Delta, z, zbar):
    """
    Conformal block for celestial CFT (continuous-spin representation)
    """
    # Use recursion relation or differential equation
    return hypergeometric_solution(Delta, z, zbar)
\end{lstlisting}

\textbf{OPE decomposition:}


\begin{lstlisting}
def ope_expansion(celestial_4pt, channel='s'):
    """
    Decompose 4-point function into conformal blocks

    Ã₄ = Σ_Δ C²(Δ) G_Δ(z, z̄)
    """
    blocks = [celestial_conformal_block(Delta, z, zbar) for Delta in spectrum]
    coefficients = fit_ope_coefficients(celestial_4pt, blocks)
    return {Delta: C for Delta, C in zip(spectrum, coefficients)}
\end{lstlisting}


\subsubsection{Phase 3: Crossing Equations (Months 4-6)}

\textbf{Set up crossing symmetry:}


\begin{lstlisting}
def crossing_equation(ope_data_s, ope_data_t):
    """
    Verify/impose s-channel = t-channel OPE

    Σ_Δs C_s(Δs) G_Δs = Σ_Δt C_t(Δt) G_Δt F_st(Δt)
    """
    lhs = sum(C_s[Delta] * block_s(Delta) for Delta in ope_data_s)
    rhs = sum(C_t[Delta] * block_t(Delta) * crossing_kernel(Delta)
              for Delta in ope_data_t)

    return lhs - rhs  # Should equal zero
\end{lstlisting}

\textbf{Soft theorem constraints:}


\begin{lstlisting}
def impose_soft_theorem(ope_data, Delta_soft):
    """
    Ã(Δ→0) ~ S⁰/Δ + subleading

    Weinberg: S⁰ = (ε₁·ε₂)/(z₁-z₂) + perms
    """
    # Extract residue at Δ=0
    residue = extract_residue(ope_data, Delta_soft, pole=1)

    # Check matches Weinberg's soft factor
    assert is_close(residue, weinberg_soft_factor())
\end{lstlisting}


\subsubsection{Phase 4: Bootstrap SDP (Months 6-9)}

\textbf{Formulate optimization:}


\begin{lstlisting}
import cvxpy as cp

# Variables: OPE coefficients (continuous spectrum!)
# Discretize Δ axis
Delta_grid = np.linspace(0.1, 10, num=1000)
C_s = cp.Variable(len(Delta_grid))
C_t = cp.Variable(len(Delta_grid))

# Constraints
constraints = []

# 1. Positivity: |C(Δ)|² ≥ 0 (automatic for real C)
constraints.append(C_s >= 0)
constraints.append(C_t >= 0)

# 2. Crossing symmetry (discretized)
for z_i in z_grid:
    crossing_residual = compute_crossing_residual(C_s, C_t, z_i)
    constraints.append(crossing_residual == 0)  # Or ≈ 0

# 3. Soft theorem
constraints.append(enforce_soft_behavior(C_s, C_t))

# Objective: search for extremal functionals or bound OPE data
# (Similar to conformal bootstrap)
\end{lstlisting}


\subsubsection{Phase 5: Extract Results (Months 9-12)}

\textbf{Allowed regions:}


\begin{lstlisting}
def scan_celestial_cft_space():
    """
    Scan over assumptions (e.g., minimal Δ_gap) and map
    allowed vs. forbidden regions
    """
    results = {}
    for Delta_gap in np.linspace(1, 10, 50):
        try:
            ope_data = solve_bootstrap(Delta_gap)
            results[Delta_gap] = {"status": "feasible", "data": ope_data}
        except InfeasibleError as e:
            results[Delta_gap] = {"status": "infeasible", "certificate": e.dual}

    return results
\end{lstlisting}


\bigskip\hrule\bigskip


\subsection{Example Starting Prompt}

\begin{lstlisting}
I need you to implement the celestial CFT bootstrap from scratch.

GOAL: Determine if a consistent celestial CFT exists for graviton scattering
with a minimal conformal weight gap Δ_gap = 2.

PHASE 1 - Celestial amplitudes:
1. Write a function that takes a momentum-space amplitude A(ωᵢ, zᵢ)
   and computes its Mellin transform to celestial amplitude Ã(Δᵢ, zᵢ).

2. Implement the 3-point graviton amplitude and compute its celestial version.
   Verify it has the right conformal weight structure.

3. Compute the 4-point MHV graviton amplitude's celestial transform.

PHASE 2 - Conformal blocks:
4. Implement celestial conformal blocks G_Δ(z, z̄) for continuous-spin
   representations.

5. Decompose the celestial 4-point amplitude into blocks:
   Ã₄ = Σ_Δ C²(Δ) G_Δ(z, z̄)

PHASE 3 - Crossing & soft theorems:
6. Write down the s-t channel crossing equation.

7. Impose the Weinberg soft graviton theorem as a constraint on Δ→0 behavior.

PHASE 4 - Bootstrap:
8. Formulate as SDP: find OPE data {C(Δ)} satisfying crossing + positivity
   with Δ ≥ Δ_gap.

9. Either find a feasible solution OR extract an extremal functional
   proving no solution exists.

Please use high-precision arithmetic and verify all conformal symmetry
transformations explicitly.
\end{lstlisting}


\bigskip\hrule\bigskip


\subsection{Success Criteria}


\subsubsection{Minimum Viable Result (6 months)}

✅ \textbf{Celestial amplitude machinery working:}

\begin{itemize}
\item 3-pt and 4-pt amplitudes computed celestially

\item SL(2,ℂ) covariance verified numerically

\item Soft theorems checked


\end{itemize}

✅ \textbf{First bootstrap result:}

\begin{itemize}
\item Crossing equation setup for simple subsector (e.g., MHV only)

\item Either: allowed OPE data found, or no-go region certified



\subsubsection{Strong Result (9 months)}

\end{itemize}

✅ \textbf{Multi-channel bootstrap:}

\begin{itemize}
\item Include all helicity sectors

\item Crossing equations in all channels

\item Soft+subsubleading soft theorems


\end{itemize}

✅ \textbf{Islands/bounds:}

\begin{itemize}
\item Rigorous allowed region in OPE space

\item Or: exclusion of certain conformal weight ranges



\subsubsection{Publication-Quality Result (12 months)}

\end{itemize}

✅ \textbf{Comprehensive classification:}

\begin{itemize}
\item Full celestial CFT space mapped for graviton scattering

\item Phase diagram of consistent theories

\item Novel predictions or no-go theorems


\end{itemize}

✅ \textbf{Formal verification:}

\begin{itemize}
\item Certificates exported and verified

\item Amplitude calculations cross-checked

\item Lean formalization of key results



\bigskip\hrule\bigskip


\subsection{Verification Protocol}

\begin{lstlisting}
def verify_celestial_cft(ope_data):
    # 1. Verify SL(2,C) covariance
    for transformation in sl2c_generators():
        transformed = apply_transformation(ope_data, transformation)
        assert is_equivalent(transformed, ope_data)

    # 2. Check crossing symmetry
    for z_point in test_points:
        s_channel = ope_sum(ope_data['s'], z_point)
        t_channel = ope_sum(ope_data['t'], z_point)
        assert abs(s_channel - t_channel) < 1e-10

    # 3. Verify soft theorem
    soft_behavior = extract_small_delta(ope_data, Delta=1e-6)
    weinberg = compute_weinberg_soft()
    assert is_close(soft_behavior, weinberg, rtol=1e-8)

    # 4. Unitarity (positive spectral density)
    assert all(c >= 0 for c in ope_data['C_squared'])

    return "VERIFIED"
\end{lstlisting}


\bigskip\hrule\bigskip


\subsection{Milestone Checklist}

\item [ ] Mellin transform calculator implemented

\item [ ] 3-pt graviton celestial amplitude computed

\item [ ] 4-pt MHV celestial amplitude computed

\item [ ] SL(2,ℂ) covariance verified

\item [ ] Celestial conformal blocks implemented

\item [ ] OPE decomposition working

\item [ ] Crossing equations formulated

\item [ ] Soft theorem constraints imposed

\item [ ] Bootstrap SDP solver running

\item [ ] First allowed/forbidden result obtained

\item [ ] Extremal functionals extracted (if applicable)

\item [ ] Results formally verified



\bigskip\hrule\bigskip

\end{itemize}

\textbf{Next Steps:} Begin with implementing Mellin transforms for known tree-level amplitudes and verify conformal covariance before attempting the bootstrap.


\end{document}
