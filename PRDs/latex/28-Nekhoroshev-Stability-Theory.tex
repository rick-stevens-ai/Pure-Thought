\documentclass[11pt,a4paper]{article}

% Packages
\usepackage[utf8]{inputenc}
\usepackage[T1]{fontenc}
\usepackage{lmodern}
\usepackage[english]{babel}
\usepackage{amsmath,amssymb,amsthm}
\usepackage{mathtools}
\usepackage{physics}
\usepackage{graphicx}
\usepackage{xcolor}
\usepackage{listings}
\usepackage{hyperref}
\usepackage{geometry}
\usepackage{fancyhdr}
\usepackage{tocloft}
\usepackage{enumitem}
\usepackage{booktabs}
\usepackage{algorithm}
\usepackage{algpseudocode}

% Page geometry
\geometry{
    a4paper,
    left=25mm,
    right=25mm,
    top=30mm,
    bottom=30mm,
}

% Header and footer
\pagestyle{fancy}
\fancyhf{}
\fancyhead[L]{Pure Thought Challenge 28}
\fancyhead[R]{\thepage}
\renewcommand{\headrulewidth}{0.4pt}

% Hyperref setup
\hypersetup{
    colorlinks=true,
    linkcolor=blue,
    citecolor=blue,
    urlcolor=blue,
    pdfauthor={Pure Thought AI Challenges},
    pdftitle={PRD 28: Nekhoroshev Stability and Exponential Timescales},
}

% Code listing style
\definecolor{codegray}{rgb}{0.95,0.95,0.95}
\definecolor{codegreen}{rgb}{0,0.6,0}
\definecolor{codepurple}{rgb}{0.58,0,0.82}

\lstdefinestyle{pythonstyle}{
    language=Python,
    backgroundcolor=\color{codegray},
    commentstyle=\color{codegreen},
    keywordstyle=\color{blue},
    stringstyle=\color{codepurple},
    basicstyle=\ttfamily\small,
    breaklines=true,
    breakatwhitespace=true,
    captionpos=b,
    frame=single,
    numbers=left,
    numberstyle=\tiny\color{gray},
    tabsize=4,
    showstringspaces=false,
}

\lstset{style=pythonstyle}

% Theorem environments
\newtheorem{theorem}{Theorem}[section]
\newtheorem{lemma}[theorem]{Lemma}
\newtheorem{proposition}[theorem]{Proposition}
\newtheorem{corollary}[theorem]{Corollary}
\theoremstyle{definition}
\newtheorem{definition}[theorem]{Definition}
\newtheorem{example}[theorem]{Example}
\theoremstyle{remark}
\newtheorem{remark}[theorem]{Remark}

% Custom commands
\newcommand{\checklist}[1]{\item[$\square$] #1}
\newcommand{\R}{\mathbb{R}}
\newcommand{\C}{\mathbb{C}}
\newcommand{\Z}{\mathbb{Z}}
\newcommand{\N}{\mathbb{N}}

% Title information
\title{\textbf{PRD 28: Nekhoroshev Stability and Exponential Timescales} \\
\large Pure Thought AI Challenge 28}
\author{Pure Thought AI Challenges Project}
\date{\today}

\begin{document}

\maketitle
\thispagestyle{empty}

\begin{abstract}
This document presents a comprehensive Product Requirement Document (PRD) for implementing a pure-thought computational challenge. The problem can be tackled using only symbolic mathematics, exact arithmetic, and fresh code---no experimental data or materials databases required until final verification. All results must be accompanied by machine-checkable certificates.
\end{abstract}

\clearpage
\tableofcontents
\clearpage


\textbf{Domain}: Celestial Mechanics & Hamiltonian Dynamics

\textbf{Timeline}: 6-9 months

\textbf{Difficulty}: High

\textbf{Prerequisites}: Hamiltonian mechanics, perturbation theory, Fourier analysis, symplectic geometry



\bigskip\hrule\bigskip


\subsection{1. Problem Statement}


\subsubsection{Scientific Context}

\textbf{Nekhoroshev stability theory} (1977) provides a fundamental complement to KAM theory for understanding long-term stability in nearly-integrable Hamiltonian systems. While KAM theory guarantees eternal stability on measure-large invariant tori, it fails near resonances where tori break down. Nekhoroshev theory fills this gap by proving that even in resonant regions, the system exhibits \textbf{super-exponentially slow diffusion} over timescales that grow exponentially with the inverse perturbation strength.


For a near-integrable Hamiltonian H = H₀(I) + εH₁(I,θ), Nekhoroshev's theorem states:


\textbf{Main Result}: If H₀ satisfies a \textit{steepness} (quasi-convexity) condition, then for all initial conditions and times |t| < T_exp ~ exp(ε^{-a}), the action variables remain close to their initial values:


|I(t) - I(0)| < ε^b


where a, b > 0 depend on dimension and the steepness properties of H₀.


This result has profound implications for \textbf{solar system stability}: with ε ~ 10^{-3} (ratio of planetary to solar mass), the exponential timescale exp((10^{-3})^{-1/2n}) far exceeds the age of the solar system for n ≥ 5 degrees of freedom, providing rigorous mathematical explanation for why planets don't fly off into interstellar space despite strong gravitational perturbations.



\subsubsection{Core Question}

\textbf{Can we rigorously verify Nekhoroshev stability conditions for realistic Hamiltonian systems and compute explicit exponential stability timescales?}


Key challenges:

\begin{itemize}
\item \textbf{Steepness verification}: Checking H₀ is steep (quasi-convex) requires proving det(∂²H₀/∂I²) > C > 0 globally

\item \textbf{Optimal exponents}: Constants a, b depend on dimension and steepness in complex ways

\item \textbf{Resonance structure}: Exponential time depends on Fourier spectrum of H₁

\item \textbf{N-planet problem}: Solar system requires handling multiple gravitational perturbations

\item \textbf{Certificate generation}: Stability bounds must be machine-checkable with interval arithmetic



\subsubsection{Why This Matters}

\item \textbf{Celestial mechanics}: Explains stability of solar system over Gyr timescales

\item \textbf{Accelerator physics}: Particle beam stability in synchrotrons, colliders

\item \textbf{Plasma confinement}: Charged particle motion in tokamaks

\item \textbf{Astrodynamics}: Long-term satellite orbit prediction

\item \textbf{Mathematical physics}: Universal mechanism for slow chaos in Hamiltonian systems



\subsubsection{Pure Thought Advantages}

\end{itemize}

Nekhoroshev theory is \textbf{ideal for pure thought investigation}:

\begin{itemize}
\item ✅ Based on \textbf{symbolic perturbation theory} (no numerical integration needed)

\item ✅ Steepness conditions verifiable via \textbf{computer algebra} (exact Hessian computation)

\item ✅ Exponential estimates computed from \textbf{Fourier coefficients} (symbolic)

\item ✅ All bounds \textbf{certified via interval arithmetic} (rigorous error control)

\item ❌ NO numerical simulations until final verification phase

\item ❌ NO empirical fitting of stability times



\bigskip\hrule\bigskip


\subsection{2. Mathematical Formulation}


\subsubsection{Hamiltonian Setup}

\end{itemize}

Consider a nearly-integrable Hamiltonian on the phase space (I,θ) ∈ ℝ^n × 𝕋^n:


H(I,θ) = H₀(I) + εH₁(I,θ)


where:

\begin{itemize}
\item H₀(I): integrable part (e.g., Kepler Hamiltonian for planets)

\item H₁(I,θ): perturbation (e.g., planet-planet gravitational interactions)

\item ε > 0: small parameter (typically ε ~ 10^{-3} for solar system)


\end{itemize}

Hamilton's equations:

\begin{lstlisting}
dI/dt = -∂H/∂θ = -ε∂H₁/∂θ
dθ/dt = ∂H/∂I = ω(I) + ε∂H₁/∂I
\end{lstlisting}

where ω(I) = ∂H₀/∂I are the unperturbed frequencies.



\subsubsection{Steepness Conditions}

\textbf{Definition (Steepness)}: H₀ is \textbf{steep} (or quasi-convex) if there exists a convex function S(I) such that:


|∂^|α| S/∂I^α| ≤ C_α


for all multi-indices |α| ≤ 3, and the Hessian satisfies:


det(∂²S/∂I²) ≥ m > 0


uniformly on a domain D ⊂ ℝ^n.


\textbf{Key Examples}:

\begin{itemize}
\item \textbf{Strictly convex}: H₀(I) = ½⟨I, AI⟩ with A positive definite (harmonic oscillators)

\item \textbf{Kepler problem}: H₀ = -μ/(2I) (steep in I > 0)

\item \textbf{Non-convex but steep}: Many physical Hamiltonians satisfy weaker quasi-convexity


\end{itemize}

\textbf{Verification Strategy}: Use symbolic differentiation to compute ∂²H₀/∂I² exactly, then prove positivity via:

\begin{itemize}
\item Interval arithmetic bounds on eigenvalues

\item SOS (sum-of-squares) decomposition

\item Gröbner basis elimination



\subsubsection{Nekhoroshev Theorem}

\end{itemize}

\textbf{Theorem (Nekhoroshev 1977)}: Let H = H₀(I) + εH₁(I,θ) with H₀ steep and H₁ real-analytic. Then there exist constants a, b, C, ε₀ > 0 such that for all ε < ε₀ and all initial conditions (I₀,θ₀) ∈ D × 𝕋^n:


|I(t) - I₀| < ε^b  for all  |t| < T_exp := C exp(ε^{-a})


\textbf{Exponents}:

\begin{itemize}
\item \textbf{Steep case} (convex): a = 1/(2n), b = 1/(2n)

\item \textbf{Super-steep case} (exponentially convex): a = 1/2, b → 1/2

\item \textbf{General quasi-convex}: a ∼ 1/(2n log(1/ε))


\end{itemize}

\textbf{Interpretation}: Actions diffuse at most ε^b over exponentially long times. For ε = 10^{-3}, n = 5, T_exp ~ exp(10^{3/10}) ~ 10^13 years ≫ age of universe.



\subsubsection{Resonance Width Formula}

Near a resonance k·ω(I) ≈ 0 (k ∈ ℤ^n integer vector), the perturbation H₁ has significant Fourier component:


H₁\textit{k(I) = ∫₀^{2π} ... ∫₀^{2π} H₁(I,θ) e^{-ik·θ} dθ₁...dθ}n


\textbf{Resonance width} (in action space):


Δ\textit{k ~ (ε|H₁}k|)^{1/2}


\textbf{Diffusion mechanism}: Actions can drift by Δ_k when trajectory spends time ~ 1/|k·ω| near the resonance.


\textbf{Nekhoroshev's key insight}: Exponential growth exp(ε^{-a}) arises from the \textit{number of resonances} the system must cross to diffuse distance O(1), which grows exponentially due to Diophantine conditions on frequency vectors.



\subsubsection{Certificates}

All results must come with \textbf{machine-checkable certificates}:


\begin{itemize}
\item \textbf{Steepness certificate}: Interval arithmetic proof that min eigenvalue(∂²H₀/∂I²) > m > 0 on domain

\item \textbf{Fourier bound certificate}: Rigorous bounds on |H₁\textit{k| for all |k| ≤ K}max

\item \textbf{Exponential time certificate}: Lower bound T\textit{exp ≥ T}min from certified constants a, b, C

\item \textbf{Diffusion bound certificate}: Upper bound sup_{t ≤ T} |I(t) - I(0)| < ε^b with error margins


\end{itemize}

\textbf{Export format}: JSON with rational/interval arithmetic entries:

\begin{lstlisting}
{
  "steepness_constant": {"lower": "0.95", "upper": "1.05"},
  "exponent_a": {"value": "0.1", "precision": "1e-3"},
  "exponential_time_years": {"lower": "1e15", "infinite": false},
  "diffusion_bound_AU": {"value": "1e-8", "certified": true}
}
\end{lstlisting}


\bigskip\hrule\bigskip


\subsection{3. Implementation Approach}


\subsubsection{Phase 1 (Months 1-2): Steepness Verification}

\textbf{Goal}: Symbolically compute Hessian of H₀ and prove steepness.


\begin{lstlisting}
import sympy as sp
import numpy as np
from mpmath import mp
mp.dps = 100  # 100-digit precision

def compute_steepness_certificate(H0_symbolic: sp.Expr,
                                  action_vars: list,
                                  domain: dict) -> dict:
    """
    Verify H₀ is steep by proving ∂²H₀/∂I² is positive definite.

    Args:
        H0_symbolic: Symbolic expression for H₀(I)
        action_vars: List of action variables [I1, I2, ..., In]
        domain: Dictionary {I1: (min, max), I2: (min, max), ...}

    Returns:
        Certificate with minimum eigenvalue bounds
    """
    n = len(action_vars)

    # Compute Hessian symbolically
    hessian = sp.Matrix(n, n, lambda i, j:
                       sp.diff(H0_symbolic, action_vars[i], action_vars[j]))

    print(f"Symbolic Hessian computed: {hessian}")

    # Eigenvalue bounds via interval arithmetic
    from mpmath import iv

    min_eigenvalue = float('inf')
    max_eigenvalue = float('-inf')

    # Sample domain with interval arithmetic grid
    n_samples = 20
    for I_point in generate_interval_grid(domain, n_samples):
        # Substitute interval values
        hessian_interval = evaluate_matrix_interval(hessian, action_vars, I_point)

        # Compute eigenvalue bounds
        eigvals_interval = compute_eigenvalue_bounds_interval(hessian_interval)

        min_eigenvalue = min(min_eigenvalue, eigvals_interval['min'])
        max_eigenvalue = max(max_eigenvalue, eigvals_interval['max'])

    is_steep = min_eigenvalue > 0

    return {
        'is_steep': is_steep,
        'min_eigenvalue': min_eigenvalue,
        'max_eigenvalue': max_eigenvalue,
        'steepness_constant': min_eigenvalue if is_steep else None,
        'certificate_type': 'interval_arithmetic',
        'precision_digits': mp.dps
    }


def kepler_hamiltonian_steepness(n_planets: int) -> dict:
    """
    Verify steepness for n-planet Kepler Hamiltonian.

    H₀ = Σᵢ -μᵢ/(2Iᵢ)  (Kepler terms for each planet)

    Hessian: ∂²H₀/∂Iᵢ∂Iⱼ = μᵢ/Iᵢ³ δᵢⱼ (diagonal, positive definite)
    """
    # Symbolic variables
    I_vars = sp.symbols(f'I1:{n_planets+1}', positive=True, real=True)
    mu_vars = sp.symbols(f'mu1:{n_planets+1}', positive=True, real=True)

    # Kepler Hamiltonian
    H0 = sum(-mu_vars[i] / (2*I_vars[i]) for i in range(n_planets))

    # Compute steepness
    domain = {I_vars[i]: (0.1, 10.0) for i in range(n_planets)}  # AU-scale actions

    cert = compute_steepness_certificate(H0, I_vars, domain)

    return cert


def interval_eigenvalue_bound_symmetric(A_intervals: np.ndarray) -> dict:
    """
    Compute rigorous eigenvalue bounds for symmetric interval matrix.

    Uses Gershgorin circle theorem with interval arithmetic.
    """
    n = A_intervals.shape[0]

    lambda_min = float('inf')
    lambda_max = float('-inf')

    for i in range(n):
        # Gershgorin disk center: diagonal entry
        center = A_intervals[i, i]

        # Radius: sum of off-diagonal absolute values
        radius = sum(abs(A_intervals[i, j]) for j in range(n) if j != i)

        lambda_min = min(lambda_min, center.a - radius.b)  # Lower bound
        lambda_max = max(lambda_max, center.b + radius.b)  # Upper bound

    return {'min': lambda_min, 'max': lambda_max}
\end{lstlisting}

\textbf{Validation}: Test on harmonic oscillator H₀ = ½Σᵢωᵢ²Iᵢ² (should give min eigenvalue = min ωᵢ²).



\subsubsection{Phase 2 (Months 2-3): Resonance Analysis}

\textbf{Goal}: Compute Fourier spectrum of perturbation H₁ and estimate resonance widths.


\begin{lstlisting}
def compute_fourier_coefficients_perturbation(H1_symbolic: sp.Expr,
                                              angle_vars: list,
                                              k_max: int = 10) -> dict:
    """
    Compute Fourier coefficients H₁ₖ(I) for |k| ≤ k_max.

    H₁(I,θ) = Σₖ H₁ₖ(I) e^{ik·θ}
    """
    n = len(angle_vars)

    fourier_coeffs = {}

    for k in generate_integer_vectors(n, k_max):
        # Integrate H₁(I,θ) * e^{-ik·θ} over 𝕋^n
        integrand = H1_symbolic * sp.exp(-sp.I * sum(k[i]*angle_vars[i]
                                                     for i in range(n)))

        # Symbolic integration (may be expensive)
        H1_k = (1/(2*sp.pi)**n) * sp.integrate(integrand,
                                               *[(theta, 0, 2*sp.pi)
                                                 for theta in angle_vars])

        fourier_coeffs[tuple(k)] = H1_k

    return fourier_coeffs


def planetary_perturbation_hamiltonian(planets: list) -> sp.Expr:
    """
    Construct H₁ for planet-planet gravitational perturbations.

    H₁ = -G Σᵢ<ⱼ mᵢmⱼ / |rᵢ - rⱼ|

    Expand in Legendre polynomials.
    """
    n = len(planets)

    # Action-angle coordinates
    I_vars = sp.symbols(f'I1:{n+1}', positive=True)
    theta_vars = sp.symbols(f'theta1:{n+1}', real=True)

    # Convert to Cartesian (via Delaunay elements)
    positions = [action_angle_to_cartesian(I_vars[i], theta_vars[i])
                 for i in range(n)]

    H1 = 0
    for i in range(n):
        for j in range(i+1, n):
            r_ij = positions[i] - positions[j]
            r_ij_norm = sp.sqrt(r_ij.dot(r_ij))

            H1 += -planets[i]['G'] * planets[i]['mass'] * planets[j]['mass'] / r_ij_norm

    # Expand to desired order in ε
    H1_expanded = sp.series(H1, planets[0]['mass']/planets[0]['M_sun'], 0, n=3).removeO()

    return H1_expanded


def resonance_width_estimate(k: np.ndarray,
                             epsilon: float,
                             H1_k: float) -> float:
    """
    Width of k-resonance in action space.

    Δₖ ~ √(ε|H₁ₖ|) / |k|
    """
    width = np.sqrt(epsilon * abs(H1_k)) / np.linalg.norm(k)

    return width


def resonance_overlap_criterion(resonance_widths: dict,
                                frequency_map: callable) -> bool:
    """
    Check if resonances overlap (Chirikov criterion).

    Overlap ⟺ Δₖ₁ + Δₖ₂ > |I_{k₁} - I_{k₂}|

    where I_k is center of k-resonance.
    """
    k_vectors = list(resonance_widths.keys())

    for i, k1 in enumerate(k_vectors):
        for k2 in k_vectors[i+1:]:
            # Resonance centers (solve k·ω(I) = 0)
            I_k1 = find_resonance_center(k1, frequency_map)
            I_k2 = find_resonance_center(k2, frequency_map)

            if I_k1 is None or I_k2 is None:
                continue

            # Check overlap
            separation = np.linalg.norm(I_k1 - I_k2)
            combined_width = resonance_widths[k1] + resonance_widths[k2]

            if combined_width > separation:
                return True  # Resonances overlap → no Nekhoroshev stability

    return False  # Well-separated resonances
\end{lstlisting}

\textbf{Validation}: Compute Fourier spectrum for Jupiter-Saturn perturbation, verify dominant k = (5, -2) Great Inequality resonance.



\subsubsection{Phase 3 (Months 3-4): Exponential Time Estimates}

\textbf{Goal}: Compute optimal exponents a, b and stability time T_exp.


\begin{lstlisting}
def nekhoroshev_exponents_optimal(dimension: int,
                                  steepness_type: str,
                                  epsilon: float) -> dict:
    """
    Compute optimal Nekhoroshev exponents a, b.

    Args:
        dimension: Number of degrees of freedom n
        steepness_type: 'convex', 'steep', 'quasi_convex'
        epsilon: Perturbation parameter

    Returns:
        Exponents a, b and constants C
    """
    if steepness_type == 'convex':
        # Best case: strictly convex H₀
        a = 1 / (2 * dimension)
        b = 1 / (2 * dimension)
        C = 1.0

    elif steepness_type == 'steep':
        # Quasi-convex (most physical systems)
        a = 1 / (2 * dimension)
        b = 1 / (4 * dimension)  # Worse diffusion bound
        C = 0.5

    elif steepness_type == 'super_steep':
        # Exponentially convex (rare)
        a = 1 / 2
        b = 1 / 2
        C = 2.0

    else:
        # Generic quasi-convex with logarithmic corrections
        log_factor = np.log(1/epsilon) if epsilon > 0 else 1
        a = 1 / (2 * dimension * log_factor)
        b = 1 / (4 * dimension)
        C = 0.1

    return {'a': a, 'b': b, 'C': C, 'type': steepness_type}


def compute_exponential_stability_time(epsilon: float,
                                       exponents: dict,
                                       time_unit: str = 'years') -> dict:
    """
    Compute T_exp = C exp((ε₀/ε)^a).
    """
    a = exponents['a']
    C = exponents['C']

    # Characteristic scale ε₀ (depends on system)
    epsilon_0 = 1.0  # Normalized units

    # Exponential time
    if epsilon > 0:
        T_exp_normalized = C * np.exp((epsilon_0 / epsilon)**a)
    else:
        T_exp_normalized = float('inf')

    # Convert to physical units
    if time_unit == 'years':
        orbital_period = 1.0  # Normalize to 1 year for outer planets
        T_exp_years = T_exp_normalized * orbital_period
    else:
        T_exp_years = T_exp_normalized

    return {
        'T_exp_normalized': T_exp_normalized,
        'T_exp_years': T_exp_years,
        'exponent_a': a,
        'log_T_exp': a * np.log(1/epsilon) if epsilon > 0 else float('inf')
    }


def solar_system_nekhoroshev_stability() -> dict:
    """
    Apply Nekhoroshev theory to the solar system.

    Key parameters:
    - n = 8 planets (neglect Mercury as interior planet)
    - ε ~ m_Jupiter / m_Sun ~ 10^{-3}
    - H₀: sum of Kepler Hamiltonians (steep)
    - H₁: planetary perturbations (real-analytic)
    """
    # System parameters
    n_planets = 8
    epsilon = 1e-3  # Jupiter mass / Sun mass

    # Nekhoroshev exponents for 8-planet system
    exponents = nekhoroshev_exponents_optimal(
        dimension=n_planets,
        steepness_type='steep',  # Kepler H₀ is steep
        epsilon=epsilon
    )

    # Compute exponential time
    stability = compute_exponential_stability_time(epsilon, exponents)

    # Compare to solar system age
    age_solar_system_years = 4.5e9
    age_universe_years = 13.8e9

    stability_margin = stability['T_exp_years'] / age_solar_system_years

    return {
        'dimension': n_planets,
        'perturbation_parameter': epsilon,
        'exponent_a': exponents['a'],
        'exponent_b': exponents['b'],
        'T_exp_years': stability['T_exp_years'],
        'age_solar_system_years': age_solar_system_years,
        'stability_margin': stability_margin,
        'verdict': 'STABLE' if stability_margin > 10 else 'UNSTABLE'
    }
\end{lstlisting}

\textbf{Validation}: Reproduce T_exp ~ exp(10^{3/10}) ~ 10^13 years for solar system (matches literature estimates).



\subsubsection{Phase 4 (Months 4-6): Action Diffusion Bounds}

\textbf{Goal}: Prove rigorous upper bounds |I(t) - I(0)| < ε^b for t < T_exp.


\begin{lstlisting}
def action_diffusion_bound_certificate(H0: callable,
                                       H1: callable,
                                       epsilon: float,
                                       time_horizon: float,
                                       initial_action: np.ndarray) -> dict:
    """
    Generate certificate for action diffusion bound.

    Proves: |I(t) - I₀| < ε^b for all t ≤ T_horizon
    """
    # Compute Nekhoroshev constants
    dimension = len(initial_action)
    exponents = nekhoroshev_exponents_optimal(dimension, 'steep', epsilon)

    a, b = exponents['a'], exponents['b']

    # Check time is within exponential bound
    T_exp = compute_exponential_stability_time(epsilon, exponents)['T_exp_normalized']

    if time_horizon > T_exp:
        return {
            'certified': False,
            'reason': f'Time horizon {time_horizon} exceeds T_exp = {T_exp}'
        }

    # Diffusion bound: |I(t) - I₀| < ε^b
    diffusion_bound = epsilon**b

    # Certificate using interval arithmetic propagation
    I_interval = propagate_actions_interval_arithmetic(
        H0, H1, epsilon, initial_action, time_horizon
    )

    max_deviation = max(abs(I_interval[i].b - initial_action[i])
                       for i in range(dimension))

    certified = max_deviation < diffusion_bound

    return {
        'certified': certified,
        'diffusion_bound': diffusion_bound,
        'max_deviation_computed': max_deviation,
        'time_horizon': time_horizon,
        'T_exp': T_exp,
        'safety_margin': diffusion_bound / max_deviation if certified else 0
    }


def propagate_actions_interval_arithmetic(H0: callable,
                                         H1: callable,
                                         epsilon: float,
                                         I0: np.ndarray,
                                         T: float,
                                         n_steps: int = 1000) -> list:
    """
    Propagate actions using interval arithmetic to get rigorous bounds.

    dI/dt = -ε ∂H₁/∂θ

    Returns: List of interval boxes [I_min, I_max] at time T
    """
    from mpmath import iv

    dt = T / n_steps
    dimension = len(I0)

    # Initialize interval boxes
    I_intervals = [iv.mpf([I0[i], I0[i]]) for i in range(dimension)]

    for step in range(n_steps):
        # Compute RHS of dI/dt using interval arithmetic
        # (requires interval evaluation of ∂H₁/∂θ)
        dI_dt_intervals = compute_action_derivative_interval(
            H1, I_intervals, epsilon
        )

        # Euler step with interval arithmetic
        for i in range(dimension):
            I_intervals[i] += dt * dI_dt_intervals[i]

    return I_intervals
\end{lstlisting}

\textbf{Validation}: Verify bounds for 2-planet system (Jupiter-Saturn) over 1 Gyr, compare to numerical integration.



\subsubsection{Phase 5 (Months 6-8): Optimal Constants and Sharpness}

\textbf{Goal}: Determine optimal (smallest) exponents a achieving given stability time.


\begin{lstlisting}
def optimize_nekhoroshev_constants(H0: callable,
                                  H1: callable,
                                  epsilon: float,
                                  desired_time: float) -> dict:
    """
    Find optimal constants a, b, C in Nekhoroshev estimate.

    Goal: Maximize a (sharper result) subject to T_exp ≥ desired_time.
    """
    from scipy.optimize import minimize_scalar

    dimension = estimate_dimension(H0)

    def objective(a_trial):
        # For given a, compute achievable T_exp
        C = 1.0  # Fix normalization
        T_exp = C * np.exp((1/epsilon)**a_trial)

        # Penalty if T_exp < desired_time
        if T_exp < desired_time:
            return 1e10  # Infeasible
        else:
            return -a_trial  # Maximize a

    # Optimize over reasonable range
    result = minimize_scalar(objective, bounds=(0.01, 1.0), method='bounded')

    a_optimal = result.x
    b_optimal = a_optimal  # Typically b ~ a for optimal results

    return {
        'a_optimal': a_optimal,
        'b_optimal': b_optimal,
        'T_exp_achieved': np.exp((1/epsilon)**a_optimal),
        'desired_time': desired_time,
        'optimality': 'sharp' if result.fun < -0.1 else 'conservative'
    }


def compare_to_numerical_integration(H_total: callable,
                                     initial_conditions: np.ndarray,
                                     T_max: float,
                                     nekhoroshev_bound: float) -> dict:
    """
    Validate Nekhoroshev bound against numerical integration.

    Integrate Hamilton's equations and check |I(t) - I(0)| < bound.
    """
    from scipy.integrate import solve_ivp

    def hamiltonian_flow(t, y):
        # y = [I, θ]
        dimension = len(y) // 2
        I, theta = y[:dimension], y[dimension:]

        dI_dt = -compute_dH_dtheta(H_total, I, theta)
        dtheta_dt = compute_dH_dI(H_total, I, theta)

        return np.concatenate([dI_dt, dtheta_dt])

    # Integrate
    sol = solve_ivp(hamiltonian_flow,
                   (0, T_max),
                   initial_conditions,
                   method='DOP853',  # High-accuracy
                   rtol=1e-12, atol=1e-14)

    # Extract action variables
    dimension = len(initial_conditions) // 2
    I_trajectory = sol.y[:dimension, :]
    I_initial = initial_conditions[:dimension]

    # Compute maximum deviation
    max_deviation = np.max(np.linalg.norm(I_trajectory - I_initial[:, np.newaxis], axis=0))

    # Compare to Nekhoroshev bound
    bound_satisfied = max_deviation < nekhoroshev_bound

    return {
        'max_deviation': max_deviation,
        'nekhoroshev_bound': nekhoroshev_bound,
        'bound_satisfied': bound_satisfied,
        'safety_factor': nekhoroshev_bound / max_deviation if max_deviation > 0 else float('inf'),
        'integration_time': T_max,
        'n_timesteps': len(sol.t)
    }
\end{lstlisting}


\subsubsection{Phase 6 (Months 8-9): Certificate Generation and Export}

\textbf{Goal}: Generate machine-checkable certificates for all stability results.


\begin{lstlisting}
import json
from dataclasses import dataclass, asdict

@dataclass
class NekhoroshevCertificate:
    """Complete Nekhoroshev stability certificate."""

    # System identification
    hamiltonian_name: str
    dimension: int
    perturbation_parameter: float

    # Steepness certificate
    is_steep: bool
    steepness_constant: float
    steepness_proof_method: str  # 'interval_arithmetic', 'SOS', 'symbolic'

    # Nekhoroshev constants
    exponent_a: float
    exponent_b: float
    constant_C: float

    # Stability estimates
    exponential_time_normalized: float
    exponential_time_years: float
    diffusion_bound: float

    # Verification
    numerical_validation: bool
    max_deviation_observed: float
    integration_time_years: float

    # Metadata
    computation_date: str
    precision_digits: int
    certificate_version: str

    def export_json(self, filename: str):
        """Export certificate to JSON."""
        with open(filename, 'w') as f:
            json.dump(asdict(self), f, indent=2)

    def verify(self) -> bool:
        """Self-check certificate validity."""
        checks = [
            self.is_steep,
            self.steepness_constant > 0,
            self.exponent_a > 0,
            self.exponent_b > 0,
            self.exponential_time_normalized > 0,
            self.diffusion_bound > 0
        ]

        if self.numerical_validation:
            checks.append(self.max_deviation_observed < self.diffusion_bound)

        return all(checks)


def generate_solar_system_certificate() -> NekhoroshevCertificate:
    """
    Generate complete Nekhoroshev certificate for solar system.
    """
    # Run all computations
    steepness = kepler_hamiltonian_steepness(n_planets=8)
    stability = solar_system_nekhoroshev_stability()

    # Numerical validation (expensive—use reduced time)
    validation_time = 1e6  # 1 Myr (much less than T_exp but feasible)
    # validation = compare_to_numerical_integration(...)  # Commented for speed

    cert = NekhoroshevCertificate(
        hamiltonian_name='Solar System (8 planets)',
        dimension=8,
        perturbation_parameter=1e-3,

        is_steep=steepness['is_steep'],
        steepness_constant=steepness['steepness_constant'],
        steepness_proof_method='interval_arithmetic',

        exponent_a=stability['exponent_a'],
        exponent_b=stability['exponent_b'],
        constant_C=1.0,

        exponential_time_normalized=stability['T_exp_years'] / 1e9,  # In Gyr
        exponential_time_years=stability['T_exp_years'],
        diffusion_bound=1e-3**stability['exponent_b'],  # AU

        numerical_validation=False,  # Set to True after running validation
        max_deviation_observed=0.0,
        integration_time_years=validation_time,

        computation_date='2026-01-17',
        precision_digits=100,
        certificate_version='1.0'
    )

    return cert
\end{lstlisting}

\textbf{Validation}: Export certificate, verify all fields satisfy logical constraints.



\bigskip\hrule\bigskip


\subsection{4. Example Starting Prompt}

\textbf{Prompt for AI System}:


You are tasked with applying Nekhoroshev stability theory to verify exponential-time stability of the solar system. Your goal is to:


\begin{itemize}
\item \textbf{Verify Steepness} (Months 1-2):

\item Construct the integrable Hamiltonian H₀ = Σᵢ -GMₛᵤₙmᵢ/(2Iᵢ) for 8 planets

\item Compute the Hessian ∂²H₀/∂I² symbolically using SymPy

\item Prove steepness by showing all eigenvalues are positive using interval arithmetic

\item Generate a steepness certificate with rigorous bounds: min eigenvalue > C > 0


\item \textbf{Analyze Perturbations} (Months 2-3):

\item Construct the perturbation Hamiltonian H₁ representing planet-planet gravitational interactions

\item Expand H₁ in action-angle coordinates using Delaunay elements

\item Compute Fourier coefficients H₁ₖ for |k| ≤ 10 using symbolic integration

\item Identify dominant resonances (e.g., Jupiter-Saturn 5:2 Great Inequality)

\item Estimate resonance widths: Δₖ = √(ε|H₁ₖ|)


\item \textbf{Compute Exponential Times} (Months 3-4):

\item Determine optimal Nekhoroshev exponents for n=8 dimensions: a = 1/(2n) = 1/16

\item Calculate exponential stability time: T_exp = exp((1/ε)^a) with ε = 10^{-3}

\item Convert to physical units: T_exp ≈ 10^13 years

\item Compare to solar system age (4.5 Gyr) and verify stability margin > 10^3


\item \textbf{Prove Diffusion Bounds} (Months 4-6):

\item Use interval arithmetic to propagate action variables forward in time

\item Prove |I(t) - I(0)| < ε^b = (10^{-3})^{1/16} ≈ 0.7 AU for t < T_exp

\item Generate certificate with rigorous error bounds using mpmath (100-digit precision)

\item Validate against numerical integration of Hamilton's equations over 1 Myr


\item \textbf{Optimize Constants} (Months 6-8):

\item Search for optimal (largest) exponent a achieving desired stability time

\item Compare to best-known theoretical results (Niederman 2004, Guzzo et al. 2011)

\item Identify sharpness: is a = 1/(2n) optimal or can it be improved?


\item \textbf{Certificate Generation} (Months 8-9):

\item Create NekhoroshevCertificate object containing all results

\item Export to JSON with interval arithmetic bounds and metadata

\item Self-verify certificate: check all constraints satisfied

\item Compare to literature: reproduce Guzzo et al. (2005) T_exp estimates for Jupiter-Saturn


\end{itemize}

\textbf{Success Criteria}:

\begin{itemize}
\item Minimum Viable Result (2-4 months): Steepness verified for Kepler Hamiltonian, basic exponential time estimate

\item Strong Result (6-8 months): Full solar system analysis with rigorous diffusion bounds and numerical validation

\item Publication-Quality Result (9 months): Optimal exponents, comparison to literature, machine-checkable certificates


\end{itemize}

\textbf{Key Constraints}:

\begin{itemize}
\item Use ONLY symbolic mathematics and interval arithmetic (no floating-point until final validation)

\item All bounds must be certified with explicit error margins

\item Compare to at least 3 literature sources (Nekhoroshev 1977, Niederman 2004, Guzzo+ 2011)

\item Generate JSON export for certificate database


\end{itemize}

\textbf{References}:

\begin{itemize}
\item Nekhoroshev (1977): Original theorem and proof outline

\item Niederman (2004): Optimal exponents and steepness conditions

\item Guzzo, Lega, Froeschlé (2005): Solar system application and numerical validation

\item Morbidelli (2002): Modern Celestial Mechanics textbook treatment


\end{itemize}

Begin by symbolically computing the Hessian of the Kepler Hamiltonian and proving steepness using interval arithmetic.



\bigskip\hrule\bigskip


\subsection{5. Success Criteria}


\subsubsection{Minimum Viable Result (Months 1-4)}

\textbf{Core Achievements}:

\begin{itemize}
\item ✅ Symbolic Hessian computation for n-planet Kepler Hamiltonian

\item ✅ Steepness verification: min eigenvalue > 0 certified via interval arithmetic

\item ✅ Basic exponential time estimate: T_exp = exp((1/ε)^{1/(2n)}) for solar system

\item ✅ Comparison to solar system age: verify T_exp ≫ 4.5 Gyr


\end{itemize}

\textbf{Validation}:

\begin{itemize}
\item Reproduce steepness for 2-planet system (Jupiter-Saturn)

\item Match literature value T_exp ~ 10^13 years for 8-planet system


\end{itemize}

\textbf{Deliverables}:

\begin{itemize}
\item Python module \texttt{nekhoroshev.py} with steepness checker and exponential time calculator

\item Jupyter notebook demonstrating solar system application

\item JSON certificate for Jupiter-Saturn system



\subsubsection{Strong Result (Months 4-8)}

\end{itemize}

\textbf{Extended Capabilities}:

\begin{itemize}
\item ✅ Fourier analysis of planetary perturbation Hamiltonian H₁

\item ✅ Resonance width calculations for all |k| ≤ 10

\item ✅ Rigorous action diffusion bounds: |I(t) - I(0)| < ε^b certified via interval propagation

\item ✅ Numerical validation: integrate Hamilton's equations over 1 Myr, verify bound satisfied

\item ✅ Comparison to 3+ literature sources (Nekhoroshev 1977, Niederman 2004, Guzzo+ 2005)


\end{itemize}

\textbf{Publications Benchmark}:

\begin{itemize}
\item Reproduce Figures 2-4 from Guzzo et al. (2005) showing action diffusion vs time

\item Match resonance widths to within 10% of published values


\end{itemize}

\textbf{Deliverables}:

\begin{itemize}
\item Full \texttt{NekhoroshevCertificate} for 8-planet solar system

\item Validation report comparing analytical bounds to numerical integration

\item Database of resonance widths for 100+ resonances



\subsubsection{Publication-Quality Result (Months 8-9)}

\end{itemize}

\textbf{Novel Contributions}:

\begin{itemize}
\item ✅ Optimal exponent determination: maximize a subject to T_exp ≥ 10 Gyr constraint

\item ✅ Sharpness analysis: compare a_optimal to theoretical lower bounds

\item ✅ Extension to other planetary systems: apply to extrasolar systems (e.g., Kepler-90, TRAPPIST-1)

\item ✅ Formal verification: translate steepness proofs to Lean or Isabelle

\item ✅ Public database: 50+ Nekhoroshev certificates for diverse Hamiltonian systems


\end{itemize}

\textbf{Beyond Literature}:

\begin{itemize}
\item Improve exponents beyond Niederman (2004) for specific system classes

\item Discover new resonances affecting long-term stability

\item Develop automated pipeline: Hamiltonian → certificate (no human intervention)


\end{itemize}

\textbf{Deliverables}:

\begin{itemize}
\item Arxiv preprint: "Rigorous Nekhoroshev Stability Certificates for Planetary Systems"

\item GitHub repository with 500+ test cases

\item Interactive web tool: input planetary masses/orbits → get T_exp estimate



\bigskip\hrule\bigskip


\subsection{6. Verification Protocol}

\begin{lstlisting}
def verify_nekhoroshev_results(certificate: NekhoroshevCertificate) -> dict:
    """
    Automated verification of Nekhoroshev certificate.

    Checks:
    1. Steepness constraint satisfied
    2. Exponents in valid range
    3. Exponential time formula correct
    4. Diffusion bound formula correct
    5. Numerical validation matches bound
    """
    results = {}

    # Check 1: Steepness
    results['steepness_valid'] = (
        certificate.is_steep and
        certificate.steepness_constant > 0
    )

    # Check 2: Exponents
    n = certificate.dimension
    a_expected = 1 / (2 * n)
    results['exponent_a_reasonable'] = (
        0.01 < certificate.exponent_a <= a_expected
    )

    results['exponent_b_reasonable'] = (
        0 < certificate.exponent_b <= certificate.exponent_a
    )

    # Check 3: Exponential time formula
    epsilon = certificate.perturbation_parameter
    a = certificate.exponent_a
    T_exp_recomputed = certificate.constant_C * np.exp((1/epsilon)**a)

    results['exponential_time_correct'] = (
        abs(T_exp_recomputed - certificate.exponential_time_normalized) /
        certificate.exponential_time_normalized < 0.01
    )

    # Check 4: Diffusion bound formula
    b = certificate.exponent_b
    diffusion_bound_recomputed = epsilon**b

    results['diffusion_bound_correct'] = (
        abs(diffusion_bound_recomputed - certificate.diffusion_bound) /
        certificate.diffusion_bound < 0.01
    )

    # Check 5: Numerical validation
    if certificate.numerical_validation:
        results['numerical_bound_satisfied'] = (
            certificate.max_deviation_observed < certificate.diffusion_bound
        )
    else:
        results['numerical_bound_satisfied'] = None  # Not tested

    # Overall verdict
    results['all_checks_passed'] = all(
        v for v in results.values() if v is not None
    )

    return results


def compare_to_literature_benchmarks(our_results: dict,
                                     source: str = 'Guzzo2005') -> dict:
    """
    Compare our Nekhoroshev results to published benchmarks.
    """
    benchmarks = {
        'Guzzo2005': {
            'system': 'Jupiter-Saturn',
            'T_exp_years': 1e13,
            'exponent_a': 0.1,
            'diffusion_bound_AU': 1e-2
        },
        'Niederman2004': {
            'exponent_a_theoretical': lambda n: 1/(2*n),
            'exponent_b_theoretical': lambda n: 1/(2*n)
        }
    }

    if source not in benchmarks:
        return {'error': f'Unknown source {source}'}

    benchmark = benchmarks[source]

    comparison = {}
    for key, value in benchmark.items():
        if key in our_results:
            our_value = our_results[key]
            relative_error = abs(our_value - value) / value
            comparison[key] = {
                'ours': our_value,
                'literature': value,
                'relative_error': relative_error,
                'match': relative_error < 0.1  # 10% tolerance
            }

    return comparison
\end{lstlisting}

\end{itemize}

\textbf{Validation Procedure}:

\begin{itemize}
\item Run \texttt{verify\textit{nekhoroshev}results()} on generated certificate

\item Compare to Guzzo et al. (2005) benchmark values

\item Numerical integration: evolve 2-planet system for 1 Myr, check diffusion < ε^b

\item Cross-check exponents with Niederman (2004) theoretical bounds



\bigskip\hrule\bigskip


\subsection{7. Resources and Milestones}


\subsubsection{Essential References}

\item \textbf{Original Papers}:

\item Nekhoroshev (1977): "An exponential estimate of the time of stability of nearly-integrable Hamiltonian systems"

\item Niederman (2004): "Stability over exponentially long times in the planetary problem"

\item Guzzo, Lega, Froeschlé (2005): "First numerical evidence of global Arnold diffusion in quasi-integrable systems"


\item \textbf{Textbooks}:

\item Morbidelli (2002): \textit{Modern Celestial Mechanics}

\item Arnold, Kozlov, Neishtadt (2006): \textit{Mathematical Aspects of Classical and Celestial Mechanics}

\item Giorgilli (2003): "Exponential stability of Hamiltonian systems"


\item \textbf{Solar System Applications}:

\item Laskar (1989): "A numerical experiment on the chaotic behaviour of the Solar System"

\item Murray & Dermott (1999): \textit{Solar System Dynamics}



\subsubsection{Common Pitfalls}

\item \textbf{Steepness too restrictive}: Not all physical Hamiltonians are convex; use quasi-convex definition

\item \textbf{Exponent optimality}: a = 1/(2n) is not always optimal; dimension-dependent improvements possible

\item \textbf{Resonance overlap}: If resonances overlap (Chirikov criterion), Nekhoroshev theory fails

\item \textbf{Numerical validation expensive}: Integrating N-body systems for Myr timescales requires high-precision symplectic integrators

\item \textbf{Certificate validity}: Interval arithmetic bounds can become loose after many propagation steps



\subsubsection{Milestone Checklist}

\item [ ] \textbf{Month 1}: Symbolic Hessian computed for Kepler Hamiltonian

\item [ ] \textbf{Month 2}: Steepness certified via interval arithmetic for 2-planet system

\item [ ] \textbf{Month 3}: Fourier coefficients H₁ₖ computed for planetary perturbations

\item [ ] \textbf{Month 3}: Resonance widths estimated for |k| ≤ 10

\item [ ] \textbf{Month 4}: Exponential time T_exp computed for 8-planet solar system

\item [ ] \textbf{Month 5}: Action diffusion bounds |I(t) - I(0)| < ε^b proven rigorously

\item [ ] \textbf{Month 6}: Numerical validation: integrate Hamilton's equations for 1 Myr

\item [ ] \textbf{Month 7}: Comparison to 3+ literature sources (errors < 10%)

\item [ ] \textbf{Month 8}: Optimal exponents a, b determined via optimization

\item [ ] \textbf{Month 9}: Complete certificate exported to JSON, self-verification passed

\item [ ] \textbf{Month 9}: Public database: 10+ planetary systems analyzed



\subsubsection{Extensions}

\end{itemize}

\textbf{Immediate Extensions} (post-MVR):

\begin{itemize}
\item Non-convex Hamiltonians: develop quasi-convexity checkers for general systems

\item Symplectic integrators: implement high-order methods for long-time validation

\item Multi-scale perturbations: handle systems with disparate timescales (e.g., inner+outer planets)


\end{itemize}

\textbf{Research Frontiers}:

\begin{itemize}
\item Improve exponents: can a > 1/(2n) be achieved for special classes?

\item Formal verification: translate steepness proofs to Lean/Isabelle

\item Machine learning: train models to predict T_exp from Hamiltonian structure

\item Quantum systems: extend Nekhoroshev theory to quantum Hamiltonians (FKPP theorem)



\bigskip\hrule\bigskip


\subsection{8. Implementation Notes}


\subsubsection{Computational Requirements}

\item \textbf{Symbolic computation}: SymPy for Hessians, Fourier integrals (may be slow for n > 3)

\item \textbf{Interval arithmetic}: mpmath with 100-digit precision for certified bounds

\item \textbf{Numerical integration}: SciPy's \texttt{solve_ivp} with DOP853 for validation (rtol=1e-12)

\item \textbf{Optimization}: SciPy's \texttt{minimize_scalar} for optimal exponent search


\end{itemize}

\textbf{Estimated Runtimes}:

\begin{itemize}
\item Steepness verification: 1 minute (symbolic), 10 minutes (interval arithmetic)

\item Fourier coefficients: 1 hour per resonance (symbolic integration expensive)

\item Exponential time: instant (formula evaluation)

\item Numerical validation (1 Myr): 1 hour on single core (can parallelize)



\subsubsection{Software Dependencies}

\begin{lstlisting}
# requirements.txt
sympy>=1.12
numpy>=1.24
scipy>=1.11
mpmath>=1.3
matplotlib>=3.7
cvxpy>=1.4  # For SDP optimization (future extension)
\end{lstlisting}


\subsubsection{Testing Strategy}

\item \textbf{Unit tests}: Each function validated on toy Hamiltonians (harmonic oscillator, pendulum)

\item \textbf{Integration tests}: Full pipeline tested on 2-planet system (Jupiter-Saturn)

\item \textbf{Regression tests}: Compare to cached results from literature

\item \textbf{Property tests}: Verify mathematical identities (e.g., symplectic flow preserves H)



\bigskip\hrule\bigskip

\end{itemize}

\textbf{End of PRD 28}


\end{document}
