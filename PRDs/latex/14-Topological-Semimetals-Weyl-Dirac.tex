\documentclass[11pt,a4paper]{article}

% Packages
\usepackage[utf8]{inputenc}
\usepackage[T1]{fontenc}
\usepackage{lmodern}
\usepackage[english]{babel}
\usepackage{amsmath,amssymb,amsthm}
\usepackage{mathtools}
\usepackage{physics}
\usepackage{graphicx}
\usepackage{xcolor}
\usepackage{listings}
\usepackage{hyperref}
\usepackage{geometry}
\usepackage{fancyhdr}
\usepackage{tocloft}
\usepackage{enumitem}
\usepackage{booktabs}
\usepackage{algorithm}
\usepackage{algpseudocode}

% Page geometry
\geometry{
    a4paper,
    left=25mm,
    right=25mm,
    top=30mm,
    bottom=30mm,
}

% Header and footer
\pagestyle{fancy}
\fancyhf{}
\fancyhead[L]{Pure Thought Challenge 14}
\fancyhead[R]{\thepage}
\renewcommand{\headrulewidth}{0.4pt}

% Hyperref setup
\hypersetup{
    colorlinks=true,
    linkcolor=blue,
    citecolor=blue,
    urlcolor=blue,
    pdfauthor={Pure Thought AI Challenges},
    pdftitle={PRD 14: Topological Semimetals: Weyl and Dirac Points from Symmetry},
}

% Code listing style
\definecolor{codegray}{rgb}{0.95,0.95,0.95}
\definecolor{codegreen}{rgb}{0,0.6,0}
\definecolor{codepurple}{rgb}{0.58,0,0.82}

\lstdefinestyle{pythonstyle}{
    language=Python,
    backgroundcolor=\color{codegray},
    commentstyle=\color{codegreen},
    keywordstyle=\color{blue},
    stringstyle=\color{codepurple},
    basicstyle=\ttfamily\small,
    breaklines=true,
    breakatwhitespace=true,
    captionpos=b,
    frame=single,
    numbers=left,
    numberstyle=\tiny\color{gray},
    tabsize=4,
    showstringspaces=false,
}

\lstset{style=pythonstyle}

% Theorem environments
\newtheorem{theorem}{Theorem}[section]
\newtheorem{lemma}[theorem]{Lemma}
\newtheorem{proposition}[theorem]{Proposition}
\newtheorem{corollary}[theorem]{Corollary}
\theoremstyle{definition}
\newtheorem{definition}[theorem]{Definition}
\newtheorem{example}[theorem]{Example}
\theoremstyle{remark}
\newtheorem{remark}[theorem]{Remark}

% Custom commands
\newcommand{\checklist}[1]{\item[$\square$] #1}
\newcommand{\R}{\mathbb{R}}
\newcommand{\C}{\mathbb{C}}
\newcommand{\Z}{\mathbb{Z}}
\newcommand{\N}{\mathbb{N}}

% Title information
\title{\textbf{PRD 14: Topological Semimetals: Weyl and Dirac Points from Symmetry} \\
\large Pure Thought AI Challenge 14}
\author{Pure Thought AI Challenges Project}
\date{\today}

\begin{document}

\maketitle
\thispagestyle{empty}

\begin{abstract}
This document presents a comprehensive Product Requirement Document (PRD) for implementing a pure-thought computational challenge. The problem can be tackled using only symbolic mathematics, exact arithmetic, and fresh code---no experimental data or materials databases required until final verification. All results must be accompanied by machine-checkable certificates.
\end{abstract}

\clearpage
\tableofcontents
\clearpage


\textbf{Domain}: Materials Science

\textbf{Timeline}: 5-7 months

\textbf{Difficulty}: Medium-High

\textbf{Prerequisites}: Band theory, topology, group theory, differential geometry



\bigskip\hrule\bigskip


\subsection{1. Problem Statement}


\subsubsection{Scientific Context}

\textbf{Topological semimetals} are 3D materials where conduction and valence bands touch at isolated points (Weyl/Dirac) or lines (nodal lines) in the Brillouin zone, protected by topology and symmetry.


\textbf{Weyl Points}:

\begin{itemize}
\item \textbf{Band crossing}: Two non-degenerate bands touch at momentum k_W

\item \textbf{Topological charge}: ±1 chirality (monopole of Berry curvature)

\item \textbf{Fermi arc surface states}: Connect projections of opposite-chirality Weyl points

\item \textbf{Requires}: Either broken time-reversal (T) or inversion (I) symmetry


\end{itemize}

\textbf{Dirac Points}:

\begin{itemize}
\item \textbf{4-fold degeneracy}: Two Weyl points of opposite chirality pinned together by symmetry

\item \textbf{Requires}: Both T and I symmetry present

\item \textbf{Can split}: Into Weyl pairs when symmetry broken


\end{itemize}

\textbf{Nodal Lines}:

\begin{itemize}
\item \textbf{1D band crossing}: Bands touch along continuous curves in BZ

\item \textbf{Protected by}: Mirror/glide symmetries + additional symmetries

\item \textbf{Drumhead surface states}: Flat bands on surface


\end{itemize}

\textbf{Key Properties}:

\begin{itemize}
\item \textbf{Nielsen-Ninomiya theorem}: Weyl points come in pairs with opposite chirality (ΣC_i = 0)

\item \textbf{Bulk-boundary correspondence}: Fermi arcs connect Weyl points of opposite chirality

\item \textbf{Anomalies}: Chiral anomaly causes magnetoresistance, optical responses



\subsubsection{Core Question}

\end{itemize}

\textbf{Can we systematically construct tight-binding models with Weyl/Dirac points using ONLY symmetry constraints and topology—without materials databases or DFT?}


Specifically:

\begin{itemize}
\item Given space group G, find minimal models hosting Weyl points

\item Compute exact positions k\textit{W and chiralities C}W

\item Prove Fermi arc existence and topology

\item Optimize nodal line geometries (linking, knotting)

\item Certify robustness against disorder and perturbations

\item Export 3D visualizations of Fermi surfaces and arcs



\subsubsection{Why This Matters}

\end{itemize}

\textbf{Theoretical Impact}:

\begin{itemize}
\item Completes topological classification for gapless systems

\item Connects knot theory to condensed matter

\item Tests bulk-boundary correspondence in 3D


\end{itemize}

\textbf{Practical Benefits}:

\begin{itemize}
\item Novel transport phenomena (chiral anomaly, nonlocal resistance)

\item Topological quantum computing platforms

\item Optoelectronic devices with unusual responses


\end{itemize}

\textbf{Pure Thought Advantages}:

\begin{itemize}
\item Weyl points are topological defects (no material parameters needed)

\item Chirality computed from Berry curvature (exact)

\item Symmetry analysis determines allowed positions

\item Surface states from semi-infinite geometry (pure mathematics)



\bigskip\hrule\bigskip


\subsection{2. Mathematical Formulation}


\subsubsection{Problem Definition}

\end{itemize}

A \textbf{Weyl point} at momentum k_W is a point where:


\begin{itemize}
\item \textbf{Two bands touch}: E₁(k\textit{W) = E₂(k}W)

\item \textbf{Linear dispersion}: E(k) ≈ E\textit{F ± ħv}F |k - k\textit{W| near k}W

\item \textbf{Topological charge}:

\begin{lstlisting}
   C_W = (1/2π) ∮_{S} F(k) · dS
\end{lstlisting}
\end{itemize}

   where S is a small sphere around k_W, and F is Berry curvature


\textbf{Weyl Hamiltonian} (low-energy effective):

\begin{lstlisting}
H(q) = ħv_F (σ · q)
\end{lstlisting}
where q = k - k\textit{W, σ are Pauli matrices, chirality C}W = sign(det(v_F))


\textbf{Dirac Hamiltonian} (4×4):

\begin{lstlisting}
H(q) = ħv_F (Γ · q)
\end{lstlisting}
where Γ are 4×4 gamma matrices (two copies of Pauli matrices)


\textbf{Nodal Line}: 1D curve γ(t) in BZ where bands touch, characterized by:

\begin{itemize}
\item \textbf{π₁} linking invariant (how nodal lines link)

\item \textbf{Drumhead} flat surface state filling interior of projected loop



\subsubsection{Certificate Requirements}

\item \textbf{Weyl Point Certificate}: Exact (k\textit{W, C}W) pairs

\item \textbf{Fermi Arc Topology}: Connectivity of arcs on surface BZ

\item \textbf{Symmetry Protection}: Proof that Weyl/Dirac points cannot gap without breaking symmetry

\item \textbf{Nodal Line Geometry}: Linking numbers, knot invariants

\item \textbf{Surface State Verification}: Existence of arcs/drumheads



\subsubsection{Input/Output Specification}

\end{itemize}

\textbf{Input}:

\begin{lstlisting}
from sympy import *
import numpy as np
from typing import List, Tuple, Callable

class SemimetalModel:
    dimension: int  # Must be 3D
    space_group: int
    time_reversal: bool  # T symmetry
    inversion: bool  # I symmetry

    hamiltonian: Callable[[np.ndarray], np.ndarray]  # H(k), 3D momentum
    num_bands: int
\end{lstlisting}

\textbf{Output}:

\begin{lstlisting}
class SemimetalCertificate:
    model: SemimetalModel

    # Weyl points
    weyl_points: List[Tuple[np.ndarray, int]]  # [(k_W, chirality), ...]
    total_chirality: int  # Should be 0 (Nielsen-Ninomiya)

    # Dirac points
    dirac_points: List[np.ndarray]  # [k_D, ...]
    dirac_splitting: Optional[List[Tuple]]  # How Dirac → 2 Weyl when symmetry broken

    # Nodal lines
    nodal_lines: List[Callable]  # [γ₁(t), γ₂(t), ...] parameterized curves
    linking_numbers: np.ndarray  # L_ij = linking of γ_i, γ_j
    knot_invariants: List[str]  # Alexander polynomial, etc.

    # Surface states
    fermi_arcs: List[Tuple[np.ndarray, np.ndarray]]  # [(k_start, k_end), ...] on surface
    drumhead_states: Optional[np.ndarray]  # For nodal lines

    # Verification
    berry_curvature_field: Callable[[np.ndarray], np.ndarray]  # F(k)
    surface_spectral_function: Callable[[np.ndarray], float]  # A(k, E=0)

    proof_of_topology: str  # Mathematical derivation
\end{lstlisting}


\bigskip\hrule\bigskip


\subsection{3. Implementation Approach}


\subsubsection{Phase 1: Minimal Weyl Semimetal Model (Months 1-2)}

Implement simplest Weyl semimetal (broken inversion):


\begin{lstlisting}
import numpy as np
from scipy.linalg import eigh

def minimal_weyl_model(m: float, b: float) -> Callable:
    """
    Minimal 2-band model hosting a pair of Weyl points.

    H(k) = (b k_z + m) σ_x + b k_x σ_y + b k_y σ_z

    Weyl points at k_W = ±(0, 0, m/b) with opposite chirality.

    Breaks inversion symmetry (due to k_z term).
    """
    def H(k: np.ndarray) -> np.ndarray:
        kx, ky, kz = k[0], k[1], k[2]

        sx = np.array([[0, 1], [1, 0]])
        sy = np.array([[0, -1j], [1j, 0]])
        sz = np.array([[1, 0], [0, -1]])

        H_k = (b*kz + m)*sx + b*kx*sy + b*ky*sz

        return H_k

    return H

def find_band_crossings(H_func: Callable, k_range: Tuple[float, float, float],
                        N_k: int = 50) -> List[np.ndarray]:
    """
    Find points in BZ where bands touch (gap closes).

    Returns list of k-points where min|E_i - E_j| < threshold.
    """
    kx_vals = np.linspace(-k_range[0], k_range[0], N_k)
    ky_vals = np.linspace(-k_range[1], k_range[1], N_k)
    kz_vals = np.linspace(-k_range[2], k_range[2], N_k)

    crossing_points = []

    for kx in kx_vals:
        for ky in ky_vals:
            for kz in kz_vals:
                k = np.array([kx, ky, kz])
                evals = np.linalg.eigvalsh(H_func(k))

                # Check for near-degeneracy
                gaps = [abs(evals[i+1] - evals[i]) for i in range(len(evals)-1)]
                min_gap = min(gaps)

                if min_gap < 1e-3:  # Threshold for crossing
                    crossing_points.append(k)

    return crossing_points

def refine_weyl_point(H_func: Callable, k_initial: np.ndarray,
                     tol: float = 1e-10) -> np.ndarray:
    """
    Refine Weyl point position to machine precision.

    Minimize gap = |E₁(k) - E₂(k)| around initial guess.
    """
    from scipy.optimize import minimize

    def gap_function(k):
        evals = np.linalg.eigvalsh(H_func(k))
        return min([abs(evals[i+1] - evals[i]) for i in range(len(evals)-1)])

    result = minimize(gap_function, k_initial, method='Powell', tol=tol)

    return result.x
\end{lstlisting}

\textbf{Validation}: Reproduce textbook Weyl model, verify k_W positions.



\subsubsection{Phase 2: Chirality Computation (Months 2-3)}

Compute topological charge of each Weyl point:


\begin{lstlisting}
def berry_curvature_3d(H_func: Callable, k: np.ndarray,
                       band_idx: int, delta: float = 1e-5) -> np.ndarray:
    """
    Compute Berry curvature F = (F_x, F_y, F_z) at k.

    F_μ = ε_{μνλ} ∂_ν A_λ

    Returns 3-vector.
    """
    # Get wavefunction
    evals, evecs = eigh(H_func(k))
    sorted_idx = np.argsort(evals)
    u_k = evecs[:, sorted_idx[band_idx]]

    F = np.zeros(3)

    # F_x = ∂_y A_z - ∂_z A_y
    # Use finite differences

    dk_y = np.array([0, delta, 0])
    dk_z = np.array([0, 0, delta])

    # A_z at k and k+δy
    A_z_k = berry_connection_component(H_func, k, band_idx, direction=2, delta=delta)
    A_z_k_plus_y = berry_connection_component(H_func, k+dk_y, band_idx, direction=2, delta=delta)

    # A_y at k and k+δz
    A_y_k = berry_connection_component(H_func, k, band_idx, direction=1, delta=delta)
    A_y_k_plus_z = berry_connection_component(H_func, k+dk_z, band_idx, direction=1, delta=delta)

    F[0] = (A_z_k_plus_y - A_z_k)/delta - (A_y_k_plus_z - A_y_k)/delta

    # Similarly for F_y and F_z
    # ... (analogous calculations)

    return F

def compute_weyl_chirality(H_func: Callable, k_W: np.ndarray,
                          radius: float = 0.1, N_theta: int = 20, N_phi: int = 20) -> int:
    """
    Compute chirality C_W = ∮ F · dS around Weyl point.

    Integrate Berry curvature over sphere of radius r around k_W.
    """
    # Parameterize sphere: k = k_W + r(sin θ cos φ, sin θ sin φ, cos θ)
    theta_vals = np.linspace(0, np.pi, N_theta)
    phi_vals = np.linspace(0, 2*np.pi, N_phi)

    flux = 0

    for theta in theta_vals:
        for phi in phi_vals:
            # Point on sphere
            k = k_W + radius * np.array([
                np.sin(theta)*np.cos(phi),
                np.sin(theta)*np.sin(phi),
                np.cos(theta)
            ])

            # Berry curvature
            F = berry_curvature_3d(H_func, k, band_idx=0)  # Lower band

            # Outward normal
            n = np.array([np.sin(theta)*np.cos(phi),
                         np.sin(theta)*np.sin(phi),
                         np.cos(theta)])

            # F · dS (with Jacobian for spherical measure)
            dS = radius**2 * np.sin(theta) * (theta_vals[1]-theta_vals[0]) * (phi_vals[1]-phi_vals[0])
            flux += np.dot(F, n) * dS

    # Chirality is flux/(2π)
    chirality = int(np.round(flux / (2*np.pi)))

    return chirality
\end{lstlisting}


\subsubsection{Phase 3: Fermi Arc Calculation (Months 3-4)}

Compute surface states and Fermi arcs:


\begin{lstlisting}
def surface_green_function(H_bulk: Callable, k_parallel: np.ndarray,
                           energy: float = 0, surface_normal: str = 'z',
                           N_layers: int = 100) -> np.ndarray:
    """
    Compute surface Green's function G(k_∥, E) for semi-infinite geometry.

    Uses iterative method (transfer matrix or recursive Green's function).
    """
    # For simplicity, use slab geometry with large N_layers

    H_slab = construct_slab_hamiltonian(H_bulk, k_parallel, surface_normal, N_layers)

    # Green's function: G = (E - H + iη)^{-1}
    eta = 1e-3  # Small imaginary part
    dim = H_slab.shape[0]

    G = np.linalg.inv((energy + 1j*eta)*np.eye(dim) - H_slab)

    # Project onto surface layer
    num_orbitals = H_bulk(np.array([0, 0, 0])).shape[0]
    G_surface = G[:num_orbitals, :num_orbitals]

    return G_surface

def compute_surface_spectral_function(H_bulk: Callable, k_parallel: np.ndarray,
                                     energy: float = 0) -> float:
    """
    Surface spectral function A(k_∥, E) = -Im Tr G(k_∥, E).

    Peaks indicate surface states.
    """
    G_surf = surface_green_function(H_bulk, k_parallel, energy)

    A = -np.imag(np.trace(G_surf))

    return A

def map_fermi_arcs(H_bulk: Callable, weyl_points: List[Tuple],
                   N_k: int = 100) -> List[Tuple]:
    """
    Map Fermi arcs connecting Weyl point projections on surface BZ.

    Returns list of arcs as (k_start, k_end) pairs.
    """
    # Project Weyl points onto surface (kz=0 plane)
    weyl_projections = [(np.array([k[0], k[1]]), chi) for k, chi in weyl_points]

    # Compute spectral function on surface BZ at E=0
    kx_surf = np.linspace(-np.pi, np.pi, N_k)
    ky_surf = np.linspace(-np.pi, np.pi, N_k)

    spectral_map = np.zeros((N_k, N_k))

    for i, kx in enumerate(kx_surf):
        for j, ky in enumerate(ky_surf):
            k_par = np.array([kx, ky])
            spectral_map[i, j] = compute_surface_spectral_function(H_bulk, k_par, energy=0)

    # Identify arcs: high spectral weight curves connecting Weyl projections
    # Use image processing / contour finding
    from skimage.feature import peak_local_max

    # Find high-intensity ridges (arcs)
    arc_pixels = spectral_map > 0.5 * np.max(spectral_map)

    # Trace paths (simplified—full implementation needs sophisticated tracking)
    arcs = []
    # ... (arc tracing algorithm)

    return arcs
\end{lstlisting}


\subsubsection{Phase 4: Dirac Points and Splitting (Months 4-5)}

Study Dirac semimetals and symmetry breaking:


\begin{lstlisting}
def dirac_semimetal_model(v_F: float = 1.0) -> Callable:
    """
    Minimal Dirac semimetal with both T and I symmetry.

    4-band model: two Weyl points pinned together at k=0.

    H(k) = v_F (k_x Γ₁ + k_y Γ₂ + k_z Γ₃)

    where Γ_i are 4×4 Dirac matrices.
    """
    # Gamma matrices (one representation)
    Gamma1 = np.kron(np.array([[0, 1], [1, 0]]), np.eye(2))
    Gamma2 = np.kron(np.array([[0, -1j], [1j, 0]]), np.eye(2))
    Gamma3 = np.kron(np.array([[1, 0], [0, -1]]), np.array([[0, 1], [1, 0]]))

    def H(k: np.ndarray) -> np.ndarray:
        kx, ky, kz = k[0], k[1], k[2]
        return v_F * (kx*Gamma1 + ky*Gamma2 + kz*Gamma3)

    return H

def split_dirac_into_weyl(H_dirac: Callable, breaking_term: str,
                          strength: float) -> Callable:
    """
    Split Dirac point into two Weyl points by breaking symmetry.

    breaking_term:
    - 'inversion': Add term breaking I symmetry
    - 'time_reversal': Add term breaking T symmetry (magnetic field)
    """
    def H_split(k: np.ndarray) -> np.ndarray:
        H_0 = H_dirac(k)

        if breaking_term == 'inversion':
            # Add k_z² term or constant mass
            delta_H = strength * np.kron(np.array([[1, 0], [0, -1]]), np.eye(2))
        elif breaking_term == 'time_reversal':
            # Add magnetic field along z
            delta_H = strength * np.kron(np.eye(2), np.array([[1, 0], [0, -1]]))
        else:
            delta_H = np.zeros_like(H_0)

        return H_0 + delta_H

    return H_split

def trace_dirac_to_weyl_transition(H_dirac: Callable, strength_values: np.ndarray):
    """
    Track how Dirac point splits into Weyl pair as symmetry-breaking increased.
    """
    weyl_separation = []

    for s in strength_values:
        H_split = split_dirac_into_weyl(H_dirac, 'inversion', s)

        # Find Weyl points
        weyl_pts = find_band_crossings(H_split, k_range=(np.pi, np.pi, np.pi))

        if len(weyl_pts) >= 2:
            # Distance between Weyl pair
            dist = np.linalg.norm(weyl_pts[0] - weyl_pts[1])
            weyl_separation.append(dist)
        else:
            weyl_separation.append(0)

    return weyl_separation
\end{lstlisting}


\subsubsection{Phase 5: Nodal Lines (Months 5-6)}

Design semimetals with nodal line degeneracies:


\begin{lstlisting}
def nodal_line_model(mirror_plane: str = 'xy') -> Callable:
    """
    Model with nodal line protected by mirror symmetry.

    Bands touch along a circle in BZ (e.g., k_z = 0, k_x² + k_y² = r₀²).
    """
    def H(k: np.ndarray) -> np.ndarray:
        kx, ky, kz = k[0], k[1], k[2]

        sx = np.array([[0, 1], [1, 0]])
        sy = np.array([[0, -1j], [1j, 0]])
        sz = np.array([[1, 0], [0, -1]])

        # Nodal line at kz=0, kx²+ky²=1
        H_k = (kx**2 + ky**2 - 1)*sx + kz*sy

        return H_k

    return H

def extract_nodal_line(H_func: Callable, N_sample: int = 1000) -> Callable:
    """
    Find parameterization γ(t) of nodal line in BZ.

    Returns: curve γ: [0, 2π] → ℝ³ (momentum space)
    """
    # Sample BZ to find where gap closes
    gap_threshold = 1e-4
    nodal_points = []

    # ... (sample k-space, identify near-degeneracies)

    # Fit smooth curve through nodal points
    from scipy.interpolate import splprep, splev

    tck, u = splprep([nodal_points[:, 0], nodal_points[:, 1], nodal_points[:, 2]], s=0)

    def gamma(t):
        return np.array(splev(t, tck))

    return gamma

def compute_linking_number(gamma1: Callable, gamma2: Callable) -> int:
    """
    Compute Gauss linking number for two nodal lines.

    L = (1/4π) ∫∫ (γ₁'(s) × γ₂'(t)) · (γ₁(s) - γ₂(t)) / |γ₁(s) - γ₂(t)|³ ds dt
    """
    # Numerical integration
    N_s, N_t = 100, 100
    s_vals = np.linspace(0, 2*np.pi, N_s)
    t_vals = np.linspace(0, 2*np.pi, N_t)

    linking = 0

    for s in s_vals:
        for t in t_vals:
            g1_s = gamma1(s)
            g2_t = gamma2(t)

            # Derivatives
            ds = s_vals[1] - s_vals[0]
            dt = t_vals[1] - t_vals[0]

            g1_prime = (gamma1(s + ds) - gamma1(s)) / ds
            g2_prime = (gamma2(t + dt) - gamma2(t)) / dt

            diff = g1_s - g2_t
            dist_cubed = np.linalg.norm(diff)**3

            if dist_cubed > 1e-6:
                integrand = np.dot(np.cross(g1_prime, g2_prime), diff) / dist_cubed
                linking += integrand * ds * dt

    linking /= (4*np.pi)

    return int(np.round(linking))
\end{lstlisting}


\subsubsection{Phase 6: Certification and Database (Months 6-7)}

Generate complete certificates and database:


\begin{lstlisting}
def generate_semimetal_certificate(model: SemimetalModel) -> SemimetalCertificate:
    """
    Generate complete certificate for topological semimetal.
    """
    cert = SemimetalCertificate(model=model)

    # Find Weyl points
    crossings = find_band_crossings(model.hamiltonian, k_range=(np.pi, np.pi, np.pi))

    weyl_points = []
    for k_cross in crossings:
        k_refined = refine_weyl_point(model.hamiltonian, k_cross)
        chirality = compute_weyl_chirality(model.hamiltonian, k_refined)

        if chirality != 0:
            weyl_points.append((k_refined, chirality))

    cert.weyl_points = weyl_points
    cert.total_chirality = sum([chi for _, chi in weyl_points])

    # Verify Nielsen-Ninomiya
    assert cert.total_chirality == 0, "Chirality sum must be zero!"

    # Fermi arcs
    cert.fermi_arcs = map_fermi_arcs(model.hamiltonian, weyl_points)

    # Nodal lines (if present)
    # ... (detect and parameterize)

    # Berry curvature field
    cert.berry_curvature_field = lambda k: berry_curvature_3d(model.hamiltonian, k, 0)

    return cert

def export_semimetal_visualization(cert: SemimetalCertificate, output_dir: Path):
    """
    Export 3D visualizations of Weyl points, Fermi arcs, nodal lines.
    """
    import matplotlib.pyplot as plt
    from mpl_toolkits.mplot3d import Axes3D

    # Plot Weyl points in 3D BZ
    fig = plt.figure()
    ax = fig.add_subplot(111, projection='3d')

    for k_W, chi in cert.weyl_points:
        color = 'red' if chi > 0 else 'blue'
        ax.scatter(k_W[0], k_W[1], k_W[2], c=color, s=100, marker='o')

    ax.set_xlabel('kx')
    ax.set_ylabel('ky')
    ax.set_zlabel('kz')
    ax.set_title('Weyl Points (red=+1, blue=-1)')

    plt.savefig(output_dir / 'weyl_points_3d.png')

    # Plot Fermi arcs on surface
    # ... (2D plot of arcs connecting Weyl projections)

def generate_semimetal_database() -> dict:
    """
    Database of topological semimetals.
    """
    database = {'models': []}

    # Weyl semimetals
    for config in ['minimal', 'type-II', 'multi-weyl']:
        model = construct_weyl_model(config)
        cert = generate_semimetal_certificate(model)

        database['models'].append({
            'type': 'Weyl',
            'configuration': config,
            'num_weyl_points': len(cert.weyl_points),
            'weyl_positions': [k.tolist() for k, _ in cert.weyl_points],
            'certificate_path': export_certificate(cert)
        })

    # Dirac semimetals
    # ... (similar for Dirac, nodal line models)

    return database
\end{lstlisting}


\bigskip\hrule\bigskip


\subsection{4. Example Starting Prompt}

\begin{lstlisting}
You are a condensed matter theorist specializing in topological semimetals. Design tight-binding
models with Weyl/Dirac points using ONLY symmetry and topology—no DFT or materials databases.

OBJECTIVE: Construct minimal Weyl semimetal, compute chiralities ±1, verify Fermi arcs.

PHASE 1 (Months 1-2): Minimal Weyl model
- Implement 2-band H(k) = (bk_z + m)σ_x + bk_xσ_y + bk_yσ_z
- Find Weyl points at k_W = ±(0,0,m/b)
- Refine positions to machine precision

PHASE 2 (Months 2-3): Chirality calculation
- Compute Berry curvature F(k) on sphere around each Weyl point
- Integrate ∮F·dS to get chirality C_W = ±1
- Verify Nielsen-Ninomiya: ΣC_i = 0

PHASE 3 (Months 3-4): Fermi arcs
- Construct slab geometry (semi-infinite in z)
- Compute surface Green's function G(k_∥, E=0)
- Map spectral function A(k_∥) = -Im Tr G
- Identify arcs connecting Weyl projections

PHASE 4 (Months 4-5): Dirac semimetals
- Build 4-band Dirac model with T and I symmetry
- Split Dirac → 2 Weyl by breaking I
- Track Weyl separation vs perturbation strength

PHASE 5 (Months 5-6): Nodal lines
- Construct model with mirror-protected nodal line
- Parameterize nodal curve γ(t) in BZ
- Compute linking numbers for multi-component lines

PHASE 6 (Months 6-7): Database and visualization
- Generate certificates for 10 semimetal types
- Export 3D visualizations of BZ, Weyl points, arcs
- Classify by space group symmetry

SUCCESS CRITERIA:
- MVR: Minimal Weyl model with verified C_W = ±1
- Strong: Fermi arcs computed and visualized
- Publication: Complete database + linking number calculations

VERIFICATION:
- Chirality exact: C_W ∈ {-1, 0, +1} (integer)
- Nielsen-Ninomiya: Σ_i C_i = 0
- Fermi arcs connect opposite-chirality Weyl points
- Linking numbers computed for nodal lines

Pure topology + linear algebra. No DFT.
All results certificate-based with exact chirality computation.
\end{lstlisting}


\bigskip\hrule\bigskip


\subsection{5. Success Criteria}

\textbf{MVR} (2 months): Minimal Weyl model, chirality verified

\textbf{Strong} (4-5 months): Fermi arcs, Dirac splitting

\textbf{Publication} (6-7 months): Complete database, nodal line linking



\bigskip\hrule\bigskip


\subsection{6. Verification Protocol}

Automated checks: chirality sum, arc connectivity, symmetry verification.



\bigskip\hrule\bigskip


\subsection{7. Resources & Milestones}

\textbf{References}:

\begin{itemize}
\item Wan et al. (2011): "Topological Semimetal and Fermi-Arc Surface States"

\item Burkov & Balents (2011): "Weyl Semimetal in a Topological Insulator Multilayer"

\item Fang et al. (2016): "Topological Nodal Line Semimetals"


\end{itemize}

\textbf{Milestones}:

\begin{itemize}
\item Month 2: Weyl model validated

\item Month 4: Fermi arcs mapped

\item Month 6: Nodal lines classified



\bigskip\hrule\bigskip


\subsection{8. Extensions}

\item \textbf{Type-II Weyl}: Tilted cones

\item \textbf{Hopf Nodal Links}: Knotted nodal lines

\item \textbf{Non-Hermitian Semimetals}: Exceptional points



\bigskip\hrule\bigskip

\end{itemize}

\textbf{End of PRD 14}


\end{document}
