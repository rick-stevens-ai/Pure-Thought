\documentclass[11pt,a4paper]{article}

% Packages
\usepackage[utf8]{inputenc}
\usepackage[T1]{fontenc}
\usepackage{lmodern}
\usepackage[english]{babel}
\usepackage{amsmath,amssymb,amsthm}
\usepackage{mathtools}
\usepackage{physics}
\usepackage{graphicx}
\usepackage{xcolor}
\usepackage{listings}
\usepackage{hyperref}
\usepackage{geometry}
\usepackage{fancyhdr}
\usepackage{tocloft}
\usepackage{enumitem}
\usepackage{booktabs}
\usepackage{algorithm}
\usepackage{algpseudocode}

% Page geometry
\geometry{
    a4paper,
    left=25mm,
    right=25mm,
    top=30mm,
    bottom=30mm,
}

% Header and footer
\pagestyle{fancy}
\fancyhf{}
\fancyhead[L]{Pure Thought Challenge 27}
\fancyhead[R]{\thepage}
\renewcommand{\headrulewidth}{0.4pt}

% Hyperref setup
\hypersetup{
    colorlinks=true,
    linkcolor=blue,
    citecolor=blue,
    urlcolor=blue,
    pdfauthor={Pure Thought AI Challenges},
    pdftitle={PRD 27: N-Body Problem and Central Configurations},
}

% Code listing style
\definecolor{codegray}{rgb}{0.95,0.95,0.95}
\definecolor{codegreen}{rgb}{0,0.6,0}
\definecolor{codepurple}{rgb}{0.58,0,0.82}

\lstdefinestyle{pythonstyle}{
    language=Python,
    backgroundcolor=\color{codegray},
    commentstyle=\color{codegreen},
    keywordstyle=\color{blue},
    stringstyle=\color{codepurple},
    basicstyle=\ttfamily\small,
    breaklines=true,
    breakatwhitespace=true,
    captionpos=b,
    frame=single,
    numbers=left,
    numberstyle=\tiny\color{gray},
    tabsize=4,
    showstringspaces=false,
}

\lstset{style=pythonstyle}

% Theorem environments
\newtheorem{theorem}{Theorem}[section]
\newtheorem{lemma}[theorem]{Lemma}
\newtheorem{proposition}[theorem]{Proposition}
\newtheorem{corollary}[theorem]{Corollary}
\theoremstyle{definition}
\newtheorem{definition}[theorem]{Definition}
\newtheorem{example}[theorem]{Example}
\theoremstyle{remark}
\newtheorem{remark}[theorem]{Remark}

% Custom commands
\newcommand{\checklist}[1]{\item[$\square$] #1}
\newcommand{\R}{\mathbb{R}}
\newcommand{\C}{\mathbb{C}}
\newcommand{\Z}{\mathbb{Z}}
\newcommand{\N}{\mathbb{N}}

% Title information
\title{\textbf{PRD 27: N-Body Problem and Central Configurations} \\
\large Pure Thought AI Challenge 27}
\author{Pure Thought AI Challenges Project}
\date{\today}

\begin{document}

\maketitle
\thispagestyle{empty}

\begin{abstract}
This document presents a comprehensive Product Requirement Document (PRD) for implementing a pure-thought computational challenge. The problem can be tackled using only symbolic mathematics, exact arithmetic, and fresh code---no experimental data or materials databases required until final verification. All results must be accompanied by machine-checkable certificates.
\end{abstract}

\clearpage
\tableofcontents
\clearpage


\textbf{Domain}: Celestial Mechanics & Dynamical Systems

\textbf{Timeline}: 6-9 months

\textbf{Difficulty}: High

\textbf{Prerequisites}: Classical mechanics, algebraic geometry, computational topology, group theory



\bigskip\hrule\bigskip


\subsection{1. Problem Statement}


\subsubsection{Scientific Context}

The \textbf{N-body problem} is one of the oldest and most fundamental problems in mathematical physics: given N point masses interacting via Newtonian gravity, determine their future motion. While the 2-body problem has exact solutions (Kepler ellipses), N ≥ 3 is generally non-integrable and chaotic. However, there exist special initial configurations called \textbf{central configurations} where all bodies maintain the same geometric shape while the configuration rotates or expands/contracts homothetically.


\textbf{Central configurations} are solutions to the algebraic equations:


∇ᵢU = λmᵢrᵢ  for all i = 1,...,N


where:

\begin{itemize}
\item U = -Σᵢ₍ⱼ mᵢmⱼ/|rᵢ - rⱼ| is the gravitational potential

\item λ is a scalar Lagrange multiplier (related to angular velocity)

\item rᵢ ∈ ℝᵈ are the positions (typically d = 2 or 3)

\item mᵢ > 0 are the masses


\end{itemize}

These configurations generate \textbf{homothetic solutions}: r(t) = α(t)r₀ where the shape r₀ is preserved. Famous examples include:

\begin{itemize}
\item \textbf{Lagrange equilateral triangle} (3-body, L4/L5 points)

\item \textbf{Euler collinear solution} (3-body, L1/L2/L3 points)

\item \textbf{Figure-eight choreography} (3-body with equal masses discovered by Moore 1993, Chenciner-Montgomery 2000)



\subsubsection{Core Question}

\end{itemize}

\textbf{Can we enumerate all central configurations for N bodies with given masses, classify them by symmetry and stability, and certify them using computational algebraic geometry?}


Key challenges:

\begin{itemize}
\item \textbf{Finiteness conjecture}: For planar N-body with generic masses, are there finitely many central configurations? (Known for N ≤ 4, open for N ≥ 5)

\item \textbf{Enumeration}: Explicitly find all solutions to the nonlinear system ∇ᵢU = λmᵢrᵢ

\item \textbf{Stability classification}: Which central configs generate stable periodic orbits?

\item \textbf{Symmetry analysis}: How do symmetry groups (rotations, reflections) constrain solutions?

\item \textbf{Certification}: Generate algebraic certificates proving completeness of enumeration



\subsubsection{Why This Matters}

\item \textbf{Celestial mechanics}: Trojan asteroids at Jupiter's L4/L5 points, Earth-Moon-Sun Lagrange points (JWST at L2)

\item \textbf{Spacecraft mission design}: Halo orbits, Lissajous trajectories around Lagrange points

\item \textbf{Choreography solutions}: Beautiful periodic N-body orbits with remarkable symmetries

\item \textbf{Pure mathematics}: Connection to algebraic geometry (Groebner bases), Morse theory (topology of configuration space)

\item \textbf{Computational topology}: Certified enumeration via homotopy continuation and symbolic computation



\subsubsection{Pure Thought Advantages}

\end{itemize}

Central configurations are \textbf{ideal for pure thought investigation}:

\begin{itemize}
\item ✅ Based on \textbf{algebraic equations} (polynomial system in positions)

\item ✅ Solutions computable via \textbf{Groebner bases} and \textbf{homotopy continuation}

\item ✅ Finiteness provable using \textbf{Bezout's theorem} and \textbf{dimension theory}

\item ✅ Stability determined by \textbf{Hessian eigenvalues} (symbolic computation)

\item ✅ All results \textbf{certified via interval arithmetic} or \textbf{exact algebraic numbers}

\item ❌ NO numerical integration of trajectories until verification phase

\item ❌ NO empirical search or Monte Carlo sampling



\bigskip\hrule\bigskip


\subsection{2. Mathematical Formulation}


\subsubsection{N-Body Equations}

\end{itemize}

The Newton equations for N gravitating bodies:


mᵢr̈ᵢ = -∇ᵢU(r)


where U(r) = -Σᵢ₍ⱼ mᵢmⱼ/|rᵢ - rⱼ| is the gravitational potential.


\textbf{Homothetic solutions}: Seek r(t) = α(t)c where c = (c₁,...,c_N) is a fixed shape. Substituting:


mᵢ(α̈cᵢ + 2α̇ċᵢ + αc̈ᵢ) = -α⁻²∇ᵢU(c)


For c̈ᵢ = 0 (fixed shape in rotating frame) and α̈ = -ω²α (harmonic oscillator), we get:


∇ᵢU(c) = λmᵢcᵢ


where λ = ω² > 0. This is the \textbf{central configuration equation}.



\subsubsection{Central Configuration Equations}

\textbf{Definition}: Positions r = (r₁,...,r_N) ∈ (ℝᵈ)ᴺ form a central configuration if:


\begin{lstlisting}
∇ᵢU(r) = λmᵢrᵢ  for all i = 1,...,N
\end{lstlisting}

with λ ∈ ℝ and Σᵢmᵢrᵢ = 0 (center of mass at origin).


\textbf{Equivalent formulation} (Albouy-Chenciner):

\begin{lstlisting}
Σⱼ≠ᵢ mⱼ(rᵢ - rⱼ)/|rᵢ - rⱼ|³ = λrᵢ  for all i
\end{lstlisting}

This is a \textbf{polynomial system} after clearing denominators (multiply by Πᵢ₍ⱼ|rᵢ - rⱼ|³).


\textbf{Degrees of freedom}:

\begin{itemize}
\item (ℝᵈ)ᴺ positions → dN variables

\item Subtract d for center of mass → dN - d

\item Subtract 1 for scale invariance (r ~ αr gives same config) → dN - d - 1

\item Subtract SO(d) rotations → dN - d - 1 - d(d-1)/2


\end{itemize}

For planar (d=2), N=3: 2·3 - 2 - 1 - 1 = 2 DOF

For planar N=4: 2·4 - 2 - 1 - 1 = 4 DOF

For planar N=5: 2·5 - 2 - 1 - 1 = 6 DOF



\subsubsection{Key Examples}

\begin{itemize}
\item \textbf{Euler collinear (3-body)}: All bodies on a line r₁ < r₂ < r₃. 5 solutions for generic masses (3 with m₂ in middle, 2 with m₁ or m₃ in middle).


\item \textbf{Lagrange equilateral (3-body)}: r₁, r₂, r₃ form equilateral triangle. Unique up to rotations.


\item \textbf{Kite configuration (4-body)}: Diamond shape with symmetry axis. Family parametrized by mass ratios.


\item \textbf{Square (4-body, equal masses)}: r₁, r₂, r₃, r₄ at vertices of square. Unstable.


\item \textbf{Figure-eight (3-body, equal masses)}: Choreography where bodies chase each other along figure-eight curve. Period ~ 6.3 time units.



\subsubsection{Stability Analysis}

\end{itemize}

\textbf{Linearization}: Perturb r(t) = c + ε(t) and linearize equations of motion:


mᵢε̈ᵢ = -∇ᵢⱼU|_c εⱼ + ω²mᵢεᵢ


where ∇ᵢⱼU is the Hessian of U.


\textbf{Stability criterion} (Lyapunov): Central configuration is stable if all eigenvalues of the Hessian (in rotating frame) have non-negative real parts, excluding zero modes (translations, rotations).



\subsubsection{Certificates}

All results must come with \textbf{machine-checkable certificates}:


\begin{itemize}
\item \textbf{Existence certificate}: Interval arithmetic proof that solution lies in certified box

\item \textbf{Uniqueness certificate}: Homotopy continuation tracking all solution branches

\item \textbf{Finiteness certificate}: Groebner basis showing ideal is zero-dimensional

\item \textbf{Stability certificate}: Interval bounds on all Hessian eigenvalues


\end{itemize}

\textbf{Export format}: JSON with exact algebraic numbers (via SymPy):

\begin{lstlisting}
{
  "n_bodies": 4,
  "dimension": 2,
  "positions": [
    {"x": "1/2", "y": "sqrt(3)/2"},
    {"x": "-1/2", "y": "sqrt(3)/2"},
    ...
  ],
  "lambda": "3",
  "stability": "stable",
  "symmetry_group": "D3"
}
\end{lstlisting}


\bigskip\hrule\bigskip


\subsection{3. Implementation Approach}


\subsubsection{Phase 1 (Months 1-2): Lagrange Points (3-Body)}

\textbf{Goal}: Find and analyze L1-L5 Lagrange points in circular restricted 3-body problem.


\begin{lstlisting}
import numpy as np
from scipy.optimize import fsolve
import sympy as sp
from mpmath import mp
mp.dps = 100  # 100-digit precision

def lagrange_points_circular_restricted(m1: float, m2: float) -> dict:
    """
    Find all five Lagrange points for circular restricted 3-body problem.

    Setup: m1 and m2 orbit their barycenter, m3 → 0 (test particle).
    Rotating frame: m1 at (-μ, 0), m2 at (1-μ, 0), μ = m2/(m1+m2).

    L1, L2, L3: collinear (on x-axis)
    L4, L5: equilateral triangle vertices
    """
    mu = m2 / (m1 + m2)

    # Effective potential in rotating frame
    def U_eff(x, y):
        r1 = np.sqrt((x + mu)**2 + y**2)
        r2 = np.sqrt((x - 1 + mu)**2 + y**2)
        return 0.5 * (x**2 + y**2) + (1 - mu)/r1 + mu/r2

    # Collinear points: solve ∂U_eff/∂x = 0 with y = 0
    def collinear_equation(x):
        r1 = abs(x + mu)
        r2 = abs(x - 1 + mu)
        return x - (1 - mu) * (x + mu) / r1**3 - mu * (x - 1 + mu) / r2**3

    # L1: between m1 and m2
    x_L1 = fsolve(collinear_equation, -mu + 0.5*(1-mu), full_output=True)[0][0]

    # L2: beyond m2
    x_L2 = fsolve(collinear_equation, 1-mu + 0.1, full_output=True)[0][0]

    # L3: beyond m1
    x_L3 = fsolve(collinear_equation, -mu - 0.5, full_output=True)[0][0]

    # L4, L5: equilateral triangles (analytical)
    x_L4 = 0.5 - mu
    y_L4 = np.sqrt(3) / 2

    x_L5 = 0.5 - mu
    y_L5 = -np.sqrt(3) / 2

    return {
        'L1': np.array([x_L1, 0.0]),
        'L2': np.array([x_L2, 0.0]),
        'L3': np.array([x_L3, 0.0]),
        'L4': np.array([x_L4, y_L4]),
        'L5': np.array([x_L5, y_L5]),
        'mu': mu
    }


def lagrange_equilateral_3body_symbolic(m1, m2, m3) -> dict:
    """
    Symbolic computation of equilateral triangle central configuration.

    For arbitrary masses m1, m2, m3.
    """
    # Symbolic variables
    x1, y1, x2, y2, x3, y3 = sp.symbols('x1 y1 x2 y2 x3 y3', real=True)
    lam = sp.symbols('lambda', positive=True)

    # Positions
    r1 = sp.Matrix([x1, y1])
    r2 = sp.Matrix([x2, y2])
    r3 = sp.Matrix([x3, y3])

    # Gravitational potential gradient
    def grad_U_i(ri, rj, mi, mj):
        r_ij = ri - rj
        dist = sp.sqrt(r_ij.dot(r_ij))
        return mi * mj * r_ij / dist**3

    # Central configuration equations
    eqs = []

    # Body 1
    grad_U_1 = grad_U_i(r1, r2, m1, m2) + grad_U_i(r1, r3, m1, m3)
    eqs.extend(grad_U_1 - lam * m1 * r1)

    # Body 2
    grad_U_2 = grad_U_i(r2, r1, m2, m1) + grad_U_i(r2, r3, m2, m3)
    eqs.extend(grad_U_2 - lam * m2 * r2)

    # Body 3
    grad_U_3 = grad_U_i(r3, r1, m3, m1) + grad_U_i(r3, r2, m3, m2)
    eqs.extend(grad_U_3 - lam * m3 * r3)

    # Center of mass constraint
    eqs.append(m1*x1 + m2*x2 + m3*x3)
    eqs.append(m1*y1 + m2*y2 + m3*y3)

    # Equilateral constraint
    eqs.append((x2-x1)**2 + (y2-y1)**2 - 1)  # |r2-r1| = 1 (normalize)
    eqs.append((x3-x1)**2 + (y3-y1)**2 - 1)  # |r3-r1| = 1
    eqs.append((x3-x2)**2 + (y3-y2)**2 - 1)  # |r3-r2| = 1

    # Solve using Groebner basis
    solution = sp.solve(eqs, [x1, y1, x2, y2, x3, y3, lam])

    return solution


def verify_central_configuration(positions: np.ndarray,
                                 masses: np.ndarray,
                                 tolerance: float = 1e-10) -> dict:
    """
    Verify that given positions form a central configuration.

    Checks: ∇ᵢU = λmᵢrᵢ for all i with same λ.
    """
    N = len(masses)
    grad_U = np.zeros_like(positions)

    # Compute potential gradient for each body
    for i in range(N):
        for j in range(N):
            if i == j:
                continue
            r_ij = positions[i] - positions[j]
            dist = np.linalg.norm(r_ij)
            grad_U[i] += masses[i] * masses[j] * r_ij / dist**3

    # Extract λ from first body (non-zero position)
    i_nonzero = np.argmax(np.linalg.norm(positions, axis=1))
    lambda_computed = np.dot(grad_U[i_nonzero], positions[i_nonzero]) / (
        masses[i_nonzero] * np.dot(positions[i_nonzero], positions[i_nonzero])
    )

    # Check if ∇ᵢU = λmᵢrᵢ for all i
    residuals = []
    for i in range(N):
        expected = lambda_computed * masses[i] * positions[i]
        residual = np.linalg.norm(grad_U[i] - expected)
        residuals.append(residual)

    max_residual = max(residuals)
    is_central = max_residual < tolerance

    return {
        'is_central_configuration': is_central,
        'lambda': lambda_computed,
        'max_residual': max_residual,
        'residuals': residuals
    }
\end{lstlisting}

\textbf{Validation}: Verify Lagrange L4/L5 for Earth-Moon system (μ ≈ 0.012), check equilateral triangle property.



\subsubsection{Phase 2 (Months 2-4): 4-Body Central Configurations}

\textbf{Goal}: Find all planar central configurations for 4 bodies using numerical continuation.


\begin{lstlisting}
from scipy.optimize import fsolve, root
from sympy.solvers import solve
from sympy.polys.groebnertools import groebner

def find_4body_central_configurations(masses: np.ndarray,
                                     initial_guesses: list = None) -> list:
    """
    Find all planar central configurations for 4 bodies.

    Uses Albouy-Chenciner formulation + homotopy continuation.
    """
    N = 4
    m1, m2, m3, m4 = masses

    def central_config_equations(x):
        """
        System of equations for central configuration.

        x: [x1, y1, x2, y2, x3, y3, x4, y4, lambda]
        (9 variables for 4 bodies in 2D + λ)
        """
        # Extract positions and λ
        positions = x[:8].reshape(4, 2)
        lam = x[8]

        # Center of mass at origin
        com_x = sum(masses[i] * positions[i, 0] for i in range(N))
        com_y = sum(masses[i] * positions[i, 1] for i in range(N))

        # Compute gradient of potential
        equations = []

        for i in range(N):
            grad_U_i = np.zeros(2)
            for j in range(N):
                if i == j:
                    continue
                r_ij = positions[i] - positions[j]
                dist = np.linalg.norm(r_ij)
                if dist < 1e-10:
                    return np.ones(9) * 1e10  # Collision, invalid

                grad_U_i += masses[j] * r_ij / dist**3

            # Central config condition: grad_U_i = λ * m_i * r_i
            eqs_i = grad_U_i - lam * positions[i]
            equations.extend(eqs_i)

        # Add center of mass constraints
        equations.append(com_x)
        equations.append(com_y)

        # Normalization: fix scale (e.g., |r1| = 1)
        equations.append(np.linalg.norm(positions[0]) - 1.0)

        return np.array(equations)

    # Generate initial guesses if not provided
    if initial_guesses is None:
        initial_guesses = generate_4body_initial_guesses(masses)

    # Find solutions using multiple initial conditions
    solutions = []

    for guess in initial_guesses:
        try:
            sol = fsolve(central_config_equations, guess, full_output=True)
            x_sol = sol[0]
            info = sol[1]

            # Check if solution converged
            if info['fvec'].max() < 1e-8:
                # Verify it's a new solution (not duplicate)
                is_new = True
                for prev_sol in solutions:
                    if np.linalg.norm(x_sol - prev_sol) < 1e-6:
                        is_new = False
                        break

                if is_new:
                    solutions.append(x_sol)
        except:
            continue

    return solutions


def generate_4body_initial_guesses(masses: np.ndarray) -> list:
    """
    Generate initial guesses for 4-body central configurations.

    Known families:
    1. Collinear (all on a line)
    2. Isosceles trapezoid
    3. Kite (diamond with symmetry axis)
    4. Equilateral triangle + 1 body at centroid
    """
    guesses = []

    # Guess 1: Collinear
    positions = np.array([[-1.5, 0], [-0.5, 0], [0.5, 0], [1.5, 0]])
    guesses.append(np.concatenate([positions.flatten(), [1.0]]))

    # Guess 2: Square (equal masses)
    positions = np.array([[-1, -1], [1, -1], [1, 1], [-1, 1]]) / np.sqrt(2)
    guesses.append(np.concatenate([positions.flatten(), [1.0]]))

    # Guess 3: Isosceles trapezoid
    positions = np.array([[-1, -0.5], [1, -0.5], [0.5, 0.5], [-0.5, 0.5]])
    guesses.append(np.concatenate([positions.flatten(), [1.0]]))

    # Guess 4: Kite
    positions = np.array([[0, -1], [-0.5, 0], [0, 1], [0.5, 0]])
    guesses.append(np.concatenate([positions.flatten(), [1.0]]))

    # Guess 5: Equilateral triangle + centroid
    positions = np.array([[0, 0], [1, 0], [0.5, np.sqrt(3)/2], [0.5, np.sqrt(3)/6]])
    guesses.append(np.concatenate([positions.flatten(), [1.0]]))

    return guesses


def classify_4body_configuration_symmetry(positions: np.ndarray,
                                         tolerance: float = 1e-6) -> str:
    """
    Classify 4-body configuration by symmetry group.

    Returns: Symmetry type (C1, C2, D2, D4, etc.)
    """
    N = 4

    # Check for reflection symmetries
    has_reflection_x = check_reflection_symmetry(positions, axis='x', tol=tolerance)
    has_reflection_y = check_reflection_symmetry(positions, axis='y', tol=tolerance)
    has_rotation_90 = check_rotational_symmetry(positions, angle=90, tol=tolerance)

    if has_rotation_90:
        return 'D4'  # Square
    elif has_reflection_x and has_reflection_y:
        return 'D2'  # Rectangle or kite
    elif has_reflection_x or has_reflection_y:
        return 'C2'  # Single axis of symmetry
    else:
        return 'C1'  # No symmetry


def check_reflection_symmetry(positions: np.ndarray, axis: str, tol: float) -> bool:
    """Check if configuration has reflection symmetry about given axis."""
    if axis == 'x':
        reflected = positions.copy()
        reflected[:, 1] *= -1
    elif axis == 'y':
        reflected = positions.copy()
        reflected[:, 0] *= -1
    else:
        raise ValueError(f"Unknown axis {axis}")

    # Check if reflected positions match original (up to permutation)
    for perm in permutations(range(len(positions))):
        if np.allclose(positions, reflected[list(perm), :], atol=tol):
            return True

    return False
\end{lstlisting}

\textbf{Validation}: Reproduce known 4-body central configs from Albouy & Moeckel (2005), verify count matches literature.



\subsubsection{Phase 3 (Months 4-5): Stability Analysis}

\textbf{Goal}: Compute Hessian eigenvalues and classify stability of central configurations.


\begin{lstlisting}
def stability_analysis_central_config(positions: np.ndarray,
                                     masses: np.ndarray,
                                     lambda_cc: float) -> dict:
    """
    Analyze linear stability of central configuration.

    Computes eigenvalues of Hessian in rotating frame.
    """
    N = len(masses)
    dim = positions.shape[1]  # 2 for planar, 3 for spatial

    # Compute Hessian of gravitational potential
    H = np.zeros((N * dim, N * dim))

    for i in range(N):
        for j in range(N):
            if i == j:
                # Diagonal block: H_ii = -Σ_{k≠i} m_k * (I/r_ik³ - 3*r_ik⊗r_ik/r_ik⁵)
                for k in range(N):
                    if k == i:
                        continue
                    r_ik = positions[i] - positions[k]
                    dist = np.linalg.norm(r_ik)

                    I_block = np.eye(dim)
                    outer_block = np.outer(r_ik, r_ik) / dist**2

                    H_ik_block = masses[k] * (I_block / dist**3 - 3 * outer_block / dist**3)

                    # Add to diagonal block
                    H[i*dim:(i+1)*dim, i*dim:(i+1)*dim] -= H_ik_block

            else:
                # Off-diagonal block: H_ij = m_j * (I/r_ij³ - 3*r_ij⊗r_ij/r_ij⁵)
                r_ij = positions[i] - positions[j]
                dist = np.linalg.norm(r_ij)

                I_block = np.eye(dim)
                outer_block = np.outer(r_ij, r_ij) / dist**2

                H_ij_block = masses[j] * (I_block / dist**3 - 3 * outer_block / dist**3)

                H[i*dim:(i+1)*dim, j*dim:(j+1)*dim] = H_ij_block

    # Add centrifugal term from rotating frame: + λ * I
    H += lambda_cc * np.eye(N * dim)

    # Compute eigenvalues
    eigvals = np.linalg.eigvalsh(H)

    # Filter out zero modes (translations, rotations)
    # Expect dim + dim(dim-1)/2 zero eigenvalues
    n_zero_modes = dim + dim * (dim - 1) // 2

    eigvals_nonzero = eigvals[n_zero_modes:]

    # Stability: all non-zero eigenvalues should be non-negative
    is_stable = all(eigvals_nonzero >= -1e-8)

    # Count unstable modes (negative eigenvalues)
    n_unstable = sum(1 for ev in eigvals_nonzero if ev < -1e-8)

    return {
        'eigenvalues_all': eigvals,
        'eigenvalues_nonzero': eigvals_nonzero,
        'is_stable': is_stable,
        'n_unstable_modes': n_unstable,
        'stability_type': 'stable' if is_stable else f'unstable ({n_unstable} modes)'
    }


def morse_index_central_config(positions: np.ndarray,
                               masses: np.ndarray) -> int:
    """
    Compute Morse index of central configuration.

    Morse index = number of negative eigenvalues of Hessian of U
    (without centrifugal term).
    """
    N = len(masses)
    dim = positions.shape[1]

    # Compute Hessian of potential U (no λ term)
    H_U = compute_potential_hessian(positions, masses)

    # Eigenvalues
    eigvals = np.linalg.eigvalsh(H_U)

    # Morse index: count negative eigenvalues (exclude zero modes)
    n_zero_modes = dim + dim * (dim - 1) // 2
    eigvals_nonzero = eigvals[n_zero_modes:]

    morse_index = sum(1 for ev in eigvals_nonzero if ev < -1e-8)

    return morse_index
\end{lstlisting}

\textbf{Validation}: Verify Lagrange L4/L5 are stable (all eigenvalues ≥ 0), L1/L2/L3 are unstable (at least one negative eigenvalue).



\subsubsection{Phase 4 (Months 5-7): 5-Body Enumeration via Groebner Bases}

\textbf{Goal}: Enumerate all planar 5-body central configurations using computational algebraic geometry.


\begin{lstlisting}
import sympy as sp
from sympy.polys.groebnertools import groebner

def enumerate_5body_central_configs_symbolic(masses: list) -> list:
    """
    Enumerate all planar 5-body central configurations using Groebner bases.

    CAUTION: Computationally expensive for N=5.
    """
    N = 5
    m1, m2, m3, m4, m5 = masses

    # Symbolic variables
    coords = []
    for i in range(N):
        coords.extend([sp.Symbol(f'x{i}'), sp.Symbol(f'y{i}')])

    lam = sp.Symbol('lambda')

    # Build polynomial equations
    equations = []

    for i in range(N):
        x_i, y_i = coords[2*i], coords[2*i+1]
        r_i = sp.Matrix([x_i, y_i])

        # Compute gradient of U for body i
        grad_U_i = sp.Matrix([0, 0])

        for j in range(N):
            if i == j:
                continue
            x_j, y_j = coords[2*j], coords[2*j+1]
            r_j = sp.Matrix([x_j, y_j])

            r_ij = r_i - r_j
            dist_sq = r_ij.dot(r_ij)

            # grad_U_i += masses[j] * r_ij / dist^3
            # Clear denominator: multiply by dist^3 = (dist_sq)^(3/2)
            # Use dist_sq instead to keep polynomial
            grad_U_i += masses[j] * r_ij * sp.sqrt(dist_sq)**(-3)

        # Central config condition: grad_U_i = λ * masses[i] * r_i
        eqs_i = grad_U_i - lam * masses[i] * r_i
        equations.extend(eqs_i)

    # Center of mass constraints
    equations.append(sum(masses[i] * coords[2*i] for i in range(N)))
    equations.append(sum(masses[i] * coords[2*i+1] for i in range(N)))

    # Normalization constraint (fix scale)
    equations.append(coords[0]**2 + coords[1]**2 - 1)

    # Compute Groebner basis (WARNING: very slow for N=5)
    print("Computing Groebner basis (this may take hours)...")
    gb = groebner(equations, coords + [lam], order='lex')

    # Extract solutions from univariate polynomial in gb
    solutions = sp.solve(gb, coords + [lam])

    return solutions


def finiteness_certificate_central_configs(masses: np.ndarray, N: int) -> dict:
    """
    Certify finiteness of central configurations for N bodies with given masses.

    Uses Bezout's theorem and dimension analysis.
    """
    # Dimension of configuration space (after removing symmetries)
    dim_config_space = 2 * N - 4  # Planar, remove translations (2), rotations (1), scale (1)

    # Number of equations in central config system
    n_equations = 2 * N + 2  # 2N for central config, 2 for center of mass

    # Bezout bound (very loose upper bound on number of solutions)
    # For generic masses, expect finite number of solutions
    # if system is square (n_equations = dim_config_space)

    is_square = (n_equations == dim_config_space)

    return {
        'dimension': dim_config_space,
        'n_equations': n_equations,
        'is_square_system': is_square,
        'expected_finite': is_square,
        'bezout_bound_estimate': 'exponential in N (not computed)'
    }
\end{lstlisting}

\textbf{Validation}: Compare to literature counts for specific mass ratios (Hampton & Moeckel 2006).



\subsubsection{Phase 5 (Months 7-8): Symmetry Classification}

\textbf{Goal}: Classify all central configurations by their symmetry groups.


\begin{lstlisting}
from itertools import permutations
import networkx as nx

def classify_symmetry_group(positions: np.ndarray,
                           masses: np.ndarray,
                           tolerance: float = 1e-6) -> dict:
    """
    Determine the symmetry group of a central configuration.

    Returns: Group type (Cn, Dn, etc.) and generators.
    """
    N = len(masses)

    # Find all symmetries (isometries preserving configuration)
    symmetries = []

    # Check rotations
    for k in range(1, N):
        angle = 2 * np.pi * k / N
        rot_matrix = np.array([[np.cos(angle), -np.sin(angle)],
                               [np.sin(angle), np.cos(angle)]])

        is_symmetry = check_isometry_symmetry(positions, masses, rot_matrix, tolerance)
        if is_symmetry:
            symmetries.append(('rotation', angle))

    # Check reflections
    for angle in np.linspace(0, np.pi, 20):
        # Reflection about line through origin with angle θ
        refl_matrix = np.array([[np.cos(2*angle), np.sin(2*angle)],
                                [np.sin(2*angle), -np.cos(2*angle)]])

        is_symmetry = check_isometry_symmetry(positions, masses, refl_matrix, tolerance)
        if is_symmetry:
            symmetries.append(('reflection', angle))

    # Determine group type from symmetries
    n_rotations = sum(1 for s in symmetries if s[0] == 'rotation') + 1  # Include identity
    n_reflections = sum(1 for s in symmetries if s[0] == 'reflection')

    if n_reflections > 0 and n_rotations > 1:
        group_type = f'D{n_rotations}'  # Dihedral group
    elif n_rotations > 1:
        group_type = f'C{n_rotations}'  # Cyclic group
    elif n_reflections > 0:
        group_type = 'C_s'  # Single reflection
    else:
        group_type = 'C_1'  # No symmetry

    return {
        'symmetry_group': group_type,
        'n_rotational_symmetries': n_rotations,
        'n_reflection_symmetries': n_reflections,
        'all_symmetries': symmetries
    }


def check_isometry_symmetry(positions: np.ndarray,
                            masses: np.ndarray,
                            transformation: np.ndarray,
                            tolerance: float) -> bool:
    """
    Check if isometry (rotation/reflection) is a symmetry.

    A symmetry must preserve both positions AND masses.
    """
    N = len(masses)

    # Apply transformation
    transformed_positions = (transformation @ positions.T).T

    # Check if transformed positions match original under some permutation
    for perm in permutations(range(N)):
        # Check positions
        pos_match = np.allclose(positions, transformed_positions[list(perm), :],
                               atol=tolerance)

        # Check masses
        mass_match = np.allclose(masses, masses[list(perm)], atol=tolerance)

        if pos_match and mass_match:
            return True

    return False
\end{lstlisting}

\textbf{Validation}: Verify Lagrange equilateral triangle has D3 symmetry, square configuration has D4.



\subsubsection{Phase 6 (Months 8-9): Certificate Generation and Database}

\textbf{Goal}: Generate complete database of certified central configurations.


\begin{lstlisting}
import json
from dataclasses import dataclass, asdict
from typing import List

@dataclass
class CentralConfigurationCertificate:
    """Complete certificate for a central configuration."""

    # System parameters
    n_bodies: int
    dimension: int
    masses: List[float]

    # Configuration data
    positions: List[List[float]]  # N x d array
    lambda_value: float

    # Verification
    is_verified: bool
    max_residual: float
    verification_method: str  # 'numerical', 'symbolic', 'interval_arithmetic'

    # Stability
    is_stable: bool
    eigenvalues: List[float]
    morse_index: int

    # Symmetry
    symmetry_group: str
    n_rotational_symmetries: int
    n_reflection_symmetries: int

    # Classification
    configuration_type: str  # 'collinear', 'planar', 'spatial', 'equilateral', etc.

    # Metadata
    computation_date: str
    precision_digits: int

    def export_json(self, filename: str):
        """Export certificate to JSON."""
        with open(filename, 'w') as f:
            json.dump(asdict(self), f, indent=2)

    def verify(self) -> bool:
        """Self-check certificate validity."""
        checks = [
            self.n_bodies > 0,
            len(self.masses) == self.n_bodies,
            len(self.positions) == self.n_bodies,
            self.max_residual < 1e-6,
            len(self.eigenvalues) > 0
        ]
        return all(checks)


def generate_central_config_database(N_max: int = 5) -> list:
    """
    Generate database of all known central configurations for N ≤ N_max.
    """
    database = []

    # N=3 configurations
    for mass_ratio in [0.5, 1.0, 2.0, 5.0, 10.0]:
        masses = np.array([1.0, mass_ratio, mass_ratio])

        # Equilateral
        config_eq = compute_lagrange_equilateral(masses)
        cert_eq = create_certificate(config_eq, masses)
        database.append(cert_eq)

        # Collinear (multiple solutions)
        configs_col = compute_euler_collinear(masses)
        for config in configs_col:
            cert = create_certificate(config, masses)
            database.append(cert)

    # N=4 configurations
    if N_max >= 4:
        for mass_config in generate_4body_mass_configurations():
            configs = find_4body_central_configurations(mass_config)
            for config_array in configs:
                positions = config_array[:8].reshape(4, 2)
                cert = create_certificate(positions, mass_config)
                database.append(cert)

    return database
\end{lstlisting}

\textbf{Validation}: Export database to JSON, verify all certificates pass self-check.



\bigskip\hrule\bigskip


\subsection{4. Example Starting Prompt}

\textbf{Prompt for AI System}:


You are tasked with enumerating and classifying central configurations for the N-body problem. Your goal is to:


\begin{itemize}
\item \textbf{Lagrange Points (Months 1-2)}:

\item Implement circular restricted 3-body problem solver

\item Find all five Lagrange points (L1-L5) for Earth-Moon system

\item Verify L4/L5 form equilateral triangles

\item Analyze stability: compute Hessian eigenvalues in rotating frame


\item \textbf{4-Body Configurations (Months 2-4)}:

\item Implement Albouy-Chenciner equations: ∇ᵢU = λmᵢrᵢ

\item Generate initial guesses: collinear, square, kite, trapezoid

\item Use numerical continuation to find all solutions for given masses

\item Verify each solution: check residuals < 10⁻⁸


\item \textbf{Stability Analysis (Months 4-5)}:

\item Compute Hessian of gravitational potential + centrifugal term

\item Extract eigenvalues, filter zero modes (translations, rotations)

\item Classify: stable (all eigenvalues ≥ 0) vs unstable

\item Compute Morse index for each configuration


\item \textbf{Symmetry Classification (Months 5-7)}:

\item Check rotational symmetries: C2, C3, C4, ...

\item Check reflection symmetries: σᵥ, σₕ, σ_d

\item Determine symmetry group: C1, Cn, Dn

\item Verify: Lagrange equilateral has D3, square has D4


\item \textbf{5-Body Enumeration (Months 7-8)}:

\item Formulate central config equations as polynomial system

\item Use Groebner basis to reduce system (WARNING: computationally expensive)

\item Apply homotopy continuation for numerical solutions

\item Certify finiteness: verify system is zero-dimensional


\item \textbf{Certificate Generation (Months 8-9)}:

\item Create CentralConfigurationCertificate for each solution

\item Include: positions (exact algebraic numbers), λ, eigenvalues, symmetry group

\item Export database to JSON

\item Verify all certificates satisfy central config equations


\end{itemize}

\textbf{Success Criteria}:

\begin{itemize}
\item Minimum Viable Result (2-4 months): L1-L5 for Earth-Moon, basic 4-body configs

\item Strong Result (6-8 months): Complete 4-body classification, stability analysis

\item Publication-Quality Result (9 months): 5-body enumeration, certified database


\end{itemize}

\textbf{Key Constraints}:

\begin{itemize}
\item Use symbolic computation (SymPy) for exact solutions where possible

\item Numerical solutions must have residuals < 10⁻⁸

\item All eigenvalues computed with interval arithmetic bounds

\item Database must include at least 50 certified configurations


\end{itemize}

\textbf{References}:

\begin{itemize}
\item Albouy & Chenciner (2001): "Le problème des n corps et les distances mutuelles"

\item Hampton & Moeckel (2006): "Finiteness of relative equilibria in the planar four-body problem"

\item Saari (2005): "Collisions, Rings, and Other Newtonian N-Body Problems"


\end{itemize}

Begin by implementing the circular restricted 3-body solver and finding L1-L5 for μ = 0.012 (Earth-Moon).



\bigskip\hrule\bigskip


\subsection{5. Success Criteria}


\subsubsection{Minimum Viable Result (Months 1-4)}

\textbf{Core Achievements}:

\begin{itemize}
\item ✅ L1-L5 Lagrange points computed for Earth-Moon system

\item ✅ Verification: L4/L5 are equilateral triangles

\item ✅ Basic 4-body central configurations found (collinear, square, kite)

\item ✅ Stability analysis: eigenvalue computation implemented


\end{itemize}

\textbf{Validation}:

\begin{itemize}
\item Match L1 position to within 0.1% of NASA data

\item Reproduce Hampton-Moeckel count for equal masses (N=4)


\end{itemize}

\textbf{Deliverables}:

\begin{itemize}
\item Python module \texttt{central_configs.py} with solvers

\item Jupyter notebook demonstrating Earth-Moon L4/L5

\item JSON database with 10+ certified configurations



\subsubsection{Strong Result (Months 4-8)}

\end{itemize}

\textbf{Extended Capabilities}:

\begin{itemize}
\item ✅ Complete 4-body enumeration for 5+ mass ratios

\item ✅ Symmetry classification: C1, Cn, Dn groups

\item ✅ Morse index computation for all configurations

\item ✅ Stability boundaries: identify bifurcations as masses vary

\item ✅ Comparison to 3+ literature sources


\end{itemize}

\textbf{Publications Benchmark}:

\begin{itemize}
\item Reproduce Figures from Albouy & Moeckel (2005)

\item Match eigenvalue spectra to within 1%


\end{itemize}

\textbf{Deliverables}:

\begin{itemize}
\item Database with 50+ configurations

\item Stability diagrams (mass ratio vs eigenvalues)

\item Symmetry classification report



\subsubsection{Publication-Quality Result (Months 8-9)}

\end{itemize}

\textbf{Novel Contributions}:

\begin{itemize}
\item ✅ 5-body central config enumeration for specific masses

\item ✅ Finiteness certificate via Groebner basis dimension

\item ✅ Formal verification: translate proofs to Lean/Coq

\item ✅ Interactive visualization tool (3D plots, animations)

\item ✅ Public database: 100+ configurations with certificates


\end{itemize}

\textbf{Beyond Literature}:

\begin{itemize}
\item Discover new 5-body configurations not in literature

\item Improve computational methods (faster Groebner basis)

\item Extend to spatial (3D) configurations


\end{itemize}

\textbf{Deliverables}:

\begin{itemize}
\item Arxiv preprint: "Complete Classification of Planar N-Body Central Configurations"

\item GitHub repository with database and solvers

\item Web interface: input masses → visualize all central configs



\bigskip\hrule\bigskip


\subsection{6. Verification Protocol}

\begin{lstlisting}
def verify_central_config_database(database: list) -> dict:
    """
    Automated verification of entire database.
    """
    results = {
        'n_total': len(database),
        'n_verified': 0,
        'n_failed': 0,
        'failed_certificates': []
    }

    for cert in database:
        # Check 1: Self-verification
        if not cert.verify():
            results['n_failed'] += 1
            results['failed_certificates'].append(cert)
            continue

        # Check 2: Central config equations satisfied
        positions = np.array(cert.positions)
        masses = np.array(cert.masses)

        verification = verify_central_configuration(positions, masses)

        if not verification['is_central_configuration']:
            results['n_failed'] += 1
            results['failed_certificates'].append(cert)
            continue

        # Check 3: Eigenvalue count matches dimension
        n_expected_eigenvalues = cert.n_bodies * cert.dimension
        if len(cert.eigenvalues) != n_expected_eigenvalues:
            results['n_failed'] += 1
            results['failed_certificates'].append(cert)
            continue

        results['n_verified'] += 1

    return results
\end{lstlisting}


\bigskip\hrule\bigskip


\subsection{7. Resources and Milestones}


\subsubsection{Essential References}

\item \textbf{Classical Papers}:

\item Euler (1767): Collinear solutions

\item Lagrange (1772): Equilateral triangle

\item Moulton (1910): Families of periodic orbits


\item \textbf{Modern Theory}:

\item Albouy & Chenciner (2001): "Le problème des n corps"

\item Hampton & Moeckel (2006): "Finiteness of relative equilibria"

\item Moeckel & Simó (1995): "Bifurcation of spatial central configurations"


\item \textbf{Textbooks}:

\item Saari (2005): \textit{Collisions, Rings, and Other Newtonian N-Body Problems}

\item Meyer, Hall, Offin (2009): \textit{Introduction to Hamiltonian Dynamical Systems}



\subsubsection{Milestone Checklist}

\item [ ] \textbf{Month 1}: L1-L5 computed for Earth-Moon

\item [ ] \textbf{Month 2}: 3-body collinear solutions verified

\item [ ] \textbf{Month 3}: 4-body solver implemented

\item [ ] \textbf{Month 4}: First 10 configurations certified

\item [ ] \textbf{Month 5}: Stability analysis complete

\item [ ] \textbf{Month 6}: Symmetry classification implemented

\item [ ] \textbf{Month 7}: 50+ configurations in database

\item [ ] \textbf{Month 8}: 5-body enumeration begun

\item [ ] \textbf{Month 9}: Final database exported, all certificates verified



\bigskip\hrule\bigskip

\end{itemize}

\textbf{End of PRD 27}


\end{document}
