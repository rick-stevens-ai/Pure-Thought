\documentclass[11pt,a4paper]{article}

% Packages
\usepackage[utf8]{inputenc}
\usepackage[T1]{fontenc}
\usepackage{lmodern}
\usepackage[english]{babel}
\usepackage{amsmath,amssymb,amsthm}
\usepackage{mathtools}
\usepackage{physics}
\usepackage{graphicx}
\usepackage{xcolor}
\usepackage{listings}
\usepackage{hyperref}
\usepackage{geometry}
\usepackage{fancyhdr}
\usepackage{tocloft}
\usepackage{enumitem}
\usepackage{booktabs}
\usepackage{algorithm}
\usepackage{algpseudocode}

% Page geometry
\geometry{
    a4paper,
    left=25mm,
    right=25mm,
    top=30mm,
    bottom=30mm,
}

% Header and footer
\pagestyle{fancy}
\fancyhf{}
\fancyhead[L]{Pure Thought Challenge 26}
\fancyhead[R]{\thepage}
\renewcommand{\headrulewidth}{0.4pt}

% Hyperref setup
\hypersetup{
    colorlinks=true,
    linkcolor=blue,
    citecolor=blue,
    urlcolor=blue,
    pdfauthor={Pure Thought AI Challenges},
    pdftitle={PRD 26: KAM Theory and Planetary Stability},
}

% Code listing style
\definecolor{codegray}{rgb}{0.95,0.95,0.95}
\definecolor{codegreen}{rgb}{0,0.6,0}
\definecolor{codepurple}{rgb}{0.58,0,0.82}

\lstdefinestyle{pythonstyle}{
    language=Python,
    backgroundcolor=\color{codegray},
    commentstyle=\color{codegreen},
    keywordstyle=\color{blue},
    stringstyle=\color{codepurple},
    basicstyle=\ttfamily\small,
    breaklines=true,
    breakatwhitespace=true,
    captionpos=b,
    frame=single,
    numbers=left,
    numberstyle=\tiny\color{gray},
    tabsize=4,
    showstringspaces=false,
}

\lstset{style=pythonstyle}

% Theorem environments
\newtheorem{theorem}{Theorem}[section]
\newtheorem{lemma}[theorem]{Lemma}
\newtheorem{proposition}[theorem]{Proposition}
\newtheorem{corollary}[theorem]{Corollary}
\theoremstyle{definition}
\newtheorem{definition}[theorem]{Definition}
\newtheorem{example}[theorem]{Example}
\theoremstyle{remark}
\newtheorem{remark}[theorem]{Remark}

% Custom commands
\newcommand{\checklist}[1]{\item[$\square$] #1}
\newcommand{\R}{\mathbb{R}}
\newcommand{\C}{\mathbb{C}}
\newcommand{\Z}{\mathbb{Z}}
\newcommand{\N}{\mathbb{N}}

% Title information
\title{\textbf{PRD 26: KAM Theory and Planetary Stability} \\
\large Pure Thought AI Challenge 26}
\author{Pure Thought AI Challenges Project}
\date{\today}

\begin{document}

\maketitle
\thispagestyle{empty}

\begin{abstract}
This document presents a comprehensive Product Requirement Document (PRD) for implementing a pure-thought computational challenge. The problem can be tackled using only symbolic mathematics, exact arithmetic, and fresh code---no experimental data or materials databases required until final verification. All results must be accompanied by machine-checkable certificates.
\end{abstract}

\clearpage
\tableofcontents
\clearpage


\textbf{Domain}: Celestial Mechanics & Dynamical Systems

\textbf{Timeline}: 6-9 months

\textbf{Difficulty}: High

\textbf{Prerequisites}: Hamiltonian mechanics, perturbation theory, measure theory, symplectic geometry



\bigskip\hrule\bigskip


\subsection{1. Problem Statement}


\subsubsection{Scientific Context}

\textbf{KAM (Kolmogorov-Arnold-Moser) theory} is one of the deepest results in dynamical systems, providing a rigorous mathematical explanation for the long-term stability of planetary orbits. The classical problem dates to Newton: given N gravitating bodies with small perturbations (planet-planet interactions), do orbits remain stable forever, or do planets eventually escape or collide?


The breakthrough came in three stages:

\begin{itemize}
\item \textbf{Kolmogorov (1954)}: Announced that "most" invariant tori of integrable systems survive small perturbations

\item \textbf{Arnold (1963)}: Proved the theorem for analytic Hamiltonians

\item \textbf{Moser (1962)}: Extended to smooth (C^k) systems with weaker differentiability


\end{itemize}

\textbf{KAM Theorem (simplified)}: Consider a nearly-integrable Hamiltonian H = H₀(I) + εH₁(I,θ) where H₀ is integrable and ε is small. If:

\begin{itemize}
\item Frequencies ω(I) = ∂H₀/∂I satisfy \textbf{Diophantine conditions} (non-resonant)

\item H₁ is sufficiently smooth


\end{itemize}

Then for ε < ε₀, there exists a \textbf{Cantor set} of invariant tori (measure → full as ε → 0) on which motion is \textbf{quasi-periodic} with frequencies ω(I).



\subsubsection{Core Question}

\textbf{Can we numerically verify KAM conditions for realistic planetary systems and certify their long-term stability?}


Key challenges:

\begin{itemize}
\item \textbf{Action-angle transformation}: Convert Keplerian orbits to (I,θ) coordinates

\item \textbf{Diophantine verification}: Check |k·ω| ≥ α/|k|^τ for infinitely many k

\item \textbf{KAM iteration}: Iteratively eliminate resonant terms via canonical transformations

\item \textbf{Measure estimates}: Compute fraction of phase space with surviving tori

\item \textbf{Solar system application}: Analyze real planetary data (Jupiter, Saturn, etc.)



\subsubsection{Why This Matters}

\item \textbf{Planetary stability}: Rigorous proof that solar system is stable over Gyr timescales

\item \textbf{Accelerator physics}: Stability of particle beams in synchrotrons

\item \textbf{Plasma confinement}: Magnetic field line structure in tokamaks

\item \textbf{General dynamical systems}: Paradigm for persistence of structure under perturbations

\item \textbf{Chaos theory}: Boundary between regular (KAM tori) and chaotic (Arnold diffusion) motion



\subsubsection{Pure Thought Advantages}

\end{itemize}

KAM theory is \textbf{ideal for pure thought investigation}:

\begin{itemize}
\item ✅ Based on \textbf{symbolic perturbation theory} (action-angle variables)

\item ✅ Diophantine conditions \textbf{verifiable algorithmically} (continued fractions)

\item ✅ KAM iteration \textbf{computable via computer algebra} (Lie series)

\item ✅ All results \textbf{certified via interval arithmetic} (rigorous error bounds)

\item ❌ NO numerical orbit integration until verification phase

\item ❌ NO empirical stability estimates



\bigskip\hrule\bigskip


\subsection{2. Mathematical Formulation}


\subsubsection{Integrable Systems and Invariant Tori}

\end{itemize}

\textbf{Integrable Hamiltonian}: H₀(I) depends only on action variables I = (I₁,...,Iₙ) ∈ ℝⁿ.


Hamilton's equations:

\begin{lstlisting}
dI/dt = -∂H₀/∂θ = 0    (actions constant)
dθ/dt = ∂H₀/∂I = ω(I)  (angles evolve linearly)
\end{lstlisting}

\textbf{Invariant tori}: Phase space (I,θ) ∈ ℝⁿ × 𝕋ⁿ foliated into n-tori {I = const}, each with quasi-periodic motion.


\textbf{Frequencies}: ω(I) = ∂H₀/∂I = (ω₁(I),...,ωₙ(I))


\textbf{Example (Kepler problem)}: H₀ = -μ/(2I) → ω = μ²/I³ (single frequency, 1D torus = circle).



\subsubsection{Perturbation and Resonances}

\textbf{Perturbed Hamiltonian}: H = H₀(I) + εH₁(I,θ) where ε ≪ 1.


\textbf{Fourier expansion}:

\begin{lstlisting}
H₁(I,θ) = Σₖ H₁ₖ(I) e^{ik·θ}
\end{lstlisting}

\textbf{Resonance}: Frequency vector ω(I) is \textbf{resonant} if k·ω(I) ≈ 0 for some k ∈ ℤⁿ \ {0}.


\textbf{Small divisors problem}: Perturbation series for quasi-periodic solutions involves denominators k·ω, which vanish at resonances → series diverges.


\textbf{KAM insight}: Avoid resonances by restricting to \textbf{Diophantine frequencies}.



\subsubsection{Diophantine Conditions}

\textbf{Definition}: Frequency vector ω ∈ ℝⁿ is \textbf{Diophantine} with parameters (α, τ) if:


|k·ω| ≥ α/|k|^τ  for all k ∈ ℤⁿ \ {0}


where |k| = |k₁| + ... + |kₙ|.


\textbf{Interpretation}: Frequencies are "sufficiently irrational"—they avoid rational resonances by a margin that decays slower than polynomially.


\textbf{Measure}: Diophantine frequencies have full measure (Lebesgue) in ℝⁿ for τ > n-1.


\textbf{Example (golden ratio)}: ω = ((√5-1)/2, 1) satisfies Diophantine conditions with τ = 2.



\subsubsection{KAM Theorem (Precise Statement)}

\textbf{Theorem (Arnold 1963)}: Let H = H₀(I) + εH₁(I,θ) be a real-analytic Hamiltonian on ℝⁿ × 𝕋ⁿ. Assume:


\begin{itemize}
\item \textbf{Non-degeneracy}: det(∂²H₀/∂I²) ≠ 0 (frequencies change with actions)

\item \textbf{Diophantine}: ω(I₀) = ∂H₀/∂I|_{I₀} satisfies |k·ω| ≥ α/|k|^τ for τ = n+1

\item \textbf{Smallness}: ε < ε₀ (depends on α, τ, analyticity radius)


\end{itemize}

Then there exists a \textbf{Cantor set} K ⊂ ℝⁿ of actions with measure |K| → |ℝⁿ| as ε → 0, such that for I ∈ K:

\begin{itemize}
\item The invariant torus T_I = {(I,θ) : θ ∈ 𝕋ⁿ} survives the perturbation

\item Motion on T_I is quasi-periodic with frequencies ω(I)


\end{itemize}

\textbf{Certificate}: To certify stability, verify:

\begin{itemize}
\item Diophantine condition for initial frequencies

\item Non-degeneracy: Hessian det ≠ 0

\item Perturbation ε below threshold ε₀ (computed via KAM estimates)



\subsubsection{Certificates}

\end{itemize}

All results must come with \textbf{machine-checkable certificates}:


\begin{itemize}
\item \textbf{Diophantine certificate}: Interval arithmetic proof that |k·ω| ≥ α/|k|^τ for |k| ≤ K_max

\item \textbf{Non-degeneracy certificate}: Hessian eigenvalues bounded away from zero

\item \textbf{KAM convergence certificate}: Iterative scheme converges with certified error bounds

\item \textbf{Measure certificate}: Lower bound on volume of surviving tori


\end{itemize}

\textbf{Export format}: JSON with exact algebraic numbers:

\begin{lstlisting}
{
  "system": "Jupiter-Saturn",
  "frequencies": {"omega1": "2.831e-4", "omega2": "1.152e-4"},
  "diophantine_alpha": 0.001,
  "diophantine_tau": 3,
  "epsilon": 0.001,
  "kam_converged": true,
  "stable_tori_measure": 0.95,
  "certified": true
}
\end{lstlisting}


\bigskip\hrule\bigskip


\subsection{3. Implementation Approach}


\subsubsection{Phase 1 (Months 1-2): Action-Angle Variables}

\textbf{Goal}: Convert Keplerian elements to action-angle coordinates.


\begin{lstlisting}
import numpy as np
import sympy as sp
from mpmath import mp
mp.dps = 100

def kepler_to_action_angle(a: float, e: float, i: float,
                           mu: float = 1.0) -> tuple:
    """
    Convert Keplerian orbital elements to Delaunay action-angle variables.

    Args:
        a: semi-major axis
        e: eccentricity
        i: inclination
        mu: gravitational parameter (G*M_sun)

    Returns:
        (actions, angles, frequencies)
        Actions: (L, G, H) where
          L = sqrt(μa)  (mean longitude action)
          G = L*sqrt(1-e²)  (angular momentum)
          H = G*cos(i)  (vertical angular momentum)
    """
    # Delaunay actions
    L = np.sqrt(mu * a)
    G = L * np.sqrt(1 - e**2)
    H = G * np.cos(i)

    actions = np.array([L, G, H])

    # Conjugate angles: (l, g, h) where
    # l = mean anomaly
    # g = argument of perihelion
    # h = longitude of ascending node

    # Frequencies ω = ∂H₀/∂I
    # For Kepler: H₀ = -μ²/(2L²)
    omega_L = mu**2 / L**3  # Mean motion n = sqrt(μ/a³)
    omega_G = 0  # Axisymmetric
    omega_H = 0  # No precession in unperturbed Kepler

    frequencies = np.array([omega_L, omega_G, omega_H])

    return actions, frequencies


def action_angle_to_cartesian(actions: np.ndarray,
                              angles: np.ndarray,
                              mu: float = 1.0) -> tuple:
    """
    Convert action-angle variables back to Cartesian positions and velocities.

    Inverse of kepler_to_action_angle.
    """
    L, G, H = actions
    l, g, h = angles

    # Reconstruct Keplerian elements
    a = L**2 / mu
    e = np.sqrt(1 - (G/L)**2)
    i = np.arccos(H/G)

    # Convert to Cartesian (standard formulas)
    # ... (omitted for brevity)

    return position, velocity


def compute_action_angle_transformation_jacobian(actions: np.ndarray) -> np.ndarray:
    """
    Compute Jacobian ∂(q,p)/∂(θ,I) of action-angle to Cartesian transformation.

    Used for verifying symplecticity: J^T Ω J = Ω where Ω = [[0, I], [-I, 0]].
    """
    L, G, H = actions

    # Symbolic computation
    L_sym, G_sym, H_sym = sp.symbols('L G H', positive=True)
    l_sym, g_sym, h_sym = sp.symbols('l g h', real=True)

    # ... (compute transformation symbolically, then differentiate)

    jacobian = sp.Matrix([[...]])  # 6x6 matrix

    # Evaluate numerically
    J_numeric = np.array(jacobian.subs({L_sym: L, G_sym: G, H_sym: H}).evalf())

    return J_numeric
\end{lstlisting}

\textbf{Validation}: Verify transformation is canonical (symplectic) by checking J^T Ω J = Ω.



\subsubsection{Phase 2 (Months 2-4): Diophantine Conditions}

\textbf{Goal}: Verify frequency vectors satisfy Diophantine inequality.


\begin{lstlisting}
from mpmath import mp, mpf
from fractions import Fraction

def check_diophantine_condition(omega: np.ndarray,
                                alpha: float = 0.001,
                                tau: float = 3.0,
                                k_max: int = 100) -> dict:
    """
    Verify Diophantine condition |k·ω| ≥ α/|k|^τ for all k with |k| ≤ k_max.

    Returns:
        Certificate with worst-case k vector and margin.
    """
    n = len(omega)
    worst_margin = float('inf')
    worst_k = None

    for k in generate_integer_lattice(n, k_max):
        if np.all(k == 0):
            continue

        k_norm = np.sum(np.abs(k))
        k_dot_omega = abs(np.dot(k, omega))

        threshold = alpha / (k_norm ** tau)

        if k_dot_omega < threshold:
            return {
                'is_diophantine': False,
                'resonant_k': k.tolist(),
                'violation': k_dot_omega / threshold
            }

        margin = k_dot_omega / threshold
        if margin < worst_margin:
            worst_margin = margin
            worst_k = k

    return {
        'is_diophantine': True,
        'worst_k': worst_k.tolist(),
        'safety_margin': worst_margin,
        'alpha': alpha,
        'tau': tau,
        'k_max': k_max
    }


def generate_integer_lattice(n: int, k_max: int) -> list:
    """Generate all integer vectors k ∈ ℤⁿ with |k| ≤ k_max."""
    from itertools import product
    vectors = []

    for k in product(range(-k_max, k_max+1), repeat=n):
        if sum(abs(ki) for ki in k) <= k_max:
            vectors.append(np.array(k))

    return vectors


def estimate_diophantine_alpha(omega: np.ndarray,
                               tau: float = 3.0,
                               k_max: int = 1000) -> float:
    """
    Estimate optimal α for given ω.

    Find largest α such that Diophantine condition holds for all |k| ≤ k_max.
    """
    min_ratio = float('inf')

    for k in generate_integer_lattice(len(omega), k_max):
        if np.all(k == 0):
            continue

        k_norm = np.sum(np.abs(k))
        k_dot_omega = abs(np.dot(k, omega))

        ratio = k_dot_omega * (k_norm ** tau)
        if ratio < min_ratio:
            min_ratio = ratio

    alpha_optimal = min_ratio

    return alpha_optimal


def brjuno_function(omega: np.ndarray) -> float:
    """
    Compute Brjuno function B(ω) measuring how close ω is to resonances.

    B(ω) = Σₙ log(qₙ₊₁) / qₙ

    where qₙ are denominators in continued fraction expansion.

    KAM theorem requires B(ω) < ∞ (weaker than Diophantine).
    """
    # Compute continued fraction for ω₁/ω₂ (2D case)
    omega_ratio = omega[0] / omega[1]

    continued_fraction = compute_continued_fraction(omega_ratio, max_terms=50)

    # Compute Brjuno sum
    denominators = continued_fraction_denominators(continued_fraction)

    brjuno_sum = 0
    for n in range(len(denominators) - 1):
        q_n = denominators[n]
        q_n1 = denominators[n+1]

        brjuno_sum += np.log(q_n1) / q_n

    return brjuno_sum
\end{lstlisting}

\textbf{Validation}: Test on known Diophantine frequencies (golden ratio, etc.).



\subsubsection{Phase 3 (Months 4-6): KAM Iteration}

\textbf{Goal}: Implement KAM iterative scheme to construct invariant tori.


\begin{lstlisting}
def kam_iteration(H0_freq: callable,
                 H1_fourier: dict,
                 epsilon: float,
                 max_iterations: int = 20,
                 tolerance: float = 1e-12) -> dict:
    """
    KAM iterative procedure to eliminate non-resonant terms.

    Algorithm (Kolmogorov):
    1. Start with H = H₀ + εH₁
    2. Find generating function S solving homological equation {S, H₀} = H₁^{non-res}
    3. Apply canonical transformation via Lie series
    4. New Hamiltonian H' = H₀' + ε²H₁' + O(ε³)
    5. Repeat until convergence

    Args:
        H0_freq: Function I → ω(I) giving frequencies
        H1_fourier: Dictionary {k: H₁ₖ(I)} of Fourier coefficients
        epsilon: Perturbation parameter
        max_iterations: Maximum KAM steps
        tolerance: Convergence threshold

    Returns:
        Certificate with final Hamiltonian and error estimates
    """
    # Initial data
    I0 = np.array([1.0, 0.9, 0.8])  # Reference action
    omega = H0_freq(I0)

    # Check Diophantine
    dioph_check = check_diophantine_condition(omega)
    if not dioph_check['is_diophantine']:
        return {
            'converged': False,
            'reason': 'resonance',
            'resonant_k': dioph_check['resonant_k']
        }

    # KAM iteration
    H1_current = H1_fourier.copy()
    epsilon_current = epsilon

    for iteration in range(max_iterations):
        # Solve homological equation: ik·ω Sₖ = H₁ₖ
        S_fourier = {}

        for k, H1_k in H1_current.items():
            k_dot_omega = np.dot(k, omega)

            if abs(k_dot_omega) > 1e-10:  # Non-resonant
                S_fourier[k] = H1_k / (1j * k_dot_omega)

        # Compute new H₁' via Lie series: H₁' = H₁ + {S, H₁} + ...
        H1_new = compute_poisson_bracket_fourier(S_fourier, H1_current, omega)

        # Estimate size of new perturbation
        H1_norm = sum(abs(H1_k) for H1_k in H1_new.values())

        print(f"Iteration {iteration}: ||H₁'|| = {H1_norm:.3e}, ε² = {epsilon_current**2:.3e}")

        if H1_norm < tolerance:
            return {
                'converged': True,
                'iterations': iteration,
                'final_perturbation_norm': H1_norm,
                'epsilon_effective': epsilon_current
            }

        # Update for next iteration
        H1_current = H1_new
        epsilon_current = epsilon_current ** 2  # Quadratic convergence

    return {
        'converged': False,
        'reason': 'max_iterations_reached',
        'final_perturbation_norm': H1_norm
    }


def compute_poisson_bracket_fourier(S_fourier: dict,
                                   H1_fourier: dict,
                                   omega: np.ndarray) -> dict:
    """
    Compute {S, H₁} in Fourier space.

    {S, H₁} = i Σₖ₁,ₖ₂ (k₁·ω) Sₖ₁ H₁,ₖ₂ e^{i(k₁+k₂)·θ}
    """
    result = {}

    for k1, S_k1 in S_fourier.items():
        for k2, H1_k2 in H1_fourier.items():
            k_sum = tuple(np.array(k1) + np.array(k2))

            k1_dot_omega = np.dot(k1, omega)

            term = 1j * k1_dot_omega * S_k1 * H1_k2

            if k_sum in result:
                result[k_sum] += term
            else:
                result[k_sum] = term

    return result
\end{lstlisting}

\textbf{Validation}: Test on pendulum (analytically solvable) and verify convergence.



\subsubsection{Phase 4 (Months 6-8): Solar System Application}

\textbf{Goal}: Apply KAM theory to analyze real planetary system stability.


\begin{lstlisting}
def solar_system_kam_stability() -> dict:
    """
    Analyze KAM stability for the solar system.

    Focus on outer planets: Jupiter, Saturn, Uranus, Neptune.
    """
    # Planetary data (semi-major axis in AU, eccentricity, inclination)
    planets = {
        'Jupiter': (5.20, 0.048, 1.31),
        'Saturn': (9.54, 0.054, 2.49),
        'Uranus': (19.19, 0.047, 0.77),
        'Neptune': (30.07, 0.009, 1.77)
    }

    # Convert to action-angle
    actions = {}
    frequencies = {}

    for name, (a, e, i) in planets.items():
        I, omega = kepler_to_action_angle(a, e, np.deg2rad(i))
        actions[name] = I
        frequencies[name] = omega

    # Extract mean motions (first component of frequency vector)
    n_jupiter = frequencies['Jupiter'][0]
    n_saturn = frequencies['Saturn'][0]

    # Famous 5:2 resonance (near miss)
    resonance_ratio = n_jupiter / n_saturn
    print(f"Jupiter/Saturn frequency ratio: {resonance_ratio:.4f} (ideal 5:2 = {5/2})")

    # Check Diophantine for combined system
    omega_combined = np.array([frequencies[p][0] for p in planets.keys()])

    dioph_cert = check_diophantine_condition(omega_combined, alpha=1e-4, tau=4, k_max=20)

    # Estimate perturbation strength
    epsilon = 0.001  # m_Jupiter / m_Sun ~ 10^{-3}

    # Apply KAM iteration (simplified—would need full perturbation Hamiltonian)
    # kam_result = kam_iteration(H0_freq, H1_fourier, epsilon)

    return {
        'planets': list(planets.keys()),
        'frequencies': {name: omega[0] for name, omega in frequencies.items()},
        'diophantine_check': dioph_cert,
        'perturbation_epsilon': epsilon,
        'conclusion': 'STABLE' if dioph_cert['is_diophantine'] else 'RESONANT'
    }


def find_resonances_in_solar_system(planets: dict,
                                   max_order: int = 10) -> list:
    """
    Find all low-order mean-motion resonances k₁n₁ + k₂n₂ ≈ 0.

    Famous examples:
    - Jupiter-Saturn: 5:2 (5n_J - 2n_S ≈ 0)
    - Neptune-Pluto: 3:2
    """
    resonances = []

    planet_names = list(planets.keys())

    for i, name1 in enumerate(planet_names):
        for name2 in planet_names[i+1:]:
            n1 = planets[name1]['mean_motion']
            n2 = planets[name2]['mean_motion']

            # Search for k1, k2 such that |k1*n1 + k2*n2| < tolerance
            for k1 in range(-max_order, max_order+1):
                for k2 in range(-max_order, max_order+1):
                    if k1 == 0 and k2 == 0:
                        continue

                    resonance_value = abs(k1 * n1 + k2 * n2)

                    if resonance_value < 1e-5:  # Near resonance
                        resonances.append({
                            'planets': (name1, name2),
                            'order': (k1, k2),
                            'mismatch': resonance_value
                        })

    return resonances
\end{lstlisting}

\textbf{Validation}: Reproduce Laskar (1989) stability estimates for Jupiter-Saturn system.



\subsubsection{Phase 5 (Months 8-9): Measure Estimates and Certificates}

\textbf{Goal}: Compute volume of phase space occupied by KAM tori.


\begin{lstlisting}
from dataclasses import dataclass, asdict
import json

@dataclass
class KAMCertificate:
    """Complete KAM stability certificate."""

    # System identification
    system_name: str
    n_bodies: int
    perturbation_epsilon: float

    # Frequency data
    frequencies: dict
    is_diophantine: bool
    diophantine_alpha: float
    diophantine_tau: float

    # KAM iteration
    kam_converged: bool
    kam_iterations: int
    final_perturbation_norm: float

    # Measure estimates
    surviving_tori_fraction: float  # Fraction of phase space with stable tori

    # Stability conclusion
    is_stable: bool
    stability_timescale_years: float

    # Metadata
    computation_date: str
    precision_digits: int

    def export_json(self, filename: str):
        """Export certificate to JSON."""
        with open(filename, 'w') as f:
            json.dump(asdict(self), f, indent=2)

    def verify(self) -> bool:
        """Self-check certificate validity."""
        checks = [
            self.n_bodies > 0,
            self.perturbation_epsilon > 0,
            self.diophantine_alpha > 0,
            0 <= self.surviving_tori_fraction <= 1,
            self.stability_timescale_years > 0
        ]
        return all(checks)


def generate_kam_certificate_solar_system() -> KAMCertificate:
    """
    Generate complete KAM certificate for solar system.
    """
    stability_analysis = solar_system_kam_stability()

    cert = KAMCertificate(
        system_name='Solar System (Jupiter-Neptune)',
        n_bodies=4,
        perturbation_epsilon=0.001,
        frequencies={name: freq for name, freq in stability_analysis['frequencies'].items()},
        is_diophantine=stability_analysis['diophantine_check']['is_diophantine'],
        diophantine_alpha=stability_analysis['diophantine_check']['alpha'],
        diophantine_tau=stability_analysis['diophantine_check']['tau'],
        kam_converged=True,  # Would come from KAM iteration
        kam_iterations=15,
        final_perturbation_norm=1e-12,
        surviving_tori_fraction=0.95,  # Estimate from KAM measure theory
        is_stable=True,
        stability_timescale_years=5e9,  # Age of solar system
        computation_date='2026-01-17',
        precision_digits=100
    )

    return cert
\end{lstlisting}

\textbf{Validation}: Export certificates, verify all self-checks pass.



\bigskip\hrule\bigskip


\subsection{4. Example Starting Prompt}

\textbf{Prompt for AI System}:


You are tasked with applying KAM theory to verify planetary stability. Your goals:


\begin{itemize}
\item \textbf{Action-Angle Transformation (Months 1-2)}:

\item Convert Keplerian elements (a, e, i) to Delaunay actions (L, G, H)

\item Compute frequencies ω = ∂H₀/∂I

\item Verify transformation is canonical (symplectic)


\item \textbf{Diophantine Verification (Months 2-4)}:

\item Check |k·ω| ≥ α/|k|^τ for all |k| ≤ 100

\item Estimate optimal α for Jupiter-Saturn system

\item Compute Brjuno function B(ω)


\item \textbf{KAM Iteration (Months 4-6)}:

\item Implement homological equation solver

\item Apply Lie series canonical transformations

\item Verify convergence to O(ε²) perturbation


\item \textbf{Solar System Application (Months 6-8)}:

\item Analyze Jupiter, Saturn, Uranus, Neptune

\item Find all resonances with order ≤ 10

\item Estimate perturbation ε ~ 10^{-3}


\item \textbf{Certificate Generation (Months 8-9)}:

\item Create KAMCertificate with all parameters

\item Export to JSON with interval arithmetic bounds

\item Verify stability timescale > age of solar system


\end{itemize}

\textbf{Success Criteria}:

\begin{itemize}
\item MVR (2-4 months): Action-angle for 2-body, Diophantine checks

\item Strong (6-8 months): KAM iteration converges, Jupiter-Saturn analysis complete

\item Publication (9 months): Full solar system certificate, measure estimates


\end{itemize}

\textbf{References}:

\begin{itemize}
\item Arnold (1963): Proof of KAM theorem

\item Laskar (1989): Numerical chaos in solar system

\item Celletti & Chierchia (2007): KAM stability for realistic models


\end{itemize}

Begin by implementing action-angle transformation for Jupiter orbit.



\bigskip\hrule\bigskip


\subsection{5. Success Criteria}


\subsubsection{Minimum Viable Result (Months 1-4)}

\textbf{Core Achievements}:

\begin{itemize}
\item ✅ Action-angle transformation for Kepler problem

\item ✅ Diophantine verification for 2D frequency vectors

\item ✅ Basic KAM iteration (3-5 steps) for toy Hamiltonian

\item ✅ Certificate generation framework


\end{itemize}

\textbf{Validation}:

\begin{itemize}
\item Canonical transformation verified (Jacobian check)

\item Diophantine condition tested on golden ratio

\item KAM iteration reduces perturbation by factor 100


\end{itemize}

\textbf{Deliverables}:

\begin{itemize}
\item Python module \texttt{kam_theory.py}

\item Jupyter notebook: Jupiter-Saturn resonance analysis

\item JSON certificate for simple 2-body system



\subsubsection{Strong Result (Months 4-8)}

\end{itemize}

\textbf{Extended Capabilities}:

\begin{itemize}
\item ✅ Full KAM iteration with 10+ steps

\item ✅ Solar system stability analysis (Jupiter-Neptune)

\item ✅ Resonance finding algorithm

\item ✅ Measure estimates: fraction of surviving tori

\item ✅ Comparison to Laskar (1989) results


\end{itemize}

\textbf{Publications Benchmark}:

\begin{itemize}
\item Reproduce Laskar stability timescales

\item Match Diophantine parameters to within 10%


\end{itemize}

\textbf{Deliverables}:

\begin{itemize}
\item Database of certificates for 10+ planetary configurations

\item Resonance map (frequency space plot)

\item Stability report: timescales vs perturbation strength



\subsubsection{Publication-Quality Result (Months 8-9)}

\end{itemize}

\textbf{Novel Contributions}:

\begin{itemize}
\item ✅ Rigorous error bounds on KAM iteration

\item ✅ Optimal Diophantine parameters for solar system

\item ✅ Extension to 3-body resonances (secular dynamics)

\item ✅ Formal verification: Coq/Lean proofs of key lemmas

\item ✅ Interactive visualization: invariant tori in phase space


\end{itemize}

\textbf{Beyond Literature}:

\begin{itemize}
\item Improve KAM convergence rates

\item Discover new stability islands in phase space

\item Apply to exoplanetary systems


\end{itemize}

\textbf{Deliverables}:

\begin{itemize}
\item Arxiv preprint: "Certified KAM Stability for the Solar System"

\item GitHub repository with all code and certificates

\item Web tool: check KAM stability for arbitrary planetary systems



\bigskip\hrule\bigskip


\subsection{6. Verification Protocol}

\begin{lstlisting}
def verify_kam_certificate(cert: KAMCertificate) -> dict:
    """
    Automated verification of KAM certificate.
    """
    results = {}

    # Check 1: Diophantine condition
    omega_array = np.array(list(cert.frequencies.values()))
    dioph_recheck = check_diophantine_condition(omega_array, cert.diophantine_alpha, cert.diophantine_tau)
    results['diophantine_verified'] = dioph_recheck['is_diophantine']

    # Check 2: KAM convergence
    results['kam_converged'] = cert.kam_converged

    # Check 3: Measure estimate
    results['measure_reasonable'] = (0.5 < cert.surviving_tori_fraction <= 1.0)

    # Check 4: Stability conclusion
    results['stability_consistent'] = (
        cert.is_stable == (cert.is_diophantine and cert.kam_converged)
    )

    # Overall verdict
    results['all_checks_passed'] = all(
        v for v in results.values() if isinstance(v, bool)
    )

    return results
\end{lstlisting}


\bigskip\hrule\bigskip


\subsection{7. Resources and Milestones}


\subsubsection{Essential References}

\item \textbf{Foundational Papers}:

\item Kolmogorov (1954): "On conservation of conditionally periodic motions"

\item Arnold (1963): "Proof of A.N. Kolmogorov's theorem"

\item Moser (1962): "On invariant curves of area-preserving mappings"


\item \textbf{Modern Developments}:

\item Celletti & Chierchia (2007): "KAM stability and celestial mechanics"

\item Laskar (1989): "A numerical experiment on the chaotic behaviour of the Solar System"

\item Féjoz (2004): "Démonstration du 'théorème d'Arnold' sur la stabilité du système planétaire"


\item \textbf{Textbooks}:

\item Arnold (1989): \textit{Mathematical Methods of Classical Mechanics}

\item Broer & Sevryuk (2007): "KAM theory: quasi-periodicity in dynamical systems"



\subsubsection{Milestone Checklist}

\item [ ] \textbf{Month 1}: Action-angle transformation implemented

\item [ ] \textbf{Month 2}: Diophantine verifier working for n ≤ 4

\item [ ] \textbf{Month 3}: KAM iteration converges for pendulum

\item [ ] \textbf{Month 4}: Jupiter-Saturn frequencies computed

\item [ ] \textbf{Month 5}: Diophantine verified for solar system

\item [ ] \textbf{Month 6}: KAM iteration for planetary Hamiltonian

\item [ ] \textbf{Month 7}: Resonance map generated

\item [ ] \textbf{Month 8}: Measure estimates computed

\item [ ] \textbf{Month 9}: Full certificate database exported



\bigskip\hrule\bigskip

\end{itemize}

\textbf{End of PRD 26}


\end{document}
