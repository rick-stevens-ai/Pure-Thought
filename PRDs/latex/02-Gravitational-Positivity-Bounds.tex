\documentclass[11pt,a4paper]{article}

% Packages
\usepackage[utf8]{inputenc}
\usepackage[T1]{fontenc}
\usepackage{lmodern}
\usepackage[english]{babel}
\usepackage{amsmath,amssymb,amsthm}
\usepackage{mathtools}
\usepackage{physics}
\usepackage{graphicx}
\usepackage{xcolor}
\usepackage{listings}
\usepackage{hyperref}
\usepackage{geometry}
\usepackage{fancyhdr}
\usepackage{tocloft}
\usepackage{enumitem}
\usepackage{booktabs}
\usepackage{algorithm}
\usepackage{algpseudocode}

% Page geometry
\geometry{
    a4paper,
    left=25mm,
    right=25mm,
    top=30mm,
    bottom=30mm,
}

% Header and footer
\pagestyle{fancy}
\fancyhf{}
\fancyhead[L]{Pure Thought Challenge 02}
\fancyhead[R]{\thepage}
\renewcommand{\headrulewidth}{0.4pt}

% Hyperref setup
\hypersetup{
    colorlinks=true,
    linkcolor=blue,
    citecolor=blue,
    urlcolor=blue,
    pdfauthor={Pure Thought AI Challenges},
    pdftitle={Challenge 02: Gravitational Positivity & Causality Bounds},
}

% Code listing style
\definecolor{codegray}{rgb}{0.95,0.95,0.95}
\definecolor{codegreen}{rgb}{0,0.6,0}
\definecolor{codepurple}{rgb}{0.58,0,0.82}

\lstdefinestyle{pythonstyle}{
    language=Python,
    backgroundcolor=\color{codegray},
    commentstyle=\color{codegreen},
    keywordstyle=\color{blue},
    stringstyle=\color{codepurple},
    basicstyle=\ttfamily\small,
    breaklines=true,
    breakatwhitespace=true,
    captionpos=b,
    frame=single,
    numbers=left,
    numberstyle=\tiny\color{gray},
    tabsize=4,
    showstringspaces=false,
}

\lstset{style=pythonstyle}

% Theorem environments
\newtheorem{theorem}{Theorem}[section]
\newtheorem{lemma}[theorem]{Lemma}
\newtheorem{proposition}[theorem]{Proposition}
\newtheorem{corollary}[theorem]{Corollary}
\theoremstyle{definition}
\newtheorem{definition}[theorem]{Definition}
\newtheorem{example}[theorem]{Example}
\theoremstyle{remark}
\newtheorem{remark}[theorem]{Remark}

% Custom commands
\newcommand{\checklist}[1]{\item[$\square$] #1}
\newcommand{\R}{\mathbb{R}}
\newcommand{\C}{\mathbb{C}}
\newcommand{\Z}{\mathbb{Z}}
\newcommand{\N}{\mathbb{N}}

% Title information
\title{\textbf{Challenge 02: Gravitational Positivity & Causality Bounds} \\
\large Pure Thought AI Challenge 02}
\author{Pure Thought AI Challenges Project}
\date{\today}

\begin{document}

\maketitle
\thispagestyle{empty}

\begin{abstract}
This document presents a comprehensive Product Requirement Document (PRD) for implementing a pure-thought computational challenge. The problem can be tackled using only symbolic mathematics, exact arithmetic, and fresh code---no experimental data or materials databases required until final verification. All results must be accompanied by machine-checkable certificates.
\end{abstract}

\clearpage
\tableofcontents
\clearpage


\textbf{Domain:} Quantum Gravity & Particle Physics

\textbf{Difficulty:} High

\textbf{Timeline:} 3-9 months

\textbf{Prerequisites:} Scattering amplitudes, dispersion relations, convex optimization



\bigskip\hrule\bigskip


\subsection{Problem Statement}


\subsubsection{Scientific Context}
Effective field theories (EFTs) with gravity cannot be arbitrary. Quantum consistency, unitarity, causality, and analyticity impose stringent constraints on the coefficients of higher-derivative terms like R², R³ in the gravitational action. These constraints distinguish theories that can be UV-completed into consistent quantum gravity from those in the "swampland."



\subsubsection{The Core Question}
\textbf{What are the sharp bounds on Wilson coefficients of higher-derivative gravitational operators imposed purely by consistency conditions?}


For example, for corrections to Einstein gravity:

\begin{lstlisting}
S = ∫ d⁴x √g [M_pl² R + a₁ R² + a₂ R_μν R^μν + a₃ R_μνρσ R^μνρσ + ...]
\end{lstlisting}

What ranges of {a₁, a₂, a₃, ...} are consistent with:

\begin{itemize}
\item Unitarity (positive-norm states)

\item Causality (no superluminal propagation)

\item Analyticity (dispersion relations)

\item Crossing symmetry



\subsubsection{Why This Matters}
\item \textbf{Swampland program:} Determines which low-energy theories can couple to gravity

\item \textbf{Phenomenology:} Constrains quantum gravity corrections to GR

\item \textbf{Rigorous:} Produces mathematical


\end{itemize}

 no-go theorems, not heuristics



\bigskip\hrule\bigskip


\subsection{Mathematical Formulation}


\subsubsection{Problem Definition}

Consider graviton-graviton scattering amplitude M(s,t) where s,t are Mandelstam variables.


\textbf{Constraints from physics:}


\begin{itemize}
\item \textbf{Unitarity:} Im M(s,t) ≥ 0 in physical region

\item \textbf{Analyticity:} M(s,t) is analytic except on physical cuts

\item \textbf{Crossing:} M(s,t) = M(t,s) = M(u,t) where u = -s-t

\item \textbf{Causality (shockwave):} Time delay δt ≥ 0 for shockwave scattering

\item Translates to: certain combinations of Wilson coefficients must be positive


\item \textbf{Dispersion relation:} For fixed t < 0,

\begin{lstlisting}
   M(s,t) = M(0,t) + s²/π ∫_{s_th}^∞ ds' Im M(s',t)/(s'²(s'-s))
\end{lstlisting}

\end{itemize}

\textbf{Optimization formulation:}


Given a set of Wilson coefficients {a₁, a₂, ..., aₙ}, determine feasibility:


\begin{lstlisting}
Minimize/Maximize: aᵢ
Subject to:
  - Dispersion relation holds
  - Im M(s,t) ≥ 0 (unitarity)
  - Crossing symmetry  satisfied
  - Causality bounds satisfied
  - Regge bound: M(s,t) ~ s² at large s
\end{lstlisting}

This is a \textbf{Sum-of-Squares (SoS) or Semidefinite Program (SDP)} that can be solved with certificates.



\subsubsection{Certificate of Correctness}

\textbf{Allowed region certificate:}

\begin{itemize}
\item Explicit Wilson coefficients {a₁\textit{, a₂}, ...}

\item Explicit scattering amplitude M(s,t) satisfying all constraints

\item Verification: compute amplitude, check unitarity numerically


\end{itemize}

\textbf{Forbidden region certificate:}

\begin{itemize}
\item Dual SoS certificate: polynomial p(s,t) such that:

\item p(s,t) is positive on physical region

\item ∫ p(s,t) [violated constraint] ds dt < 0

\item This proves the region is impossible



\bigskip\hrule\bigskip


\subsection{Implementation Approach}


\subsubsection{Phase 1: 2→2 Graviton Scattering at Tree Level (Month 1-2)}

\end{itemize}

\textbf{Build amplitude calculator:}


\begin{itemize}
\item \textbf{Einstein gravity amplitude}

\begin{lstlisting}
   def einstein_amplitude(s, t, M_pl):
       """
       Pure GR amplitude for graviton-graviton → graviton-graviton
       """
       u = -s - t
       return (s*t*u) / M_pl**2  # Schematic
\end{lstlisting}

\item \textbf{Higher-derivative corrections}

\begin{lstlisting}
   def corrected_amplitude(s, t, M_pl, wilson_coeffs):
       """
       Amplitude including R², R³, ... corrections
       """
       M_0 = einstein_amplitude(s, t, M_pl)
       M_R2 = wilson_coeffs['a1'] * (s**2 + t**2 + u**2)
       M_R3 = wilson_coeffs['a2'] * (s**3 + t**3 + u**3)
       return M_0 + M_R2 / M_pl**4 + M_R3 / M_pl**6 + ...
\end{lstlisting}

\item \textbf{Verify crossing symmetry}

\begin{lstlisting}
   assert abs(M(s,t) - M(t,s)) < 1e-10
   assert abs(M(s,t) - M(u,t)) < 1e-10
\end{lstlisting}


\subsubsection{Phase 2: Dispersion Relations (Month 2-3)}

\end{itemize}

\textbf{Implement forward dispersion relation:}


\begin{lstlisting}
def dispersion_relation(M, s, t, s_min, s_max):
    """
    Check if M(s,t) satisfies dispersion relation

    M(s,t) = poly(s,t) + s^N/π ∫ ds' Im M(s',t)/(s'^N(s'-s))
    """
    # Subtracted dispersion relation
    lhs = M(s, t)

    # Polynomial subtraction
    poly_part = sum(c_n * s**n for n, c_n in subtractions)

    # Dispersive integral
    def integrand(s_prime):
        return imag(M(s_prime + 1j*epsilon, t)) / (s_prime**N * (s_prime - s))

    dispersive_part = s**N / np.pi * quad(integrand, s_min, s_max)[0]

    rhs = poly_part + dispersive_part

    return abs(lhs - rhs)  # Should be ~ 0
\end{lstlisting}


\subsubsection{Phase 3: Causality from Shockwave Scattering (Month 3-4)}

\textbf{Eikonal phase shift:}


The shockwave time delay is related to the eikonal phase:

\begin{lstlisting}
δ(b) = (1/2s) ∫ d²q/(2π)² e^{iq·b} M(s, t=-q²)
\end{lstlisting}

\textbf{Causality:} δ(b) ≥ 0 for all impact parameters b


This translates to positivity constraints on M(s,t) at fixed t.



\subsubsection{Phase 4: SDP Formulation & Solver (Month 4-6)}

\textbf{Set up SDP:}


\begin{lstlisting}
import cvxpy as cp

# Variables: Wilson coefficients
a = cp.Variable(n_coeffs)

# Constraints
constraints = []

# 1. Unitarity: Im M ≥ 0
for s_i, t_i in grid_points:
    Im_M = imaginary_part_amplitude(s_i, t_i, a)
    constraints.append(Im_M >= 0)

# 2. Causality: shockwave positivity
for b_i in impact_parameters:
    delta = eikonal_phase(b_i, a)
    constraints.append(delta >= 0)

# 3. Dispersion relation (approximate as polynomial constraints)
# ... implement as sum-of-squares

# Objective: maximize/minimize specific coefficient
objective = cp.Maximize(a[0])  # e.g., bound on a₁

problem = cp.Problem(objective, constraints)
problem.solve(solver=cp.MOSEK, verbose=True)

# Extract dual certificate
dual_cert = constraints[i].dual_value
\end{lstlisting}


\subsubsection{Phase 5: Exact Certificates (Month 6-9)}

\textbf{Generate SoS certificates:}


Use Sum-of-Squares decomposition to prove bounds:

\begin{lstlisting}
from sympy import symbols, expand
from sympy.polys.polytools import Poly

s, t = symbols('s t', real=True)

# Claim: a₁ ≥ a₁_min
# Proof: Show that (a₁ - a₁_min) can be written as SoS

def find_sos_certificate(constraint_poly, variables):
    """
    Find polynomials {p_i} such that:
    constraint_poly = Σ p_i² + (physical constraints)
    """
    # Use SDP to find {p_i}
    # Return explicit polynomials
    pass
\end{lstlisting}


\bigskip\hrule\bigskip


\subsection{Example Starting Prompt}

\begin{lstlisting}
I need you to derive and implement bounds on gravitational Wilson coefficients
from causality and unitarity.

GOAL: Bound the coefficient a₁ of the R² term in the gravitational action.

PHASE 1 - Set up the amplitude:
1. Write down the tree-level graviton-graviton scattering amplitude
   in Einstein gravity (just GR).

2. Add the R² correction term and write the corrected amplitude M(s,t; a₁).

3. Verify crossing symmetry: M(s,t) = M(t,s) = M(u,t).

PHASE 2 - Implement dispersion relation:
4. Write the forward dispersion relation (t=0) for graviton scattering.

5. Express the dispersive integral in terms of Im M(s, 0).

6. Use unitarity: Im M(s,0) ≥ 0 for s > threshold.

PHASE 3 - Causality constraint:
7. Implement the eikonal phase shift δ(b) from the amplitude.

8. Impose δ(b) ≥ 0 for all impact parameters b.

9. Translate this into a constraint on a₁.

PHASE 4 - Solve the optimization:
10. Formulate as SDP: maximize a₁ subject to all constraints.

11. Also minimize a₁ to get the allowed range.

12. Extract dual certificate from the SDP solver.

PHASE 5 - Verify:
13. Check that the certificate is a valid SoS decomposition.

14. Export certificate in machine-readable format.

Please implement this step-by-step, verifying each constraint independently
before combining them.
\end{lstlisting}


\bigskip\hrule\bigskip


\subsection{Success Criteria}


\subsubsection{Minimum Viable Result (3 months)}

✅ \textbf{Single coefficient bound:}

\begin{itemize}
\item Rigorous bounds on a₁ (R² coefficient)

\item Both upper and lower bounds

\item Dual certificate extracted and verified


\end{itemize}

✅ \textbf{Validation:}

\begin{itemize}
\item Reproduce known bounds from literature (if any)

\item Independent verification of certificate



\subsubsection{Strong Result (6 months)}

\end{itemize}

✅ \textbf{Multi-parameter bounds:}

\begin{itemize}
\item Simultaneous bounds on {a₁, a₂, a₃}

\item Allowed region in 3D parameter space

\item Boundary of region certified with SoS


\end{itemize}

✅ \textbf{Novel bounds:}

\begin{itemize}
\item Tighter than previous literature, or

\item New multi-coupling constraints

\item Machine-checkable certificates



\subsubsection{Publication-Quality Result (9 months)}

\end{itemize}

✅ \textbf{Complete EFT space:}

\begin{itemize}
\item All Wilson coefficients up to dimension 8 operators

\item Full allowed region characterized

\item Phase diagram of allowed vs. forbidden regions


\end{itemize}

✅ \textbf{Formal proofs:}

\begin{itemize}
\item Certificates formalized in Lean/Isabelle

\item Automated theorem proving for no-go regions

\item Published with proof repository



\bigskip\hrule\bigskip


\subsection{Verification Protocol}


\subsubsection{Automated Checks}

\begin{lstlisting}
def verify_bound_certificate(coeff_bounds, dual_certificate):
    """
    Verify that the dual certificate proves the bound.
    """
    # 1. Check SoS decomposition
    sos_polynomials = extract_sos_from_certificate(dual_certificate)
    reconstructed = sum(p**2 for p in sos_polynomials)
    assert is_equivalent(reconstructed, constraint_polynomial)

    # 2. Verify positivity on physical region
    test_points = generate_physical_region_samples(n=1000)
    for s, t in test_points:
        assert eval_certificate(dual_certificate, s, t) >= -1e-10

    # 3. Check bound is saturated correctly
    critical_amplitude = construct_amplitude_from_certificate(dual_certificate)
    verify_unitarity(critical_amplitude)
    verify_causality(critical_amplitude)
    verify_crossing(critical_amplitude)

    return "VERIFIED"
\end{lstlisting}


\subsubsection{Exported Artifacts}

\item \textbf{Bound certificate:} \texttt{bound_a1.sos}

\item Sum-of-Squares decomposition

\item Exact rational coefficients

\item Verifiable by independent SoS checkers


\item \textbf{Allowed region:} \texttt{allowed_region.json}

\begin{lstlisting}
   {
     "coefficients": ["a1", "a2", "a3"],
     "bounds": {
       "a1": {"min": -0.5, "max": 0.5},
       "a2": {"min": 0.0, "max": 1.0}
     },
     "constraints": ["unitarity", "causality", "crossing"]
   }
\end{lstlisting}

\item \textbf{Proof script:} \texttt{gravitational_bounds.lean}



\bigskip\hrule\bigskip


\subsection{Milestone Checklist}

\item [ ] Tree-level amplitude calculator implemented

\item [ ] Crossing symmetry verified numerically

\item [ ] Forward dispersion relation implemented

\item [ ] Unitarity bounds imposed

\item [ ] Eikonal/shockwave causality constraints derived

\item [ ] SDP solver setup with certificate extraction

\item [ ] First single-coefficient bound obtained

\item [ ] Dual certificate verified independently

\item [ ] Multi-parameter bounds computed

\item [ ] Formal verification initiated

\item [ ] Publication draft ready



\bigskip\hrule\bigskip

\end{itemize}

\textbf{Next Steps:} Start with implementing the tree-level amplitude and verify crossing symmetry before adding any constraints.


\end{document}
