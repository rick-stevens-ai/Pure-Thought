\documentclass[11pt,a4paper]{article}

% Packages
\usepackage[utf8]{inputenc}
\usepackage[T1]{fontenc}
\usepackage{lmodern}
\usepackage[english]{babel}
\usepackage{amsmath,amssymb,amsthm}
\usepackage{mathtools}
\usepackage{physics}
\usepackage{graphicx}
\usepackage{xcolor}
\usepackage{listings}
\usepackage{hyperref}
\usepackage{geometry}
\usepackage{fancyhdr}
\usepackage{tocloft}
\usepackage{enumitem}
\usepackage{booktabs}
\usepackage{algorithm}
\usepackage{algpseudocode}

% Page geometry
\geometry{
    a4paper,
    left=25mm,
    right=25mm,
    top=30mm,
    bottom=30mm,
}

% Header and footer
\pagestyle{fancy}
\fancyhf{}
\fancyhead[L]{Pure Thought Challenge 01}
\fancyhead[R]{\thepage}
\renewcommand{\headrulewidth}{0.4pt}

% Hyperref setup
\hypersetup{
    colorlinks=true,
    linkcolor=blue,
    citecolor=blue,
    urlcolor=blue,
    pdfauthor={Pure Thought AI Challenges},
    pdftitle={Challenge 01: AdS₃ Pure Gravity via the Modular Bootstrap},
}

% Code listing style
\definecolor{codegray}{rgb}{0.95,0.95,0.95}
\definecolor{codegreen}{rgb}{0,0.6,0}
\definecolor{codepurple}{rgb}{0.58,0,0.82}

\lstdefinestyle{pythonstyle}{
    language=Python,
    backgroundcolor=\color{codegray},
    commentstyle=\color{codegreen},
    keywordstyle=\color{blue},
    stringstyle=\color{codepurple},
    basicstyle=\ttfamily\small,
    breaklines=true,
    breakatwhitespace=true,
    captionpos=b,
    frame=single,
    numbers=left,
    numberstyle=\tiny\color{gray},
    tabsize=4,
    showstringspaces=false,
}

\lstset{style=pythonstyle}

% Theorem environments
\newtheorem{theorem}{Theorem}[section]
\newtheorem{lemma}[theorem]{Lemma}
\newtheorem{proposition}[theorem]{Proposition}
\newtheorem{corollary}[theorem]{Corollary}
\theoremstyle{definition}
\newtheorem{definition}[theorem]{Definition}
\newtheorem{example}[theorem]{Example}
\theoremstyle{remark}
\newtheorem{remark}[theorem]{Remark}

% Custom commands
\newcommand{\checklist}[1]{\item[$\square$] #1}
\newcommand{\R}{\mathbb{R}}
\newcommand{\C}{\mathbb{C}}
\newcommand{\Z}{\mathbb{Z}}
\newcommand{\N}{\mathbb{N}}

% Title information
\title{\textbf{Challenge 01: AdS₃ Pure Gravity via the Modular Bootstrap} \\
\large Pure Thought AI Challenge 01}
\author{Pure Thought AI Challenges Project}
\date{\today}

\begin{document}

\maketitle
\thispagestyle{empty}

\begin{abstract}
This document presents a comprehensive Product Requirement Document (PRD) for implementing a pure-thought computational challenge. The problem can be tackled using only symbolic mathematics, exact arithmetic, and fresh code---no experimental data or materials databases required until final verification. All results must be accompanied by machine-checkable certificates.
\end{abstract}

\clearpage
\tableofcontents
\clearpage


\textbf{Domain:} Quantum Gravity & Particle Physics

\textbf{Difficulty:} High

\textbf{Timeline:} 6-12 months

\textbf{Prerequisites:} Conformal field theory, modular forms, semidefinite programming



\bigskip\hrule\bigskip


\subsection{Problem Statement}


\subsubsection{Scientific Context}

The \textbf{AdS/CFT correspondence} (Anti-de Sitter / Conformal Field Theory), discovered by Maldacena in 1997, stands as one of the most profound insights in theoretical physics. It posits an exact duality between quantum gravity in (d+1)-dimensional Anti-de Sitter space and a d-dimensional conformal field theory living on the boundary. For \textbf{AdS₃/CFT₂}—the three-dimensional bulk with a two-dimensional boundary—this correspondence is particularly tractable due to the infinite-dimensional Virasoro symmetry of 2D CFTs.


\textbf{Pure gravity} in AdS₃ contains only the graviton and no additional matter fields. According to AdS/CFT, such a theory should be dual to an \textbf{extremal CFT}: a 2D conformal field theory where the spectrum is maximally sparse, containing only the stress tensor (and its Virasoro descendants) up to a large conformal dimension gap Δ_gap. For central charge c = 24, this extremal theory is unique and corresponds to the \textbf{Monstrous Moonshine} CFT, whose partition function is given by the \textbf{j-invariant} and whose symmetry group is the \textbf{Monster group} M—the largest sporadic finite simple group.


The \textbf{modular bootstrap} approach exploits the fact that the torus partition function Z(τ) of any 2D CFT must be invariant under the modular group SL(2,ℤ), which acts on the modular parameter τ by Möbius transformations. This \textbf{modular invariance}, combined with \textbf{unitarity} (all operator degeneracies are non-negative) and \textbf{crossing symmetry}, imposes powerful constraints that can be formulated as a linear or semidefinite programming problem. By solving this optimization problem, one can either:


\begin{itemize}
\item \textbf{Construct} explicit extremal CFTs by finding consistent spectra

\item \textbf{Rule out} existence via dual certificates proving no solution exists


\end{itemize}

For c = 24k with k > 1, the question of whether extremal CFTs exist is \textbf{open}. If they exist, they would provide consistent quantum gravity theories in AdS₃ at those central charges. If they provably do not exist, this would constrain which gravitational theories are mathematically consistent—shedding light on the \textbf{Swampland program} distinguishing low-energy effective theories that can be UV-completed into quantum gravity from those that cannot.



\subsubsection{The Core Question}

\textbf{Do extremal 2D CFTs exist for central charge c = 24k (k > 1) with only Virasoro primaries below the gap Δ_gap ≈ c/12?}


For k = 1 (c = 24), the \textbf{Monster CFT} provides an explicit example with Δ_gap = 2. For higher k, no such theories are known, and numerical evidence suggests they may not exist for certain gaps. Proving existence or impossibility for k ≥ 2 would constitute a major result in quantum gravity and conformal field theory.



\subsubsection{Why This Matters}

\begin{itemize}
\item \textbf{Existence:} Constructing explicit extremal CFTs would prove pure AdS₃ gravity theories exist at these central charges and potentially reveal new moonshine phenomena connecting number theory and physics

\item \textbf{Impossibility:} Rigorous no-go theorems would constrain the landscape of quantum gravity theories and support Swampland conjectures about which effective theories can be UV-completed

\item \textbf{Method:} The modular bootstrap uses only fundamental axioms (modular invariance, unitarity, integrality)—no phenomenological input or experimental data required

\item \textbf{Moonshine:} Connections between modular forms, sporadic groups, and physics have led to profound mathematical discoveries; extremal CFTs at higher c might reveal new instances



\bigskip\hrule\bigskip


\subsection{Mathematical Formulation}


\subsubsection{Virasoro Algebra and Characters}

\end{itemize}

A 2D CFT with central charge c is characterized by its Virasoro algebra:

\begin{lstlisting}
[L_m, L_n] = (m - n) L_{m+n} + (c/12) m(m² - 1) δ_{m+n,0}
\end{lstlisting}

Primary operators |h⟩ satisfy L\textit{0 |h⟩ = h |h⟩ (conformal dimension h) and L}m |h⟩ = 0 for m > 0. The \textbf{Virasoro character} at level h is:

\begin{lstlisting}
χ_h(q) = Tr_{V_h} q^{L_0 - c/24}
       = q^{h - c/24} / ∏_{n=1}^∞ (1 - q^n)
       = q^{h - c/24} η(τ)^{-1}
\end{lstlisting}
where q = exp(2πiτ), τ is the modular parameter, and η(τ) is the \textbf{Dedekind eta function}:

\begin{lstlisting}
η(τ) = q^{1/24} ∏_{n=1}^∞ (1 - q^n)
\end{lstlisting}

The torus \textbf{partition function} is:

\begin{lstlisting}
Z(τ, τ̄) = Σ_h d(h) |χ_h(τ)|²
\end{lstlisting}
where d(h) is the degeneracy of primary operators at conformal dimension h.



\subsubsection{Modular Invariance}

The partition function must be invariant under the modular group PSL(2,ℤ) = SL(2,ℤ)/{±I}. The key generators are:

\begin{lstlisting}
S: τ → -1/τ
T: τ → τ + 1
\end{lstlisting}

For holomorphic CFTs (no anti-holomorphic dependence), we have:

\begin{lstlisting}
Z(τ) = χ_0(τ) + Σ_{h > 0} d(h) χ_h(τ)
\end{lstlisting}

\textbf{Modular S-transformation} relates characters via:

\begin{lstlisting}
χ_h(-1/τ) = Σ_{h'} S_{h,h'} χ_{h'}(τ)
\end{lstlisting}

For Virasoro characters, the S-matrix is:

\begin{lstlisting}
S_{h,h'} = i exp(-2πi√(hh'))  (approximately, for large c)
\end{lstlisting}

Modular invariance Z(τ) = Z(-1/τ) imposes \textbf{infinitely many linear constraints} on the degeneracies {d(h)}.



\subsubsection{Extremality Condition}

An extremal CFT has \textbf{no primaries in the gap} (0, Δ_gap) except the vacuum:

\begin{lstlisting}
d(0) = 1  (vacuum)
d(h) = 0  for 0 < h < Δ_gap
d(h) ≥ 0  for h ≥ Δ_gap
\end{lstlisting}

For c = 24k, a natural gap choice is Δ_gap = c/12 = 2k, though other gaps can be explored.



\subsubsection{Optimization Problem Formulation}

The modular bootstrap can be cast as a \textbf{linear programming} (LP) or \textbf{semidefinite programming} (SDP) feasibility problem:


\textbf{Primal Problem:}

\begin{lstlisting}
Find: {d(h) ∈ ℤ, d(h) ≥ 0} for h ≥ Δ_gap
Subject to:
  1. Z(τ) - Z(-1/τ) = 0  (modular invariance)
  2. d(h) ≥ 0           (unitarity)
  3. d(h) ∈ ℤ           (integrality)
\end{lstlisting}

In practice, we truncate the spectrum at some large h_max (justified by asymptotic growth bounds) and solve:

\begin{lstlisting}
Minimize: 0  (feasibility problem)
Variables: d(h) for h ∈ {Δ_gap, Δ_gap + 1, ..., h_max}
Constraints: Modular invariance equations + d(h) ≥ 0
\end{lstlisting}

\textbf{Dual Problem:}


If the primal is infeasible, the LP dual provides a \textbf{certificate of impossibility}: a functional α(h) such that:

\begin{lstlisting}
Σ_h α(h) [modular constraint]_h < 0
α(h) ≥ 0 for allowed h
\end{lstlisting}

This proves mathematically that no solution exists.



\subsubsection{Certificate of Correctness}

\textbf{If feasible (extremal CFT exists):}

\begin{itemize}
\item Explicit list of degeneracies {d(h)} for h ∈ [Δ\textit{gap, h}max]

\item Verification: Compute Z(τ) and check |Z(τ) - Z(-1/τ)| < ε numerically at many τ points

\item Verification: All d(h) are non-negative integers

\item Export to JSON with exact integer values


\end{itemize}

\textbf{If infeasible (no extremal CFT):}

\begin{itemize}
\item SDP/LP dual certificate: functional α(h) with exact rational coefficients

\item Verification: Check α certifies infeasibility via Farkas lemma

\item Export to SMT-LIB or LP certificate format for independent verification

\item Optionally: Formalize in Lean/Isabelle for machine-checked proof



\bigskip\hrule\bigskip


\subsection{Implementation Approach}


\subsubsection{Phase 1: Virasoro Characters and Modular Forms (Months 1-2)}

\end{itemize}

\textbf{Goal:} Build high-precision calculator for Virasoro characters and modular transformations.


\textbf{Key Components:}


\begin{itemize}
\item \textbf{Dedekind eta function:}

\begin{lstlisting}
from mpmath import mp, exp, pi, I, qfrom, prod

mp.dps = 150  # 150 decimal places

def dedekind_eta(tau: complex) -> complex:
    """
    Compute η(τ) = q^{1/24} ∏_{n=1}^∞ (1 - q^n)

    Uses q-series truncation with error control.
    """
    q = mp.exp(2 * mp.pi * I * tau)

    # Product truncation (error ~ q^{N_max})
    N_max = 100
    product = mp.mpf(1)

    for n in range(1, N_max + 1):
        product *= (1 - q**n)

    eta = q**(mp.mpf(1)/24) * product

    return complex(eta)

def test_eta_modular():
    """
    Verify η(-1/τ) = √(-iτ) η(τ)
    """
    tau = 0.3 + 0.5j

    eta_tau = dedekind_eta(tau)
    eta_S_tau = dedekind_eta(-1/tau)

    # Expected relation
    expected = mp.sqrt(-I * tau) * eta_tau

    assert abs(eta_S_tau - expected) < 1e-50
    print(f"η modular check passed: error = {abs(eta_S_tau - expected)}")
\end{lstlisting}

\item \textbf{Virasoro character:}

\begin{lstlisting}
def virasoro_character(c: float, h: float, tau: complex) -> complex:
    """
    Compute χ_h(τ) = q^{h - c/24} / η(τ)

    Args:
        c: central charge
        h: conformal dimension
        τ: modular parameter (Im(τ) > 0)

    Returns:
        Character value χ_h(τ)
    """
    q = mp.exp(2 * mp.pi * I * tau)

    eta_tau = dedekind_eta(tau)

    chi = q**(h - c/24) / eta_tau

    return complex(chi)

def character_grid(c: float, h_values: List[float], tau: complex) -> np.ndarray:
    """
    Compute characters for multiple h values.
    """
    chi = np.array([virasoro_character(c, h, tau) for h in h_values], dtype=complex)
    return chi
\end{lstlisting}

\item \textbf{Partition function:}

\begin{lstlisting}
from typing import Dict

def partition_function(c: float, spectrum: Dict[float, int], tau: complex) -> complex:
    """
    Compute Z(τ) = χ_0(τ) + Σ_h d(h) χ_h(τ)

    Args:
        c: central charge
        spectrum: dictionary {h: d(h)} of degeneracies
        tau: modular parameter

    Returns:
        Partition function Z(τ)
    """
    # Vacuum contribution
    Z = virasoro_character(c, 0, tau)

    # Sum over primaries
    for h, dh in spectrum.items():
        if h > 0:
            Z += dh * virasoro_character(c, h, tau)

    return Z

def verify_modular_invariance(c: float, spectrum: Dict[float, int],
                               tau_samples: List[complex], tol: float = 1e-50) -> bool:
    """
    Verify Z(τ) = Z(-1/τ) at multiple points.
    """
    for tau in tau_samples:
        Z_tau = partition_function(c, spectrum, tau)
        Z_S_tau = partition_function(c, spectrum, -1/tau)

        error = abs(Z_tau - Z_S_tau)
        if error > tol:
            print(f"Modular invariance violated at τ={tau}: error={error}")
            return False

    return True
\end{lstlisting}

\end{itemize}

\textbf{Validation:} Test on known modular forms (j-invariant, E₄, E₆).



\subsubsection{Phase 2: Modular S-Matrix and Constraints (Month 2-3)}

\textbf{Goal:} Compute S-matrix S_{h,h'} and formulate modular invariance as linear equations.


\textbf{S-Matrix Computation:}


For large central charge c, the Virasoro S-matrix can be approximated, but for exact results we compute it from:

\begin{lstlisting}
χ_h(-1/τ) = Σ_{h'} S_{h,h'} χ_{h'}(τ)
\end{lstlisting}

\begin{lstlisting}
def compute_s_matrix(c: float, h_values: List[float], tau_ref: complex = 0.1 + 0.5j) -> np.ndarray:
    """
    Compute S-matrix S_{h,h'} by evaluating characters.

    S_{h,h'} = coefficient of χ_{h'}(τ) in expansion of χ_h(-1/τ)
    """
    N = len(h_values)
    S = np.zeros((N, N), dtype=complex)

    # Compute characters at reference τ
    chi_tau = np.array([virasoro_character(c, h, tau_ref) for h in h_values])

    # Compute characters at -1/τ
    tau_S = -1 / tau_ref
    chi_S_tau = np.array([virasoro_character(c, h, tau_S) for h in h_values])

    # Solve for S: chi_S_tau = S @ chi_tau
    # In practice, use multiple τ points and least squares

    for i, h in enumerate(h_values):
        chi_h_S = virasoro_character(c, h, tau_S)

        # Express as linear combination of chi_{h'}(tau_ref)
        # Solve via least squares (overdetermined system with multiple τ)
        # For now, use direct inversion (requires N tau points)
        pass

    # Simplified: use asymptotic formula for large c
    for i, h in enumerate(h_values):
        for j, hp in enumerate(h_values):
            S[i, j] = I * mp.exp(-2 * mp.pi * I * mp.sqrt(h * hp))

    return S
\end{lstlisting}

\textbf{Modular Constraints:}


Modular invariance Z(τ) = Z(-1/τ) gives:

\begin{lstlisting}
χ_0(-1/τ) + Σ_h d(h) χ_h(-1/τ) = χ_0(τ) + Σ_h d(h) χ_h(τ)
\end{lstlisting}

Using S-matrix:

\begin{lstlisting}
Σ_{h'} S_{0,h'} χ_{h'}(τ) + Σ_h d(h) Σ_{h'} S_{h,h'} χ_{h'}(τ) = χ_0(τ) + Σ_h d(h) χ_h(τ)
\end{lstlisting}

Matching coefficients of χ_{h'}(τ) for each h':

\begin{lstlisting}
S_{0,h'} + Σ_h d(h) S_{h,h'} = δ_{h',0} + d(h')
\end{lstlisting}

Rearranging:

\begin{lstlisting}
Σ_h [S_{h,h'} - δ_{h,h'}] d(h) = δ_{h',0} - S_{0,h'}
\end{lstlisting}

This is a \textbf{system of linear equations} in the degeneracies {d(h)}.


\begin{lstlisting}
def setup_modular_constraints(c: float, gap: float, h_max: float,
                               h_values: List[float]) -> Tuple[np.ndarray, np.ndarray]:
    """
    Set up Ax = b for modular invariance.

    Variables: x = [d(h_1), d(h_2), ..., d(h_N)]
    where h_i ∈ [gap, h_max]

    Constraints: one equation per h' in h_values
    """
    N = len(h_values)

    # Compute S-matrix
    S = compute_s_matrix(c, h_values)

    # Build constraint matrix A and RHS b
    A = np.zeros((N, N), dtype=complex)
    b = np.zeros(N, dtype=complex)

    for i, hp in enumerate(h_values):
        # Equation for h'=hp
        for j, h in enumerate(h_values):
            A[i, j] = S[j, i] - (1 if h == hp else 0)

        # RHS
        b[i] = (1 if hp == 0 else 0) - S[0, i]

    return A, b
\end{lstlisting}


\subsubsection{Phase 3: Linear Programming and Optimization (Months 3-4)}

\textbf{Goal:} Solve for non-negative integer degeneracies or certify infeasibility.


\textbf{Relaxed LP (continuous variables):}


\begin{lstlisting}
import cvxpy as cp

def solve_modular_bootstrap_lp(c: float, gap: float, h_max: float) -> Dict:
    """
    Solve modular bootstrap as linear program.

    Minimize: 0  (feasibility problem)
    Subject to: A @ d = b, d ≥ 0
    """
    h_values = np.arange(gap, h_max + 1, 1.0)
    N = len(h_values)

    A, b = setup_modular_constraints(c, gap, h_max, h_values)

    # Convert to real system (separate real/imaginary parts)
    A_real = np.vstack([A.real, A.imag])
    b_real = np.hstack([b.real, b.imag])

    # Define variables
    d = cp.Variable(N, nonneg=True)

    # Constraints
    constraints = [A_real @ d == b_real]

    # Solve
    problem = cp.Problem(cp.Minimize(0), constraints)
    problem.solve(solver=cp.SCS, verbose=True)

    if problem.status == cp.OPTIMAL:
        spectrum = {h: d.value[i] for i, h in enumerate(h_values)}
        return {
            'status': 'feasible',
            'spectrum': spectrum,
            'dual_certificate': None
        }
    elif problem.status == cp.INFEASIBLE:
        # Extract dual certificate
        dual = constraints[0].dual_value
        return {
            'status': 'infeasible',
            'spectrum': None,
            'dual_certificate': dual
        }
    else:
        return {'status': 'unknown'}
\end{lstlisting}

\textbf{Integer Programming:}


For exact integrality, use branch-and-bound or rounding + verification:


\begin{lstlisting}
from scipy.optimize import milp, LinearConstraint, Bounds

def solve_modular_bootstrap_milp(c: float, gap: float, h_max: float) -> Dict:
    """
    Solve with integer constraints using MILP.
    """
    h_values = np.arange(gap, h_max + 1, 1.0)
    N = len(h_values)

    A, b = setup_modular_constraints(c, gap, h_max, h_values)
    A_real = np.vstack([A.real, A.imag])
    b_real = np.hstack([b.real, b.imag])

    # MILP setup
    constraints = LinearConstraint(A_real, b_real, b_real)
    bounds = Bounds(lb=0, ub=1e6)
    integrality = np.ones(N)  # All variables integer

    result = milp(c=np.zeros(N), constraints=constraints, bounds=bounds,
                  integrality=integrality)

    if result.success:
        spectrum = {h: int(round(result.x[i])) for i, h in enumerate(h_values)}
        return {'status': 'feasible', 'spectrum': spectrum}
    else:
        return {'status': 'infeasible'}
\end{lstlisting}


\subsubsection{Phase 4: Extremal CFT Search at c=24k (Months 4-6)}

\textbf{Goal:} Systematically search for extremal CFTs at c = 48, 72, 96, ...


\textbf{Monster CFT Validation (c=24, k=1):}


\begin{lstlisting}
def test_monster_cft():
    """
    Verify we recover the Monster CFT at c=24.

    Known spectrum:
    d(1) = 0 (gap at Δ=2)
    d(2) = 196884
    d(3) = 21493760
    ...
    """
    c = 24
    gap = 2
    h_max = 10

    result = solve_modular_bootstrap_milp(c, gap, h_max)

    assert result['status'] == 'feasible'

    # Check first few degeneracies
    monster_spectrum = {
        2: 196884,
        3: 21493760,
        4: 864299970
    }

    for h, d_expected in monster_spectrum.items():
        d_computed = result['spectrum'][h]
        assert abs(d_computed - d_expected) < 1
        print(f"d({h}) = {d_computed} (expected {d_expected})")

    print("Monster CFT validation: PASSED")
\end{lstlisting}

\textbf{k=2 Search (c=48):}


\begin{lstlisting}
def search_extremal_c48():
    """
    Search for extremal CFT at c=48 with gap Δ=4.
    """
    c = 48
    gap = 4
    h_max = 20

    print(f"Searching for extremal CFT at c={c}, gap={gap}")

    result = solve_modular_bootstrap_milp(c, gap, h_max)

    if result['status'] == 'feasible':
        print("FOUND FEASIBLE SPECTRUM:")
        for h, d in sorted(result['spectrum'].items()):
            if d > 0.01:  # Only print non-zero
                print(f"  d({h}) = {d}")

        # Verify modular invariance
        tau_samples = [0.1 + 0.5j, 0.2 + 0.8j, 0.5 + 1.0j]
        verified = verify_modular_invariance(c, result['spectrum'], tau_samples)
        print(f"Modular invariance verification: {verified}")

        # Export
        export_spectrum(c, gap, result['spectrum'])

    else:
        print("INFEASIBLE: No extremal CFT exists at this gap")

        # Export dual certificate
        if result['dual_certificate'] is not None:
            export_impossibility_certificate(c, gap, result['dual_certificate'])

    return result
\end{lstlisting}

\textbf{Phase Diagram:}


\begin{lstlisting}
def compute_phase_diagram(k_max: int = 5):
    """
    Map out (c, gap) phase diagram: feasible vs infeasible.
    """
    results = {}

    for k in range(1, k_max + 1):
        c = 24 * k

        for gap in [c/12 - 1, c/12, c/12 + 1]:
            print(f"\nTesting c={c}, gap={gap}")

            result = solve_modular_bootstrap_milp(c, gap, h_max=int(c/2))
            results[(c, gap)] = result['status']

            print(f"Result: {result['status']}")

    # Visualize
    import matplotlib.pyplot as plt

    fig, ax = plt.subplots()

    for (c, gap), status in results.items():
        color = 'green' if status == 'feasible' else 'red'
        ax.scatter(c, gap, c=color, s=100)

    ax.set_xlabel('Central charge c')
    ax.set_ylabel('Gap Δ_gap')
    ax.set_title('Extremal CFT Phase Diagram')
    plt.savefig('phase_diagram.png')

    return results
\end{lstlisting}


\subsubsection{Phase 5: Dual Certificates and Impossibility Proofs (Months 6-8)}

\textbf{Goal:} Extract machine-verifiable certificates when no solution exists.


\textbf{Certificate Extraction:}


\begin{lstlisting}
def extract_dual_certificate(c: float, gap: float, h_max: float) -> Optional[np.ndarray]:
    """
    Solve dual LP to get impossibility certificate.

    Dual problem:
    Maximize: b^T y
    Subject to: A^T y ≤ 0

    If dual is unbounded, primal is infeasible.
    """
    h_values = np.arange(gap, h_max + 1, 1.0)
    A, b = setup_modular_constraints(c, gap, h_max, h_values)

    A_real = np.vstack([A.real, A.imag])
    b_real = np.hstack([b.real, b.imag])

    M = A_real.shape[0]
    y = cp.Variable(M)

    objective = cp.Maximize(b_real @ y)
    constraints = [A_real.T @ y <= 0]

    problem = cp.Problem(objective, constraints)
    problem.solve()

    if problem.status == cp.OPTIMAL and problem.value > 1e-6:
        # Found certificate
        return y.value
    else:
        return None

def verify_dual_certificate(A: np.ndarray, b: np.ndarray, y: np.ndarray) -> bool:
    """
    Verify that y is a valid dual certificate.

    Check:
    1. A^T y ≤ 0
    2. b^T y > 0
    """
    # Check dual feasibility
    dual_slack = A.T @ y
    if not np.all(dual_slack <= 1e-10):
        print("Dual certificate violates feasibility")
        return False

    # Check proves infeasibility
    objective = b.T @ y
    if objective <= -1e-10:
        print(f"Dual objective = {objective} ≤ 0, proves infeasibility")
        return True
    else:
        print(f"Dual objective = {objective} not sufficient")
        return False
\end{lstlisting}

\textbf{Export to SMT-LIB:}


\begin{lstlisting}
def export_certificate_smtlib(c: float, gap: float, y: np.ndarray, filename: str):
    """
    Export dual certificate to SMT-LIB format for independent verification.
    """
    with open(filename, 'w') as f:
        f.write("; Impossibility certificate for extremal CFT\n")
        f.write(f"; c = {c}, gap = {gap}\n")
        f.write("(set-logic QF_LRA)\n\n")

        # Declare dual variables
        for i in range(len(y)):
            f.write(f"(declare-const y{i} Real)\n")

        # Encode A^T y <= 0
        # (implementation details)

        # Encode b^T y > 0
        # (proves infeasibility)

        f.write("(check-sat)\n")
        f.write("(get-model)\n")

    print(f"Certificate exported to {filename}")
\end{lstlisting}


\subsubsection{Phase 6: Formal Verification and Publication (Months 8-12)}

\textbf{Goal:} Formalize results in Lean for machine-checked proofs.


\textbf{Lean Formalization Template:}


\begin{lstlisting}
import Mathlib.Analysis.Complex.Basic
import Mathlib.LinearAlgebra.Matrix.Spectrum

-- Define Virasoro character
def virasoro_character (c h : ℝ) (τ : ℂ) : ℂ := sorry

-- Modular invariance axiom
axiom modular_invariance (c : ℝ) (Z : ℂ → ℂ) :
  (∀ τ, Z τ = Z (-1/τ)) → ModularInvariant Z

-- Extremal CFT theorem
theorem no_extremal_cft_c48_gap4 :
  ∀ (spectrum : ℝ → ℕ),
    (∀ h, 0 < h ∧ h < 4 → spectrum h = 0) →  -- gap condition
    (∀ h, spectrum h ≥ 0) →                   -- unitarity
    ¬(ModularInvariant (partition_function 48 spectrum)) := by
  intro spectrum h_gap h_unit
  -- Proof using dual certificate
  sorry
\end{lstlisting}

\textbf{Certificate Verification Code:}


\begin{lstlisting}
def generate_lean_proof(c: float, gap: float, certificate: np.ndarray, filename: str):
    """
    Generate Lean proof script from dual certificate.
    """
    with open(filename, 'w') as f:
        f.write("-- Impossibility proof for extremal CFT\n")
        f.write(f"-- c = {c}, gap = {gap}\n\n")

        # Define certificate as explicit vector
        f.write("def dual_certificate : Vector ℝ := ")
        f.write(f"![{', '.join(map(str, certificate))}]\n\n")

        # State and prove theorem
        f.write("theorem extremal_cft_impossible :\n")
        f.write(f"  no_extremal_CFT {c} {gap} := by\n")
        f.write("  apply dual_certificate_implies_infeasible\n")
        f.write("  exact dual_certificate\n")
        f.write("  -- Verification steps\n")
        f.write("  sorry\n")

    print(f"Lean proof template written to {filename}")
\end{lstlisting}


\bigskip\hrule\bigskip


\subsection{Example Starting Prompt}

\begin{lstlisting}
I need you to implement a complete modular bootstrap solver for 2D CFTs to search
for extremal theories corresponding to pure AdS₃ gravity. This is a research-level
problem in quantum gravity that requires exact arithmetic and rigorous certificates.

SCIENTIFIC GOAL:
Determine whether extremal CFTs exist at central charge c = 48 (and higher c = 24k)
with only the vacuum operator below a conformal dimension gap Δ_gap ≈ c/12.

For c=24, the Monster CFT is the unique extremal theory. For c=48, existence is unknown.
We will either construct an explicit spectrum or prove impossibility via dual certificates.

IMPLEMENTATION PHASES:

PHASE 1 - Virasoro Characters (Weeks 1-2):
1. Implement dedekind_eta(tau) computing η(τ) = q^{1/24} ∏(1-q^n) to 100+ digit precision
   Use mpmath with mp.dps = 150

2. Implement virasoro_character(c, h, tau) computing χ_h(τ) = q^{h-c/24}/η(τ)

3. Test modular transformation: verify η(-1/τ) = √(-iτ) η(τ) numerically

4. Implement partition_function(c, spectrum, tau) computing Z(τ) = Σ_h d(h) χ_h(τ)

PHASE 2 - Modular Constraints (Weeks 2-4):
5. Compute modular S-matrix S_{h,h'} relating χ_h(-1/τ) = Σ S_{h,h'} χ_{h'}(τ)
   For large c, use asymptotic formula S_{h,h'} ≈ i exp(-2πi√(hh'))

6. Formulate modular invariance Z(τ) = Z(-1/τ) as linear system A @ d = b
   where d = [d(Δ_gap), d(Δ_gap+1), ..., d(h_max)]

7. Export constraint matrix to exact rational arithmetic

PHASE 3 - Linear Programming (Weeks 4-8):
8. Solve LP relaxation: find d ≥ 0 satisfying Ad = b using cvxpy
   Start with c=24, gap=2 to validate against Monster CFT

9. Verify solution: d(2) should equal 196884 for Monster

10. If LP is infeasible, extract dual certificate proving impossibility

PHASE 4 - Integer Programming (Weeks 8-12):
11. Implement integer rounding: round LP solution and verify constraints

12. If rounding fails, use MILP solver (scipy.optimize.milp) with integrality constraints

13. For c=24, confirm exact Monster spectrum:
    d(2) = 196884, d(3) = 21493760, d(4) = 864299970

PHASE 5 - k=2 Search (Weeks 12-16):
14. Apply solver to c=48, gap=4, h_max=20

15. If feasible: export spectrum, verify modular invariance at 10+ random τ points

16. If infeasible: extract and export dual certificate in SMT-LIB format

17. Verify certificate independently using Z3 or external LP solver

PHASE 6 - Classification (Weeks 16-24):
18. Repeat for k=3,4,5 (c=72,96,120)

19. Try multiple gap values for each c

20. Build phase diagram: plot (c, gap) colored by feasible/infeasible

21. Export all certificates to certificates/ directory

TECHNICAL REQUIREMENTS:
- Use mpmath with mp.dps >= 150 for all character computations
- All degeneracies must be exact non-negative integers
- Modular invariance verified to |Z(τ) - Z(-1/τ)| < 10^{-50}
- Export certificates in JSON (spectra) and SMT-LIB (impossibility proofs)
- Every result must include a machine-checkable certificate

SUCCESS CRITERIA:
MINIMUM (3 months): Monster CFT reproduced, one new result for c=48
STRONG (6 months): Complete results for k=2,3,4 with verified certificates
PUBLICATION (12 months): k up to 10, Lean formalization, novel CFTs or no-go theorems

START HERE:
Begin by implementing Phase 1. Write dedekind_eta() and test it against known values.
Then move to virasoro_character(). Do not proceed to Phase 2 until Phase 1 passes
all numerical tests to 100+ digit precision.
\end{lstlisting}


\bigskip\hrule\bigskip


\subsection{Success Criteria}


\subsubsection{Minimum Viable Result (3-4 months)}

✅ \textbf{Infrastructure validated:}

\begin{itemize}
\item Virasoro character calculator accurate to 100+ decimal digits

\item Modular S-transformation verified numerically with error < 10⁻⁵⁰

\item Successfully reproduce Monster CFT spectrum at c=24: d(2)=196884, d(3)=21493760, d(4)=864299970


\end{itemize}

✅ \textbf{One new rigorous result:}

\begin{itemize}
\item Either: Feasible spectrum for c=48 at gap Δ=4 with all degeneracies verified as non-negative integers

\item Or: Dual certificate proving impossibility of extremal CFT at c=48 for gap Δ=4

\item Certificate exported to machine-verifiable format (JSON for spectrum, SMT-LIB for impossibility)



\subsubsection{Strong Result (6-8 months)}

\end{itemize}

✅ \textbf{Multiple cases resolved:}

\begin{itemize}
\item Complete results for k = 2, 3, 4 (c = 48, 72, 96)

\item For each: either explicit spectrum with d(h) verified or impossibility certificate

\item All modular invariance constraints satisfied to machine precision (|Z(τ) - Z(-1/τ)| < 10⁻⁵⁰)


\end{itemize}

✅ \textbf{Rigorous certificates:}

\begin{itemize}
\item All spectra exported with exact integer degeneracies in JSON format

\item All impossibility proofs exported as LP dual certificates in SMT-LIB

\item Independent verification: certificates checked by Z3 or external LP solver

\item Documentation of certificate format and verification procedure


\end{itemize}

✅ \textbf{Phase diagram initiated:}

\begin{itemize}
\item Scan over gaps Δ ∈ [c/12 - 2, c/12 + 2] for each c

\item Phase diagram plotting (c, Δ) with feasible/infeasible regions identified

\item Patterns or bounds on maximum gap for extremality observed



\subsubsection{Publication-Quality Result (9-12 months)}

\end{itemize}

✅ \textbf{Comprehensive classification:}

\begin{itemize}
\item Results for k up to 10 (c up to 240)

\item Refined gap scans: Δ in increments of 0.1 near boundaries

\item Phase diagram revealing clear structure (if any) in (c, Δ) space

\item Database of all extremal CFTs found or impossibility certificates generated


\end{itemize}

✅ \textbf{Formal verification:}

\begin{itemize}
\item Key impossibility theorems formalized in Lean 4 or Isabelle/HOL

\item Dual certificates imported and verified in proof assistant

\item Machine-checked proofs: "No extremal CFT exists at c=48 with gap Δ=4" (or similar)

\item Formal specification of modular invariance and extremality axioms


\end{itemize}

✅ \textbf{Novel insights:}

\begin{itemize}
\item New extremal CFTs discovered beyond Monster (if they exist), or

\item Systematic impossibility theorems: "For c=24k, k>1, no extremal CFT exists for Δ < c/12" (if true)

\item Connections to moonshine: check if any found CFTs exhibit sporadic group symmetries

\item Comparison with holographic bounds: verify extremal CFTs saturate known bounds on gap from AdS₃ gravity


\end{itemize}

✅ \textbf{Publication-ready artifacts:}

\begin{itemize}
\item ArXiv preprint with full results and certificate repository

\item Public database: spectra or impossibility proofs for all (c, Δ) pairs tested

\item Lean formalization repository with verified theorems

\item Reproducible Jupyter notebooks for all computations



\bigskip\hrule\bigskip


\subsection{Verification Protocol}


\subsubsection{Automated Checks}

\end{itemize}

For each claimed result, the following checks must pass:


\textbf{If claiming feasibility (extremal CFT exists):}


\begin{lstlisting}
def verify_extremal_cft(c: float, gap: float, spectrum: Dict[float, int],
                         tau_samples: Optional[List[complex]] = None) -> Dict:
    """
    Comprehensive verification of extremal CFT spectrum.

    Returns:
        Dictionary with verification results and error metrics
    """
    if tau_samples is None:
        # Default: 10 random points in fundamental domain
        tau_samples = [0.05 + 0.5j, 0.2 + 0.8j, 0.5 + 1.0j,
                       -0.4 + 0.7j, 0.3 + 1.2j] * 2

    results = {
        'integrality_passed': True,
        'unitarity_passed': True,
        'gap_passed': True,
        'modular_invariance_passed': True,
        'max_error': 0.0
    }

    # 1. Check integrality
    for h, d in spectrum.items():
        if not isinstance(d, int) or d < 0:
            results['integrality_passed'] = False
            print(f"FAIL: d({h}) = {d} is not a non-negative integer")

    # 2. Check unitarity (redundant with integrality check, but explicit)
    if not all(d >= 0 for d in spectrum.values()):
        results['unitarity_passed'] = False

    # 3. Check gap condition
    if any(0 < h < gap for h in spectrum.keys()):
        results['gap_passed'] = False
        print(f"FAIL: Primaries found in gap (0, {gap})")

    # 4. Check modular invariance
    for tau in tau_samples:
        Z_tau = partition_function(c, spectrum, tau)
        Z_S_tau = partition_function(c, spectrum, -1/tau)

        error = abs(Z_tau - Z_S_tau)
        results['max_error'] = max(results['max_error'], error)

        if error > 1e-30:
            results['modular_invariance_passed'] = False
            print(f"FAIL: Modular invariance at τ={tau}: error={error}")

    # Overall status
    all_passed = all([
        results['integrality_passed'],
        results['unitarity_passed'],
        results['gap_passed'],
        results['modular_invariance_passed']
    ])

    results['status'] = 'VERIFIED' if all_passed else 'FAILED'

    return results

def export_verified_spectrum(c: float, gap: float, spectrum: Dict[float, int],
                               verification: Dict, filename: str):
    """
    Export spectrum with verification metadata to JSON.
    """
    import json
    from datetime import datetime

    data = {
        'central_charge': c,
        'gap': gap,
        'spectrum': {str(h): int(d) for h, d in spectrum.items()},
        'verification': verification,
        'timestamp': datetime.now().isoformat(),
        'precision': f'{mp.dps} decimal digits'
    }

    with open(filename, 'w') as f:
        json.dump(data, f, indent=2)

    print(f"Spectrum exported to {filename}")
\end{lstlisting}

\textbf{If claiming infeasibility:}


\begin{lstlisting}
def verify_impossibility_certificate(c: float, gap: float, h_max: float,
                                      dual_certificate: np.ndarray) -> bool:
    """
    Verify that dual certificate proves infeasibility.

    Certificate y must satisfy:
    1. A^T y ≤ 0  (dual feasibility)
    2. b^T y > 0  (proves primal infeasible via Farkas lemma)

    Returns:
        True if certificate is valid
    """
    h_values = np.arange(gap, h_max + 1, 1.0)
    A, b = setup_modular_constraints(c, gap, h_max, h_values)

    A_real = np.vstack([A.real, A.imag])
    b_real = np.hstack([b.real, b.imag])

    y = dual_certificate

    # Check dual feasibility: A^T y ≤ 0
    dual_slack = A_real.T @ y

    if not np.all(dual_slack <= 1e-8):
        max_violation = np.max(dual_slack)
        print(f"FAIL: Dual feasibility violated, max violation = {max_violation}")
        return False

    # Check proves infeasibility: b^T y > 0
    # (For standard LP, opposite sign convention may apply)
    objective = b_real @ y

    if objective <= -1e-10:
        print(f"SUCCESS: Dual objective = {objective} < 0 proves infeasibility")
        return True
    else:
        print(f"FAIL: Dual objective = {objective} does not prove infeasibility")
        return False

def export_impossibility_certificate(c: float, gap: float, certificate: np.ndarray,
                                       filename: str):
    """
    Export dual certificate to JSON and SMT-LIB formats.
    """
    import json

    # JSON format
    json_file = filename + '.json'
    data = {
        'central_charge': c,
        'gap': gap,
        'result': 'infeasible',
        'dual_certificate': certificate.tolist(),
        'verification': 'Use verify_impossibility_certificate()'
    }

    with open(json_file, 'w') as f:
        json.dump(data, f, indent=2)

    print(f"Certificate exported to {json_file}")

    # SMT-LIB format (for independent verification)
    smtlib_file = filename + '.smt2'
    export_certificate_smtlib(c, gap, certificate, smtlib_file)
\end{lstlisting}


\subsubsection{Human-Reviewable Artifacts}

\begin{itemize}
\item \textbf{Spectrum file} (if feasible): \texttt{spectrum\textit{c{c}}gap{gap}.json}

\begin{lstlisting}
   {
     "central_charge": 48,
     "gap": 4,
     "spectrum": {
       "4": 12345678,
       "5": 98765432,
       "6": 111222333
     },
     "verification": {
       "status": "VERIFIED",
       "max_error": 1.2e-51,
       "integrality_passed": true,
       "modular_invariance_passed": true
     },
     "timestamp": "2026-01-17T10:30:00",
     "precision": "150 decimal digits"
   }
\end{lstlisting}

\item \textbf{Impossibility certificate} (if infeasible): \texttt{certificate\textit{c{c}}gap{gap}.smt2}

\item SMT-LIB format encoding dual feasibility and infeasibility proof

\item Can be verified independently using Z3: \texttt{z3 certificate\textit{c48}gap4.smt2}


\item \textbf{Proof script}: \texttt{proof\textit{c{c}}gap{gap}.lean}

\item Formal Lean 4 proof importing certificate and verifying impossibility

\item Type-checks in Lean to confirm mathematical correctness


\item \textbf{Phase diagram}: \texttt{phase_diagram.png}

\item Plot of (c, Δ) space with color-coded feasible/infeasible regions

\item Includes all tested points with markers



\bigskip\hrule\bigskip


\subsection{Resources & References}


\subsubsection{Essential Papers}

\item \textbf{Hartman, Keller, Stoica (2014):} "Universal Spectrum of 2d Conformal Field Theory in the Large c Limit"

\end{itemize}

   [arXiv:1405.5137] - Establishes universal bounds on spectrum using modular bootstrap at large c


\begin{itemize}
\item \textbf{Afkhami-Jeddi, Cohn, Hartman, Tajdini (2020):} "Free Partition Functions and an Averaged Holographic Duality"

\end{itemize}

   [arXiv:2006.04839] - Uses modular bootstrap to constrain averaged CFT ensembles


\begin{itemize}
\item \textbf{Collier, Lin, Yin (2019):} "Modular Bootstrap Revisited"

\end{itemize}

   [arXiv:1608.06241] - Systematic treatment of modular constraints for rational CFTs


\begin{itemize}
\item \textbf{Hellerman (2011):} "A Universal Inequality for CFT and Quantum Gravity"

\end{itemize}

   [arXiv:0902.2790] - Lower bound on gap using modular invariance


\begin{itemize}
\item \textbf{Friedan, Keller, Yin (2013):} "A Remark on AdS/CFT for the Extremal Virasoro Partition Function"

\end{itemize}

   [arXiv:1312.1536] - Analysis of extremal CFTs and connections to AdS₃ gravity



\subsubsection{Code Libraries}

\begin{itemize}
\item \textbf{mpmath:} Arbitrary precision arithmetic in Python - \texttt{pip install mpmath}

\item \textbf{SymPy:} Symbolic mathematics - \texttt{pip install sympy}

\item \textbf{CVXPY:} Convex optimization with SDP/LP solvers - \texttt{pip install cvxpy}

\item \textbf{SciPy:} Scientific computing including MILP solvers - \texttt{pip install scipy}

\item \textbf{Lean 4:} Proof assistant for formal verification - https://lean-lang.org



\subsubsection{Mathematical Background}

\item \textbf{Modular Forms:} Serre's "A Course in Arithmetic", Diamond & Shurman "A First Course in Modular Forms"

\item \textbf{Virasoro Algebra:} Di Francesco et al. "Conformal Field Theory" (Yellow Book)

\item \textbf{Optimization:} Boyd & Vandenberghe "Convex Optimization" (Chapter on LP duality)

\item \textbf{AdS/CFT:} Aharony et al. "Large N Field Theories, String Theory and Gravity" [arXiv:hep-th/9905111]



\subsubsection{Key Concepts to Master}

\item \textbf{Dedekind eta function} and its modular transformation properties

\item \textbf{Virasoro minimal models} and character formulas

\item \textbf{Modular group} PSL(2,ℤ) and fundamental domain

\item \textbf{Linear programming duality} and Farkas lemma (for impossibility certificates)

\item \textbf{q-series} and asymptotic expansions for characters

\item \textbf{Partition functions} and genus expansion in CFT



\bigskip\hrule\bigskip


\subsection{Common Pitfalls & How to Avoid Them}


\subsubsection{Numerical Precision Issues}
\end{itemize}

❌ \textbf{Problem:} Modular invariance appears satisfied due to rounding errors; false positives in feasibility

✅ \textbf{Solution:}

\begin{itemize}
\item Use \texttt{mpmath} with \texttt{mp.dps = 150} or higher for all character computations

\item Verify modular invariance to at least 50 decimal digits: |Z(τ) - Z(-1/τ)| < 10⁻⁵⁰

\item Cross-check with multiple τ points in fundamental domain



\subsubsection{Fake Spectra from LP Relaxation}
\end{itemize}

❌ \textbf{Problem:} LP solution has d(h) = 123.7 (non-integer) accepted as valid

✅ \textbf{Solution:}

\begin{itemize}
\item Always enforce strict integrality using MILP or branch-and-bound

\item Round and verify: if rounding LP solution, check all constraints still satisfied

\item Export only exact integer degeneracies



\subsubsection{Incomplete or Invalid Dual Certificates}
\end{itemize}

❌ \textbf{Problem:} Dual certificate extracted but doesn't rigorously prove infeasibility

✅ \textbf{Solution:}

\begin{itemize}
\item Verify certificate satisfies A^T y ≤ 0 and b^T y > 0 (or appropriate sign convention)

\item Use exact rational arithmetic for certificate verification

\item Export to SMT-LIB and verify independently with Z3



\subsubsection{Truncation Effects in Spectrum}
\end{itemize}

❌ \textbf{Problem:} Setting h_max too small misses important high-dimension operators

✅ \textbf{Solution:}

\begin{itemize}
\item Start with h_max = c (usually sufficient for extremal CFTs)

\item Check sensitivity: increase h_max and verify solution doesn't change

\item Use asymptotic bounds on degeneracies to justify truncation



\subsubsection{S-Matrix Approximation Errors}
\end{itemize}

❌ \textbf{Problem:} Using asymptotic S-matrix formula S_{h,h'} ≈ i exp(-2πi√(hh')) introduces errors

✅ \textbf{Solution:}

\begin{itemize}
\item Compute S-matrix exactly by evaluating characters at multiple τ points

\item For large c, asymptotic formula is accurate; verify error < 10⁻¹⁰

\item Cross-check S-matrix satisfies unitarity: S S† = I



\subsubsection{Confusing Modular S and T Transformations}
\end{itemize}

❌ \textbf{Problem:} Implementing only S: τ → -1/τ but ignoring T: τ → τ+1

✅ \textbf{Solution:}

\begin{itemize}
\item S and T generate full modular group; both must be satisfied

\item For extremal CFTs, T-invariance is automatic (weights in ℤ), but verify explicitly

\item Test full modular orbit: check Z(γτ) = Z(τ) for multiple γ ∈ PSL(2,ℤ)



\bigskip\hrule\bigskip


\subsection{Milestone Checklist}

\end{itemize}

\textbf{Infrastructure (Months 1-2):}

\begin{itemize}
\item [ ] Dedekind eta function η(τ) implemented with 100+ digit precision

\item [ ] Modular transformation η(-1/τ) = √(-iτ) η(τ) verified numerically

\item [ ] Virasoro character χ_h(τ) calculator tested on known values

\item [ ] Partition function Z(τ) builder with arbitrary spectrum input

\item [ ] Modular S-matrix S_{h,h'} computed and verified for unitarity


\end{itemize}

\textbf{Validation (Month 2):}

\begin{itemize}
\item [ ] Monster CFT (c=24, gap=2) spectrum reproduced exactly:

\item [ ] d(2) = 196884 ✓

\item [ ] d(3) = 21493760 ✓

\item [ ] d(4) = 864299970 ✓

\item [ ] Modular invariance of Monster partition function verified to 10⁻⁵⁰


\end{itemize}

\textbf{Optimization Solvers (Months 2-3):}

\begin{itemize}
\item [ ] LP relaxation solver (cvxpy) working and tested

\item [ ] MILP solver (scipy.optimize.milp) enforcing integrality

\item [ ] Dual certificate extraction from infeasible LP implemented

\item [ ] Certificate verification functions tested


\end{itemize}

\textbf{New Results (Months 3-6):}

\begin{itemize}
\item [ ] c=48, gap=4: Result obtained (feasible or impossibility certificate)

\item [ ] c=48 result verified independently

\item [ ] c=72, gap=6: Result obtained

\item [ ] c=96, gap=8: Result obtained

\item [ ] All certificates exported to JSON/SMT-LIB formats


\end{itemize}

\textbf{Classification (Months 6-9):}

\begin{itemize}
\item [ ] Phase diagram for k=1 to 5 complete

\item [ ] Gap scan: multiple Δ values tested for each c

\item [ ] Database of all spectra and impossibility proofs assembled

\item [ ] Visualization: (c, Δ) phase diagram plotted


\end{itemize}

\textbf{Formal Verification (Months 9-12):}

\begin{itemize}
\item [ ] Lean 4 formalization of modular invariance begun

\item [ ] First impossibility theorem formalized and type-checked in Lean

\item [ ] All dual certificates imported and verified in proof assistant

\item [ ] Publication draft with machine-checkable proofs prepared


\end{itemize}

\textbf{Publication (Month 12):}

\begin{itemize}
\item [ ] ArXiv preprint submitted with full results

\item [ ] Public repository with all certificates and verification code

\item [ ] Reproducible Jupyter notebooks for all computations

\item [ ] All certificates publicly available and independently verified



\bigskip\hrule\bigskip

\end{itemize}

\textbf{Next Steps:}


Begin with Phase 1 infrastructure. Implement \texttt{dedekind\textit{eta()} and test it against known values (e.g., η(i) = π^{1/4} / Γ(1/4)^{1/2}). Then implement \texttt{virasoro}character()} and verify numerical values against published tables. Do \textbf{not} proceed to Phase 2 until Phase 1 passes all tests to 100+ digit precision. Validate thoroughly on the Monster CFT (k=1) before attempting any new cases (k≥2).


Focus on building robust, high-precision code with comprehensive testing. Every result must be accompanied by a machine-verifiable certificate—either an explicit spectrum or a dual certificate proving impossibility.


\end{document}
