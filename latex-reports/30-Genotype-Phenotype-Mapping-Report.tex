%% =============================================================================
%% GENOTYPE-PHENOTYPE MAPPING AND EVOLUTIONARY LANDSCAPES
%% Comprehensive Technical Report
%% =============================================================================

\documentclass[11pt,a4paper]{article}

%% =============================================================================
%% PACKAGE IMPORTS
%% =============================================================================
\usepackage[utf8]{inputenc}
\usepackage[T1]{fontenc}
\usepackage{amsmath,amssymb,amsthm}
\usepackage{mathtools}
\usepackage{physics}
\usepackage{graphicx}
\usepackage{xcolor}
\usepackage{tcolorbox}
\usepackage{listings}
\usepackage{algorithm}
\usepackage{algpseudocode}
\usepackage{booktabs}
\usepackage{hyperref}
\usepackage{cleveref}
\usepackage{geometry}
\usepackage{fancyhdr}
\usepackage{titlesec}
\usepackage{enumitem}
\usepackage{caption}
\usepackage{subcaption}
\usepackage{float}

\geometry{margin=1in}

%% =============================================================================
%% TCOLORBOX ENVIRONMENT DEFINITIONS
%% =============================================================================
\tcbuselibrary{skins,breakable}

% Annotation box (gray) - for side notes and clarifications
\newtcolorbox{annotation}{
    colback=gray!10,
    colframe=gray!50,
    coltitle=black,
    title=\textbf{Annotation},
    fonttitle=\bfseries,
    breakable,
    enhanced,
    boxrule=1pt,
    left=5pt,
    right=5pt,
    top=5pt,
    bottom=5pt
}

% Pursuit box (green/yellow) - for key insights and goals
\newtcolorbox{pursuitbox}{
    colback=green!5,
    colframe=green!50!yellow!80,
    coltitle=black,
    title=\textbf{Key Pursuit},
    fonttitle=\bfseries,
    breakable,
    enhanced,
    boxrule=1.5pt,
    left=5pt,
    right=5pt,
    top=5pt,
    bottom=5pt
}

% Warning box (red) - for cautions and critical notes
\newtcolorbox{warningbox}{
    colback=red!5,
    colframe=red!70,
    coltitle=white,
    title=\textbf{Warning},
    fonttitle=\bfseries,
    breakable,
    enhanced,
    boxrule=1.5pt,
    left=5pt,
    right=5pt,
    top=5pt,
    bottom=5pt
}

% Physics box (purple) - for mathematical and physical derivations
\newtcolorbox{physicsbox}{
    colback=purple!5,
    colframe=purple!70,
    coltitle=white,
    title=\textbf{Mathematical Derivation},
    fonttitle=\bfseries,
    breakable,
    enhanced,
    boxrule=1.5pt,
    left=5pt,
    right=5pt,
    top=5pt,
    bottom=5pt
}

%% =============================================================================
%% LISTINGS CONFIGURATION FOR CODE
%% =============================================================================
\lstdefinestyle{pythonstyle}{
    language=Python,
    basicstyle=\ttfamily\small,
    keywordstyle=\color{blue}\bfseries,
    stringstyle=\color{red},
    commentstyle=\color{green!60!black}\itshape,
    numberstyle=\tiny\color{gray},
    numbers=left,
    numbersep=5pt,
    breaklines=true,
    showstringspaces=false,
    frame=single,
    framesep=5pt,
    backgroundcolor=\color{gray!5},
    rulecolor=\color{gray!50},
    tabsize=4,
    captionpos=b,
    morekeywords={self,True,False,None,as,with,yield,async,await}
}

\lstset{style=pythonstyle}

%% =============================================================================
%% THEOREM ENVIRONMENTS
%% =============================================================================
\newtheorem{theorem}{Theorem}[section]
\newtheorem{lemma}[theorem]{Lemma}
\newtheorem{proposition}[theorem]{Proposition}
\newtheorem{corollary}[theorem]{Corollary}
\newtheorem{definition}[theorem]{Definition}
\newtheorem{example}[theorem]{Example}
\newtheorem{remark}[theorem]{Remark}

%% =============================================================================
%% CUSTOM COMMANDS
%% =============================================================================
\newcommand{\genotype}{\mathcal{G}}
\newcommand{\phenotype}{\mathcal{P}}
\newcommand{\gpmap}{\Phi}
\newcommand{\neutralnet}{\mathcal{N}}
\newcommand{\fitness}{\mathcal{F}}
\newcommand{\alphabet}{\Sigma}
\newcommand{\seqlength}{L}
\newcommand{\hamming}{d_H}
\newcommand{\expect}{\mathbb{E}}
\newcommand{\prob}{\mathbb{P}}
\newcommand{\real}{\mathbb{R}}
\newcommand{\nat}{\mathbb{N}}
\newcommand{\complexity}{\mathcal{O}}

%% =============================================================================
%% DOCUMENT METADATA
%% =============================================================================
\title{
    \textbf{Genotype-Phenotype Mapping and Evolutionary Landscapes}\\
    \large A Comprehensive Technical Analysis
}
\author{
    Pure Thought AI Research Initiative\\
    \texttt{research@purethought.ai}
}
\date{\today}

%% =============================================================================
%% DOCUMENT BEGIN
%% =============================================================================
\begin{document}

\maketitle

\begin{abstract}
This report provides a comprehensive examination of genotype-phenotype (GP) mapping
and evolutionary fitness landscapes. We explore the mathematical foundations of GP
maps, with particular emphasis on RNA secondary structure as a canonical model system.
The report covers fundamental algorithms including the Nussinov algorithm for maximizing
base pairs and the Zuker algorithm for minimizing free energy. We analyze neutral
networks, their percolation properties, and implications for evolutionary dynamics.
Fitness landscape theory is developed including measures of ruggedness, autocorrelation
functions, and epistasis coefficients. The NK model framework for tunable ruggedness
is presented alongside classical population genetics models (Wright-Fisher and Moran).
We conclude with formal verification protocols and Python implementations for all
major computational components.
\end{abstract}

\tableofcontents
\newpage

%% =============================================================================
%% SECTION 1: INTRODUCTION AND FUNDAMENTALS
%% =============================================================================
\section{Introduction to Genotype-Phenotype Mapping}

\subsection{Foundational Concepts}

The relationship between genotype and phenotype lies at the heart of evolutionary
biology. The \textbf{genotype-phenotype map} (GP map) describes how genetic information
is translated into observable characteristics.

\begin{definition}[Genotype-Phenotype Map]
A genotype-phenotype map is a function $\gpmap: \genotype \to \phenotype$ where
$\genotype = \alphabet^{\seqlength}$ is the genotype space (sequences of length
$\seqlength$ over alphabet $\alphabet$) and $\phenotype$ is the phenotype space.
\end{definition}

\begin{pursuitbox}
The central pursuit in GP mapping research is understanding how the structure of
the map $\gpmap$ influences evolutionary dynamics, particularly:
\begin{itemize}
    \item The accessibility of phenotypes from different genotypes
    \item The existence and structure of neutral networks
    \item The relationship between robustness and evolvability
\end{itemize}
\end{pursuitbox}

\subsection{Mathematical Formalization}

Let us formalize the key mathematical structures underlying GP mapping.

\begin{physicsbox}
The genotype space $\genotype = \alphabet^{\seqlength}$ has cardinality:
\begin{equation}
    |\genotype| = |\alphabet|^{\seqlength}
\end{equation}
For RNA sequences with $\alphabet = \{A, U, G, C\}$ and typical lengths
$\seqlength \sim 100$, we have:
\begin{equation}
    |\genotype| = 4^{100} \approx 10^{60}
\end{equation}
This astronomical size makes exhaustive enumeration impossible, necessitating
statistical approaches and efficient algorithms.
\end{physicsbox}

\subsubsection{Hamming Distance and Sequence Space}

The natural metric on genotype space is the Hamming distance.

\begin{definition}[Hamming Distance]
For sequences $s, s' \in \alphabet^{\seqlength}$, the Hamming distance is:
\begin{equation}
    \hamming(s, s') = \sum_{i=1}^{\seqlength} \mathbb{1}[s_i \neq s'_i]
\end{equation}
where $\mathbb{1}[\cdot]$ is the indicator function.
\end{definition}

\begin{annotation}
The Hamming distance counts the number of positions at which two sequences differ.
It satisfies metric axioms: non-negativity, identity of indiscernibles, symmetry,
and triangle inequality. The maximum distance between any two sequences is $\seqlength$.
\end{annotation}

The sequence space forms a \textbf{Hamming graph} where vertices are sequences and
edges connect sequences at Hamming distance 1.

\begin{theorem}[Connectivity of Hamming Graph]
The Hamming graph $H(\seqlength, |\alphabet|)$ is connected. Any two sequences
$s, s' \in \alphabet^{\seqlength}$ can be connected by a path of length at most
$\seqlength$.
\end{theorem}

\begin{proof}
Given sequences $s = (s_1, \ldots, s_{\seqlength})$ and $s' = (s'_1, \ldots, s'_{\seqlength})$,
construct the path by sequentially changing each position $i$ from $s_i$ to $s'_i$.
Each step changes exactly one position, so each edge in the path has Hamming distance 1.
The path length equals $\hamming(s, s')$, which is at most $\seqlength$.
\end{proof}

\subsection{Properties of GP Maps}

\begin{definition}[Degeneracy]
A GP map $\gpmap$ exhibits \textbf{degeneracy} when multiple genotypes map to the
same phenotype:
\begin{equation}
    \exists s, s' \in \genotype: s \neq s' \land \gpmap(s) = \gpmap(s')
\end{equation}
\end{definition}

\begin{definition}[Phenotype Frequency]
The frequency of phenotype $\phi \in \phenotype$ is:
\begin{equation}
    f(\phi) = \frac{|\gpmap^{-1}(\phi)|}{|\genotype|} = \frac{|\{s \in \genotype : \gpmap(s) = \phi\}|}{|\alphabet|^{\seqlength}}
\end{equation}
\end{definition}

\begin{warningbox}
Phenotype frequencies in real GP maps are highly non-uniform. In RNA folding,
a small number of ``common'' structures have exponentially more sequences folding
to them than ``rare'' structures. This bias strongly affects evolutionary search.
\end{warningbox}

%% =============================================================================
%% SECTION 2: RNA SECONDARY STRUCTURE AS MODEL GP SYSTEM
%% =============================================================================
\section{RNA Secondary Structure as Model GP System}

\subsection{Why RNA?}

RNA secondary structure prediction provides an ideal model GP system because:
\begin{enumerate}
    \item The genotype space is well-defined: $\genotype = \{A, U, G, C\}^{\seqlength}$
    \item The phenotype (secondary structure) is computationally tractable
    \item The folding map can be computed efficiently via dynamic programming
    \item It captures essential GP map properties found in more complex systems
\end{enumerate}

\begin{pursuitbox}
RNA folding exemplifies the key principle that the GP map creates structure in
evolutionary search space. Understanding this structure enables predictions about
evolutionary dynamics that would be impossible from sequence analysis alone.
\end{pursuitbox}

\subsection{RNA Secondary Structure Formalism}

\begin{definition}[RNA Secondary Structure]
An RNA secondary structure $S$ on a sequence of length $\seqlength$ is a set of
base pairs $(i, j)$ with $1 \leq i < j \leq \seqlength$ satisfying:
\begin{enumerate}
    \item \textbf{No sharp hairpins}: $j - i \geq 4$ (minimum loop size)
    \item \textbf{No base triples}: Each position participates in at most one pair
    \item \textbf{No pseudoknots}: For pairs $(i,j)$ and $(k,l)$ with $i < k$,
          either $j < k$ or $l < j$ (no crossing pairs)
\end{enumerate}
\end{definition}

\begin{physicsbox}
The number of valid secondary structures grows asymptotically as:
\begin{equation}
    N(\seqlength) \sim \frac{1.104366}{\seqlength^{3/2}} \cdot 1.84892^{\seqlength}
\end{equation}
This result, derived by Stein and Waterman (1978), shows exponential growth but
much slower than the $4^{\seqlength}$ growth of sequence space, indicating massive
degeneracy in the GP map.
\end{physicsbox}

\subsection{Base Pairing Rules}

Watson-Crick base pairs form between complementary bases:
\begin{itemize}
    \item Adenine (A) pairs with Uracil (U)
    \item Guanine (G) pairs with Cytosine (C)
    \item G-U ``wobble'' pairs are also allowed in some models
\end{itemize}

\begin{definition}[Complementarity Function]
Define the complementarity function $\delta: \alphabet \times \alphabet \to \{0, 1\}$:
\begin{equation}
    \delta(a, b) = \begin{cases}
        1 & \text{if } (a,b) \in \{(A,U), (U,A), (G,C), (C,G)\} \\
        0 & \text{otherwise}
    \end{cases}
\end{equation}
For models including wobble pairs:
\begin{equation}
    \delta_{wc+gu}(a, b) = \begin{cases}
        1 & \text{if } (a,b) \in \{(A,U), (U,A), (G,C), (C,G), (G,U), (U,G)\} \\
        0 & \text{otherwise}
    \end{cases}
\end{equation}
\end{definition}

%% =============================================================================
%% SECTION 3: NUSSINOV ALGORITHM
%% =============================================================================
\section{The Nussinov Algorithm: Maximizing Base Pairs}

\subsection{Problem Formulation}

The Nussinov algorithm (1978) finds the secondary structure with maximum base pairs.

\begin{definition}[Maximum Base Pairing Problem]
Given an RNA sequence $s = s_1 s_2 \ldots s_{\seqlength}$, find the secondary
structure $S^*$ that maximizes:
\begin{equation}
    S^* = \arg\max_{S \in \mathcal{S}} |S|
\end{equation}
where $\mathcal{S}$ is the set of valid secondary structures and $|S|$ is the
number of base pairs.
\end{definition}

\subsection{Dynamic Programming Formulation}

\begin{physicsbox}
Define $M(i, j)$ as the maximum number of base pairs in the subsequence $s_i \ldots s_j$.

\textbf{Base case:}
\begin{equation}
    M(i, j) = 0 \quad \text{for } j - i < 4
\end{equation}

\textbf{Recurrence relation:}
\begin{equation}
    M(i, j) = \max \begin{cases}
        M(i, j-1) & \text{(j unpaired)} \\
        \max_{i \leq k < j-4} \{M(i, k-1) + M(k+1, j-1) + \delta(s_k, s_j)\} & \text{(j pairs with k)}
    \end{cases}
\end{equation}

The optimal solution is $M(1, \seqlength)$.
\end{physicsbox}

\begin{theorem}[Nussinov Complexity]
The Nussinov algorithm has time complexity $\complexity(\seqlength^3)$ and space
complexity $\complexity(\seqlength^2)$.
\end{theorem}

\begin{proof}
The DP table has $\complexity(\seqlength^2)$ entries. Each entry requires examining
$\complexity(\seqlength)$ possible pairing partners. Thus, total time is
$\complexity(\seqlength^3)$. Space is dominated by storing the $\seqlength \times \seqlength$
matrix, giving $\complexity(\seqlength^2)$.
\end{proof}

\subsection{Algorithm Implementation}

\begin{lstlisting}[caption={Nussinov Algorithm for Maximum Base Pairing}]
import numpy as np
from typing import List, Tuple, Set

def nussinov_fold(sequence: str, min_loop_size: int = 4) -> Tuple[int, Set[Tuple[int, int]]]:
    """
    Compute maximum base pairing using Nussinov algorithm.

    Args:
        sequence: RNA sequence string (A, U, G, C)
        min_loop_size: Minimum number of unpaired bases in hairpin loop

    Returns:
        Tuple of (max_base_pairs, set_of_pairs)
    """
    n = len(sequence)

    # Complementarity check
    def can_pair(a: str, b: str) -> bool:
        pairs = {('A', 'U'), ('U', 'A'), ('G', 'C'), ('C', 'G')}
        return (a, b) in pairs

    # Initialize DP table
    M = np.zeros((n, n), dtype=int)

    # Fill table diagonally
    for length in range(min_loop_size + 1, n):
        for i in range(n - length):
            j = i + length

            # Case 1: j is unpaired
            M[i, j] = M[i, j - 1]

            # Case 2: j pairs with some k
            for k in range(i, j - min_loop_size):
                if can_pair(sequence[k], sequence[j]):
                    score = 1  # Base pair contribution
                    if k > i:
                        score += M[i, k - 1]
                    if k + 1 < j:
                        score += M[k + 1, j - 1]
                    M[i, j] = max(M[i, j], score)

    # Traceback to recover structure
    def traceback(i: int, j: int) -> Set[Tuple[int, int]]:
        if j - i < min_loop_size:
            return set()

        pairs = set()

        # Check if j is unpaired
        if M[i, j] == M[i, j - 1]:
            return traceback(i, j - 1)

        # Find pairing partner
        for k in range(i, j - min_loop_size):
            if can_pair(sequence[k], sequence[j]):
                score = 1
                left_score = M[i, k - 1] if k > i else 0
                right_score = M[k + 1, j - 1] if k + 1 < j else 0

                if M[i, j] == score + left_score + right_score:
                    pairs.add((k, j))
                    if k > i:
                        pairs.update(traceback(i, k - 1))
                    if k + 1 < j:
                        pairs.update(traceback(k + 1, j - 1))
                    return pairs

        return pairs

    base_pairs = traceback(0, n - 1)
    return M[0, n - 1], base_pairs


def structure_to_dot_bracket(length: int, pairs: Set[Tuple[int, int]]) -> str:
    """Convert base pair set to dot-bracket notation."""
    structure = ['.'] * length
    for i, j in pairs:
        structure[i] = '('
        structure[j] = ')'
    return ''.join(structure)


# Example usage
if __name__ == "__main__":
    sequence = "GGGAAAUCC"
    max_pairs, pairs = nussinov_fold(sequence)
    structure = structure_to_dot_bracket(len(sequence), pairs)

    print(f"Sequence:  {sequence}")
    print(f"Structure: {structure}")
    print(f"Max pairs: {max_pairs}")
    print(f"Pairs: {pairs}")
\end{lstlisting}

\begin{annotation}
The Nussinov algorithm provides optimal solutions for the maximum base pairing
problem but does not account for thermodynamic stability. Real RNA structures
minimize free energy rather than maximize base pairs, leading to the more
sophisticated Zuker algorithm.
\end{annotation}

%% =============================================================================
%% SECTION 4: ZUKER ALGORITHM
%% =============================================================================
\section{The Zuker Algorithm: Minimizing Free Energy}

\subsection{Thermodynamic Model}

The minimum free energy (MFE) approach models RNA folding as a thermodynamic
process seeking the lowest energy configuration.

\begin{definition}[Free Energy of Secondary Structure]
The free energy $\Delta G(S)$ of a secondary structure $S$ decomposes into
contributions from individual loops:
\begin{equation}
    \Delta G(S) = \sum_{\ell \in \text{loops}(S)} \Delta G(\ell)
\end{equation}
\end{definition}

\begin{physicsbox}
\textbf{Loop Energy Contributions:}

\textbf{1. Stacking pairs:}
\begin{equation}
    \Delta G_{\text{stack}}(i,j; i+1,j-1) = \varepsilon_{\text{stack}}(s_i s_{i+1}, s_{j-1} s_j)
\end{equation}

\textbf{2. Hairpin loops:}
\begin{equation}
    \Delta G_{\text{hairpin}}(i,j) = a + b \cdot n + c \cdot \text{special}
\end{equation}
where $n = j - i - 1$ is the loop size.

\textbf{3. Interior loops:}
\begin{equation}
    \Delta G_{\text{interior}}(i,j; i',j') = f(n_1, n_2) + \text{mismatch terms}
\end{equation}
where $n_1 = i' - i - 1$ and $n_2 = j - j' - 1$.

\textbf{4. Bulge loops:}
\begin{equation}
    \Delta G_{\text{bulge}}(n) = a + b \cdot \ln(n)
\end{equation}

\textbf{5. Multi-branch loops:}
\begin{equation}
    \Delta G_{\text{multi}} = a + b \cdot k + c \cdot n
\end{equation}
where $k$ is the number of branches and $n$ is the number of unpaired bases.
\end{physicsbox}

\subsection{Zuker Recursion}

The Zuker algorithm uses multiple interrelated DP tables.

\begin{definition}[Zuker DP Variables]
\begin{itemize}
    \item $W(i)$: MFE of subsequence $s_1 \ldots s_i$
    \item $V(i,j)$: MFE of subsequence $s_i \ldots s_j$ where $(i,j)$ forms a base pair
    \item $VBI(i,j)$: MFE when $(i,j)$ closes an internal/bulge loop
    \item $VM(i,j)$: MFE when $(i,j)$ closes a multi-loop
\end{itemize}
\end{definition}

\begin{physicsbox}
\textbf{Zuker Recurrence Relations:}

\textbf{1. External loop:}
\begin{equation}
    W(j) = \min \begin{cases}
        W(j-1) \\
        \min_{1 \leq i < j} \{W(i-1) + V(i,j)\}
    \end{cases}
\end{equation}

\textbf{2. Closed pair:}
\begin{equation}
    V(i,j) = \min \begin{cases}
        E_{\text{hairpin}}(i,j) \\
        \min_{i<i'<j'<j} \{E_{\text{stack/interior/bulge}}(i,j;i',j') + V(i',j')\} \\
        VM(i,j)
    \end{cases}
\end{equation}

\textbf{3. Multi-loop:}
\begin{equation}
    VM(i,j) = \min_{i<k<j} \{VM'(i,k) + VM'(k+1,j)\} + a
\end{equation}
where $VM'$ accounts for the multi-loop decomposition.
\end{physicsbox}

\begin{theorem}[Zuker Complexity]
The standard Zuker algorithm has time complexity $\complexity(\seqlength^4)$ and
space complexity $\complexity(\seqlength^2)$. With Lyngsø optimizations for
internal loops, time reduces to $\complexity(\seqlength^3)$.
\end{theorem}

\subsection{Implementation}

\begin{lstlisting}[caption={Simplified Zuker Algorithm Implementation}]
import numpy as np
from typing import Dict, Tuple, Set
from dataclasses import dataclass

@dataclass
class EnergyParameters:
    """Turner energy parameters (simplified)."""
    stack: Dict[str, float]  # Stacking energies
    hairpin_init: float = 4.1  # Hairpin initiation
    hairpin_slope: float = 1.75  # Per unpaired base
    interior_init: float = 1.7
    bulge_init: float = 3.8
    multi_init: float = 3.4
    multi_branch: float = 0.4
    multi_unpaired: float = 0.0

    @staticmethod
    def default() -> 'EnergyParameters':
        """Return default Turner-like parameters."""
        stack = {
            'AU_AU': -0.9, 'AU_CG': -2.2, 'AU_GC': -2.1, 'AU_GU': -0.6,
            'AU_UA': -1.1, 'AU_UG': -1.4, 'CG_AU': -2.1, 'CG_CG': -3.3,
            'CG_GC': -2.4, 'CG_GU': -1.4, 'CG_UA': -2.1, 'CG_UG': -2.1,
            'GC_AU': -2.4, 'GC_CG': -3.4, 'GC_GC': -3.3, 'GC_GU': -1.5,
            'GC_UA': -2.2, 'GC_UG': -2.5, 'GU_AU': -1.3, 'GU_CG': -2.5,
            'GU_GC': -2.1, 'GU_GU': -0.5, 'GU_UA': -0.6, 'GU_UG': 1.3,
            'UA_AU': -1.3, 'UA_CG': -2.4, 'UA_GC': -2.1, 'UA_GU': -1.0,
            'UA_UA': -0.9, 'UA_UG': -1.3, 'UG_AU': -1.0, 'UG_CG': -1.5,
            'UG_GC': -1.4, 'UG_GU': 0.3, 'UG_UA': -0.6, 'UG_UG': -0.5,
        }
        return EnergyParameters(stack=stack)


def zuker_fold(sequence: str, params: EnergyParameters = None) -> Tuple[float, Set[Tuple[int, int]]]:
    """
    Compute minimum free energy structure using Zuker algorithm.

    Args:
        sequence: RNA sequence string
        params: Energy parameters (default Turner parameters)

    Returns:
        Tuple of (MFE in kcal/mol, set of base pairs)
    """
    if params is None:
        params = EnergyParameters.default()

    n = len(sequence)
    INF = float('inf')
    MIN_LOOP = 3

    # Base pair check
    def can_pair(i: int, j: int) -> bool:
        pairs = {('A', 'U'), ('U', 'A'), ('G', 'C'), ('C', 'G'),
                 ('G', 'U'), ('U', 'G')}
        return (sequence[i], sequence[j]) in pairs

    def pair_type(i: int, j: int) -> str:
        return f"{sequence[i]}{sequence[j]}"

    def stack_energy(i: int, j: int, ip: int, jp: int) -> float:
        key = f"{pair_type(i,j)}_{pair_type(ip,jp)}"
        return params.stack.get(key, 0.0)

    def hairpin_energy(i: int, j: int) -> float:
        size = j - i - 1
        return params.hairpin_init + params.hairpin_slope * np.log(size + 1)

    def interior_energy(i: int, j: int, ip: int, jp: int) -> float:
        n1 = ip - i - 1
        n2 = j - jp - 1
        size = n1 + n2
        asym = abs(n1 - n2)
        return params.interior_init + 0.3 * size + 0.5 * asym

    def bulge_energy(size: int) -> float:
        return params.bulge_init + 1.75 * np.log(size + 1)

    # Initialize DP tables
    V = np.full((n, n), INF)
    W = np.zeros(n)

    # Fill V table
    for length in range(MIN_LOOP + 2, n):
        for i in range(n - length):
            j = i + length

            if not can_pair(i, j):
                continue

            # Hairpin loop
            V[i, j] = hairpin_energy(i, j)

            # Stacking / interior / bulge
            for ip in range(i + 1, j - MIN_LOOP):
                for jp in range(ip + MIN_LOOP + 1, j):
                    if not can_pair(ip, jp):
                        continue

                    n1 = ip - i - 1
                    n2 = j - jp - 1

                    if n1 == 0 and n2 == 0:
                        # Stacking
                        energy = stack_energy(i, j, ip, jp) + V[ip, jp]
                    elif n1 == 0 or n2 == 0:
                        # Bulge
                        energy = bulge_energy(n1 + n2) + V[ip, jp]
                    else:
                        # Interior
                        energy = interior_energy(i, j, ip, jp) + V[ip, jp]

                    V[i, j] = min(V[i, j], energy)

    # Fill W table
    for j in range(n):
        W[j] = W[j - 1] if j > 0 else 0

        for i in range(j - MIN_LOOP - 1):
            if V[i, j] < INF:
                prev = W[i - 1] if i > 0 else 0
                W[j] = min(W[j], prev + V[i, j])

    # Traceback
    def traceback_W(j: int) -> Set[Tuple[int, int]]:
        if j < 0:
            return set()

        if j > 0 and abs(W[j] - W[j - 1]) < 1e-6:
            return traceback_W(j - 1)

        for i in range(j - MIN_LOOP - 1):
            if V[i, j] < INF:
                prev = W[i - 1] if i > 0 else 0
                if abs(W[j] - (prev + V[i, j])) < 1e-6:
                    pairs = {(i, j)}
                    pairs.update(traceback_V(i, j))
                    pairs.update(traceback_W(i - 1))
                    return pairs

        return set()

    def traceback_V(i: int, j: int) -> Set[Tuple[int, int]]:
        if V[i, j] >= INF:
            return set()

        # Check hairpin
        if abs(V[i, j] - hairpin_energy(i, j)) < 1e-6:
            return set()

        # Check stacking/interior/bulge
        for ip in range(i + 1, j - MIN_LOOP):
            for jp in range(ip + MIN_LOOP + 1, j):
                if not can_pair(ip, jp) or V[ip, jp] >= INF:
                    continue

                n1 = ip - i - 1
                n2 = j - jp - 1

                if n1 == 0 and n2 == 0:
                    energy = stack_energy(i, j, ip, jp) + V[ip, jp]
                elif n1 == 0 or n2 == 0:
                    energy = bulge_energy(n1 + n2) + V[ip, jp]
                else:
                    energy = interior_energy(i, j, ip, jp) + V[ip, jp]

                if abs(V[i, j] - energy) < 1e-6:
                    pairs = {(ip, jp)}
                    pairs.update(traceback_V(ip, jp))
                    return pairs

        return set()

    mfe = W[n - 1]
    pairs = traceback_W(n - 1)

    return mfe, pairs
\end{lstlisting}

\begin{warningbox}
The simplified Zuker implementation above omits many details of the full Turner
energy model, including:
\begin{itemize}
    \item Dangling end contributions
    \item Terminal mismatches
    \item Special hairpin loop sequences (tetraloops, triloops)
    \item Coaxial stacking in multi-loops
\end{itemize}
For production use, established tools like ViennaRNA or RNAfold should be used.
\end{warningbox}

%% =============================================================================
%% SECTION 5: NEUTRAL NETWORKS
%% =============================================================================
\section{Neutral Networks}

\subsection{Definition and Properties}

\begin{definition}[Neutral Network]
The neutral network $\neutralnet(\phi)$ of phenotype $\phi$ is the set of all
genotypes mapping to that phenotype:
\begin{equation}
    \neutralnet(\phi) = \{s \in \genotype : \gpmap(s) = \phi\} = \gpmap^{-1}(\phi)
\end{equation}
\end{definition}

\begin{definition}[Neutrality]
The neutrality $\rho(s)$ of a sequence $s$ is the fraction of single-mutant
neighbors that share its phenotype:
\begin{equation}
    \rho(s) = \frac{|\{s' : \hamming(s, s') = 1 \land \gpmap(s') = \gpmap(s)\}|}{|\{s' : \hamming(s, s') = 1\}|}
\end{equation}
The denominator equals $\seqlength(|\alphabet| - 1)$ for sequences over alphabet $\alphabet$.
\end{definition}

\begin{physicsbox}
For RNA secondary structure with $\alphabet = 4$ and $\seqlength = 100$:
\begin{equation}
    |\{s' : \hamming(s, s') = 1\}| = 100 \times 3 = 300
\end{equation}
If a sequence has neutrality $\rho = 0.3$, it has approximately 90 neutral
neighbors that fold to the same structure.
\end{physicsbox}

\begin{pursuitbox}
Neutral networks enable evolutionary search by allowing populations to explore
genotype space while maintaining phenotypic function. The structure of neutral
networks - their size, connectivity, and extent - fundamentally shapes evolutionary
dynamics and the discovery of novel phenotypes.
\end{pursuitbox}

\subsection{Network Connectivity and Giant Components}

\begin{theorem}[Neutral Network Graph]
The neutral network $\neutralnet(\phi)$ inherits graph structure from the Hamming
graph. Two genotypes $s, s' \in \neutralnet(\phi)$ are connected in the neutral
network graph if and only if $\hamming(s, s') = 1$.
\end{theorem}

\begin{definition}[Giant Connected Component]
The giant connected component (GCC) of $\neutralnet(\phi)$ is the largest connected
subgraph, containing a fraction $\gamma$ of all vertices in $\neutralnet(\phi)$.
\end{definition}

\begin{physicsbox}
\textbf{Percolation Theory for Neutral Networks}

Consider a random graph model where each potential edge in the Hamming graph
exists with probability $p$. The percolation threshold $p_c$ marks the transition
where a giant component emerges.

For the Hamming graph $H(\seqlength, |\alphabet|)$ with degree
$k = \seqlength(|\alphabet| - 1)$:
\begin{equation}
    p_c \approx \frac{1}{k} = \frac{1}{\seqlength(|\alphabet| - 1)}
\end{equation}

The neutral network analogy: if the average neutrality $\bar{\rho}$ exceeds $p_c$:
\begin{equation}
    \bar{\rho} > \frac{1}{\seqlength(|\alphabet| - 1)}
\end{equation}
then the neutral network percolates and contains a GCC spanning much of sequence space.
\end{physicsbox}

\begin{theorem}[Percolation in RNA Neutral Networks]
For common RNA secondary structures, neutral networks percolate across sequence
space. The giant component typically contains $> 99\%$ of sequences in the neutral
network.
\end{theorem}

\begin{annotation}
This remarkable result means that for most biologically relevant RNA structures,
evolution can reach essentially any neutral sequence from any other through single
point mutations, never leaving the neutral network. This ``neutral evolution''
enables exploration without fitness penalty.
\end{annotation}

\subsection{Neutral Network Statistics}

\begin{lstlisting}[caption={Neutral Network Analysis Implementation}]
import numpy as np
from typing import Dict, List, Set, Tuple
from collections import defaultdict
import random

def generate_neighbors(sequence: str, alphabet: str = 'AUGC') -> List[str]:
    """Generate all single-mutation neighbors of a sequence."""
    neighbors = []
    for i in range(len(sequence)):
        for base in alphabet:
            if base != sequence[i]:
                neighbor = sequence[:i] + base + sequence[i+1:]
                neighbors.append(neighbor)
    return neighbors


def compute_neutrality(sequence: str,
                       fold_function,
                       alphabet: str = 'AUGC') -> float:
    """
    Compute neutrality of a sequence.

    Args:
        sequence: RNA sequence
        fold_function: Function mapping sequence to structure
        alphabet: Nucleotide alphabet

    Returns:
        Neutrality (fraction of neutral neighbors)
    """
    original_structure = fold_function(sequence)
    neighbors = generate_neighbors(sequence, alphabet)

    neutral_count = sum(
        1 for neighbor in neighbors
        if fold_function(neighbor) == original_structure
    )

    return neutral_count / len(neighbors)


def neutral_network_statistics(seed_sequence: str,
                               fold_function,
                               max_samples: int = 1000,
                               alphabet: str = 'AUGC') -> Dict:
    """
    Estimate neutral network statistics via random walk sampling.

    Args:
        seed_sequence: Starting sequence
        fold_function: Folding function
        max_samples: Maximum sequences to sample
        alphabet: Nucleotide alphabet

    Returns:
        Dictionary of network statistics
    """
    target_structure = fold_function(seed_sequence)

    # Sample neutral network via random walk
    visited = {seed_sequence}
    current = seed_sequence
    neutralities = []

    for _ in range(max_samples):
        # Compute neutrality
        rho = compute_neutrality(current, fold_function, alphabet)
        neutralities.append(rho)

        # Random walk step
        neighbors = generate_neighbors(current, alphabet)
        neutral_neighbors = [
            n for n in neighbors
            if fold_function(n) == target_structure
        ]

        if neutral_neighbors:
            current = random.choice(neutral_neighbors)
            visited.add(current)
        else:
            # Stuck - restart from random visited sequence
            current = random.choice(list(visited))

    # Estimate network extent (max Hamming distance observed)
    sample_list = list(visited)
    max_distance = 0
    for i in range(min(100, len(sample_list))):
        for j in range(i + 1, min(100, len(sample_list))):
            dist = hamming_distance(sample_list[i], sample_list[j])
            max_distance = max(max_distance, dist)

    return {
        'target_structure': target_structure,
        'samples_collected': len(visited),
        'mean_neutrality': np.mean(neutralities),
        'std_neutrality': np.std(neutralities),
        'min_neutrality': np.min(neutralities),
        'max_neutrality': np.max(neutralities),
        'estimated_extent': max_distance,
        'sequence_length': len(seed_sequence)
    }


def hamming_distance(s1: str, s2: str) -> int:
    """Compute Hamming distance between two sequences."""
    return sum(c1 != c2 for c1, c2 in zip(s1, s2))


def estimate_phenotype_frequency(structure: str,
                                 sequence_length: int,
                                 inverse_fold_function,
                                 num_samples: int = 1000) -> float:
    """
    Estimate frequency of a phenotype in sequence space.

    Uses inverse folding to sample sequences with given structure,
    then estimates based on success rate.
    """
    successful = 0
    for _ in range(num_samples):
        try:
            seq = inverse_fold_function(structure)
            if seq is not None:
                successful += 1
        except:
            pass

    # Rough estimate based on inverse fold success rate
    return successful / num_samples
\end{lstlisting}

%% =============================================================================
%% SECTION 6: FITNESS LANDSCAPES
%% =============================================================================
\section{Fitness Landscapes}

\subsection{Definition and Visualization}

\begin{definition}[Fitness Landscape]
A fitness landscape is a function $\fitness: \genotype \to \real$ assigning a
fitness value to each genotype. Combined with the neighborhood structure from
Hamming distance, this defines a landscape that evolution navigates.
\end{definition}

\begin{physicsbox}
The composite fitness landscape through GP mapping:
\begin{equation}
    \fitness: \alphabet^{\seqlength} \xrightarrow{\gpmap} \phenotype \xrightarrow{F_{\phi}} \real
\end{equation}
where $F_{\phi}: \phenotype \to \real$ assigns fitness to phenotypes. The
composite $\fitness = F_{\phi} \circ \gpmap$ defines fitness on genotype space.
\end{physicsbox}

\subsection{Local Optima and Ruggedness}

\begin{definition}[Local Optimum]
A genotype $s^*$ is a local optimum if it has higher fitness than all neighbors:
\begin{equation}
    \fitness(s^*) \geq \fitness(s') \quad \forall s': \hamming(s^*, s') = 1
\end{equation}
A strict local optimum satisfies the inequality strictly.
\end{definition}

\begin{definition}[Global Optimum]
A genotype $s^{**}$ is a global optimum if:
\begin{equation}
    \fitness(s^{**}) \geq \fitness(s) \quad \forall s \in \genotype
\end{equation}
\end{definition}

\begin{definition}[Ruggedness]
The ruggedness of a fitness landscape quantifies the density and depth of local
optima. A highly rugged landscape has many local optima, making global optimization
difficult.
\end{definition}

\begin{warningbox}
Rugged fitness landscapes pose fundamental challenges for evolution:
\begin{itemize}
    \item Populations may get trapped at suboptimal local peaks
    \item The number of local optima can grow exponentially with sequence length
    \item Neutral evolution becomes critical for escaping local optima
\end{itemize}
\end{warningbox}

\subsection{Autocorrelation Function}

\begin{definition}[Fitness Autocorrelation]
The autocorrelation function $r(d)$ measures fitness correlation between sequences
at Hamming distance $d$:
\begin{equation}
    r(d) = \frac{\expect[\fitness(s)\fitness(s')] - \expect[\fitness(s)]^2}{\text{Var}[\fitness(s)]}
\end{equation}
where the expectation is over all pairs $(s, s')$ with $\hamming(s, s') = d$.
\end{definition}

\begin{physicsbox}
\textbf{Properties of Autocorrelation:}
\begin{enumerate}
    \item $r(0) = 1$ (perfect self-correlation)
    \item $r(d)$ typically decreases with $d$
    \item For additive landscapes: $r(d) = (1 - 2d/\seqlength)$ (linear decay)
    \item For random landscapes: $r(d) = 0$ for $d > 0$
\end{enumerate}

The \textbf{correlation length} $\ell$ characterizes decay:
\begin{equation}
    r(d) \approx e^{-d/\ell}
\end{equation}
Small $\ell$ indicates high ruggedness; large $\ell$ indicates smoothness.
\end{physicsbox}

\subsection{Epistasis}

\begin{definition}[Epistasis]
Epistasis occurs when the fitness effect of a mutation depends on the genetic
background. For positions $i$ and $j$:
\begin{equation}
    \varepsilon_{ij} = \fitness(s_{ij}^{11}) - \fitness(s_{ij}^{10}) - \fitness(s_{ij}^{01}) + \fitness(s_{ij}^{00})
\end{equation}
where $s_{ij}^{ab}$ has allele $a$ at position $i$ and allele $b$ at position $j$.
\end{definition}

\begin{physicsbox}
\textbf{Types of Epistasis:}
\begin{itemize}
    \item $\varepsilon_{ij} = 0$: No epistasis (additive)
    \item $\varepsilon_{ij} > 0$: Positive (synergistic) epistasis
    \item $\varepsilon_{ij} < 0$: Negative (antagonistic) epistasis
    \item Sign epistasis: Mutation beneficial in one background, deleterious in another
    \item Reciprocal sign epistasis: Both mutations show sign epistasis
\end{itemize}
\end{physicsbox}

\begin{theorem}[Epistasis and Local Optima]
A fitness landscape has local optima if and only if it exhibits reciprocal sign
epistasis.
\end{theorem}

\begin{lstlisting}[caption={Fitness Landscape Analysis Functions}]
import numpy as np
from typing import Callable, List, Tuple, Dict
from itertools import product

def compute_autocorrelation(fitness_func: Callable[[str], float],
                           sequences: List[str],
                           max_distance: int = None) -> Dict[int, float]:
    """
    Compute fitness autocorrelation function.

    Args:
        fitness_func: Function mapping sequence to fitness
        sequences: Sample of sequences
        max_distance: Maximum distance to compute

    Returns:
        Dictionary mapping distance to correlation
    """
    n = len(sequences[0])
    if max_distance is None:
        max_distance = n

    # Compute fitness values
    fitness_values = {seq: fitness_func(seq) for seq in sequences}
    mean_fitness = np.mean(list(fitness_values.values()))
    var_fitness = np.var(list(fitness_values.values()))

    if var_fitness == 0:
        return {d: 1.0 if d == 0 else 0.0 for d in range(max_distance + 1)}

    # Compute correlations by distance
    correlations = defaultdict(list)

    for i, s1 in enumerate(sequences):
        for s2 in sequences[i:]:
            d = hamming_distance(s1, s2)
            if d <= max_distance:
                f1, f2 = fitness_values[s1], fitness_values[s2]
                correlations[d].append((f1 - mean_fitness) * (f2 - mean_fitness))

    return {
        d: np.mean(vals) / var_fitness if vals else 0.0
        for d, vals in correlations.items()
    }


def estimate_correlation_length(autocorr: Dict[int, float]) -> float:
    """
    Estimate correlation length from autocorrelation function.

    Fits exponential decay r(d) = exp(-d/ell).
    """
    distances = sorted(autocorr.keys())
    correlations = [autocorr[d] for d in distances]

    # Find first non-positive correlation
    for i, r in enumerate(correlations):
        if r <= 0 and i > 0:
            break

    # Fit log-linear model to positive correlations
    positive_d = [d for d, r in zip(distances[:i], correlations[:i]) if r > 0]
    positive_r = [r for r in correlations[:i] if r > 0]

    if len(positive_d) < 2:
        return float('inf')

    log_r = np.log(positive_r)
    slope, _ = np.polyfit(positive_d, log_r, 1)

    return -1.0 / slope if slope < 0 else float('inf')


def compute_epistasis(fitness_func: Callable[[str], float],
                      sequence: str,
                      pos_i: int,
                      pos_j: int,
                      alphabet: str = 'AUGC') -> float:
    """
    Compute pairwise epistasis coefficient.

    Args:
        fitness_func: Fitness function
        sequence: Reference sequence
        pos_i, pos_j: Positions to analyze
        alphabet: Nucleotide alphabet

    Returns:
        Epistasis coefficient
    """
    # Get alleles at positions
    allele_i = sequence[pos_i]
    allele_j = sequence[pos_j]

    # Choose alternative alleles
    alt_i = [a for a in alphabet if a != allele_i][0]
    alt_j = [a for a in alphabet if a != allele_j][0]

    def mutate(seq, pos, new_allele):
        return seq[:pos] + new_allele + seq[pos+1:]

    # Four fitness values
    f00 = fitness_func(sequence)  # Wild type
    f10 = fitness_func(mutate(sequence, pos_i, alt_i))  # Mutant at i
    f01 = fitness_func(mutate(sequence, pos_j, alt_j))  # Mutant at j

    double_mutant = mutate(mutate(sequence, pos_i, alt_i), pos_j, alt_j)
    f11 = fitness_func(double_mutant)  # Double mutant

    return f11 - f10 - f01 + f00


def count_local_optima(fitness_func: Callable[[str], float],
                       sequences: List[str],
                       alphabet: str = 'AUGC') -> Tuple[int, List[str]]:
    """
    Count local optima in a sample of sequences.

    Returns count and list of local optima.
    """
    local_optima = []

    for seq in sequences:
        fitness = fitness_func(seq)
        neighbors = generate_neighbors(seq, alphabet)

        is_optimum = all(
            fitness >= fitness_func(neighbor)
            for neighbor in neighbors
        )

        if is_optimum:
            local_optima.append(seq)

    return len(local_optima), local_optima
\end{lstlisting}

%% =============================================================================
%% SECTION 7: NK MODEL
%% =============================================================================
\section{The NK Model: Tunable Ruggedness}

\subsection{Model Definition}

The NK model, introduced by Stuart Kauffman (1987), provides a framework for
generating fitness landscapes with tunable ruggedness.

\begin{definition}[NK Model]
An NK model is defined by:
\begin{itemize}
    \item $N$: Sequence length (number of loci)
    \item $K$: Epistatic interactions per locus ($0 \leq K \leq N-1$)
    \item For each locus $i$, a fitness contribution table $f_i$ depending on
          locus $i$ and $K$ other specified loci
\end{itemize}
\end{definition}

\begin{physicsbox}
\textbf{NK Fitness Function:}

The fitness of sequence $s = (s_1, \ldots, s_N)$ is:
\begin{equation}
    \fitness(s) = \frac{1}{N} \sum_{i=1}^{N} f_i(s_i, s_{\pi_i(1)}, \ldots, s_{\pi_i(K)})
\end{equation}
where $\pi_i = \{\pi_i(1), \ldots, \pi_i(K)\}$ specifies the $K$ loci that
epistatically interact with locus $i$.

Each $f_i: \{0,1\}^{K+1} \to [0,1]$ is typically drawn randomly from uniform distribution.
\end{physicsbox}

\begin{theorem}[NK Model Properties]
\begin{enumerate}
    \item When $K = 0$: Landscape is additive (smooth), single global optimum
    \item When $K = N - 1$: Landscape is maximally rugged (uncorrelated)
    \item Number of local optima grows with $K$
    \item Correlation length decreases with $K$
\end{enumerate}
\end{theorem}

\subsection{Interaction Structures}

\begin{definition}[Interaction Types]
\begin{itemize}
    \item \textbf{Adjacent}: $\pi_i = \{i+1, i+2, \ldots, i+K\} \mod N$
    \item \textbf{Random}: $\pi_i$ chosen uniformly at random
    \item \textbf{Block}: Interactions within contiguous blocks
\end{itemize}
\end{definition}

\begin{physicsbox}
\textbf{Expected Number of Local Optima:}

For the random interaction NK model with $K = N - 1$:
\begin{equation}
    \expect[\text{local optima}] = \frac{2^N}{N + 1}
\end{equation}

For general $K$ (asymptotic):
\begin{equation}
    \expect[\text{local optima}] \sim \binom{N}{K+1}^{-1} \cdot 2^N \cdot \left(\frac{1}{2}\right)^{N-K-1}
\end{equation}
\end{physicsbox}

\subsection{Implementation}

\begin{lstlisting}[caption={NK Model Implementation}]
import numpy as np
from typing import List, Dict, Tuple, Optional
from dataclasses import dataclass

@dataclass
class NKModel:
    """NK fitness landscape model."""

    n: int  # Sequence length
    k: int  # Epistatic interactions
    fitness_tables: List[np.ndarray]  # Fitness contribution tables
    interactions: List[List[int]]  # Interaction partners

    @classmethod
    def random(cls, n: int, k: int,
               interaction_type: str = 'random',
               seed: Optional[int] = None) -> 'NKModel':
        """
        Create random NK model.

        Args:
            n: Sequence length
            k: Epistatic interactions per locus
            interaction_type: 'random', 'adjacent', or 'block'
            seed: Random seed

        Returns:
            NKModel instance
        """
        rng = np.random.default_rng(seed)

        # Generate interaction structure
        interactions = []
        for i in range(n):
            if interaction_type == 'adjacent':
                partners = [(i + j + 1) % n for j in range(k)]
            elif interaction_type == 'random':
                others = [j for j in range(n) if j != i]
                partners = list(rng.choice(others, size=k, replace=False))
            elif interaction_type == 'block':
                block_size = k + 1
                block_start = (i // block_size) * block_size
                partners = [
                    (block_start + j) % n
                    for j in range(block_size)
                    if (block_start + j) % n != i
                ][:k]
            else:
                raise ValueError(f"Unknown interaction type: {interaction_type}")

            interactions.append(partners)

        # Generate fitness tables
        fitness_tables = []
        for i in range(n):
            # Table has 2^(k+1) entries (locus i plus k partners)
            table = rng.uniform(0, 1, size=2**(k + 1))
            fitness_tables.append(table)

        return cls(n, k, fitness_tables, interactions)

    def fitness(self, sequence: np.ndarray) -> float:
        """Compute fitness of binary sequence."""
        total = 0.0

        for i in range(self.n):
            # Get relevant bits
            bits = [sequence[i]] + [sequence[j] for j in self.interactions[i]]

            # Convert to index
            index = sum(b * (2 ** j) for j, b in enumerate(bits))

            total += self.fitness_tables[i][int(index)]

        return total / self.n

    def fitness_string(self, sequence: str) -> float:
        """Compute fitness of string sequence (0s and 1s)."""
        arr = np.array([int(c) for c in sequence])
        return self.fitness(arr)

    def random_sequence(self, rng: np.random.Generator = None) -> np.ndarray:
        """Generate random binary sequence."""
        if rng is None:
            rng = np.random.default_rng()
        return rng.integers(0, 2, size=self.n)

    def neighbors(self, sequence: np.ndarray) -> List[np.ndarray]:
        """Generate all single-mutation neighbors."""
        neighbors = []
        for i in range(self.n):
            neighbor = sequence.copy()
            neighbor[i] = 1 - neighbor[i]
            neighbors.append(neighbor)
        return neighbors

    def is_local_optimum(self, sequence: np.ndarray) -> bool:
        """Check if sequence is a local optimum."""
        current_fitness = self.fitness(sequence)
        return all(
            current_fitness >= self.fitness(neighbor)
            for neighbor in self.neighbors(sequence)
        )

    def adaptive_walk(self,
                      start: np.ndarray = None,
                      rng: np.random.Generator = None) -> Tuple[np.ndarray, float, int]:
        """
        Perform adaptive walk to local optimum.

        Returns:
            (final_sequence, final_fitness, steps)
        """
        if rng is None:
            rng = np.random.default_rng()
        if start is None:
            start = self.random_sequence(rng)

        current = start.copy()
        current_fitness = self.fitness(current)
        steps = 0

        while True:
            # Find beneficial mutations
            beneficial = []
            for neighbor in self.neighbors(current):
                neighbor_fitness = self.fitness(neighbor)
                if neighbor_fitness > current_fitness:
                    beneficial.append((neighbor, neighbor_fitness))

            if not beneficial:
                break

            # Choose random beneficial mutation
            current, current_fitness = beneficial[rng.integers(len(beneficial))]
            steps += 1

        return current, current_fitness, steps


def analyze_nk_landscape(n: int, k: int,
                         num_walks: int = 100,
                         seed: int = None) -> Dict:
    """
    Analyze NK landscape properties.

    Returns dictionary of landscape statistics.
    """
    model = NKModel.random(n, k, seed=seed)
    rng = np.random.default_rng(seed)

    # Perform adaptive walks
    final_fitness = []
    walk_lengths = []
    local_optima = set()

    for _ in range(num_walks):
        seq, fit, steps = model.adaptive_walk(rng=rng)
        final_fitness.append(fit)
        walk_lengths.append(steps)
        local_optima.add(tuple(seq))

    # Sample fitness correlations
    samples = [model.random_sequence(rng) for _ in range(min(500, 2**n))]
    fitness_values = [model.fitness(s) for s in samples]

    return {
        'n': n,
        'k': k,
        'mean_fitness': np.mean(fitness_values),
        'std_fitness': np.std(fitness_values),
        'mean_walk_length': np.mean(walk_lengths),
        'std_walk_length': np.std(walk_lengths),
        'mean_optimum_fitness': np.mean(final_fitness),
        'num_unique_optima': len(local_optima),
        'num_walks': num_walks
    }
\end{lstlisting}

\begin{annotation}
The NK model bridges the gap between fully additive ($K=0$) and fully random
($K=N-1$) landscapes. Real biological systems likely have intermediate $K$ values,
with epistatic interactions structured by physical and functional constraints.
\end{annotation}

%% =============================================================================
%% SECTION 8: EVOLUTIONARY DYNAMICS
%% =============================================================================
\section{Evolutionary Dynamics Models}

\subsection{Wright-Fisher Model}

\begin{definition}[Wright-Fisher Model]
The Wright-Fisher model describes evolution in a finite population of size $N$:
\begin{enumerate}
    \item Discrete, non-overlapping generations
    \item Each individual in generation $t+1$ is sampled with replacement from
          generation $t$ according to fitness-weighted probabilities
\end{enumerate}
\end{definition}

\begin{physicsbox}
\textbf{Selection Probability:}

For individual $i$ with fitness $f_i$, probability of being parent:
\begin{equation}
    p_i = \frac{f_i}{\sum_{j=1}^N f_j}
\end{equation}

\textbf{Transition Probability:}

For allele frequency $x$ changing to $x'$ with $Nx' = k$:
\begin{equation}
    P(k | x) = \binom{N}{k} w(x)^k (1 - w(x))^{N-k}
\end{equation}
where $w(x)$ incorporates selection:
\begin{equation}
    w(x) = \frac{x \cdot W_1}{x \cdot W_1 + (1-x) \cdot W_0}
\end{equation}
with $W_1$ and $W_0$ being fitnesses of the two alleles.
\end{physicsbox}

\subsection{Fixation Probability}

\begin{theorem}[Kimura Fixation Probability]
For a new mutation with selective advantage $s$ in a population of size $N$:
\begin{equation}
    \pi = \frac{1 - e^{-2s}}{1 - e^{-2Ns}}
\end{equation}
\end{theorem}

\begin{corollary}
\begin{itemize}
    \item Neutral mutation ($s = 0$): $\pi = 1/N$
    \item Beneficial mutation ($s > 0$, $Ns \gg 1$): $\pi \approx 2s$
    \item Deleterious mutation ($s < 0$, $N|s| \gg 1$): $\pi \approx 2|s| e^{-2N|s|}$
\end{itemize}
\end{corollary}

\subsection{Moran Model}

\begin{definition}[Moran Model]
The Moran model uses overlapping generations:
\begin{enumerate}
    \item At each time step, one individual is chosen to reproduce (proportional
          to fitness)
    \item One individual is chosen uniformly at random to die
    \item The reproducing individual's offspring replaces the dying individual
\end{enumerate}
\end{definition}

\begin{physicsbox}
\textbf{Moran Fixation Probability:}

For initial count $i$ of mutant allele with relative fitness $r$:
\begin{equation}
    \pi_i = \frac{1 - r^{-i}}{1 - r^{-N}}
\end{equation}

For a single new mutant ($i = 1$):
\begin{equation}
    \pi_1 = \frac{1 - 1/r}{1 - 1/r^N}
\end{equation}
\end{physicsbox}

\begin{theorem}[Moran-Wright-Fisher Equivalence]
The Moran model with population $N$ is equivalent in fixation probability to a
Wright-Fisher model with effective population $N_e = N/2$.
\end{theorem}

\subsection{Implementation}

\begin{lstlisting}[caption={Wright-Fisher and Moran Evolution Simulations}]
import numpy as np
from typing import List, Tuple, Callable, Optional
from dataclasses import dataclass, field

@dataclass
class EvolutionResult:
    """Results from evolutionary simulation."""
    generations: int
    final_population: List[str]
    fitness_history: List[float]
    diversity_history: List[float]
    fixations: List[Tuple[int, str, str]]  # (gen, old, new)


def wright_fisher_evolution(
    initial_sequence: str,
    fitness_func: Callable[[str], float],
    population_size: int = 100,
    generations: int = 1000,
    mutation_rate: float = 0.001,
    alphabet: str = 'AUGC',
    seed: Optional[int] = None
) -> EvolutionResult:
    """
    Simulate Wright-Fisher evolution.

    Args:
        initial_sequence: Starting sequence
        fitness_func: Fitness function
        population_size: Population size N
        generations: Number of generations
        mutation_rate: Per-site mutation rate
        alphabet: Sequence alphabet
        seed: Random seed

    Returns:
        EvolutionResult with simulation data
    """
    rng = np.random.default_rng(seed)
    seq_len = len(initial_sequence)

    # Initialize population
    population = [initial_sequence] * population_size

    # Tracking
    fitness_history = []
    diversity_history = []
    fixations = []

    def mutate(seq: str) -> str:
        """Apply mutations to sequence."""
        result = list(seq)
        for i in range(len(result)):
            if rng.random() < mutation_rate:
                alternatives = [a for a in alphabet if a != result[i]]
                result[i] = rng.choice(alternatives)
        return ''.join(result)

    def compute_diversity(pop: List[str]) -> float:
        """Compute nucleotide diversity."""
        n = len(pop)
        if n < 2:
            return 0.0

        total_diff = 0
        comparisons = 0
        for i in range(n):
            for j in range(i + 1, n):
                total_diff += hamming_distance(pop[i], pop[j])
                comparisons += 1

        return total_diff / (comparisons * seq_len) if comparisons > 0 else 0.0

    for gen in range(generations):
        # Compute fitness
        fitness_values = np.array([fitness_func(seq) for seq in population])

        # Record statistics
        fitness_history.append(np.mean(fitness_values))
        diversity_history.append(compute_diversity(population))

        # Selection: sample with replacement proportional to fitness
        if fitness_values.sum() > 0:
            probs = fitness_values / fitness_values.sum()
        else:
            probs = np.ones(population_size) / population_size

        parent_indices = rng.choice(
            population_size,
            size=population_size,
            p=probs
        )

        # Create new generation with mutation
        new_population = [mutate(population[i]) for i in parent_indices]

        # Detect fixations (simplified)
        old_consensus = max(set(population), key=population.count)
        new_consensus = max(set(new_population), key=new_population.count)
        if old_consensus != new_consensus:
            fixations.append((gen, old_consensus, new_consensus))

        population = new_population

    return EvolutionResult(
        generations=generations,
        final_population=population,
        fitness_history=fitness_history,
        diversity_history=diversity_history,
        fixations=fixations
    )


def moran_evolution(
    initial_sequence: str,
    fitness_func: Callable[[str], float],
    population_size: int = 100,
    max_events: int = 10000,
    mutation_rate: float = 0.001,
    alphabet: str = 'AUGC',
    seed: Optional[int] = None
) -> EvolutionResult:
    """
    Simulate Moran process evolution.

    Args:
        initial_sequence: Starting sequence
        fitness_func: Fitness function
        population_size: Population size N
        max_events: Maximum birth-death events
        mutation_rate: Per-site mutation rate
        alphabet: Sequence alphabet
        seed: Random seed

    Returns:
        EvolutionResult with simulation data
    """
    rng = np.random.default_rng(seed)
    seq_len = len(initial_sequence)

    # Initialize population
    population = [initial_sequence] * population_size

    # Tracking
    fitness_history = []
    diversity_history = []
    fixations = []

    def mutate(seq: str) -> str:
        """Apply single mutation to sequence."""
        if rng.random() > mutation_rate * seq_len:
            return seq

        result = list(seq)
        pos = rng.integers(len(result))
        alternatives = [a for a in alphabet if a != result[pos]]
        result[pos] = rng.choice(alternatives)
        return ''.join(result)

    def compute_diversity(pop: List[str]) -> float:
        """Compute nucleotide diversity."""
        unique = list(set(pop))
        if len(unique) < 2:
            return 0.0

        total_diff = 0
        for i, s1 in enumerate(unique):
            for s2 in unique[i+1:]:
                total_diff += hamming_distance(s1, s2)

        return total_diff / (len(unique) * (len(unique) - 1) / 2 * seq_len)

    for event in range(max_events):
        # Compute fitness
        fitness_values = np.array([fitness_func(seq) for seq in population])

        # Record statistics periodically
        if event % population_size == 0:
            fitness_history.append(np.mean(fitness_values))
            diversity_history.append(compute_diversity(population))

        # Birth: choose individual proportional to fitness
        if fitness_values.sum() > 0:
            birth_probs = fitness_values / fitness_values.sum()
        else:
            birth_probs = np.ones(population_size) / population_size

        parent_idx = rng.choice(population_size, p=birth_probs)
        offspring = mutate(population[parent_idx])

        # Death: choose uniformly at random
        death_idx = rng.integers(population_size)

        population[death_idx] = offspring

    generations = max_events // population_size

    return EvolutionResult(
        generations=generations,
        final_population=population,
        fitness_history=fitness_history,
        diversity_history=diversity_history,
        fixations=fixations
    )


def fixation_probability_simulation(
    wild_type: str,
    mutant: str,
    fitness_func: Callable[[str], float],
    population_size: int = 100,
    num_trials: int = 1000,
    max_generations: int = 10000,
    seed: Optional[int] = None
) -> float:
    """
    Estimate fixation probability by simulation.

    Args:
        wild_type: Wild type sequence
        mutant: Mutant sequence
        fitness_func: Fitness function
        population_size: Population size
        num_trials: Number of simulation trials
        max_generations: Max generations per trial
        seed: Random seed

    Returns:
        Estimated fixation probability
    """
    rng = np.random.default_rng(seed)

    w_fit = fitness_func(wild_type)
    m_fit = fitness_func(mutant)

    fixations = 0

    for trial in range(num_trials):
        # Start with one mutant
        mutant_count = 1

        for gen in range(max_generations):
            if mutant_count == 0:
                break
            if mutant_count == population_size:
                fixations += 1
                break

            # Wright-Fisher sampling
            p_mutant = (mutant_count * m_fit) / (
                mutant_count * m_fit +
                (population_size - mutant_count) * w_fit
            )
            mutant_count = rng.binomial(population_size, p_mutant)

    return fixations / num_trials
\end{lstlisting}

%% =============================================================================
%% SECTION 9: ROBUSTNESS AND EVOLVABILITY
%% =============================================================================
\section{Robustness-Evolvability Tradeoff}

\subsection{Definitions}

\begin{definition}[Mutational Robustness]
The mutational robustness $R(s)$ of genotype $s$ is the probability that a
random point mutation preserves the phenotype:
\begin{equation}
    R(s) = \frac{|\{s' : \hamming(s, s') = 1 \land \gpmap(s') = \gpmap(s)\}|}{\seqlength(|\alphabet| - 1)}
\end{equation}
This equals the neutrality $\rho(s)$ defined earlier.
\end{definition}

\begin{definition}[Evolvability]
The evolvability $E(s)$ of genotype $s$ measures access to novel phenotypes:
\begin{equation}
    E(s) = |\{\phi : \exists s', \hamming(s, s') = 1 \land \gpmap(s') = \phi \land \phi \neq \gpmap(s)\}|
\end{equation}
the number of distinct phenotypes accessible via single mutations.
\end{definition}

\begin{pursuitbox}
Understanding the relationship between robustness and evolvability is central to
evolutionary biology:
\begin{itemize}
    \item High robustness means mutations don't disrupt function
    \item High evolvability means mutations can access new functions
    \item These seem contradictory but can coexist in structured GP maps
\end{itemize}
\end{pursuitbox}

\subsection{The Paradox and Resolution}

\begin{theorem}[Robustness-Evolvability in Neutral Networks]
In GP maps with extended neutral networks:
\begin{enumerate}
    \item Neutral evolution allows populations to spread across the neutral network
    \item Different positions in the network have different phenotypic neighborhoods
    \item The population's collective evolvability can exceed individual evolvability
    \item Robustness enables exploration of the neutral network, enhancing evolvability
\end{enumerate}
\end{theorem}

\begin{physicsbox}
\textbf{Population-Level Evolvability:}

For a population $P \subseteq \neutralnet(\phi)$ distributed across the neutral network:
\begin{equation}
    E(P) = \left|\bigcup_{s \in P} \{\phi' : \exists s', \hamming(s, s') = 1 \land \gpmap(s') = \phi'\}\right|
\end{equation}

This collective evolvability satisfies:
\begin{equation}
    E(P) \geq \max_{s \in P} E(s)
\end{equation}

A spread population samples diverse phenotypic neighborhoods, increasing $E(P)$.
\end{physicsbox}

\begin{warningbox}
The robustness-evolvability relationship depends critically on:
\begin{itemize}
    \item Population size (larger populations explore more)
    \item Mutation rate (must be sufficient for neutral drift)
    \item Neutral network structure (extent and connectivity)
    \item Selective pressure (stabilizing vs. directional selection)
\end{itemize}
\end{warningbox}

%% =============================================================================
%% SECTION 10: CERTIFICATE DATA STRUCTURE
%% =============================================================================
\section{Genotype-Phenotype Certificate}

\subsection{Certificate Definition}

\begin{definition}[GenotypePhentotypeCertificate]
A formal certificate attesting to properties of a GP mapping analysis:
\begin{equation}
    \mathcal{C} = (\genotype_{\text{sample}}, \phenotype_{\text{obs}}, \mathcal{M}, \mathcal{V})
\end{equation}
where:
\begin{itemize}
    \item $\genotype_{\text{sample}}$: Sampled genotypes
    \item $\phenotype_{\text{obs}}$: Observed phenotypes
    \item $\mathcal{M}$: Computed metrics (neutrality, epistasis, etc.)
    \item $\mathcal{V}$: Verification protocols with results
\end{itemize}
\end{definition}

\subsection{Implementation}

\begin{lstlisting}[caption={GenotypePhentotypeCertificate Data Structure}]
from dataclasses import dataclass, field
from typing import Dict, List, Set, Tuple, Optional, Any
from datetime import datetime
import hashlib
import json

@dataclass
class FoldingResult:
    """Result of folding a single sequence."""
    sequence: str
    structure: str
    energy: Optional[float] = None
    algorithm: str = "nussinov"

    def to_dict(self) -> Dict:
        return {
            'sequence': self.sequence,
            'structure': self.structure,
            'energy': self.energy,
            'algorithm': self.algorithm
        }


@dataclass
class NeutralNetworkMetrics:
    """Metrics for neutral network analysis."""
    target_phenotype: str
    sample_size: int
    mean_neutrality: float
    std_neutrality: float
    estimated_extent: int
    connectivity_estimate: float

    def to_dict(self) -> Dict:
        return {
            'target_phenotype': self.target_phenotype,
            'sample_size': self.sample_size,
            'mean_neutrality': self.mean_neutrality,
            'std_neutrality': self.std_neutrality,
            'estimated_extent': self.estimated_extent,
            'connectivity_estimate': self.connectivity_estimate
        }


@dataclass
class FitnessLandscapeMetrics:
    """Metrics for fitness landscape analysis."""
    correlation_length: float
    num_local_optima_sampled: int
    mean_optimum_fitness: float
    mean_epistasis: float
    ruggedness_score: float

    def to_dict(self) -> Dict:
        return {
            'correlation_length': self.correlation_length,
            'num_local_optima_sampled': self.num_local_optima_sampled,
            'mean_optimum_fitness': self.mean_optimum_fitness,
            'mean_epistasis': self.mean_epistasis,
            'ruggedness_score': self.ruggedness_score
        }


@dataclass
class EvolutionMetrics:
    """Metrics from evolutionary simulation."""
    generations: int
    population_size: int
    final_mean_fitness: float
    final_diversity: float
    num_fixations: int
    model_type: str  # "wright_fisher" or "moran"

    def to_dict(self) -> Dict:
        return {
            'generations': self.generations,
            'population_size': self.population_size,
            'final_mean_fitness': self.final_mean_fitness,
            'final_diversity': self.final_diversity,
            'num_fixations': self.num_fixations,
            'model_type': self.model_type
        }


@dataclass
class VerificationResult:
    """Result of a verification protocol."""
    protocol_name: str
    passed: bool
    message: str
    details: Dict[str, Any] = field(default_factory=dict)

    def to_dict(self) -> Dict:
        return {
            'protocol_name': self.protocol_name,
            'passed': self.passed,
            'message': self.message,
            'details': self.details
        }


@dataclass
class GenotypePhentotypeCertificate:
    """
    Formal certificate for GP mapping analysis.

    Contains all results, metrics, and verification outcomes
    from analyzing a genotype-phenotype system.
    """

    # Identification
    certificate_id: str
    created_at: datetime
    system_description: str

    # Input data
    alphabet: str
    sequence_length: int
    num_sequences_analyzed: int

    # Folding results
    folding_algorithm: str
    folding_results: List[FoldingResult] = field(default_factory=list)

    # Neutral network analysis
    neutral_network_metrics: Optional[NeutralNetworkMetrics] = None

    # Fitness landscape analysis
    fitness_landscape_metrics: Optional[FitnessLandscapeMetrics] = None

    # Evolution simulation
    evolution_metrics: Optional[EvolutionMetrics] = None

    # Verification
    verifications: List[VerificationResult] = field(default_factory=list)

    # Summary
    all_verifications_passed: bool = False

    @classmethod
    def create(cls,
               system_description: str,
               alphabet: str = 'AUGC',
               sequence_length: int = 0) -> 'GenotypePhentotypeCertificate':
        """Create new certificate with generated ID."""
        cert_id = hashlib.sha256(
            f"{system_description}{datetime.now().isoformat()}".encode()
        ).hexdigest()[:16]

        return cls(
            certificate_id=cert_id,
            created_at=datetime.now(),
            system_description=system_description,
            alphabet=alphabet,
            sequence_length=sequence_length,
            num_sequences_analyzed=0,
            folding_algorithm="unspecified"
        )

    def add_folding_result(self, result: FoldingResult) -> None:
        """Add a folding result to the certificate."""
        self.folding_results.append(result)
        self.num_sequences_analyzed = len(self.folding_results)

    def add_verification(self, result: VerificationResult) -> None:
        """Add verification result and update overall status."""
        self.verifications.append(result)
        self.all_verifications_passed = all(
            v.passed for v in self.verifications
        )

    def to_dict(self) -> Dict:
        """Convert certificate to dictionary."""
        return {
            'certificate_id': self.certificate_id,
            'created_at': self.created_at.isoformat(),
            'system_description': self.system_description,
            'alphabet': self.alphabet,
            'sequence_length': self.sequence_length,
            'num_sequences_analyzed': self.num_sequences_analyzed,
            'folding_algorithm': self.folding_algorithm,
            'folding_results': [r.to_dict() for r in self.folding_results],
            'neutral_network_metrics': (
                self.neutral_network_metrics.to_dict()
                if self.neutral_network_metrics else None
            ),
            'fitness_landscape_metrics': (
                self.fitness_landscape_metrics.to_dict()
                if self.fitness_landscape_metrics else None
            ),
            'evolution_metrics': (
                self.evolution_metrics.to_dict()
                if self.evolution_metrics else None
            ),
            'verifications': [v.to_dict() for v in self.verifications],
            'all_verifications_passed': self.all_verifications_passed
        }

    def to_json(self, indent: int = 2) -> str:
        """Serialize certificate to JSON."""
        return json.dumps(self.to_dict(), indent=indent)

    @classmethod
    def from_json(cls, json_str: str) -> 'GenotypePhentotypeCertificate':
        """Deserialize certificate from JSON."""
        data = json.loads(json_str)

        cert = cls(
            certificate_id=data['certificate_id'],
            created_at=datetime.fromisoformat(data['created_at']),
            system_description=data['system_description'],
            alphabet=data['alphabet'],
            sequence_length=data['sequence_length'],
            num_sequences_analyzed=data['num_sequences_analyzed'],
            folding_algorithm=data['folding_algorithm']
        )

        # Reconstruct nested objects
        for fr_data in data.get('folding_results', []):
            cert.folding_results.append(FoldingResult(**fr_data))

        if data.get('neutral_network_metrics'):
            cert.neutral_network_metrics = NeutralNetworkMetrics(
                **data['neutral_network_metrics']
            )

        if data.get('fitness_landscape_metrics'):
            cert.fitness_landscape_metrics = FitnessLandscapeMetrics(
                **data['fitness_landscape_metrics']
            )

        if data.get('evolution_metrics'):
            cert.evolution_metrics = EvolutionMetrics(
                **data['evolution_metrics']
            )

        for v_data in data.get('verifications', []):
            cert.verifications.append(VerificationResult(**v_data))

        cert.all_verifications_passed = data['all_verifications_passed']

        return cert

    def summary(self) -> str:
        """Generate human-readable summary."""
        lines = [
            f"GP Mapping Certificate: {self.certificate_id}",
            f"Created: {self.created_at.strftime('%Y-%m-%d %H:%M:%S')}",
            f"System: {self.system_description}",
            f"",
            f"Analysis Summary:",
            f"  Alphabet: {self.alphabet}",
            f"  Sequence Length: {self.sequence_length}",
            f"  Sequences Analyzed: {self.num_sequences_analyzed}",
            f"  Folding Algorithm: {self.folding_algorithm}",
        ]

        if self.neutral_network_metrics:
            nn = self.neutral_network_metrics
            lines.extend([
                f"",
                f"Neutral Network Metrics:",
                f"  Target Phenotype: {nn.target_phenotype[:30]}...",
                f"  Mean Neutrality: {nn.mean_neutrality:.4f}",
                f"  Estimated Extent: {nn.estimated_extent}",
            ])

        if self.fitness_landscape_metrics:
            fl = self.fitness_landscape_metrics
            lines.extend([
                f"",
                f"Fitness Landscape Metrics:",
                f"  Correlation Length: {fl.correlation_length:.4f}",
                f"  Local Optima Found: {fl.num_local_optima_sampled}",
                f"  Ruggedness Score: {fl.ruggedness_score:.4f}",
            ])

        if self.evolution_metrics:
            ev = self.evolution_metrics
            lines.extend([
                f"",
                f"Evolution Metrics ({ev.model_type}):",
                f"  Generations: {ev.generations}",
                f"  Final Mean Fitness: {ev.final_mean_fitness:.4f}",
                f"  Final Diversity: {ev.final_diversity:.4f}",
            ])

        lines.extend([
            f"",
            f"Verifications: {len(self.verifications)}",
            f"  All Passed: {self.all_verifications_passed}",
        ])

        for v in self.verifications:
            status = "PASS" if v.passed else "FAIL"
            lines.append(f"  [{status}] {v.protocol_name}: {v.message}")

        return '\n'.join(lines)
\end{lstlisting}

%% =============================================================================
%% SECTION 11: VERIFICATION PROTOCOLS
%% =============================================================================
\section{Verification Protocols}

\subsection{Protocol Definitions}

\begin{definition}[Verification Protocol]
A verification protocol $\mathcal{V}$ consists of:
\begin{enumerate}
    \item \textbf{Preconditions}: Required inputs and their formats
    \item \textbf{Test procedure}: Computational steps to verify
    \item \textbf{Success criteria}: Conditions that must hold for verification to pass
    \item \textbf{Failure modes}: Expected failure conditions and diagnostics
\end{enumerate}
\end{definition}

\subsection{Core Verification Protocols}

\begin{lstlisting}[caption={Verification Protocol Implementations}]
from typing import Callable, Tuple
import numpy as np

def verify_folding_consistency(
    fold_function: Callable[[str], str],
    sequences: List[str],
    tolerance: float = 0.0
) -> VerificationResult:
    """
    Verify that folding function is deterministic.

    Protocol: Fold each sequence twice and compare results.
    Success: All pairs match exactly.
    """
    mismatches = []

    for seq in sequences:
        result1 = fold_function(seq)
        result2 = fold_function(seq)

        if result1 != result2:
            mismatches.append({
                'sequence': seq,
                'result1': result1,
                'result2': result2
            })

    passed = len(mismatches) == 0
    message = (
        f"Folding consistent for all {len(sequences)} sequences"
        if passed else
        f"Inconsistency detected in {len(mismatches)} sequences"
    )

    return VerificationResult(
        protocol_name="folding_consistency",
        passed=passed,
        message=message,
        details={'mismatches': mismatches}
    )


def verify_structure_validity(
    structures: List[str],
    min_loop_size: int = 3
) -> VerificationResult:
    """
    Verify that all structures are valid secondary structures.

    Checks:
    - Balanced parentheses
    - No pseudoknots
    - Minimum hairpin loop size
    """
    invalid = []

    for structure in structures:
        issues = []

        # Check balanced parentheses
        stack = []
        for i, char in enumerate(structure):
            if char == '(':
                stack.append(i)
            elif char == ')':
                if not stack:
                    issues.append(f"Unmatched ')' at position {i}")
                else:
                    j = stack.pop()
                    # Check minimum loop size
                    if i - j - 1 < min_loop_size:
                        issues.append(
                            f"Hairpin too small: positions {j}-{i}"
                        )

        if stack:
            issues.append(f"Unmatched '(' at positions {stack}")

        if issues:
            invalid.append({
                'structure': structure,
                'issues': issues
            })

    passed = len(invalid) == 0
    message = (
        f"All {len(structures)} structures valid"
        if passed else
        f"{len(invalid)} invalid structures found"
    )

    return VerificationResult(
        protocol_name="structure_validity",
        passed=passed,
        message=message,
        details={'invalid': invalid}
    )


def verify_neutrality_bounds(
    neutralities: List[float],
    min_bound: float = 0.0,
    max_bound: float = 1.0
) -> VerificationResult:
    """
    Verify neutrality values are within valid bounds.
    """
    violations = [
        n for n in neutralities
        if n < min_bound or n > max_bound
    ]

    passed = len(violations) == 0
    message = (
        f"All {len(neutralities)} neutrality values in [{min_bound}, {max_bound}]"
        if passed else
        f"{len(violations)} values outside bounds"
    )

    return VerificationResult(
        protocol_name="neutrality_bounds",
        passed=passed,
        message=message,
        details={
            'violations': violations,
            'mean': np.mean(neutralities),
            'std': np.std(neutralities)
        }
    )


def verify_fitness_landscape_properties(
    model: 'NKModel',
    num_samples: int = 100
) -> VerificationResult:
    """
    Verify NK model fitness landscape properties.

    Checks:
    - Fitness values in [0, 1]
    - Local optima exist
    - Landscape navigable
    """
    rng = np.random.default_rng()
    issues = []

    # Sample fitness values
    fitness_values = []
    for _ in range(num_samples):
        seq = model.random_sequence(rng)
        fitness = model.fitness(seq)
        fitness_values.append(fitness)

        if fitness < 0 or fitness > 1:
            issues.append(f"Fitness {fitness} outside [0,1]")

    # Verify adaptive walks find optima
    optima_found = 0
    for _ in range(10):
        seq, fitness, steps = model.adaptive_walk(rng=rng)
        if model.is_local_optimum(seq):
            optima_found += 1

    if optima_found < 10:
        issues.append(f"Only {optima_found}/10 walks found local optima")

    passed = len(issues) == 0
    message = (
        f"NK landscape (N={model.n}, K={model.k}) properties verified"
        if passed else
        f"Issues found: {'; '.join(issues[:3])}"
    )

    return VerificationResult(
        protocol_name="fitness_landscape_properties",
        passed=passed,
        message=message,
        details={
            'mean_fitness': np.mean(fitness_values),
            'optima_found': optima_found,
            'issues': issues
        }
    )


def verify_evolution_conservation(
    initial_population: List[str],
    final_population: List[str]
) -> VerificationResult:
    """
    Verify evolution simulation conserved population size.
    """
    passed = len(initial_population) == len(final_population)
    message = (
        f"Population size conserved: {len(final_population)}"
        if passed else
        f"Population size changed: {len(initial_population)} -> {len(final_population)}"
    )

    return VerificationResult(
        protocol_name="evolution_conservation",
        passed=passed,
        message=message,
        details={
            'initial_size': len(initial_population),
            'final_size': len(final_population)
        }
    )


def verify_hamming_distance_metric(
    sequences: List[str]
) -> VerificationResult:
    """
    Verify Hamming distance satisfies metric axioms.
    """
    issues = []

    for i, s1 in enumerate(sequences[:20]):  # Sample
        # Non-negativity and identity
        if hamming_distance(s1, s1) != 0:
            issues.append(f"Self-distance non-zero for {s1}")

        for s2 in sequences[i+1:i+10]:
            d12 = hamming_distance(s1, s2)
            d21 = hamming_distance(s2, s1)

            # Symmetry
            if d12 != d21:
                issues.append(f"Asymmetric: d({s1},{s2})={d12} != d({s2},{s1})={d21}")

            # Triangle inequality (sample)
            for s3 in sequences[i+10:i+15]:
                d13 = hamming_distance(s1, s3)
                d23 = hamming_distance(s2, s3)

                if d12 > d13 + d23:
                    issues.append(
                        f"Triangle violated: d12={d12} > d13={d13} + d23={d23}"
                    )

    passed = len(issues) == 0
    message = (
        "Hamming distance metric axioms verified"
        if passed else
        f"{len(issues)} metric axiom violations"
    )

    return VerificationResult(
        protocol_name="hamming_metric",
        passed=passed,
        message=message,
        details={'violations': issues[:10]}
    )


def run_all_verifications(
    certificate: GenotypePhentotypeCertificate,
    fold_function: Callable[[str], str] = None,
    model: 'NKModel' = None
) -> GenotypePhentotypeCertificate:
    """
    Run all applicable verification protocols.
    """
    # Extract sequences
    sequences = [fr.sequence for fr in certificate.folding_results]
    structures = [fr.structure for fr in certificate.folding_results]

    if sequences:
        # Hamming metric verification
        certificate.add_verification(
            verify_hamming_distance_metric(sequences)
        )

    if structures:
        # Structure validity
        certificate.add_verification(
            verify_structure_validity(structures)
        )

    if fold_function and sequences:
        # Folding consistency
        certificate.add_verification(
            verify_folding_consistency(fold_function, sequences[:50])
        )

    if model:
        # Landscape properties
        certificate.add_verification(
            verify_fitness_landscape_properties(model)
        )

    return certificate
\end{lstlisting}

\subsection{Success Criteria Summary}

\begin{table}[H]
\centering
\caption{Verification Protocol Success Criteria}
\begin{tabular}{lll}
\toprule
\textbf{Protocol} & \textbf{Criterion} & \textbf{Tolerance} \\
\midrule
Folding Consistency & Same output for same input & Exact \\
Structure Validity & Valid dot-bracket notation & None \\
Neutrality Bounds & $0 \leq \rho \leq 1$ & None \\
Fitness Bounds & $0 \leq F \leq 1$ (NK model) & None \\
Local Optima & Adaptive walks terminate & 100\% \\
Hamming Metric & All axioms satisfied & Exact \\
Population Size & Conserved across generations & Exact \\
\bottomrule
\end{tabular}
\end{table}

%% =============================================================================
%% SECTION 12: COMPLETE WORKFLOW EXAMPLE
%% =============================================================================
\section{Complete Analysis Workflow}

\subsection{End-to-End Example}

\begin{lstlisting}[caption={Complete GP Mapping Analysis Workflow}]
#!/usr/bin/env python3
"""
Complete Genotype-Phenotype Mapping Analysis Workflow

This script demonstrates the full analysis pipeline:
1. RNA folding with Nussinov algorithm
2. Neutral network exploration
3. Fitness landscape analysis with NK model
4. Evolutionary simulation
5. Certificate generation and verification
"""

import numpy as np
from typing import List, Dict, Tuple
import json

# Import all components (assuming they're in a module)
# from gp_mapping import *


def main():
    """Run complete GP mapping analysis."""

    print("=" * 60)
    print("GENOTYPE-PHENOTYPE MAPPING ANALYSIS")
    print("=" * 60)

    # =========================================
    # 1. RNA FOLDING ANALYSIS
    # =========================================
    print("\n1. RNA Folding Analysis")
    print("-" * 40)

    # Sample sequences
    test_sequences = [
        "GGGAAAUCC",
        "GCGCAAGCGC",
        "AAAAAAAAAA",
        "GCAUCGAUGC",
        "GGGGAAAACCCC"
    ]

    # Create certificate
    cert = GenotypePhentotypeCertificate.create(
        system_description="RNA Secondary Structure GP Map Analysis",
        alphabet="AUGC",
        sequence_length=10
    )
    cert.folding_algorithm = "nussinov"

    # Fold sequences
    for seq in test_sequences:
        max_pairs, pairs = nussinov_fold(seq)
        structure = structure_to_dot_bracket(len(seq), pairs)

        result = FoldingResult(
            sequence=seq,
            structure=structure,
            energy=None,  # Nussinov doesn't compute energy
            algorithm="nussinov"
        )
        cert.add_folding_result(result)

        print(f"  {seq} -> {structure} ({max_pairs} pairs)")

    # =========================================
    # 2. NEUTRAL NETWORK ANALYSIS
    # =========================================
    print("\n2. Neutral Network Analysis")
    print("-" * 40)

    # Define folding function wrapper
    def fold_to_structure(seq: str) -> str:
        _, pairs = nussinov_fold(seq)
        return structure_to_dot_bracket(len(seq), pairs)

    # Analyze neutral network from first sequence
    seed_seq = test_sequences[0]
    nn_stats = neutral_network_statistics(
        seed_sequence=seed_seq,
        fold_function=fold_to_structure,
        max_samples=200
    )

    cert.neutral_network_metrics = NeutralNetworkMetrics(
        target_phenotype=nn_stats['target_structure'],
        sample_size=nn_stats['samples_collected'],
        mean_neutrality=nn_stats['mean_neutrality'],
        std_neutrality=nn_stats['std_neutrality'],
        estimated_extent=nn_stats['estimated_extent'],
        connectivity_estimate=0.95  # Placeholder
    )

    print(f"  Target structure: {nn_stats['target_structure']}")
    print(f"  Samples collected: {nn_stats['samples_collected']}")
    print(f"  Mean neutrality: {nn_stats['mean_neutrality']:.4f}")
    print(f"  Std neutrality: {nn_stats['std_neutrality']:.4f}")
    print(f"  Estimated extent: {nn_stats['estimated_extent']}")

    # =========================================
    # 3. NK FITNESS LANDSCAPE ANALYSIS
    # =========================================
    print("\n3. NK Model Fitness Landscape")
    print("-" * 40)

    # Create NK models with different K values
    for k in [0, 2, 5]:
        model = NKModel.random(n=20, k=k, seed=42)
        stats = analyze_nk_landscape(n=20, k=k, num_walks=50, seed=42)

        print(f"  K={k}:")
        print(f"    Mean fitness: {stats['mean_fitness']:.4f}")
        print(f"    Walk length: {stats['mean_walk_length']:.1f}")
        print(f"    Unique optima: {stats['num_unique_optima']}")

    # Use K=2 for certificate
    model = NKModel.random(n=20, k=2, seed=42)
    stats = analyze_nk_landscape(n=20, k=2, num_walks=100, seed=42)

    # Compute autocorrelation
    sample_seqs = [''.join(str(x) for x in model.random_sequence())
                   for _ in range(100)]
    autocorr = compute_autocorrelation(
        model.fitness_string,
        sample_seqs,
        max_distance=10
    )
    corr_length = estimate_correlation_length(autocorr)

    cert.fitness_landscape_metrics = FitnessLandscapeMetrics(
        correlation_length=corr_length,
        num_local_optima_sampled=stats['num_unique_optima'],
        mean_optimum_fitness=stats['mean_optimum_fitness'],
        mean_epistasis=0.0,  # Would compute separately
        ruggedness_score=1.0 / corr_length if corr_length > 0 else 1.0
    )

    print(f"\n  Correlation length: {corr_length:.4f}")

    # =========================================
    # 4. EVOLUTIONARY SIMULATION
    # =========================================
    print("\n4. Evolutionary Simulation")
    print("-" * 40)

    # Define fitness function (structure similarity)
    target_structure = "(((....)))"

    def structure_fitness(seq: str) -> float:
        """Fitness based on similarity to target structure."""
        try:
            _, pairs = nussinov_fold(seq)
            actual = structure_to_dot_bracket(len(seq), pairs)

            if len(actual) != len(target_structure):
                return 0.1

            matches = sum(a == t for a, t in zip(actual, target_structure))
            return matches / len(target_structure)
        except:
            return 0.1

    # Run Wright-Fisher evolution
    initial_seq = "AAAAAAAAAA"
    wf_result = wright_fisher_evolution(
        initial_sequence=initial_seq,
        fitness_func=structure_fitness,
        population_size=50,
        generations=100,
        mutation_rate=0.01,
        seed=42
    )

    print(f"  Wright-Fisher Evolution:")
    print(f"    Initial fitness: {wf_result.fitness_history[0]:.4f}")
    print(f"    Final fitness: {wf_result.fitness_history[-1]:.4f}")
    print(f"    Fitness improvement: {wf_result.fitness_history[-1] - wf_result.fitness_history[0]:.4f}")

    cert.evolution_metrics = EvolutionMetrics(
        generations=wf_result.generations,
        population_size=50,
        final_mean_fitness=wf_result.fitness_history[-1],
        final_diversity=wf_result.diversity_history[-1],
        num_fixations=len(wf_result.fixations),
        model_type="wright_fisher"
    )

    # =========================================
    # 5. VERIFICATION
    # =========================================
    print("\n5. Verification Protocols")
    print("-" * 40)

    # Run structure validity check
    structures = [fr.structure for fr in cert.folding_results]
    v1 = verify_structure_validity(structures)
    cert.add_verification(v1)
    print(f"  [{('PASS' if v1.passed else 'FAIL')}] {v1.protocol_name}: {v1.message}")

    # Run folding consistency check
    v2 = verify_folding_consistency(fold_to_structure, test_sequences)
    cert.add_verification(v2)
    print(f"  [{('PASS' if v2.passed else 'FAIL')}] {v2.protocol_name}: {v2.message}")

    # Run Hamming metric check
    v3 = verify_hamming_distance_metric(test_sequences)
    cert.add_verification(v3)
    print(f"  [{('PASS' if v3.passed else 'FAIL')}] {v3.protocol_name}: {v3.message}")

    # Run NK landscape check
    v4 = verify_fitness_landscape_properties(model)
    cert.add_verification(v4)
    print(f"  [{('PASS' if v4.passed else 'FAIL')}] {v4.protocol_name}: {v4.message}")

    # =========================================
    # 6. CERTIFICATE OUTPUT
    # =========================================
    print("\n6. Certificate Summary")
    print("-" * 40)
    print(cert.summary())

    # Save certificate
    cert_json = cert.to_json()
    print(f"\n  Certificate JSON length: {len(cert_json)} bytes")

    return cert


if __name__ == "__main__":
    certificate = main()
    print("\n" + "=" * 60)
    print("ANALYSIS COMPLETE")
    print("=" * 60)
\end{lstlisting}

%% =============================================================================
%% SECTION 13: ADVANCED TOPICS
%% =============================================================================
\section{Advanced Topics}

\subsection{Shape Space Covering}

\begin{theorem}[Shape Space Covering]
For RNA sequences of length $\seqlength$, neutral networks of common structures
cover shape space such that any phenotype is within a few mutations of any
sufficiently large neutral network.
\end{theorem}

\begin{physicsbox}
\textbf{Covering Radius:}

The covering radius $r_c$ is the minimum radius such that:
\begin{equation}
    \bigcup_{s \in \neutralnet(\phi)} B_r(s) \supseteq \genotype \quad \text{for all } r \geq r_c
\end{equation}
where $B_r(s) = \{s' : \hamming(s, s') \leq r\}$.

For RNA with $\seqlength \approx 100$, empirically $r_c \approx 15-20$.
\end{physicsbox}

\subsection{Arrival of the Frequent}

\begin{definition}[Arrival of the Frequent]
Evolution tends to discover phenotypes with high genotypic frequency:
\begin{equation}
    \prob(\text{reach } \phi) \propto f(\phi)^{\alpha}
\end{equation}
where $f(\phi)$ is the phenotype frequency and $\alpha > 0$ depends on
population parameters.
\end{definition}

\begin{annotation}
This principle explains why certain structures evolve repeatedly in independent
lineages (convergent evolution) - they have many genotypes mapping to them,
making them more likely to be discovered.
\end{annotation}

\subsection{Mutational Load}

\begin{physicsbox}
\textbf{Haldane's Principle:}

At mutation-selection balance, the mean fitness reduction (load) is:
\begin{equation}
    L = 1 - \bar{w}/w_{\max} \approx \mu \seqlength
\end{equation}
where $\mu$ is the per-site mutation rate.

\textbf{Error Threshold:}

For RNA viruses with mutation rate $\mu$:
\begin{equation}
    \mu_{\text{crit}} \approx \frac{\ln(s)}{(\seqlength - \seqlength_0)}
\end{equation}
where $s$ is the selective advantage and $\seqlength_0$ is the number of
neutral sites.
\end{physicsbox}

%% =============================================================================
%% SECTION 14: CONCLUSION
%% =============================================================================
\section{Conclusion}

\subsection{Summary of Key Results}

This report has presented a comprehensive treatment of genotype-phenotype mapping
and evolutionary fitness landscapes:

\begin{enumerate}
    \item \textbf{GP Map Fundamentals}: The mapping from genotype space
          $\alphabet^{\seqlength}$ to phenotype space creates structure that
          fundamentally shapes evolutionary dynamics.

    \item \textbf{RNA Secondary Structure}: Provides an ideal model system with
          tractable algorithms (Nussinov, Zuker) enabling detailed study of GP
          map properties.

    \item \textbf{Neutral Networks}: Extended networks enabling drift across
          sequence space while maintaining phenotype, resolving the apparent
          paradox between robustness and evolvability.

    \item \textbf{Fitness Landscapes}: Characterized by ruggedness metrics
          (autocorrelation, epistasis, local optima density) with the NK model
          providing tunable complexity.

    \item \textbf{Evolutionary Dynamics}: Wright-Fisher and Moran models capture
          population-level evolution with mathematically tractable fixation
          probabilities.

    \item \textbf{Formal Verification}: Certificate-based approach ensures
          correctness of computational analyses.
\end{enumerate}

\subsection{Future Directions}

\begin{pursuitbox}
Key open questions in GP mapping research:
\begin{itemize}
    \item How do GP map properties scale with system complexity?
    \item Can we design GP maps with desired evolutionary properties?
    \item What role do GP maps play in major evolutionary transitions?
    \item How can GP map understanding improve directed evolution?
\end{itemize}
\end{pursuitbox}

\subsection{Practical Applications}

The theoretical framework developed here has applications in:
\begin{itemize}
    \item \textbf{Directed evolution}: Understanding landscape structure guides
          experimental design
    \item \textbf{Drug resistance}: Predicting evolutionary escape routes
    \item \textbf{Synthetic biology}: Designing robust genetic circuits
    \item \textbf{Machine learning}: Optimization in rugged landscapes
\end{itemize}

%% =============================================================================
%% APPENDICES
%% =============================================================================
\appendix

\section{Mathematical Notation Reference}

\begin{table}[H]
\centering
\caption{Symbol Reference Table}
\begin{tabular}{lll}
\toprule
\textbf{Symbol} & \textbf{Meaning} & \textbf{Type} \\
\midrule
$\genotype$ & Genotype space & Set \\
$\phenotype$ & Phenotype space & Set \\
$\gpmap$ & GP map function & $\genotype \to \phenotype$ \\
$\neutralnet(\phi)$ & Neutral network of $\phi$ & Set \\
$\rho(s)$ & Neutrality of sequence $s$ & $[0,1]$ \\
$\fitness$ & Fitness function & $\genotype \to \real$ \\
$\hamming$ & Hamming distance & $\genotype \times \genotype \to \nat$ \\
$\seqlength$ & Sequence length & $\nat$ \\
$\alphabet$ & Alphabet & Set \\
$N$ & Population size & $\nat$ \\
$K$ & NK model epistasis & $\nat$ \\
$\mu$ & Mutation rate & $[0,1]$ \\
$s$ & Selection coefficient & $\real$ \\
$\pi$ & Fixation probability & $[0,1]$ \\
$\Delta G$ & Free energy & $\real$ (kcal/mol) \\
\bottomrule
\end{tabular}
\end{table}

\section{Algorithm Complexity Summary}

\begin{table}[H]
\centering
\caption{Algorithm Complexity}
\begin{tabular}{lcc}
\toprule
\textbf{Algorithm} & \textbf{Time} & \textbf{Space} \\
\midrule
Nussinov & $\complexity(\seqlength^3)$ & $\complexity(\seqlength^2)$ \\
Zuker (standard) & $\complexity(\seqlength^4)$ & $\complexity(\seqlength^2)$ \\
Zuker (optimized) & $\complexity(\seqlength^3)$ & $\complexity(\seqlength^2)$ \\
NK fitness evaluation & $\complexity(N)$ & $\complexity(N \cdot 2^{K+1})$ \\
Adaptive walk & $\complexity(N^2 \cdot T)$ & $\complexity(N)$ \\
Wright-Fisher generation & $\complexity(N)$ & $\complexity(N)$ \\
Neutral network sampling & $\complexity(M \cdot N \cdot |\alphabet|)$ & $\complexity(M)$ \\
\bottomrule
\end{tabular}
\end{table}

\section{Energy Parameters}

\begin{table}[H]
\centering
\caption{Sample Turner Energy Parameters (kcal/mol at 37C)}
\begin{tabular}{lcccc}
\toprule
& \textbf{AU} & \textbf{CG} & \textbf{GC} & \textbf{GU} \\
\midrule
AU & -0.9 & -2.2 & -2.1 & -0.6 \\
CG & -2.1 & -3.3 & -2.4 & -1.4 \\
GC & -2.4 & -3.4 & -3.3 & -1.5 \\
GU & -1.3 & -2.5 & -2.1 & -0.5 \\
\bottomrule
\end{tabular}
\end{table}

%% =============================================================================
%% BIBLIOGRAPHY
%% =============================================================================
\section*{References}

\begin{enumerate}
    \item Nussinov, R., et al. (1978). Algorithms for loop matchings.
          \textit{SIAM Journal on Applied Mathematics}, 35(1), 68-82.

    \item Zuker, M., \& Stiegler, P. (1981). Optimal computer folding of large
          RNA sequences using thermodynamics and auxiliary information.
          \textit{Nucleic Acids Research}, 9(1), 133-148.

    \item Kauffman, S. A. (1987). Towards a general theory of adaptive walks on
          rugged landscapes. \textit{Journal of Theoretical Biology}, 128(1), 11-45.

    \item Schuster, P., et al. (1994). From sequences to shapes and back: a case
          study in RNA secondary structures. \textit{Proceedings of the Royal
          Society B}, 255(1344), 279-284.

    \item Wagner, A. (2008). Robustness and evolvability: a paradox resolved.
          \textit{Proceedings of the Royal Society B}, 275(1630), 91-100.

    \item Kimura, M. (1962). On the probability of fixation of mutant genes in a
          population. \textit{Genetics}, 47(6), 713-719.

    \item Wright, S. (1932). The roles of mutation, inbreeding, crossbreeding and
          selection in evolution. \textit{Proceedings of the Sixth International
          Congress of Genetics}, 1, 356-366.

    \item Moran, P. A. P. (1958). Random processes in genetics.
          \textit{Mathematical Proceedings of the Cambridge Philosophical Society},
          54(1), 60-71.

    \item Stadler, P. F. (2002). Fitness landscapes.
          \textit{Biological Evolution and Statistical Physics}, 183-204.

    \item Dingle, K., et al. (2015). The structure of the genotype-phenotype map
          strongly constrains the evolution of non-coding RNA.
          \textit{Interface Focus}, 5(6), 20150053.
\end{enumerate}

\end{document}
